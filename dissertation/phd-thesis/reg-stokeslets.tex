%%% -*-LaTeX-*-

\chapter{Three-Dimensional Rolling Model}
\label{cha:three-dimens-roll}

The previous models presented greatly simplify the physics and
geometry of an ellipsoidal platelet rolling in a three-dimensional
shear flow. In the 2D rolling model, the platelet is modeled as a
sphere with a thin reactive strip around the circumference. Two axes
of motion and two axes of rotation are completely discarded, along
with the relevant forces and torques. In the jump-velocity model, all
platelet geometry and all fluid forces are ignored completely, and the
bond dynamics of many receptors and ligands are simplified to simple
binding and unbinding rates.

In this chapter, we develop a fully 3D adhesive dynamics model, and
present results from this model. This model is too computationally
intensive to use for true parameter estimation from experimental data,
but of the three models presented here it is the most valuable for
developing intuition of the complex forces and motions involved in the
rolling dynamics of platelets.

\section{Model Description}
\label{sec:model-description}

In general, adhesive dynamics (AD) models describe the motion of cells
in a Stokes flow due to binding and unbinding with a ligand-coated
wall. There are two major components in AD models:
\begin{enumerate}
\item Bond dynamics---individual receptors are modeled separately,
  their positions are tracked, and they can bind and unbind
  independently of other receptors, and
\item Fluid-structure interaction---modeling the drag forces generated
  by the moving fluid on an immersed body.
\end{enumerate}
Putting these components together, AD models give us a way to
translate bond-level kinetics to cell-level behavior.

In developing this model, we focus on the forces acting on the
platelet. While there is some flow field around the platelet, we are
not interested in what this flow field actually looks like; we are
only interested in its effect on the platelet.

We consider three generalized forces which act on the platelet and
affect its motion:
\begin{enumerate}
\item $\hydrodynamicForces$---the hydrodynamic force on the cell, 
\item $\bindingForces$---the force due to binding with ligands on the wall, and
\item $\bodyForces$---various body forces acting on the cell
  (e.g. electrostatic and steric forces).
\end{enumerate}
Because we assume the flow field is a Stokes flow (i.e., no inertia is
present in the system), all of these forces balance:
$\hydrodynamicForces + \bindingForces + \bodyForces = \vect{0}$. These
vectors all represent generalized forces, so all of them are elements
of $\reals^6$; three components represent forces in each of the
coordinate directions, and three components represent torques about
the three coordinate axes.

$\bindingForces$ depends on the location and orientation of bonds
between the cell and wall at time $t$, and $\bodyForces$ depends on
the location and orientation of the cell. The hydrodynamic forces
$\hydrodynamicForces$ depend both on the location and orientation of
the cell $\pltPosition$, and also the translational and angular
velocities $\pltPosition'$. Therefore we can use this relation to find
$\pltPosition'(t)$ at each time step, and update the position and
orientation of the cell accordingly. Similar to the generalized
forces, $\pltPosition \in \reals^6$. Three components give the
position of the platelet's center of mass, and three components give
the three orientation angles of the platelet.

\subsection{Definitions and Domain}
\label{sec:definitions-domain}

We want to model the motion of a platelet (i.e. an ellipsoid) near a
plane wall, so the fluid domain we use is the upper half-space
$\fluidDomain = \{(\rsX, \rsY, \rsZ) \in \reals \mid \rsX > 0\}$. The wall
is represented by a plane at $\rsX = 0$. The background flow
$\flowField^\infty$ is a linear shear flow with a shear rate of
$\shear$: $\flowField^\infty = \shear \rsX \e{\rsZ}$.

The platelet is modeled as an ellipsoid with axes lengths of
$1.5 \mu m \times 1.5 \mu m \times 0.5 \mu m$, and the position of its
center of mass is $\pltCOM$. There are a couple of different ways to
represent the orientation of the platelet, however the simplest way to
represent it is as a unit vector pointing in the direction of the
platelet's minor axis: $\e{m}$. This uniquely defines an orientation
because the platelet is rotationally symmetric about its minor
axis. This simplification is not able to represent rotations around
the minor axis. This isn't an issue while the platelet is still
rotationally symmetric, however if there is something to break the
symmetry (like a non-symmetrical distribution of receptors for
example), this representation can be augmented into a $3 \times 3$
rotation matrix where each column represents one of the vectors in an
orthonormal coordinate system that rotates with the platelet.

Finally, as mentioned above we use Stokes equations to model the flow
field, with no-slip boundary conditions on the platelet surface and
the wall, and a far-field condition:
\begin{align}
  \Delta \flowField = \nabla \pressure, \quad \nabla \cdot \flowField
  = 0 \label{eq:stokes-eqn} \\
  \flowField(\arbLoc)|_{\partial\mathcal{P}} = \pltVelocity +
  \pltRotation \times \arbLoc \label{eq:plt-bc} \\
  \flowField(\arbLoc)|_{\rsX = 0} =
  \vect{0} \label{eq:wall-bc} \\
  \flowField(\arbLoc)|_{\|\arbLoc\| \rightarrow \infty}
  \rightarrow \flowField^\infty. \label{eq:far-field}
\end{align}

\subsection{Equations of motion}
\label{sec:equations-motion}

Let us put aside the issue of solving equations
(\ref{eq:stokes-eqn})---(\ref{eq:far-field}) for the moment, and focus
on the main structure of the model. We are interested in the motion of
the platelet, and this can be found by solving the following system of
ODEs; the equations of rigid body motion for the platelet:
\begin{align}
  \Der{\pltCOM}{t} &= \pltVelocity(\pltCOM,
                     \e{m}) \label{eq:plt-velocity} \\
  \Der{\e{m}}{t} &= \pltRotation(\pltCOM,
                   \e{m}). \label{eq:plt-rotation} 
\end{align}

We haven't defined the functions for the platelet's translational
($\pltVelocity$) and angular ($\pltRotation$) velocities yet, and all
of the numerical complexity in solving this model is wrapped up in
these functions. These velocity functions are found by enforcing force
balance on the platelet. For a given platelet position, orientation,
and set of bonds, the body forces $\bodyForces$ and bond forces
$\bindingForces$ are fixed. However the hydrodynamic forces
$\hydrodynamicForces$ exerted by the fluid on the cell depend on
$\pltVelocity$ and $\pltRotation$, as well as the position and
orientation. Therefore, we will choose the values of $\pltVelocity$
and $\pltRotation$ so that the hydrodynamic forces balance with the
other forces acting on the cell: $\hydrodynamicForces + \bodyForces +
\bindingForces = \vect{0}$.

First, let us look at $\hydrodynamicForces$ more closely. For a given
velocity field $\flowField$, the stress tensor is defined as
$\stressTensor = -p \uu{I} + \viscosity \left(\nabla \flowField +
  \left(\nabla \flowField\right)^T\right)$, and the body forces and
torques can be found, respectively, as:
\begin{align}
  \left[\hydrodynamicForces\right]_{1:3} &= \Int{\stressTensor \cdot
  \mathbf{n}}{s(\arbLoc)}{\pltSurf}{} \label{eq:body-force} \\
  \left[\hydrodynamicForces\right]_{4:6} &= \Int{(\arbLoc - \pltCOM)
  \times \stressTensor \cdot \mathbf{n}}{s(\arbLoc)}{\partial
  P}{}. \label{eq:body-torque} 
\end{align}

As mentioned in Section \ref{sec:definitions-domain},
$\hydrodynamicForces$ is generated by a background flow of
$\flowField^\infty = \shear \rsX \e{\rsZ}$. Because of the linearity
of Stokes' flow, we can decompose $\hydrodynamicForces$ into two
pieces:
\begin{itemize}
\item $\dragForces$---the drag force and torque exerted on a moving
  platelet in a stationary flow, and
\item $\shearForces$---the drag force and torque exerted on a
  stationary platelet in a background shear flow.
\end{itemize}
Again using linearity of Stokes' flow, there is a linear relation
between $\dragForces$ and the generalized velocity $\genVelocity$ of
the platelet (where
$\genVelocity = (\pltVelocity, \pltRotation)^T \in \reals^6$):
\begin{equation}
  \label{eq:resistance-matrix}
  \dragForces = \resMatrix \genVelocity.
\end{equation}
The matrix $\resMatrix$ is called the \emph{resistance tensor} and
depends only on the position, orientation, and shape of the
platelet. In order to find the resistance tensor, we need to solve 6
Stokes' flow problems with the following boundary conditions on the
platelet:
\begin{align}
  \flowField_i |_{\arbLoc \in \pltSurf} &= \delta_{ij} \e{j} \tn{
  for } i = 1, 2, 3, \tn{ and} \label{eq:translation-stokes-bcs} \\
  \flowField_i |_{\arbLoc \in \pltSurf} &= \delta_{i-3, j} \e{j}
  \times \arbLoc \tn{ for } i = 4, 5,
  6. \label{eq:rotation-stokes-bcs}
\end{align}
For the sake of completeness, the exact system being solved is the
equations (\ref{eq:stokes-eqn}), (\ref{eq:wall-bc}), and
\begin{equation}
  \flowField_i |_{\|\arbLoc\| \rightarrow \infty} \rightarrow
  \vect{0}. \label{eq:far-field-drag}
\end{equation}
with either (\ref{eq:translation-stokes-bcs}) or
(\ref{eq:rotation-stokes-bcs}) depending on $i$.

With the boundary conditions specified in equations
(\ref{eq:translation-stokes-bcs}) and (\ref{eq:rotation-stokes-bcs}),
we have six different generalized velocities: $\genVelocity_i = \e{i}$
($\e{i}$ is the $i$th canonical basis vector in $\reals^6$). Therefore
to construct $\resMatrix$, for each $\genVelocity_i$, we solve for
$v_i$ and then compute the drag forces and torques (combined in the
generalized force vector $\dragForces^i$) on the platelet using
equations (\ref{eq:body-force}) and (\ref{eq:body-torque}). Because
$\genVelocity_i = \e{i}$, the $i$ column of $\resMatrix$ is exactly
$\dragForces^i$.

The drag force $\shearForces$ due to the shear background flow is
found with a 7th solve of Stokes' equations, with the boundary and
far-field conditions
\begin{align}
  &\flowField_7 |_{\arbLoc \in \pltSurf} =
  \vect{0} \label{eq:plt-bcs-shear} \\
  &\flowField_7 |_{\|\arbLoc\| \rightarrow \infty} \rightarrow
  \shear \rsX \e{\rsZ}. \label{eq:plt-far-field-shear}
\end{align}

Putting everything together and assuming for the moment that
$\bindingForces$ and $\bodyForces$ are given, we require that
$\hydrodynamicForces + \bindingForces + \bodyForces = 0$. We can
express $\hydrodynamicForces$ as a sum of the stationary drag forces
on the platelet and the drag from the background shear flow on the
platelet: $\hydrodynamicForces = \dragForces + \shearForces =
\resMatrix \genVelocity + \shearForces$. Therefore, $\resMatrix
\genVelocity + \shearForces + \bindingForces + \bodyForces = 0$, and
solving for $\genVelocity$ we get
\begin{equation}
  \label{eq:gen-velocity-equation}
  \genVelocity = -\resMatrix\inv \left(\shearForces + \bindingForces +
    \dragForces \right).
\end{equation}

\subsubsection{Numerical method for solving Stokes' equations}
\label{sec:numer-meth-solv}

We use the method of regularized Stokeslets to solve Stokes' equations
for the different boundary conditions specified above. This is an
approach developed by Ricardo Cortez to solve fluid-structure
interaction problems in Stokes' flows which distributes regularized
point forces over the surface of the structure being modeled in order
to communicate forces between the fluid and the structure. Basically
it is a regularized Green's function approach to solving Stokes' flow.

Below, I will describe the method of regularized Stokeslets for an
unbounded flow in three dimensions. This is a simpler case than
solving Stokes' equations in the half-space with $\vect{0}$
boundary conditions on the wall, but nonetheless it has many of the
same features. This description closely follows the derivations laid
out in \cite{Cortez2001} and \cite{Cortez2005}. %Give examples?

Similar to a Green's function approach, the method of regularized
Stokeslets relies on the linearity of Stokes' equations by exploiting
the superposition principle for linear PDEs. In a true Green's
function method (i.e., the method of Stokeslets), one would solve
Stokes' equations with a singular point force:
\begin{equation}
  \label{eq:stokes-with-pt-force}
  \Delta \flowField - \nabla \pressure = -\arbForce \delta(\arbLoc
  - \evalLoc), \quad \nabla \cdot \flowField = \vect{0}.
\end{equation}
These equations have an exact solution---called the Stokeslet---and
can be used to solve Stokes' equations for an arbitrary right-hand
side.

The equation that forms the basis of the method of regularized
Stokeslets is very similar, but with the point force replaced by a
regularized point force:
\begin{equation}
  \label{eq:stokes-with-reg-force}
  \Delta \flowField - \nabla \pressure = -\arbForce
  \blobFun{\blobPar}(\arbLoc - \blobLoc), \quad \nabla \cdot
  \flowField = \vect{0}.
\end{equation}
The function $\blobFun{\blobPar}$ is usually called a ``blob
function'' or ``cutoff function,'' and $\blobPar$ is the ``blob
parameter.'' The blob function is a regularized approximation to the
$\delta$ function, and satisfies three criteria:
\begin{enumerate}
\item $\blobFun{\blobPar}$ is radially symmetric and smooth, 
\item $\Int{\blobFun{\blobPar}(\arbLoc)}{\arbLoc}{}{} = 1$ for all
  $\blobPar$, and
\item $\blobFun{\blobPar} \rightarrow \delta$ as $\blobPar \rightarrow
  0$.
\end{enumerate}
There are of course many different possible functions which satisfy
these criteria. The specific blob function we use is
\begin{equation}
  \label{eq:blob-function}
  \blobFun{\blobPar}(\arbLoc) = \frac{15 \blobPar^4}{8 \pi
    \left(\|\arbLoc\|^2 + \blobPar^2 \right)^{7/2}}.
\end{equation}
Then we can write the solution to equation
(\ref{eq:stokes-with-reg-force}) as
\begin{equation}
  \label{eq:solution-to-reg-force}
  \flowField(\arbLoc) = \frac{1}{8\pi \viscosity}
  \regStokeslet{\blobPar} (\arbLoc, \blobLoc) \arbForce
\end{equation}
where $\regStokeslet{\blobPar}$ is the regularized Stokeslet. For the
blob function defined in equation (\ref{eq:blob-function}), the
regularized Stokeslet is
\begin{equation}
  \label{eq:reg-stokeslet}
  \regStokeslet{\blobPar}(\arbLoc, \blobLoc) = \frac{\diffLoc^T \diffLoc +
    2\blobPar^2}{\left(\diffLoc^T \diffLoc + \blobPar^2 \right)^{3/2}}
  \uu{I} + \frac{\diffLoc \diffLoc^T}{\left(\diffLoc^T \diffLoc +
      \blobPar^2 \right)^{3/2}},
\end{equation}
where $\diffLoc = \arbLoc - \blobLoc$. In the limit $\blobPar
\rightarrow 0$, $\regStokeslet{\blobPar}$ approaches the singular
solution of Stokes' equations.

It is possible to derive the following boundary integral equation for
the solution of Stokes' equations around a body $\pltBody$
\cite{Cortez2005}:
\begin{equation}
  \label{eq:boundary-integral-equation}
  \Int{\flowField(\arbLoc) \blobFun{\blobPar}(\arbLoc -
    \evalLoc)}{V(\arbLoc)}{\reals^3}{} = \frac{1}{8\pi \viscosity}
  \Int{\regStokeslet{\blobPar}(\arbLoc, \blobLoc)
    \arbForce}{s(\arbLoc)}{\pltSurf}{}. 
\end{equation}
This equation is used to derive the method of regularized Stokeslets,
and two approximations are made to arrive at the numerical method. The
first approximation is to replace $\blobFun{\blobPar}$ with a singular
$\delta$-function, which eliminates the integral on the left-hand
side. The second approximation is that the integral on the right hand
side will be discretized and approximated with a quadrature
rule. These approximations give the equation
\begin{equation}
  \label{eq:method-reg-stokeslets}
  \flowField(\evalLoc) = \frac{1}{8\pi\viscosity} \sum_{n = 1}^N
  \regStokeslet{\blobPar} (\arbLoc_n, \evalLoc) \arbForce_n A_n,
\end{equation}
where $N$ is the number of Stokeslets placed on $\pltSurf$,
$\arbLoc_n$ is the location of the $n$th Stokeslet, $\arbForce_n$ is
the force vector applied at $\arbLoc_n$, and $A_n$ is the quadrature
weight of the $n$th Stokeslet.

In this method, we pick the location of each of the point forces, and
the quadrature rule used to approximate the right-hand side integral
in equation (\ref{eq:boundary-integral-equation}). The unknowns are
the Stokeslet strengths $\arbForce_n$. Because the boundary
conditions on $\pltSurf$ are known, we can choose a set of $\evalLoc$
on $\pltSurf$ and derive $\flowField(\evalLoc)$ from the boundary
conditions. By choosing $\arbLoc_n$ and $N$ evaluation points
$\evalLoc \in \pltSurf$, equation (\ref{eq:method-reg-stokeslets}) can
be written as a $3N \times 3N$ linear system:
\begin{equation}
  \label{eq:rs-linear-system}
  \hat{\flowField} = \frac{1}{8\pi\viscosity} \mathcal{A}
  \hat{\arbForce},
\end{equation}
where $\hat{\flowField} = (\flowField(\evalLoc^1),
\flowField(\evalLoc^2), \hdots, \flowField(\evalLoc^N))^T$,
$\hat{\arbForce} = (\arbForce(\arbLoc_1), \arbForce(\arbLoc_2),
\hdots, \arbForce(\arbLoc_N))^T$, and $\mathcal{A}$ is a $3N \times
3N$ matrix which depends on $\regStokeslet{\blobPar}$ and $A_n$.

We take the evaluation points $\evalLoc^n$ to be the same as the force
locations; so $\evalLoc^n = \arbLoc_n$, but other choices could be
made as well. Therefore in order to find the Stokeslet strengths
$\hat{\arbForce}$ for a given choice of boundary conditions
$\hat{\flowField}$, we have to invert $\mathcal{A}$, and because we
need to solve for $\hat{\arbForce}$ for seven different
$\hat{\flowField}$s, we compute the Cholesky factorization of
$\mathcal{A}$ so that we can re-use this work when solving the seven
linear systems.

Thus far I have glossed over some details regarding the solution of
the shear drag, that is, solving Stokes' equations with the conditions
(\ref{eq:plt-bcs-shear}) and (\ref{eq:plt-far-field-shear}). The
regularized Stokeslets vanish as $\|\arbLoc\| \rightarrow \infty$,
which is not consistent with the far-field condition
(\ref{eq:plt-far-field-shear}). Therefore, following Pozrikidis
\cite{Pozrikidis92}, we decompose $\flowField_7$ into the background
flow $\flowField^\infty$ and a disturbance flow $\flowField_d$, which
represents the disturbance to the background flow from the
platelet. The background flow $\flowField^\infty$ satifies Stokes'
equations and the boundary condition at the wall, and the disturbance
field $\flowField_d$ satisfies Stokes equations with the modified
boundary conditions:
\begin{equation}
  \label{eq:disturbance-bcs}
  \flowField_d(\arbLoc) = -\flowField^\infty(\arbLoc) \tn{ for }
  \arbLoc \in \pltSurf, \quad \flowField_d \rightarrow \vect{0} \tn{
    for } \|\arbLoc\| \rightarrow \infty.
\end{equation}
Now $\flowField_d$ can be found using regularized Stokeslets following
the same process described above, and the solution can be integrated
to find $\vect{F}_d$.

Most of this section makes no assumption on the domain of the
fluid. The theory laid out here applies for an arbitrary domain, and
to solve the Stokes' problems in different domains one only needs to
change the form of the regularized Stokeslet
$\regStokeslet{\blobPar}$. Equation (\ref{eq:reg-stokeslet}) is the
solution to Stokes' equations in $\reals^3$ with $\blobFun{}$ as in
equation (\ref{eq:blob-function}). In our case, we are solving Stokes'
equations in the upper half-space, with no slip boundary conditions at
the plane wall at $\rsX = 0$. Therefore we need to use a regularized
Stokeslet which satisfies Stokes' equations in that domain with the
necessary boundary conditions. The precise form of that regularized
Stokeslet is developed in \cite{Ainley2008}, and is constructed with a
linear combination of regularized Stokeslets and ``higher-order''
Stokeslets found by taking derivatives of the regularized
Stokeslet. This linear combination is called an \emph{image
  system}. With $\blobFun{\blobPar}$ as in (\ref{eq:blob-function}),
the appropriate image system is
\begin{multline}
  \label{eq:image-system}
  \regStokeslet{\blobPar}(\imEval, \blobLoc) = H_1(r^*) \uu{I} +
  \arbLoc^* (\arbLoc^*)^T H_2(r^*) - H_1(r) \uu{I} - \arbLoc \arbLoc^T
  H_2(r) - h^2 \left[ D_1(r) \uu{I} + \arbLoc \arbLoc^T D_2(r) \right] \uu{D} \\
  + 2h \left[ \left(\e{1} \arbLoc^T + \e{1}^T \arbLoc \uu{I}\right)
    H_2(r) + \left(\e{1} H_1'(r) + (\e{1}^T \arbLoc) \arbLoc H_2'(r)
    \right) \frac{\arbLoc^T}{r} \right] \uu{D} \\
  - 2h \left(\frac{H_1'(r)}{r} + H_2(r)\right) \left( \uu{I} \times
    \arbLoc \right) \uu{E},
\end{multline}
where $\blobLoc = (h, y, z)^T$ is the location of the regularized
Stokeslet, $\imLoc = (-h, y, z)^T$ is the location of the image points,
and the variables and functions are defined as follows:
\begin{align*}
  & \arbLoc^* = \imEval - \blobLoc, \quad \arbLoc = \imEval - \imLoc, 
  \quad r^* = \|\arbLoc^*\|, \quad r = \|\arbLoc\|, \\
  H_1(r) &= \frac{1}{8\pi \left(r^2 + \blobPar^2 \right)^{1/2}} +
           \frac{\blobPar^2}{8\pi \left(r^2 + \blobPar^2
           \right)^{3/2}}, \\
  H_2(r) &= \frac{1}{8\pi \left(r^2 + \blobPar^2 \right)^{3/2}}, \\
  D_1(r) &= \frac{1}{4\pi \left(r^2 + \blobPar^2 \right)^{3/2}} -
           \frac{3\blobPar^2}{4\pi \left(r^2 + \blobPar^2
           \right)^{5/2}} \\
  D_2(r) &= -\frac{3}{4\pi \left(r^2 + \blobPar^2 \right)^{5/2}} \\
  \uu{D} &=
           \begin{pmatrix}
             1 & 0 & 0 \\
             0 & -1 & 0 \\
             0 & 0 & -1
           \end{pmatrix} \\
  \uu{E} &=
           \begin{pmatrix}
             0 & 0 & 0 \\
             0 & 0 & 1 \\
             0 & -1 & 0
           \end{pmatrix}
\end{align*}

Briefly, this image system is constructed by first placing a
regularized Stokeslet at the Stokeslet location $\blobLoc$ in the
fluid domain. The flow from this regularized Stokeslet is captured in
the $H_1(r^*) \uu{I} + \arbLoc^* (\arbLoc^*)^T H_2(r^*)$ terms in
equation (\ref{eq:image-system}). By itself, these terms give the
regularized Stokeslet in $\reals^3$. However, we require the flow to
vanish at the wall, so we need additional terms to cancel out the flow
there and satisfy the boundary conditions. So next, a regularized
Stokeslet is placed at the image point $\imLoc$, and its contribution
to the flow is subtracted from the original regularized
Stokeslet. This is captured in the $-H_1(r) \uu{I} + \arbLoc \arbLoc^T
H_2(r)$ terms in equation (\ref{eq:image-system}). These two terms by
themselves are not sufficient to cancel the flow at the wall, and the
remaining terms are a combination of higher-order regularized
Stokeslets placed at the image point that completely cancel out the
remaining parts of the flow at the wall.

\subsection{Binding dynamics}
\label{sec:binding-dynamics}

With the fluid dynamics component of the model taken care of, let's
now add the binding and unbinding component of the model. The
assumptions made in this model are similar to those made in the 2D
rolling model developed in Chapter \ref{cha:two-dimens-roll}. We
assume that there is a finite number of receptors on the surface of
the cell, and we track their locations explicitly. We assume the
receptors have a fixed location relative to the surface of the cell,
so that the motion of every receptor is completely determined by the
motion of the platelet. Ligands on the wall are assumed to exist in
excess, and ligands are not tracked explicitly.

We assume that bonds have some nonzero rest length $\restLength$. When
the bond length is greater than $\restLength$, the bond acts like a
linear spring, and when the bond length is less than $\restLength$, no
force is exerted by the bond. Therefore the function defining the bond
force exerted by a single bond is
\begin{equation}
  \label{eq:single-bond-force}
  \bindingForces^i = \stiffness H(\|\arbLoc_l^i - \arbLoc_r^i\| -
  \restLength) \frac{\arbLoc_l^i - \arbLoc_r^i}{\|\arbLoc_l^i -
    \arbLoc_r^i\|},
\end{equation}
where $\arbLoc_l^i$ is the point where the bond attaches to a ligand
on the wall, and $\arbLoc_r^i$ is the location of the bound
receptor. Here $H$ is the Heaviside step function:
\[H(x) =
\begin{cases}
  0 & x \le 0 \\
  1 & x > 0
\end{cases}.\]
In order to find the total bond force and torque exerted on the
platelet, we compute the following sums:
\begin{align}
  \left[\bindingForces\right]_{1:3} = \sum_{i = 1}^{N_\tn{bonds}}
  \bindingForces^i \label{eq:total-bond-force} \\
  \left[\bindingForces\right]_{4:6} = \sum_{i = 1}^{N_\tn{bonds}}
  (\arbLoc_r^i - \pltCOM) \times \bindingForces^i.
\end{align}

The bond formation ($\onRate(L)$) and breaking ($\offRate(L)$) rates
are distance/length-dependent. We use the Dembo model \cite{Dembo1988}
for these rates:
\begin{align}
  &\onRate(L) = \onRate^0 \exp\left[\frac{-\stiffness^\tn{ts} \left(L -
  \restLength \right)^2}{2 \boltzmann \temp}
  \right] \label{eq:dembo-on} \\
  &\offRate(L) = \offRate^0 \exp\left[\frac{\left(\stiffness -
  \stiffness^\tn{ts} \right) \left(L - \restLength \right)^2}{2
  \boltzmann \temp} \right]. \label{eq:dembo-off}
\end{align}

These rates are derived in \cite{Bell1984} and \cite{Dembo1988} using
the free energy of a Hookean spring. The dissociation constant of a
chemical reaction is related to the change in free energy by the
equation
\begin{equation}
K_\tn{eq} = \exp\left[\frac{\Delta G^0}{\boltzmann
    T}\right], \label{eq:dissociation-const}
\end{equation}
where $\Delta G^0$ is the difference in the standard free energy of
the reactants and products, and $K_\tn{eq} = k_r / k_f$. If we
consider a binding reaction \schemestart R + L \arrow{}[0, .5] B
\schemestop \, (where R is the receptor, L is the ligand, and B is a
bond), we can write the standard free energies of the reactants and
products as $\mu_R^0$, $\mu_L^0$, and $\mu_B^0(S)$. The standard free
energy of the bond depends on its length $S$. If we assume the bond
acts like a linear spring, then
$\mu_B^0(L) = \mu_B^0(\restLength) + 1/2 \stiffness (L -
\restLength)^2$. Combining this with equation
(\ref{eq:dissociation-const}), we get the following equation for the
dissociation constant of receptor-ligand bond formation as a function
of the bond length:
\begin{multline}
  K_\tn{eq}(L) = \exp\left[\frac{\mu_B^0(L) - \mu_R^0 -
      \mu_L^0}{\boltzmann \temp} \right] =
  \exp\left[\frac{\mu_B^0(\restLength) - \mu_R^0 - \mu_L^0 + 1/2
      \stiffness (L - \restLength)^2}{\boltzmann \temp}\right] \\
  = K_L \exp\left[\frac{\stiffness (L - \restLength)^2}{2 \boltzmann
      \temp} \right]. \label{eq:final-dissoc-const}
\end{multline}

So far, we've only derived the ratio of the bond breaking and
formation rates. In order to find the above expression for
$\onRate(L)$, Dembo et. al. \cite{Dembo1988} use the Arrhenius
equation from the transition state theory of chemical reactions:
$\onRate = A \exp[-E_a / \boltzmann \temp]$. $A$ is a constant, and
$E_a$ is the activation energy of the reaction. Dembo et. al. then
assume that the transition state is a linear spring with the same rest
length as the completely-formed bond, but a different stiffness:
$\stiffness^\tn{ts}$. Then the activation energy has the form
$E_a = \mu_\tn{ts}^0(\restLength) + 1/2 \stiffness^\tn{ts} (L -
\restLength)^2$ and this gives the expression (\ref{eq:dembo-on}) for
the bond formation rate. Then the expression for the off rate in
(\ref{eq:dembo-off}) follows immediately from equations
(\ref{eq:dembo-on}) and (\ref{eq:final-dissoc-const}).

% Boltzmann = R / N_A

% Right now I haven't written down any of the numerical details, or
% experiments to verify the numerics

\section{Numerical rolling experiments without binding}
\label{sec:numer-roll-exper}

As a test of this model, we ran simulations of platelets moving in a
shear flow near a wall with binding turned off. This behavior has been
studied and quantified before \cite{Mody2005}, and our model should be
able to reproduce that known behavior. In the mentioned paper, Mody
and King identify three distinct modes of platelet motion near a plane
wall which depend on the platelet's distance from the wall. The motion
which occurs closest to the wall is a wobbling or ``surfing''
motion. The platelet primarily slides along the wall, with a small
periodic vertical and rotational motion, where the platelet alternates
tilting forward and backward in the flow. The method of regularized
Stokeslets reproduced this behavior for a range of different initial
starting heights close to the wall. Figure \ref{fig:plt-free-surfing}
shows the behavior in one of these experiments, where the center of
mass of the platelet starts at an initial height of $0.8 \mu m$ from
the wall.

\begin{figure}
  \centering
  \includegraphics[width=\textwidth]{run74.eps}
  \caption[Platelet surfing motion]{Platelet center of mass
    height and orientation in the ``surfing'' regime.}
  \label{fig:plt-surfing}
\end{figure}

The second type of motion which occurs somewhat farther from the wall
(in Mody and King's experiments, this motion occurs between 0.75 and
1.1 platelet radii from the wall) is a repeated tumbling motion. At
the start of an experiment, the platelet drifts towards the wall and
rotates forward, until the edge of the platelet contacts the wall and
the center of the platelet ``pole-vaults'' off the surface. Following
this pole-vaulting motion, the center of mass of the platelet is at a
greater height from the wall than when it started, and platelets with
different starting heights converge to the same trajectory following
the initial pole-vault. After the first pole-vault, the platelet
continues to tumble and the edge of the platelet repeatedly contacts
the wall. Our results differed somewhat with the results in Mody and
King in this case. In \cite{Mody2005}, the platelet motion was
periodic after the initial pole-vault. In particular, the platelet
center did not migrate any further from the wall on subsequent
tumbles. However with the method of regularized Stokeslets, we see a
continued slow migration away from the wall. We suggest that this is
due to the regularization of forces at the platelet wall. When the
platelet surface closely approaches the wall during a tumble, because
of the regularization the platelet surface can continue to push
against the wall even when the platelet surface and wall are not
contacting each other. The method used in \cite{Mody2005} relies on a
singular Stokeslet, and so this does not occur in their
simulations. Figure \ref{fig:plt-free-pving} shows the behavior of the
platelet in one simulation in the pole-vaulting regime, where the
center of mass of the platelet starts at an initial height of
$1.2 \mu m$ from the wall.

\begin{figure}
  \centering
  \includegraphics[width=\textwidth]{run72.eps}
  \caption[Platelet pole-vaulting motion]{Platelet center of mass
    height and orientation in the ``pole-vaulting'' regime}
  \label{fig:plt-free-pving}
\end{figure}

The third type of motion that Mody and King describe occurs at a
distance of greater than 1.2 platelet radii from the wall, and the
platelet surface does not come in close contact with the wall in this
type of motion. In particular, the platelet surface does not come
close enough to the wall to form bonds, and therefore we are not
interested in this type of platelet motion.

\section{Rolling and binding experiments}
\label{sec:roll-bind-exper}

The next thing we want to add in to the simulations is bond
dynamics. First, let us look at a couple of simple experiments with a
single bond to develop some intuition of forces and platelet motions
involved with a single bond between the platelet and surface. In these
experiments, the platelet is bound to the wall with a single bond and
the binding and unbinding rates are set to 0. Figure \ref{fig:exp1}
shows the platelet center of mass, the $\rsZ$ component of the minor
axis, and the length of the bond for an experiment where the platelet
is initially bound in a flat orientation (the minor axis is oriented
normal to the wall). Figure \ref{fig:exp2} shows the same three
quantities for an experiment where the platelet is initially bound in
a vertical orientation, with the minor axis parallel to the flow
direction.

\begin{figure}
  \centering
  \begin{subfigure}{.5\textwidth}
    \includegraphics[width=\textwidth]{com_exp1}
  \end{subfigure}
  \\
  \begin{subfigure}{.5\textwidth}
    \includegraphics[width=\textwidth]{emz_exp1}
  \end{subfigure}
  \\
  \begin{subfigure}{.5\textwidth}
    \includegraphics[width=\textwidth]{len_exp1}
  \end{subfigure}
  \caption{Bound experiment: Platelet initially bound in a flat
    orientation, no unbinding}
  \label{fig:exp1}
\end{figure}

\begin{figure}
  \centering
  \begin{subfigure}{.5\textwidth}
    \includegraphics[width=\textwidth]{com_exp2}
  \end{subfigure}
  \\
  \begin{subfigure}{.5\textwidth}
    \includegraphics[width=\textwidth]{emz_exp2}
  \end{subfigure}
  \\
  \begin{subfigure}{.5\textwidth}
    \includegraphics[width=\textwidth]{len_exp2}
  \end{subfigure}
  \caption{Bound experiment: Platelet initially bound in a vertical
    orientation, no unbinding}
  \label{fig:exp2}
\end{figure}

In Figure \ref{fig:exp1}, the platelet first is pushed downstream
(i.e. in the $\rsZ$ direction) by the flow, and is pulled towards the
wall (the $\rsX$ component of the center of mass decreases
initially). The bond length is increasing in this short initial
period, because the drag forces from the flow are greater than the
bond forces. Then the bond length reaches a local maximum, and
decreases for a short period of time. Simultaneously, the platelet's
velocity in the downstream direction decreases, and it continues to be
pulled towards the wall. When the bond length is decreasing, the bond
force is greater than the drag force on the platelet, and I suspect
what is happening is a combination of the bond force increasing
because the bond has lengthened by $20 nm$ and the drag force on the
platelet has decreased, because the platelet is being pulled closer to
the wall where the flow is slower. While this is occurring, the
platelet is tilting forward in the flow, because the $\rsZ$ component
of $\e{m}$ is increasing. This forward tilt is what drives the bond to
start lengthening again, and forces the center of mass of the platelet
away from the wall again. The rotational component of the shear flow
applies a torque on the platelet, trying to make it rotate forward,
however this begins to be counter-acted as the leading edge of the
platelet approaches the wall and the bond lengthens again to prevent
the center of mass of the platelet from moving further from the
wall. Eventually the lengthening of the bond slows as the forces from
the bond start to balance the drag forces, and the platelet will reach
a steady state.

In the second experiment shown in Figure \ref{fig:exp2}, there is a
brief initial period where the bond gets shorter and the platelet is
pulled towards the wall. Then the bond lengthens quickly, as the
platelet is pushed downstream in the flow, and the platelet rotates
forward. Meanwhile the center of mass of the platelet moves closer to
the wall due to the combined effects of the rotation of the platelet,
and the bond attaching the trailing edge of the platelet to the
wall. There is a short period where the bond length remains roughly
constant, but then it continues to lengthen as the platelet reaches a
steady state.

Next we look at a couple of examples of a platelet rolling with
multiple receptors, where both binding and unbinding is turned on. In
all of these experiments, the platelet initially starts in a flat
orientation with a center of mass height of $1.2 \mu m$. % I need to
                                % add in the parameters that I used
                                % for these simulations at some point
The center of mass position, $\rsZ$ component of $\e{m}$, and the bond
lengths for two simulations are shown in Figures \ref{fig:bd-expt0}
and \ref{fig:bd-expta}. In the bond length plot, a curve starts when a
bond forms, and it ends when that bond breaks. In both trials, several
bonds form during the initial pole-vault that occurs at 0.1 s. These bonds
``catch'' the platelet as the edge of the platelet is close to the
wall, slows the platelet down in the flow, and causes it to flatten
against the wall. Without bonds, the platelet's center of mass would
migrate away from the wall after flipping, but the presence of bonds
tethers the platelet's trailing edge to the wall and causes the
platelet's center of mass to move towards the wall as the platelet
flattens in the flow.

\begin{figure}
  \centering
  \includegraphics[width=.75\textwidth]{bd_expt}
  \caption[Binding experiment \#1]{Center of mass position, $\rsZ$
    component of the platelet minor axis, and bond lengths for the
    first binding experiment. Different colors are used in the bond
    length subplot to help distinguish the curves for different bonds,
    but don't have any other meaning.}
  \label{fig:bd-expt0}
\end{figure}

\begin{figure}
  \centering
  \includegraphics[width=.75\textwidth]{bd_expta}
  \caption[Binding experiment \#2]{Center of mass position, $\rsZ$
    component of the platelet minor axis, and bond lengths for the
    second binding experiment.}
  \label{fig:bd-expta}
\end{figure}

These initial bonds break because they get stretched far beyond their
rest length (except for the orange bond in Figure \ref{fig:bd-expt0}),
but more bonds form between the flat part of the platelet and the wall
which mostly do not get stretched beyond their rest length. The
platelets in these simulations are hyper-active: they were captured
from the flow at the first flip, and immediately formed stable
adhesions. In reality, unprimed and primed platelets are less active
than this, and do not immediately form stable adhesions, therefore
$\onConst$ should be lowered in order to get more realistic
behavior. Nonetheless, these figures show how platelets can be
captured from the flow.

\section{Statistical analysis of numerical rolling experiments}
\label{sec:stat-analys-numer}

% Some intro shit

Similar to the analysis for experimental trajectories, we extract
pauses, steps, and average velocities from our numerical
trajectories. Steps are defined as periods where there are no bonds
between the platelet and the surface, and dwells are defined as
periods where there is at least one bond. The first ``step'' (i.e. the
time between when the simulation starts, and the formation of the
first bond) is excluded from the step data, along with the final step
if the platelet is in a step phase at the end of the
simuation. Similarly, if the platelet is in a dwell at the end of a
simulation, that dwell is excluded from the dwell data.



% I also need to describe how I'm extracting the data I analyze at
% some point.. maybe in the results section

% local variables:
% TeX-master: "phd-thesis.ltx"
% End:
