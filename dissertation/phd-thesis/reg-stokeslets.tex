%%% -*-LaTeX-*-

\chapter{Three-Dimensional Rolling Model}
\label{cha:three-dimens-roll}

The previous models presented greatly simplify the physics and
geometry of an ellipsoidal platelet rolling in a three-dimensional
shear flow. In the 2D rolling model, the platelet is modeled as a
sphere with a thin reactive strip around the circumference. Two axes
of motion and two axes of rotation are completely discarded, along
with the relevant forces and torques. In the jump-velocity model, all
platelet geometry and all fluid forces are ignored completely, and the
bond dynamics of many receptors and ligands are simplified to simple
binding and unbinding rates.

In this chapter, we develop a fully 3D adhesive dynamics model, and
present results from this model. This model is too computationally
intensive to use for true parameter estimation from experimental data,
but of the three models presented here it is the most valuable for
developing intuition of the complex forces and motions involved in the
rolling dynamics of platelets.

\section{Model Description}
\label{sec:model-description}

In general, adhesive dynamics (AD) models describe the motion of cells
in a Stokes flow due to binding and unbinding with a ligand-coated
wall. There are two major components in AD models:
\begin{enumerate}
\item Bond dynamics---individual receptors are modeled separately,
  their positions are tracked, and they can bind and unbind
  independently of other receptors, and
\item Fluid-structure interaction---modeling the drag forces generated
  by the moving fluid on an immersed body.
\end{enumerate}
Putting these components together, AD models give us a way to
translate bond-level kinetics to cell-level behavior.

In developing this model, we focus on the forces acting on the
platelet. While there is some flow field around the platelet, we are
not interested in what this flow field actually looks like; we are
only interested in its effect on the platelet.

We consider three generalized forces which act on the platelet and
affect its motion:
\begin{enumerate}
\item $\hydrodynamicForces$---the hydrodynamic force on the cell, 
\item $\bindingForces$---the force due to binding with ligands on the wall, and
\item $\bodyForces$---various body forces acting on the cell
  (e.g. electrostatic and steric forces).
\end{enumerate}
Because we assume the flow field is a Stokes flow (i.e., no inertia is
present in the system), all of these forces balance:
$\hydrodynamicForces + \bindingForces + \bodyForces = \vect{0}$.

$\bindingForces$ depends on the location and orientation of bonds
between the cell and wall at time $t$, and $\bodyForces$ depends on
the location and orientation of the cell. The hydrodynamic forces
$\hydrodynamicForces$ depend both on the location and orientation of
the cell $\pltPosition$, and also the translational and angular
velocities $\pltPosition'$. Therefore we can use this relation to find
$\pltPosition'(t)$ at each time step, and update the position and
orientation of the cell accordingly.

\subsection{Definitions and Domain}
\label{sec:definitions-domain}

We want to model the motion of a platelet (i.e. an ellipsoid) near a
plane wall, so the fluid domain we use is the upper half-space
$\fluidDomain = \{(\rsX, \rsY, \rsZ) \in \reals \mid \rsX > 0\}$. The wall
is represented by a plane at $\rsX = 0$. The background flow
$\flowField^\infty$ is a linear shear flow with a shear rate of
$\shear$: $\flowField^\infty = \shear \rsX \e{\rsZ}$.

The platelet is modeled as an ellipsoid with axes lengths of
$1.5 \mu m \times 1.5 \mu m \times 0.5 \mu m$, and the position of its
center of mass is $\pltCOM$. There are a couple of different ways to
represent the orientation of the platelet, however the simplest way to
represent the platelet is as a unit vector pointing in the direction
of the platelet's minor axis: $\e{m}$. This uniquely defines an
orientation because the platelet is rotationally symmetric about its
minor axis.

Finally, as mentioned above we use Stokes equations to model the flow
field, with no-slip boundary conditions on the platelet surface and
the wall, and a far-field condition:
\begin{align}
  \Delta \flowField = \nabla \pressure, \quad \nabla \cdot \flowField
  = 0 \label{eq:stokes-eqn} \\
  \flowField(\arbLoc)|_{\partial\mathcal{P}} = \pltVelocity +
  \pltRotation \times \arbLoc \label{eq:plt-bc} \\
  \flowField(\arbLoc)|_{\rsX = 0} =
  \vect{0} \label{eq:wall-bc} \\
  \flowField(\arbLoc)|_{\|\arbLoc\| \rightarrow \infty}
  \rightarrow \flowField^\infty. \label{eq:far-field}
\end{align}

\subsection{Equations of motion}
\label{sec:equations-motion}

Let us put aside the issue of solving equations
(\ref{eq:stokes-eqn})---(\ref{eq:far-field}) for the moment, and focus
on the main structure of the model. We are interested in the motion of
the platelet, and this can be found by solving the following system of
ODEs; the equations of rigid body motion for the platelet:
\begin{align}
  \Der{\pltCOM}{t} &= \pltVelocity(\pltCOM,
                     \e{m}) \label{eq:plt-velocity} \\
  \Der{\e{m}}{t} &= \pltRotation(\pltCOM,
                   \e{m}). \label{eq:plt-rotation} 
\end{align}

We haven't defined the functions for the platelet's translational
($\pltVelocity$) and angular ($\pltRotation$) velocities yet, and all
of the numerical complexity in solving this model is wrapped up in
these functions. These velocity functions are found by enforcing force
balance on the platelet. For a given platelet position, orientation,
and set of bonds, the body forces $\bodyForces$ and bond forces
$\bindingForces$ are fixed. However the hydrodynamic forces
$\hydrodynamicForces$ exerted by the fluid on the cell depend on
$\pltVelocity$ and $\pltRotation$, as well as the position and
orientation. Therefore, we will choose the values of $\pltVelocity$
and $\pltRotation$ so that the hydrodynamic forces balance with the
other forces acting on the cell: $\hydrodynamicForces + \bodyForces +
\bindingForces = \vect{0}$.

First, let us look at $\hydrodynamicForces$ more closely. For a given
velocity field $\flowField$, the stress tensor is defined as
$\stressTensor = -p \uu{I} + \viscosity \left(\nabla \flowField +
  \left(\nabla \flowField\right)^T\right)$, and the body forces and
torques can be found, respectively, as:
\begin{align}
  \left[\hydrodynamicForces\right]_{1:3} &= \Int{\stressTensor \cdot
  \mathbf{n}}{s(\arbLoc)}{\pltSurf}{} \label{eq:body-force} \\
  \left[\hydrodynamicForces\right]_{4:6} &= \Int{(\arbLoc - \pltCOM)
  \times \stressTensor \cdot \mathbf{n}}{s(\arbLoc)}{\partial
  P}{}. \label{eq:body-torque} 
\end{align}

As mentioned in Section \ref{sec:definitions-domain},
$\hydrodynamicForces$ is generated by a background flow of
$\flowField^\infty = \shear \rsX \e{\rsZ}$. Because of the linearity
of Stokes' flow, we can decompose $\hydrodynamicForces$ into two
pieces:
\begin{itemize}
\item $\dragForces$---the drag force and torque exerted on a moving
  platelet in a stationary flow, and
\item $\shearForces$---the drag force and torque exerted on a
  stationary platelet in a background shear flow.
\end{itemize}
Again using linearity of Stokes' flow, there is a linear relation
between $\dragForces$ and the generalized velocity $\genVelocity$ of
the platelet (where
$\genVelocity = (\pltVelocity, \pltRotation)^T \in \reals^6$):
\begin{equation}
  \label{eq:resistance-matrix}
  \dragForces = \resMatrix \genVelocity.
\end{equation}
The matrix $\resMatrix$ is called the \emph{resistance tensor} and
depends only on the position, orientation, and shape of the
platelet. In order to find the resistance tensor, we need to solve 6
Stokes' flow problems with the following boundary conditions on the
platelet:
\begin{align}
  \flowField_i |_{\arbLoc \in \pltSurf} &= \delta_{ij} \e{j} \tn{
  for } i = 1, 2, 3, \tn{ and} \label{eq:translation-stokes-bcs} \\
  \flowField_i |_{\arbLoc \in \pltSurf} &= \delta_{i-3, j} \e{j}
  \times \arbLoc \tn{ for } i = 4, 5,
  6. \label{eq:rotation-stokes-bcs}
\end{align}
For the sake of completeness, the exact system being solved is the
equations (\ref{eq:stokes-eqn}), (\ref{eq:wall-bc}), and
\begin{equation}
  \flowField_i |_{\|\arbLoc\| \rightarrow \infty} \rightarrow
  \vect{0}. \label{eq:far-field-drag}
\end{equation}
with either (\ref{eq:translation-stokes-bcs}) or
(\ref{eq:rotation-stokes-bcs}) depending on $i$.

With the boundary conditions specified in equations
(\ref{eq:translation-stokes-bcs}) and (\ref{eq:rotation-stokes-bcs}),
we have six different generalized velocities: $\genVelocity_i = \e{i}$
($\e{i}$ is the $i$th canonical basis vector in $\reals^6$). Therefore
to construct $\resMatrix$, for each $\genVelocity_i$, we solve for
$v_i$ and then compute the drag forces and torques (combined in the
generalized force vector $\dragForce^i$) on the platelet using
equations (\ref{eq:body-force}) and (\ref{eq:body-torque}). Because
$\genVelocity_i = \e{i}$, the $i$ column of $\resMatrix$ is exactly
$\dragForce^i$.

The drag force $\shearForces$ due to the shear background flow is
found with a 7th solve of Stokes' equations, with the boundary and
far-field conditions
\begin{align}
  &\flowField_7 |_{\arbLoc \in \pltSurf} =
  \vect{0} \label{eq:plt-bcs-shear} \\
  &\flowField_7 |_{\|\arbLoc\| \rightarrow \infty} \rightarrow
  \shear \rsX \e{\rsZ}. \label{eq:plt-far-field-shear}
\end{align}

Putting everything together and assuming for the moment that
$\bindingForces$ and $\bodyForces$ are given, we require that
$\hydrodynamicForces + \bindingForces + \bodyForces = 0$. We can
express $\hydrodynamicForces$ as a sum of the stationary drag forces
on the platelet and the drag from the background shear flow on the
platelet: $\hydrodynamicForces = \dragForces + \shearForces =
\resMatrix \genVelocity + \shearForces$. Therefore, $\resMatrix
\genVelocity + \shearForces + \bindingForces + \bodyForces = 0$, and
solving for $\genVelocity$ we get
\begin{equation}
  \label{eq:gen-velocity-equation}
  \genVelocity = -\resMatrix\inv \left(\shearForces + \bindingForces +
    \dragForces \right).
\end{equation}

\subsubsection{Numerical method for solving Stokes' equations}
\label{sec:numer-meth-solv}

We use the method of regularized Stokeslets to solve Stokes' equations
for the different boundary conditions specified above. This is an
approach developed by Ricardo Cortez to solve fluid-structure
interaction problems in Stokes' flows which distributes regularized
point forces over the surface of the structure being modeled in order
to communicate forces between the fluid and the structure. Basically
it is a regularized Green's function approach to solving Stokes' flow.

Below, I will describe the method of regularized Stokeslets for an
unbounded flow in three dimensions. This is a simpler case than
solving Stokes' equations in the half-space with $\vect{0}$
boundary conditions on the wall, but nonetheless it has many of the
same features. This description closely follows the derivations laid
out in \cite{Cortez2001} and \cite{Cortez2005}. %Give examples?

Similar to a Green's function approach, the method of regularized
Stokeslets relies on the linearity of Stokes' equations by exploiting
the superposition principle for linear PDEs. In a true Green's
function method (i.e., the method of Stokeslets), one would solve
Stokes' equations with a singular point force:
\begin{equation}
  \label{eq:stokes-with-pt-force}
  \Delta \flowField - \nabla \pressure = -\arbForce \delta(\arbLoc
  - \evalLoc), \quad \nabla \cdot \flowField = \vect{0}.
\end{equation}
These equations have an exact solution---called the Stokeslet---and
can be used to solve Stokes' equations for an arbitrary right-hand
side.

The equation that forms the basis of the method of regularized
Stokeslets is very similar, but with the point force replaced by a
regularized point force:
\begin{equation}
  \label{eq:stokes-with-reg-force}
  \Delta \flowField - \nabla \pressure = -\arbForce
  \blobFun{\blobPar}(\arbLoc - \evalLoc), \quad \nabla \cdot
  \flowField = \vect{0}.
\end{equation}
The function $\blobFun{\blobPar}$ is usually called a ``blob
function'' or ``cutoff function,'' and $\blobPar$ is the ``blob
parameter.'' The blob function is a regularized approximation to the
$\delta$ function, and satisfies three criteria:
\begin{enumerate}
\item $\blobFun{\blobPar}$ is radially symmetric and smooth, 
\item $\Int{\blobFun{\blobPar}(\arbLoc)}{\arbLoc}{}{} = 1$ for all
  $\blobPar$, and
\item $\blobFun{\blobPar} \rightarrow \delta$ as $\blobPar \rightarrow
  0$.
\end{enumerate}
There are of course many different possible functions which satisfy
these criteria. The specific blob function we use is
\begin{equation}
  \label{eq:blob-function}
  \blobFun{\blobPar}(\arbLoc) = \frac{15 \blobPar^4}{8 \pi
    \left(\|\arbLoc\|^2 + \blobPar^2 \right)^{7/2}}.
\end{equation}
Then we can write the solution to equation
(\ref{eq:stokes-with-reg-force}) as
\begin{equation}
  \label{eq:solution-to-reg-force}
  \flowField(\arbLoc) = \frac{1}{8\pi \viscosity}
  \regStokeslet{\blobPar} \arbForce
\end{equation}
where $\regStokeslet{\blobPar}$ is the regularized Stokeslet.

It is possible to derive the following boundary integral equation for
the solution of Stokes' equations around a body $\pltBody$
\cite{Cortez2005}:
\begin{equation}
  \label{eq:boundary-integral-equation}
  \Int{\flowField(\arbLoc) \blobFun{\blobPar}(\arbLoc -
    \evalLoc)}{V(\arbLoc)}{\reals^3}{} = \frac{1}{8\pi \viscosity}
  \Int{\regStokeslet{\blobPar} \arbForce}{s(\arbLoc)}{\pltSurf}{}.
\end{equation}
This equation is used to derive the method of regularized Stokeslets,
and two approximations are made to arrive at the numerical method. The
first approximation is to replace $\blobFun{\blobPar}$ with a singular
$\delta$-function, which eliminates the integral on the left-hand
side. The second approximation is that the integral on the right hand
side will be discretized and approximated with a quadrature
rule. These approximations give the equation
\begin{equation}
  \label{eq:method-reg-stokeslets}
  \flowField(\evalLoc) = \frac{1}{8\pi\viscosity} \sum_{n = 1}^N
  \regStokeslet{\blobPar} (\arbLoc_n, \evalLoc) \arbForce_n A_n,
\end{equation}
where $N$ is the number of Stokeslets placed on $\pltSurf$,
$\arbLoc_n$ is the location of the $n$th Stokeslet, $\arbForce_n$ is
the force vector applied at $\arbLoc_n$, and $A_n$ is the quadrature
weight of the $n$th Stokeslet.

In this method, we pick the location of each of the point forces, and
the quadrature rule used to approximate the right-hand side integral
in equation (\ref{eq:boundary-integral-equation}). The unknowns are
the Stokeslet strengths $\arbForce_n$. Because the boundary
conditions on $\pltSurf$ are known, we can choose a set of $\evalLoc$
on $\pltSurf$ and derive $\flowField(\evalLoc)$ from the boundary
conditions. By choosing $\arbLoc_n$ and $N$ evaluation points
$\evalLoc \in \pltSurf$, equation (\ref{eq:method-reg-stokeslets}) can
be written as a $3N \times 3N$ linear system:
\begin{equation}
  \label{eq:rs-linear-system}
  \hat{\flowField} = \frac{1}{8\pi\viscosity} \mathcal{A}
  \hat{\arbForce},
\end{equation}
where $\hat{\flowField} = (\flowField(\evalLoc^1),
\flowField(\evalLoc^2), \hdots, \flowField(\evalLoc^N)^T$,
$\hat{\arbForce} = (\arbForce(\arbLoc_1), \arbForce(\arbLoc_2),
\hdots, \arbForce(\arbLoc_N))^T$, and $\mathcal{A}$ is a $3N \times
3N$ matrix which depends on $\regStokeslet{\blobPar}$ and $A_n$.

We take the evaluation points $\evalLoc^n$ to be the same as the force
locations; so $\evalLoc^n = \arbLoc_n$, but other choices could be
made as well. Therefore in order to find the Stokeslet strengths
$\hat{\arbForce}$ for a given choice of boundary conditions
$\hat{\flowField}$, we have to invert $\mathcal{A}$, and because we
need to solve for $\hat{\arbForce}$ for seven different
$\hat{\flowField}$s, we compute the Cholesky factorization of
$\mathcal{A}$ so that we can re-use this work when solving the seven
linear systems.

Thus far I have glossed over some details regarding the solution of
the shear drag, that is, solving Stokes' equations with the conditions
(\ref{eq:plt-bcs-shear}) and (\ref{eq:plt-far-field-shear}). The
regularized Stokeslets vanish as $\|\arbLoc\| \rightarrow \infty$,
which is not consistent with the far-field condition
(\ref{eq:plt-far-field-shear}). Therefore, following Pozrikidis
\cite{Pozrikidis92}, we decompose $\flowField_7$ into the background
flow $\flowField^\infty$ and a disturbance flow $\flowField_d$, which
represents the disturbance to the background flow from the
platelet. The background flow $\flowField^\infty$ satifies Stokes'
equations and the boundary condition at the wall, and the disturbance
field $\flowField_d$ satisfies Stokes equations with the modified
boundary conditions:
\begin{equation}
  \label{eq:disturbance-bcs}
  \flowField_d(\arbLoc) = -\flowField^\infty(\arbLoc) \tn{ for }
  \arbLoc \in \pltSurf, \quad \flowField_d \rightarrow \vect{0} \tn{
    for } \|\arbLoc\| \rightarrow \infty.
\end{equation}
Now $\flowField_d$ can be found using regularized Stokeslets following
the same process described above, and the solution can be integrated
to find $\genForces_d$.

\subsection{Binding dynamics}
\label{sec:binding-dynamics}

With the fluid dynamics component of the model taken care of, let's
now add the binding and unbinding component of the model. The
assumptions made in this model are similar to those made in the 2D
rolling model developed in Chapter \ref{cha:two-dimens-roll}. We
assume that there is a finite number of receptors on the surface of
the cell, and we track their locations explicitly. We assume the
receptors have a fixed location relative to the surface of the cell,
so that the motion of every receptor is completely determined by the
motion of the platelet. Ligands on the wall are assumed to exist in
excess, and ligands are not tracked explicitly.

We assume that bonds have some nonzero rest length $\restLength$. When
the bond length is greater than $\restLength$, the bond acts like a
linear spring, and when the bond length is less than $\restLength$, no
force is exerted by the bond. Therefore the function defining the bond
force exerted by a single bond is
\begin{equation}
  \label{eq:single-bond-force}
  \bondForces_i = \stiffness H(\|\arbLoc_l^i - \arbLoc_r^i\| -
  \restLength) \frac{\arbLoc_l^i - \arbLoc_r^i}{\|\arbLoc_l^i -
    \arbLoc_r^i\|},
\end{equation}
where $\arbLoc_l^i$ is the point where the bond attaches to a ligand
on the wall, and $\arbLoc_r^i$ is the location of the bound receptor.
In order to find the total bond force and torque exerted on the
platelet, we compute the following sums:
\begin{align}
  \left[\bondForces\right]_{1:3} = \sum_{i = 1}^{N_\tn{bonds}}
  \bondForces_i \label{eq:total-bond-force} \\
  \left[\bondForces\right]_{4:6} = \sum_{i = 1}^{N_\tn{\bonds}}
  (\arbLoc_r^i - \pltCOM) \times \bondForces_i.
\end{align}

The bond formation ($\onRate(L)$) and breaking ($\offRate(L)$) rates
are distance/length-dependent. We use the Dembo model \cite{Dembo1988}
for these rates:
\begin{align}
  \onRate(L) = \onRate^0 \exp\left[\frac{-\stiffness_\tn{ts} \left(L -
  \restLength \right)^2}{2 \boltzmann \temp}
  \right] \label{eq:dembo-on} \\
  \offRate(L) = \offRate^0 \exp\left[\frac{\left(\stiffness -
  \stiffness_\tn{ts} \right) \left(L - \restLength \right)^2}{2
  \boltzmann \temp} \right]. \label{eq:dembo-off}
\end{align}

% local variables:
% TeX-master: "phd-thesis.ltx"
% End:
