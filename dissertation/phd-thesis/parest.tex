%%% -*-LaTeX-*-

\chapter{Parameter Estimates}
\label{app:parameter-estimates}

There are 13 dimensional parameters in the dimensional version of the
PDE model, listed in Table \ref{tab:dim_pars}.

\begin{table}
  \renewcommand*{\arraystretch}{1.5}
  \centering
  \begin{tabu}{cXXX}
    \toprule Symbol & Description & Estimate & Source \\ \midrule
    $\radius$ & Platelet radius & 1 {\textmugreek}m & None \\
    $\separation$ & Wall separation distance & 0.01 {\textmugreek}m &
    None \\
    \multirow{3}{*}{$\stiffness$} & \multirow{3}{*}{Bond stiffness} &
    12 pN/nm & \cite{Litvinov2011} (\ITA{IIb}\ITB{3}--fbg) \\
    & & 0.1 pN/nm & \cite{Fitzgibbon2014} (GP1b-vWF) \\
    & & 10 pN/nm & \cite{Mody2008b} (GP1b-vWF) \\
    $\boltzmann$ & Boltzmann constant & $1.38 \times 10^{-23}$ J/K
    & None \\
    $\temp$ & Blood temperature & 310 K & None \\
    $\onConst$ & Maximum bond on rate &
    &  \\
    $\offConst$ & Unstressed bond off rate & 5 s$\inv$
    & \cite{Fitzgibbon2014} (GP1b--vWF)\\
    \multirow{2}{*}{$\refForce$} &
    \multirow{2}{*}{\parbox{\textwidth}{Characteristic
        breaking\\force}} & 6.03 pN & Estimated below from
    \cite{Mody2008b} (GP1b--vWF) \\
    & & 107 pN & Estimated below from \cite{Bhatia2003}
    (E-selectin--sLe\textsuperscript{x}) (\cite{Doggett2002} gives similar)\\
    $\receptorDensity$ & Angular receptor density & $\sim 1000$
    receptors/radian & Estimated
    below \\
    $\appliedRot$ & Fluid-imposed rotation rate & $\gamma/2$ s$\inv$ &
    \cite{Goldman1967b} \\
    $\appliedVel$ & Fluid-imposed translation velocity &
    $(R + d)\gamma$ {\textmugreek}m/s &
    \cite{Goldman1967b} \\
    $\velFriction$ & Translational drag coefficient &
    $6 \times 10^{-5}$ g/s & Estimated below
    (GP1b--vWF) \\
    $\rotFriction$ & Rotational drag coefficient & $8 \times 10^{-5}$
    {\textmugreek}m\textsuperscript{2}g/s & Estimated below \\
    \bottomrule
  \end{tabu}
  \caption{Dimensional parameters in the PDE model}
  \label{tab:dim_pars}
\end{table}

\section{Estimates for model parameters}
\label{sec:estim-model-param}

\subsection{Characteristic breaking force}
\label{sec:char-break-force}

$\refForce$ is equivalent to $\boltzmann\temp/\compliance$ in
\cite{Pospieszalska2009} and \cite{Sundd2011}, where $\compliance$ is
the reactive compliance and is estimated at 0.4 angstroms in
\cite{Bhatia2003}. Then converting units as needed:
\begin{align*}
  &\compliance = 0.4 \text{ angstroms} = 4 \times 10^{-5}
    \text{{\textmugreek}m} \text{, and} \\
  &\boltzmann = 1.38 \times 10^{-23} \text{N}\cdot\text{m}/\text{K} =
    1.38 \times 10^{-5} \text{pN}\cdot\text{{\textmugreek}m}/\text{K}.
\end{align*}
Finally, $\refForce = \frac{\boltzmann\cdot\temp}{\compliance} =
\frac{310 \cdot 1.38 \times 10^{-5}}{4 \times 10^{-5}} \text{pN} = 107
\text{pN}$.

\subsection{Angular receptor density}
\label{sec:ang-rec-dens}

$\receptorDensity$ is related to the number of receptors on the
surface of the platelet, but it is not clear how. Obviously in reality
the receptors are distributed over a 2D surface, but in the model the
platelet is 2D and its surface is 1D. There are 60,000--80,000 
\ITA{IIb}\ITB{3}  receptors on the surface of a platelet, and more are
recruited after platelet activation
\cite{Shattil1998,Litvinov2011}. If we assume all of these are on the
surface of our 2D platelet, that gives a range of 10,000--13,000 for
$\receptorDensity$. 

If we assume the reactive region is a narrow strip down the centerline
of the platelet of width $\width$, then $\receptorDensity$ is found by
first calculating the number of receptors in that reactive
strip. Approximating the reactive region with a cylinder of radius
$\radius$ and height $\width$ gives a surface area of
$2\pi\width\radius$. The surface density of receptors is given by
$\receptorNumber/(4\pi\radius^2)$, and so there are
$\receptorNumber\cdot\width/(2\radius)$ total receptors in the
reactive strip. After dividing by $2\pi$ to get the angular density of
receptors, we end up with $\receptorDensity =
\receptorNumber\cdot\width/(4\pi\radius)$. With our estimates of
$\receptorNumber$ and $\radius$ above, this gives us a range for
$\receptorDensity$ of $4,775\width$--$6,366\width$.

% \subsection{Drag coefficients}
% \label{sec:drag-coefficients}

% The simplest case is to use the Stokes' drag coefficients for a sphere
% in an unbounded uniform flow: $6\pi\viscosity\radius$ and
% $8\pi\viscosity\radius^3$. With the  viscosity of blood around 3--4
% cP (3--4 $\times 10^{-6}$ g/(\textmugreek{m} s)), this gives drag
% coefficients of $6 \times 10^{-5}$ g/s and $8 \times 10^{-5}$
% \textmugreek{m}\textsuperscript{2}g/s respectively.

% A more sophisticated treatment would use the resistance coefficients
% estimated by Goldman, Cox, and Brenner
% \cite{Goldman1967a,Goldman1967b} for the case of a sphere moving in a
% shear flow near an infinite plane wall. Based on their results, we can
% relate the hydrodynamic force and torque on a sphere to its
% translational and angular velocities with a 2x2 matrix equation:
% $\mathbf{F}^h = \underline{\underline{R}} \mathbf{U} + \mathbf{F}^s$ where
% $\mathbf{F}^h = (\, )^T$, $\mathbf{U} = (v, \omega)^T$, and $\mathbf{F}^s$
% is the force and torque generated by the shear flow. The entries of
% the resistance matrix $\underline{\underline{R}}$ are listed in Table
% \ref{tab:resistance-coefficients}. The forces and torques generated by
% the shear flow are taken to be $F^s = 6 \pi \viscosity \radius
% (\radius + \separation) \gamma \cdot 1.6160$ and $T^s = 4 \pi
% \viscosity \radius^3 \gamma \cdot 0.9537$, where $\mathbf{F}^s = (F^s,
% T^s)^T$.

% For a separation of 100 nm, we get the following values for elements
% in the resistance matrix:
% \begin{align*}
%   r_{11} &= -1.24 \times 10^{-4} \frac{g}{s} \\
%   r_{12} &= 3.08 \times 10^{-6} \frac{g \mu m}{s} \\
%   r_{21} &= 3.07 \times 10^{-6} \frac{g \mu m}{s} \\
%   r_{22} &= -9.83 \times 10^{-5} \frac{g \mu m^2}{s}.
% \end{align*}

% \begin{table}
%   \centering
%   \begin{tabular}{c|c}
%     $r_{11} = 6 \pi \viscosity \radius
%     \left(\frac{8}{15}\log(\separation/\radius) - 0.9588\right)$
%   & $r_{12} = 6 \pi \viscosity \radius^2
%     \left(-\frac{2}{15}\log(\separation/\radius) - 0.2526\right)$ \\ 
%     \hline
%     $r_{21} = 8 \pi \viscosity \radius^2
%     \left(-\frac{1}{10}\log(\separation/\radius) - 0.1895\right)$
%   & $r_{22} = 8 \pi \viscosity \radius^3
%     \left(\frac{2}{5}\log(\separation/\radius) - 0.3817\right)$
%   \end{tabular}
%   \caption{Coefficients of the resistance matrix}
%   \label{tab:resistance-coefficients}
% \end{table}

\section{Nondimensional parameters}
\label{sec:nd-params}

There are 8 parameters in the nondimensional PDE rolling model, listed
in Table \ref{tab:nd-params} with values calculated from the
dimensional parameter values listed in Table \ref{tab:dim_pars}.

\begin{table}
  \renewcommand*{\arraystretch}{1.5}
  \centering
  \begin{tabular}{ccll}
    \toprule
    Parameter & Definition & Value & Description \\ \midrule
    $\ndSeparation$ & $\separation = \radius\ndSeparation$ & 0.01
                                   & cell-surface separation \\
    $\ndAppliedRot$ & $\appliedRot = \ndAppliedRot\offConst$
                           & $(0.1 \text{ s})\shear$ & Fluid-imposed
                                                       rotation rate
    \\
    $\ndAppliedVel$ & $\appliedVel = \ndAppliedVel\radius\offConst$
                           & $(0.2 \text{ s}) (1 + \ndSeparation)
                             \shear$
                                   & Fluid-imposed translation
                                     velocity \\
    $\ndOnConst$ & $\ndOnConst = \frac{\radius\onConst}{\offConst}$
                           &  & Maximum relative on rate \\
    $\onForceScale$ & $\onForceScale =
                    \frac{\stiffness\radius^2}{\boltzmann\temp}$
                           & $2.34 \times 10^4$ & Length dependence of
                                                  on rate \\
    $\offForceScale$ & $\offForceScale = \stiffness\radius/\refForce$
                           & 16.6 & Length dependence of off rate \\
    $\ndVelFriction$ & $\ndVelFriction =
                     \frac{\offConst}{\stiffness\receptorDensity}
                       \velFriction$ & $7.36 \times 10^{-6}$
                                   & Translational drag coefficient \\
    $\ndRotFriction$ & $\ndRotFriction =
                     \frac{\offConst}{\stiffness\receptorDensity
                                      \radius^2}\rotFriction$
                           & $9.82 \times 10^{-6}$ & Rotational drag
                                                     coefficient \\
    \bottomrule
  \end{tabular}
  \caption{Nondimensional parameters}
  \label{tab:nd-params}
\end{table}

% Local Variables:
% TeX-master: "phd-thesis.ltx"
% End:
