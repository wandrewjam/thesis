%%% -*-LaTeX-*-

\chapter{Results and Analysis}
\label{cha:results}

The model described above is a fairly simple model of platelet
rolling. The geometry of a platelet is greatly idealized, the
hydrodynamic and chemical forces between the platelet and surface are
simplified, and platelet activation is not considered in the
model. All of these will be discussed in greater detail in Chapter
\ref{cha:future-work}, but we first want to verify that this model can
reproduce rolling behavior that is qualitatively similar to
experimental observations and results from previously published
rolling models. The results presented below show that the model can
produce realistic rolling behavior for biologically realistic
parameter values. In addition, the behavior of the model is sensitive
to the on rate $\ndOnConst$ and the density of receptors on the
surface of the cell $\receptorDensity$; both of which are parameters
that may change when a platelet undergoes activation.

\section{Behavior of the model at steady state}
\label{sec:behav-model-steady}

The first set of experiments show the behavior of the deterministic
model at steady state. Results at steady state are relatively quick to
generate, making it possible to look at the average behavior of
platelets over a large range of parameter values. 

The steady solution (of the nondimensional system) is given by solving
equations \eqref{eq:nd-bond-density}--\eqref{eq:nd-torque-balance}
with $\Pder{\ndBondDensity}{\ndTime} \equiv 0$ in
\eqref{eq:nd-bond-density}. This can be viewed as a 2D nonlinear
rootfinding problem
\begin{align}
  \label{eq:2D-ss-translation}
  0 &= \ndVelFriction (\ndAppliedVel - \ndVelocity) +
      \ndHorzTotalForce(\ndVelocity, \ndRotation) \\
  \label{eq:2D-ss-rotation}
  0 &= \ndRotFriction (\ndAppliedRot - \ndRotation) +
      \ndTotalTorque(\ndVelocity, \ndRotation)
\end{align}
where evaluating the functions $\ndTotalTorque$ and
$\ndHorzTotalForce$ requires solving the steady state PDE for
$\ndBondDensity$ and then integrating the result. In general, the
translation velocity $\ndVelocity$ and rotation rate $\ndRotation$ of
the platelet at steady state are both functions of the applied
velocity $\ndAppliedVel$ and applied rotation $\ndAppliedRot$, meaning
that finding the relation between $(\ndAppliedVel, \ndAppliedRot)$ and
$(\ndVelocity, \ndRotation)$ requires sweeping over a 2D space of
applied velocities. This problem is still too computationally
intensive to explore platelet rolling behavior over a wide range of
parameter, so assume further that the rolling platelet is in direct
contact with the surface and there is no slip between the surfaces of
the platelet and vessel wall.

If we assume that a cell is sitting directly on top of the surface
(i.e. $\ndSeparation = 0$), and the cell is only rolling, but not
sliding
($\velocity = \radius\rotation \implies \ndVelocity = \ndRotation$),
then solving the steady state problem reduces to a 1D rootfinding
problem. One can solve equation \eqref{eq:2D-ss-rotation} to find the
steady state value(s) of $\ndRotation$ for a given $\ndAppliedRot$. It
is also possible to use equation \eqref{eq:2D-ss-translation} to find
$\ndVelocity$ for a given $\ndAppliedVel$, and in general this will
give a different value of $\ndVelocity$ than if you solved for
$\ndRotation$ first and then set $\ndVelocity = \ndRotation$.

In practice, it is computationally difficult to find the steady state
bond distribution $\ndBondDensity$ and rotation rate $\ndRotation$ for
a given applied rotation rate $\ndAppliedRot$. Solving this
problem requires solving a nonlinear integro-PDE:
\begin{align}
  \label{eq:ndim-ss}
  0 &= \ndRotation
      \left(\frac{\partial\ndBondDensity}{\partial\ndWallDist} +
      \frac{\partial \ndBondDensity}{\partial\recAngle}\right) +
      \ndOnConst\exp\left(-\frac{\onForceScale}{2} \ndLength^2 \right)
      \left(1 - \Int{\ndBondDensity(\ndWallDist, \recAngle)}
      {\ndWallDist} {-\infty} {\infty} \right) - \exp \left(
      \offForceScale \ndLength \right) \ndBondDensity, \\
  \label{eq:ndim-torbal}
  0 &= \ndRotFriction(\ndAppliedRot - \ndRotation) + \ndTotalTorque.
\end{align}
It is easier to solve this problem in the reverse order. That is,
with a specified $\ndRotation$ equation (\ref{eq:ndim-ss}) is a linear
PDE which can be solved to find the steady state bond distribution,
$\ndBondDensity$. Then the integral to find the total torque in
equation (\ref{eq:ndim-torbal}) can simply be evaluated, and
$\ndAppliedRot$ is given by
$\ndRotation - \ndTotalTorque / \ndRotFriction$. Therefore, in order
to find steady state values of $\ndRotation$ for a range of
$\ndAppliedRot$s, we generate a bunch of
$(\ndAppliedRot, \ndRotation)$ pairs that solve the steady state
problem. Then we can use interpolation to find a steady state angular
velocity $\ndRotation$ for an unknown $\ndAppliedRot$.

\subsection{Parameter sweeps}
\label{sec:parameter-sweeps}

One feature we expect to find is a small number of bonds between the
platelet and the surface. Experimental and modeling work has shown
that stochasticity is an important feature in cell rolling, and
significant stochasticity can only exist if rolling is mediated by a
small number of bonds. Therefore, we searched through
$(\ndOnConst,\onForceScale,\offForceScale,\ndRotation)$ parameter
space to look for regions where the total number of bonds
$\Int{\Int{\ndBondDensity}{\ndWallDist}{}{}}{\recAngle}{}{}$ was in
the range $10^{-4}$--$10^{-2}$. Based on the estimate
$\receptorDensity \approx 10^3 \, \tn{receptors/radian}$ from Table
\ref{tab:dim_pars}, this corresponds to a range of $\pi \times
10^{-1}$--$\pi \times 10^1$ bonds between the platelet and surface on
average. 

The fact that we are looking for regions of parameter space where the
number of bonds is small suggests one more simplification. If
$\Int{\Int{\ndBondDensity}{\ndWallDist}{}{}}{\recAngle}{}{}$ is small,
it is reasonable to expect that receptor saturation does not play a
significant role in equation (\ref{eq:ndim-ss}). Therefore we assume
that $\Int{\ndBondDensity}{\ndWallDist}{-\infty}{\infty} << 1$ and
solve the following PDE for a range of $\ndOnConst$, $\onForceScale$,
$\offForceScale$, and $\ndRotation$ values:
\begin{equation}
  \label{eq:ndim-ss-nosat}
  0 = \ndRotation
  \left(\frac{\partial\ndBondDensity}{\partial\ndWallDist} +
    \frac{\partial \ndBondDensity}{\partial\recAngle}\right) +
  \ndOnConst\exp\left(-\frac{\onForceScale}{2} \ndLength^2 \right) -
  \exp \left(\offForceScale \ndLength \right) \ndBondDensity.
\end{equation}
It turns out it is convenient to define a modified $\ndOnConst$
parameter to more clearly separate the \emph{overall} binding rate
from the length-dependence of the binding rate captured in the
Gaussian term of equation (\ref{eq:ndim-ss-nosat}). The overall bond
formation rate per receptor is the integral of
$\ndOnConst \exp(-\onForceScale \ndLength^2/2)$ over $\ndWallDist$,
which as written depends on both $\ndOnConst$ and
$\onForceScale$. Therefore define a new parameter
$\ndOnConst = \newOnConst \sqrt{\onForceScale/(2\pi)}$, so that
$\newOnConst$ controls the overall binding rate independent of
$\onForceScale$.

For each parameter quadruple, we calculate the total number of bonds:
\begin{equation}
  \label{eq:m-tot}
  \ndBondDensity_\tn{tot}(\newOnConst, \onForceScale, \offForceScale,
  \ndRotation) = \Int{\Int{ \ndBondDensity(\ndWallDist, \recAngle)}
    {\ndWallDist} {-\infty} {\infty}} {\recAngle} {-\pi/2} {\pi/2},
\end{equation}
and then compare $\ndBondDensity_\tn{tot}$ to the desired range. This
defines some region in 4D parameter space where stochasticity is
important. This region is then visualized by looking at 2D
heatmaps, shown in Figure \ref{fig:parameter-sweeps}. For a given
heatmap, each pixel is colored according the fraction of parameters in
the other 2 dimensions that yielded an $\ndBondDensity_\tn{tot}$
within the desired range. More precisely, the $(\newOnConst,
\ndRotation)$ heatmap (for example), visualizes the function
\begin{equation}
  \label{eq:kap-om-heatmap}
  (N_\onForceScale N_\offForceScale)\inv \sum_{i=1}^{N_\onForceScale}
  \sum_{j=1}^{N_\offForceScale}
  \left[\ndBondDensity_\tn{tot}(\newOnConst, \onForceScale_i,
    \offForceScale_j, \omega) \in [10^{-4}, 10^{-2}] \right],
\end{equation}
where $N_\onForceScale$ is the number of values of $\onForceScale$
tested, $N_\offForceScale$ is the number of values of $\offForceScale$
tested, and $\left[ \ndBondDensity_\tn{tot} \in I \right]$ is an
indicator function:
\begin{equation}
  \label{eq:indicator}
  \left[ \ndBondDensity_\tn{tot} \in I \right] =
  \begin{cases}
    1 & \tn{if} \quad \ndBondDensity_\tn{tot} \in I \\
    0 & \tn{if} \quad \ndBondDensity_\tn{tot} \notin I
  \end{cases}
\end{equation}

\begin{figure}
  \centering
  \includegraphics[width=0.75\textwidth]{parameter_sweep_heatmaps.png}
  \caption[2D heatmaps of regions where stochasticity may be
  relevant.]{2D heatmaps showing the fraction of tested parameter
    values where the total number of bonds at steady state is within
    the desired range for stochasticity. The total scanned parameter
    space is 4-dimensional, so for each 2D heatmap the averages
    displayed are carried out over the other two dimensions.}
  \label{fig:parameter-sweeps}
\end{figure}

From Figure \ref{fig:parameter-sweeps}, $\newOnConst$ is the most
important parameter in controlling whether or not the number of bonds
at steady state is within the desired range. There is a large strip in
$(\newOnConst, \ndRotation)$ space where every tested value of
$\onForceScale$ and $\offForceScale$ resulted in an
$\ndBondDensity_\tn{tot}$ value in the interval $[10^{-4}, 10^{-2}]$,
and conversely there are large regions where every tested value of
$\onForceScale$ and $\offForceScale$ resulted in a total bond number
outside that interval. Compare this to the heatmaps in
$(\onForceScale, \ndRotation)$ and $(\offForceScale, \ndRotation)$
space, which are essentially uniform. Put another way, for any choice
of $\onForceScale$ and $\offForceScale$ there is a region of
$\newOnConst$ in which stochasticity is important.

\begin{figure}
  \centering
  \includegraphics[width=0.75\textwidth]{kappa_omega_sweep}
  \caption[Filled contour plot of $\ndBondDensity_\tn{tot}$ for
  $\onForceScale = 10^4$ and $\offForceScale = 16$.]{Filled contour
    plot of $\ndBondDensity_\tn{tot}$ for $\onForceScale = 10^4$ and
    $\offForceScale = 16$. The region where stochasticity is likely to
    be significant is when
    $10^{-4} < \ndBondDensity_\tn{tot} < 10^{-2}$.}
  \label{fig:kappa-omega-sweep}
\end{figure}

This analysis provides one way to estimate the on rate
$\newOnConst$. The on rate is difficult to measure experimentally. For
example, Hammer et. al. \cite{Hammer1992a} give a possible range of 5
orders of magnitude for $\ndOnConst$! Figure
\ref{fig:kappa-omega-sweep} shows values of $\ndBondDensity_\tn{tot}$
for a range of $\newOnConst$ and $\ndRotation$, with $\onForceScale$
and $\offForceScale$ values estimated from the literature in Table
\ref{tab:nd-params}. The applied rotation rate $\ndAppliedRot$ has a
significant influence on the $\newOnConst$ values that produce a small
number of total bonds. 

Figures \ref{fig:parameter-sweeps} and \ref{fig:kappa-omega-sweep}
treat $\ndRotation$ as a parameter that can be varied externally by
the modeler, but in the full nonlinear rolling model $\ndRotation$ is
calculated by balancing drag forces with bond forces. Therefore, these
figures cannot be used to choose values of $\onConst$ for specified
applied rotation rates. For example, larger values of $\newOnConst$
allow more bonds to form, which decreases $\ndRotation$ further which
allows even more bonds to form. There is some positive feedback in
this system between $\newOnConst$ and $\ndRotation$, and there may be
a threshold in which $\newOnConst$ triggers a switch from a regime
where there are many bonds with the surface, to one in which there are
few bonds. In the next section, we look directly at the relationship
between the applied rotation rate $\ndAppliedRot$ and actual rotation
rate $\ndRotation$ for a range of $\kappa$ values.

\subsection{Steady rotation rate as a function of applied rotation}
\label{sec:rotation-vs-applied}

In order to relate the platelet rotation rate $\ndRotation$ to the
applied rotation rate $\ndAppliedRot$, first choose a set of
$\ndRotation$ values. Then find the steady state bond distribution for
those $\ndRotation_i$ using (\ref{eq:ndim-ss-nosat}) and find the
torque generated by that distribution of bonds. The $\ndAppliedRot$
necessary to drive the platelet at rotation rate $\ndRotation$ is
found by solving the torque balance equation:
(\ref{eq:ndim-torbal}). This gives a set of
$(\ndAppliedRot, \ndRotation)$ pairs that can be interpolated.

Because Figure \ref{fig:kappa-omega-sweep} suggests that there are a
range of $\newOnConst$ values that may result in a small number of
bonds, we find the relationship between applied rotation rate and
platelet rotation for a range of different $\newOnConst$s in Figure
\ref{fig:steady-states}. Through some trial and error, a range of
$[.1, 2]$ was found to give interesting behavior. If $\newOnConst <
0.1$, bonds form too infrequently to affect the platelet's motion, and
if $\newOnConst > 2$ then many bonds form with the surface and the
platelet remains firmly bound for realistic values of $\ndAppliedRot$.

\begin{figure}
  \centering
  \includegraphics[width=\textwidth]{kappa-sweeps}
  \caption[Relation between $\ndRotation$ vs. $\ndTotalTorque$ and
  $\ndAppliedRot$ vs. $\ndRotation$ for varying
  $\newOnConst$.]{Relation between $\ndRotation$ vs. $\ndTotalTorque$
    (left) and $\ndAppliedRot$ vs. $\ndRotation$ (right) for
    $\newOnConst = 0.1, 0.5, 1, 2$ (top to bottom). The values of the
    other parameters are $\onForceScale = 10^4$,
    $\offForceScale = 16$, and $\ndRotFriction = 10^{-5}$.}
  \label{fig:steady-states}
\end{figure}

Figure \ref{fig:steady-states} demonstrates that there is a range of
$\newOnConst$ for which multiple steady state values of $\ndRotation$
are possible for a single $\ndAppliedRot$. The biological
interpretation of Figure \ref{fig:steady-states} is that for low
applied angular velocity (i.e. low shear rate) the only stable
behavior is for platelets to adhere to the surface and move very
slowly with respect to the fluid velocity. At some medium shear rate,
a second stable behavior arises where platelets roll along the surface
substantially faster than the adhered platelets. Finally, at a high
shear rate, all platelets are moving at a speed close to the fluid
velocity and none are adhered to the surface: the torque generated by
the fluid on the platelet is too large. While I have referred to
stable and unstable branches in Figure \ref{fig:steady-states}, it is
worth noting that this analysis doesn't show which solution branches
are stable and which are unstable.

Putting this aside, I will nonetheless refer to this region as the
bistable region for convenience. One final thing that Figure
\ref{fig:steady-states} shows is that the location and width of the
bistable region changes as $\newOnConst$ changes. In particular, as
$\newOnConst$ increases, the bistable region shifts to a higher applied
rotation rate and the region widens. This result agrees with
intuition, in that we would expect the lower branch (where the
platelet is firmly bound to the surface and moving very slowly) to
extend to larger $\ndAppliedRot$ values. It also makes sense that the
upper branch (where the platelet is rolling at a rate close to that
applied by the fluid) loses its stability for lower values of
$\ndAppliedRot$.

\subsection{Effects of activation on stable rolling behavior}
\label{sec:effects-activation}

While platelet activation isn't included in this model, it is still
possible to crudely examine the effect of activation on platelet
rolling by looking at how rolling behavior changes as you vary
parameters that can change with activation. Two parameters that could
change with platelet activation are $\receptorDensity$ and
$\onConst$.

The nondimensional bond formation rate constant, $\ndOnConst$ varies
proportionally with $\onConst$. With priming we would expect on rates
to increase, yielding an increase in $\ndOnConst$. This should result
in a higher density of bonds at steady state, generating larger
torques at the same rotation rate $\ndRotation$. As shown in Figure
\ref{fig:steady-states}, the torque magnitudes increase with
increasing $\newOnConst$. This change in the torque magnitude acts to
shift the region of bistability to larger values of $\ndAppliedRot$.

Another platelet response to activation is the recruitment of
additional adhesion mole\-cules to the surface of the cell. In the
model, this is represented as an increase in $\receptorDensity$. As
discussed in Appendix \ref{app:nondim}, $\ndRotFriction$ and
$\ndVelFriction$ are inversely proportional to $\receptorDensity$
(also in Table \ref{tab:nd-vars}). In the rolling only case,
$\ndRotFriction$ is the only parameter of these two that matters, and
the effect of changing its value can be seen by examining equations
(\ref{eq:ndim-ss}) and (\ref{eq:ndim-torbal}). First,
$\receptorDensity$ does not appear in any parameters in the
(\ref{eq:ndim-ss}), and so it does not affect the nondimensional bond
density, it only affects how much the rolling velocity $\ndRotation$
depends on $\totalTorque$. Increasing $\receptorDensity$ gives a
proportional decrease in $\ndRotFriction$, which in turn increases the
magnitude of the term $\ndTotalTorque/\ndRotFriction$ in the equation
above. As shown in Figure \ref{fig:steady-states}, this lengthens the
interval in $\ndAppliedRot$ in which bistability occurs, and the
interval is centered on higher $\ndAppliedRot$ as $\ndRotFriction$
decreases.

\begin{figure}
  \centering
  \includegraphics[width=.5\textwidth]{variable-receptors}
  \caption[Steady state rotation rates for 4 different
  $\ndRotFriction$ values.]{Steady state rotation rates for
    $\ndRotFriction = 5 \times 10^{-6}, 10^{-5}, 2 \times 10^{-5}, 5
    \times 10^{-5}$. $\onForceScale$ and $\offForceScale$ are the same
    as in Figure \ref{fig:steady-states}, and $\newOnConst = 1$.}
  \label{fig:variable-receptors}
\end{figure}

According to this analysis, either increasing the bond formation rate
or increasing the number of receptors on the surface results in the
platelet firmly binding to the surface at higher applied rotation
rates which agrees with intuition. However, the above plots also show
there is switching behavior as $\onConst$ or $\receptorDensity$ is
changed.

\section{Time-dependent experiments}
\label{sec:time-dep-exp}

Next we look at the time-dependent dynamics of the deterministic and
stochastic models. Figures
\ref{fig:small-kappa-trials}--\ref{fig:large-kappa-trials} show
results from running the deterministic and stochastic simulations at
$\ndOnConst = 1, \, 10, \, 50$ (with
$\onForceScale = 2.3 \times 10^4$, this corresponds to
$\newOnConst = 0.016, \, 0.16, \, 0.83$). There is a switch in rolling
behavior between Figure \ref{fig:small-kappa-trials} and Figure
\ref{fig:med-kappa-trials} in which most of the platelets go from
moving at the fluid velocity to being mostly bound to the surface. In
each of these figures, the left column shows the solution of the
deterministic algorithm in red, and the mean plus/minus 2 standard
errors of the mean from 1000 stochastic simulations. In the right
column, the solution of the deterministic algorithm is shown in red
and a single stochastic simulation is shown in blue.

\begin{figure}
  \centering
  \includegraphics[width=\textwidth]{small-kappa-trials}
  \caption[Deterministic and stochastic solutions with
  $\ndOnConst = 1$.]{Deterministic (red) and stochastic solutions
    (blue) with $\ndOnConst = 1$. The plots on the left compare the
    solution of the deterministic model with the mean (thick blue
    line) and mean $\pm$ $2\,\tn{SEM}$ (thin blue lines, SEM: standard
    error of the mean). The plots on the right show an individual run
    of the stochastic algorithm in blue. All other parameters are
    taken from Table \ref{tab:nd-params}.}
  \label{fig:small-kappa-trials}
\end{figure}

\begin{figure}
  \centering
  \includegraphics[width=\textwidth]{med-kappa-trials}
  \caption[Deterministic and stochastic solutions with
  $\ndOnConst = 10$]{Deterministic (red) and stochastic solutions
    (blue) with $\ndOnConst = 10$. All other parameters are taken from
    Table \ref{tab:nd-params}.}
  \label{fig:med-kappa-trials}
\end{figure}

\begin{figure}
  \centering
  \includegraphics[width=\textwidth]{large-kappa-trials}
  \caption[Deterministic and stochastic solutions with
  $\ndOnConst = 50$.]{Deterministic (red) and stochastic (blue)
    solutions with $\ndOnConst = 50$. All other parameters are taken
    from Table \ref{tab:nd-params}.}
  \label{fig:large-kappa-trials}
\end{figure}

The average number of bonds over 1000 stochastic trials tracks well
with the deterministic simulation in Figures
\ref{fig:med-kappa-trials} and \ref{fig:large-kappa-trials}. The
average translation velocity and rotation rate is also in good
agreement with the deterministic model in Figure
\ref{fig:large-kappa-trials}. However, there is a discrepancy in the
rolling velocities in Figures \ref{fig:med-kappa-trials} and
\ref{fig:small-kappa-trials}. I suspect this is because the population
velocity distribution is heavily skewed: right skewed in Figure
\ref{fig:small-kappa-trials}, and left skewed in Figure
\ref{fig:med-kappa-trials}.

The example stochastic runs plotted in each of the three figures give
an explanation for this. Because $\ndRotFriction$ is so small, a
single bond between the platelet and surface is enough to halt a
platelet, therefore a platelet can essentially exist in one of two
states: either completely unbound from the wall and moving with the
fluid, or bound to the wall and stationary. There is only a brief
transition period after an unbound platelet forms a bond with the wall
in which the bond stretches until the spring force exerted by the bond
equilibrates with the stationary drag force on the platelet. Clearly
none of this can be captured in the deterministic model, and so there
is a discrepancy between it and the stochastic simulations.

This leaping behavior exhibited by the stochastic model is
seen experimentally, and is called \emph{saltatory
  motion}. The stochastic trials plotted in Figures
\ref{fig:small-kappa-trials}--\ref{fig:large-kappa-trials} indicate
there are really two types of jumps:
\begin{enumerate}
\item jumps where the platelet releases from surface entirely and
  moves with the fluid until a new bond forms, and
\item jumps where the platelet has multiple bonds with the surface,
  and a load-bearing bond breaks resulting in the platelet quickly
  accelerating and decelerating again as another bond is put under load.
\end{enumerate}
It isn't clear if this distinction is important, but jumps of the 2nd
type can only be a fraction of a platelet radius whereas a platelet
can travel much farther if it is completely released into the flow.

\section{Summary}
\label{sec:summary-results}

A steady state analysis of a reduced version of the deterministic
model shows that the model predicts a small number of bonds
($\sim$0.1--10) for realistic values of $\onForceScale$ and
$\offForceScale$, and for values of $\ndOnConst$ restricted to a range
of a couple of orders of magnitude. The on rate $\ndOnConst$ is hard
to estimate experimentally, and the values of $\ndOnConst$ that
produce a small number of bonds are plausible. Steady state analysis
also shows that there are multiple steady state rolling behaviors for
a range of $\ndAppliedRot$, suggesting the presence of switch-like
behavior in platelet rolling as $\ndAppliedRot$, $\ndOnConst$, or
$\receptorDensity$ are varied. Finally, the full stochastic and
deterministic models produce rolling behavior that is qualitatively
similar to experimental observation.


% Local Variables:
% TeX-master: "phd-thesis.ltx"
% End:
