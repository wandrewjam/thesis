%%% -*-LaTeX-*-

\chapter{Numerical Schemes}
\label{cha:numerical-schemes}

\section{Deterministic model}
\label{sec:deterministic-model}

The PDE \eqref{eq:nd-bond-density} is a linear advection-reaction
equation for known $\ndRotation$ and $\ndVelocity$, but the whole
system including the force balance equations
(i.e. \eqref{eq:nd-bond-density}, \eqref{eq:nd-force-balance}, and
\eqref{eq:nd-torque-balance}) is nonlinear. Numerically, I linearized
the system by lagging the angular and linear velocities in order to
find the bond density distribution at the next time step. Then I
integrated over the bond distribution to find the net force and torque
generated by the bonds, and used that to update the angular and linear
velocities. As suggested by the results above, I will need to modify
this approach to handle the case where a platelet is initially firmly
adhered and unmoving, until fluid forces are applied.
% Discuss the issues with upwinding?

I solve the PDE using a second order Beam-Warming scheme in both
$\recAngle$ and $\ndWallDist$. The velocities $\ndVelocity$ and
$\ndRotation$ are always non-negative, so I approximate the spatial
derivatives with a forward difference. Equation
(\ref{eq:forward-difference}) below gives the forward difference in
$\ndWallDist$; the forward difference in $\recAngle$ is analogous.
\begin{equation}
  \label{eq:forward-difference}
  \Pder{m}{\ndWallDist}(\ndWallDist_i, \recAngle_j) \approx
  \frac{-3m(\ndWallDist_i, \recAngle_j) + 4m(\ndWallDist_{i+1},
    \recAngle_j) - m(\ndWallDist_{i+2},
    \recAngle_j)}{2(\ndWallDist_{i+1} - \ndWallDist_i)}.
\end{equation}

The bond formation term is treated explicitly, so that I can advance
time with just a matrix multiplication, instead of solving a linear
system. The bond breaking term essentially \emph{must} be treated
implicitly, because the breaking rate is very large over much of the
domain $(\ndWallDist, \recAngle)$. Luckily it only requires a scalar
division for each element of $\ndBondDensity^k$ to treat that term
implicitly.

In summary, for each timestep $k$, I do the following:
\begin{enumerate}
\item Advance the PDE in time using the velocities from the previous
  timestep:
  \begin{multline}
    \label{eq:pde-timestep}
    \ndBondDensity_{i,j}^{k+1} = \frac{1}{1 + \Delta \dTime
      \exp(\offForceScale \ndLength_{ij})} \left(\ndBondDensity_{ij}^k
      + \ndRotation^k \Delta \dTime
      \left(\frac{-3\ndBondDensity_{ij}^k + 4\ndBondDensity_{i, j+1}^k
          - \ndBondDensity_{i, j+2}^k}{2\nu}\right) + \right. \\
    \left. \ndVelocity^k
      \Delta \dTime \left(\frac{-3\ndBondDensity_{ij}^k +
          4\ndBondDensity_{i+1, j}^k - \ndBondDensity_{i+2,
            j}^k}{2h}\right) + \kappa \Delta \dTime \exp\left(-\eta
        \frac{\ndLength_{ij}}{2}\right) \left(1 - h \sum_{q = 0}^{N-1}
        \ndBondDensity_{qj}^k\right)\right)
  \end{multline}
\item Calculate $\ndHorzTotalForce^{k+1}$ and $\ndTotalTorque^{k+1}$
  from the new bond distribution $\ndBondDensity^{k+1}$ using the
  trapezoid rule:
  \begin{align}
    \label{eq:num-force-calc}
    \ndHorzTotalForce^{k+1} &= (h \nu)^{-1} \sum_{i=0}^{2M} w_i
                              \sum_{j=0}^{2N} w_j
                              \ndBondDensity_{ij}^{k+1}
                              \left(\ndWallDist_i -
                              \sin\recAngle_j\right) \\
    \ndTotalTorque^{k+1} &= (h \nu)^{-1} \sum_{i=0} ^{2M} w_i
                           \sum_{j=0}^{2N} w_j
                           \ndBondDensity_{ij}^{k+1} \left[ \left(1 -
                           \cos\recAngle_j + \ndSeparation \right)
                           \sin\recAngle_j + \left(\sin\recAngle_j -
                           \ndWallDist_i \right) \right].
  \end{align}
  Here the $w_i$s are the weights in the trapezoid rule: $\{w_i\} =
  \{1/2, 1, \hdots, 1, 1/2\}$.
\item Find the new angular and linear velocities:
  \begin{align}
    \label{eq:ang-timestep}
    \ndRotation^{k+1} &= \ndAppliedRot + \ndTotalTorque^{k+1}/\ndRotFriction \\
    \label{eq:vel-timestep}
    \ndVelocity^{k+1} &= \ndAppliedVel + \ndHorzTotalForce^{k+1}/\ndVelFriction.
  \end{align}
\item Repeat
\end{enumerate}

\section{Stochastic model}
\label{sec:app-stochastic-model}

The full modified Gillespie algorithm is described below. Parameters
of the stochastic simulation algorithm are:
\begin{itemize}
\item the total simulation time $T_\tn{end}$,
\item the maximum time step allowed $\Delta t_\tn{max}$,
\item the number $N$ of $\recAngle$-bins on the surface of the
  platelet,
\item the total number of receptors in a $\recAngle$-bin $b_\tn{max}
  \equiv \receptorDensity \pi / N$, and
\item the $\ndWallDist$ domain length $\domLength$ so that
  $\ndWallDist \in [-\domLength, \domLength]$.
\end{itemize}

\begin{enumerate}
\item \textbf{Initialize arrays}:
  \begin{align}
    \label{eq:bondList}
    \texttt{bondList} &=
    \begin{bmatrix}
      \ndWallDist_0 & j_0 \\
      \ndWallDist_1 & j_1 \\
      \vdots & \vdots \\
      \ndWallDist_{\texttt{numBonds} - 1} & j_{\texttt{numBonds}-1}
    \end{bmatrix}, \\
    \label{eq:midpoints}
    \texttt{midpoints} &= \left[ \binMidpoint{0}, \binMidpoint{1},
      \hdots, \binMidpoint{N-1} \right].
  \end{align}
\item \textbf{Initialize scalars}:
  \begin{align}
    \label{eq:t-init}
    t &= 0 \\
    \label{eq:force-init}
    \ndHorzTotalForce &= \sum_{i=0}^{\texttt{numBonds}-1}
                        \ndHorzForce\left(\ndWallDist_i,
                        \binMidpoint{j_i} \right) \\
    \label{eq:torque-init}
    \ndTotalTorque &= \sum_{i=0}^{\texttt{numBonds}-1} \ndTorque\left(
                     \ndWallDist_i, \binMidpoint{j_i} \right) \\
    \label{eq:vel-init}
    \ndVelocity &= \ndAppliedVel - \ndHorzTotalForce/\ndVelFriction \\
    \label{eq:rot-init}
    \ndRotation &= \ndAppliedRot - \ndTotalTorque/\ndRotFriction
  \end{align}
\item \textbf{Calculate the reaction propensities}:
  \begin{enumerate}
  \item The breaking propensities for each existing bond:
    \begin{equation}
      a_i = \exp(\offForceScale \ndLength(\ndWallDist_i,
      \binMidpoint{j_i})) \quad \tn{for} \quad i = 0, \hdots,
      \texttt{numBonds} - 1.
    \end{equation}
  \item The formation propensities for each $\recAngle$-bin:
    \begin{multline}
      a_i =
      \begin{cases}
        b^\tn{avail}_i \ndOnConst
        \Int{\exp\left(-\frac{\onForceScale}{2} \ndLength^2
            \left(\ndWallDist', \binMidpoint{i}\right) \right)}
        {\ndWallDist'} {-\domLength} {\domLength} & \tn{if}
        \quad \binMidpoint{i}
        \in [-\pi/2, \pi/2] \\
        0 & \tn{otherwise}
      \end{cases} \\
      \tn{for} \quad i = \texttt{numBonds}, \hdots, 2N +
      \texttt{numBonds} - 1.
    \end{multline}
  \end{enumerate}
\item \textbf{Find the next reaction}:
  \begin{enumerate}
  \item Generate two random numbers: $r_1, r_2 \sim
    \operatorname{Unif}(0, 1)$.
  \item Calculate the overall rate: $a_0 = \sum_i a^\tn{break}_i +
    \sum_i a^\tn{form}_i$.
  \item Find the time to the next reaction: $\Delta t = a_0\inv
    \ln(r_1\inv)$.
  \item Find which reaction occurred: $j = \tn{the smallest integer
      satisfying} \quad \sum_{i=0}^j a_i > r_2 a_0$.
  \item If $\Delta t_\tn{max} < \Delta t$, set $\Delta t \leftarrow
    \Delta t_\tn{max}$ and $j \leftarrow -1$.
  \end{enumerate}
\item \textbf{Update the bond list}:
  \begin{enumerate}
  \item Update bond positions $\ndWallDist_i \leftarrow \ndWallDist_i
    + \Delta t \ndVelocity$
    for $i=0, \hdots, \texttt{numBonds} - 1$,
    $\binMidpoint{i} \leftarrow \binMidpoint{i} + \Delta t \ndRotation$ for
    $i = 0, \hdots, 2N - 1$.
  \item If $j = -1$, go to step 6
  \item If $j \in [0, \texttt{numBonds})$, remove row $j$ from
    $\texttt{bondList}$. 
  \item If $j \in [\texttt{numBonds}, 2N + \texttt{numBonds})$, add a
    new row to $\texttt{bondList}$ with $j_\tn{new} = j -
    \texttt{numBonds}$ and $z_\tn{new}$ chosen from a truncated normal
    distribution with $\mu = \sin\binMidpoint{j_\tn{new}}$, $\sigma^2
    = \onForceScale\inv$, and bounds $[-\domLength, \domLength]$.
  \end{enumerate}
\item \textbf{Update velocities}:
  \begin{enumerate}
  \item Calculate the new bond forces using equations
    (\ref{eq:force-init}) and (\ref{eq:torque-init}).
  \item Calculate the new velocities using equations
    (\ref{eq:vel-init}) and (\ref{eq:rot-init}).
  \item Update the system time $t \leftarrow t + \Delta t$.
  \end{enumerate}
\item If $t < T_\tn{end}$, go to step 3. Otherwise end the simulation.
\end{enumerate}
% Local Variables:
% TeX-master: "oral-document.ltx"
% End:
