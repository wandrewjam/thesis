%%% -*-LaTeX-*-
%%% ====================================================================
%%% This file is intended to be included as a front-matter section of
%%% the sample thesis, with a command like this
%%%
%%% \optionalfront{Color in typesetting}{%%% -*-LaTeX-*-
%%% ====================================================================
%%% This file is intended to be included as a front-matter section of
%%% the sample thesis, with a command like this
%%%
%%% \optionalfront{Color in typesetting}{%%% -*-LaTeX-*-
%%% ====================================================================
%%% This file is intended to be included as a front-matter section of
%%% the sample thesis, with a command like this
%%%
%%% \optionalfront{Color in typesetting}{%%% -*-LaTeX-*-
%%% ====================================================================
%%% This file is intended to be included as a front-matter section of
%%% the sample thesis, with a command like this
%%%
%%% \optionalfront{Color in typesetting}{\input{color-notes}}
%%%
%%% just before the \uumaintext macro that begins the body of the thesis,
%%% and starts chapter numbering.
%%%
%%% [28-Jan-2018]
%%% ====================================================================

Color is ubiquitous in the commercial publishing industry, color
computer displays are now standard, and color printers are commonly
available on university campuses and in home offices.

The human visual system is acutely aware of color, and when used
carefully, color can make documents more attractive, and easier to
read.  Complex figures are often best presented in color to help
distinguish features, and color background shading in long tabular
displays can make them easier to understand, as illustrated in
Table~\vref{dfact}.

Here is how a portion of that table was written in \LaTeX{}, with
row-shading support from the \texttt{colortbl} package that is loaded
in the document preamble.

\begin{Verbatim}[formatcom = \color{brown4}]
\begin{table}[t]
    \definecolor {HeadingColor} {rgb} {0.60, 0.60, 0.90}
    \definecolor {ShadingColor} {rgb} {0.90, 0.90, 0.99}
    \newcommand{\R}{\rowcolor{ShadingColor}}
    \begin{center}
        \caption{Table of double factorials.}
        \label{dfact}
        \medskip
        % Generate table data and replace _ by \, (thin space)
        % hocd128> load("dfact")
        % hocd128> for (k = -1 ; k < 41; ++k) { m = k % 10; \
        %                if ((0 < m) && (m < 6)) s = "\\R" else s = "  " ; \
        %                printf("%2s %5d & %35.0..3f \\\\\n", s, k, dfact(k)) }
        \begin{tabular}{rr}
            \hline
            \rowcolor{HeadingColor}
            \boldmath $n$ & \multicolumn{1}{c}{\boldmath $n!!$} \\
            \hline
                  -1 &                                   1 \\
                   0 &                                   1 \\
            \R     1 &                                   1 \\
            \R     2 &                                   2 \\
            \R     3 &                                   3 \\
            \R     4 &                                   8 \\
            \R     5 &                                  15 \\
                   6 &                                  48 \\
                   7 &                                 105 \\
                   8 &                                 384 \\
                   9 &                                 945 \\
                  10 &                              3\,840 \\
            ...
        \end{tabular}
    \end{center}
\end{table}
\end{Verbatim}

\begin{table}[t]
    \definecolor {HeadingColor} {rgb} {0.60, 0.60, 0.90}
    \definecolor {ShadingColor} {rgb} {0.90, 0.90, 0.99}
    \newcommand{\R}{\rowcolor{ShadingColor}}
    \begin{center}
        \caption{Table of double factorials.}
        \label{dfact}
        \medskip
        % Generate table data and replace _ by \, (thin space)
        % hocd128> load("dfact")
        % hocd128> for (k = -1 ; k < 41; ++k) { m = k % 10; \
        %                if ((0 < m) && (m < 6)) s = "\\R" else s = "  " ; \
        %                printf("%2s %5d & %35.0..3f \\\\\n", s, k, dfact(k)) }
        \begin{tabular}{rr}
            \hline
            \rowcolor{HeadingColor}
            \boldmath $n$ & \multicolumn{1}{c}{\boldmath $n!!$} \\
            \hline
                  -1 &                                   1 \\
                   0 &                                   1 \\
            \R     1 &                                   1 \\
            \R     2 &                                   2 \\
            \R     3 &                                   3 \\
            \R     4 &                                   8 \\
            \R     5 &                                  15 \\
                   6 &                                  48 \\
                   7 &                                 105 \\
                   8 &                                 384 \\
                   9 &                                 945 \\
                  10 &                               3\,840 \\
            \R    11 &                              10\,395 \\
            \R    12 &                              46\,080 \\
            \R    13 &                             135\,135 \\
            \R    14 &                             645\,120 \\
            \R    15 &                           2\,027\,025 \\
                  16 &                          10\,321\,920 \\
                  17 &                          34\,459\,425 \\
                  18 &                         185\,794\,560 \\
                  19 &                         654\,729\,075 \\
                  20 &                       3\,715\,891\,200 \\
            \R    21 &                      13\,749\,310\,575 \\
            \R    22 &                      81\,749\,606\,400 \\
            \R    23 &                     316\,234\,143\,225 \\
            \R    24 &                   1\,961\,990\,553\,600 \\
            \R    25 &                   7\,905\,853\,580\,625 \\
                  26 &                  51\,011\,754\,393\,600 \\
                  27 &                 213\,458\,046\,676\,875 \\
                  28 &               1\,428\,329\,123\,020\,800 \\
                  29 &               6\,190\,283\,353\,629\,375 \\
                  30 &              42\,849\,873\,690\,624\,000 \\
            \R    31 &             191\,898\,783\,962\,510\,625 \\
            \R    32 &           1\,371\,195\,958\,099\,968\,000 \\
            \R    33 &           6\,332\,659\,870\,762\,850\,625 \\
            \R    34 &          46\,620\,662\,575\,398\,912\,000 \\
            \R    35 &         221\,643\,095\,476\,699\,771\,875 \\
                  36 &       1\,678\,343\,852\,714\,360\,832\,000 \\
                  37 &       8\,200\,794\,532\,637\,891\,559\,375 \\
                  38 &      63\,777\,066\,403\,145\,711\,616\,000 \\
                  39 &     319\,830\,986\,772\,877\,770\,815\,625 \\
                  40 &   2\,551\,082\,656\,125\,828\,464\,640\,000 \\
            \hline
        \end{tabular}
    \end{center}
\end{table}


The distinction between computer input and output is often improved by
showing them in different colors, as in the following indented examples.

\begin{quotation}
\begin{singlespace}
This is human input to a computer:
\definecoloralias {verbatimcolor} {utahred}
\begin{verbatim}
    % cal 1 2018
\end{verbatim}
%
Here is the computer output from that command:
%
\definecoloralias {verbatimcolor} {darkblue}
\begin{verbatim}
        January 2018      
    Su Mo Tu We Th Fr Sa  
        1  2  3  4  5  6  
     7  8  9 10 11 12 13  
    14 15 16 17 18 19 20  
    21 22 23 24 25 26 27  
    28 29 30 31           
\end{verbatim}
\end{singlespace}
\end{quotation}
%
Here is how those paragraphs were written in \LaTeX{}:
%
\begin{Verbatim}[formatcom = \color{brown4}]
\begin{quotation}
\begin{singlespace}
This is human input to a computer:
\definecoloralias {verbatimcolor} {utahred}
\begin{verbatim}
    % cal 1 2018
\end{verbatim}
%
Here is the computer output from that command:
%
\definecoloralias {verbatimcolor} {darkblue}
\begin{verbatim}
        January 2018      
    Su Mo Tu We Th Fr Sa  
        1  2  3  4  5  6  
     7  8  9 10 11 12 13  
    14 15 16 17 18 19 20  
    21 22 23 24 25 26 27  
    28 29 30 31           
\end{verbatim}
\end{singlespace}
\end{quotation}
\end{Verbatim}

\definecoloralias {verbatimcolor} {brown4}

Because verbatim displays are terminated by the first matching
\verb=\end= command, the illustrated input fragment had to be wrapped
in a second kind of verbatim supplied by the \texttt{fancyvrb} package
that is included in the \LaTeX{} preamble of this document.
%
\begin{verbatim}
\begin{Verbatim}[formatcom = \color{brown4}]
...
\end{Verbatim}
\end{verbatim}
%

Color reproduction fidelity is device dependent, and care must be
taken to avoid use of colors that should be distinct, but are not, on
some output devices, or for those humans who have some degree of color
blindness.  For many such people, red and green are just similar
shades of brown.

You should also ensure sufficient contrast between foreground and
background colors: yellow text on a white background, or blue text on
a black background, both pose severe reading difficulties.

For dissertations and theses, it is of particular concern that color
documents output on black-and-white printers should not suffer serious
loss of readability, such as a light color being printed in a
hard-to-see faint grey.
}
%%%
%%% just before the \uumaintext macro that begins the body of the thesis,
%%% and starts chapter numbering.
%%%
%%% [28-Jan-2018]
%%% ====================================================================

Color is ubiquitous in the commercial publishing industry, color
computer displays are now standard, and color printers are commonly
available on university campuses and in home offices.

The human visual system is acutely aware of color, and when used
carefully, color can make documents more attractive, and easier to
read.  Complex figures are often best presented in color to help
distinguish features, and color background shading in long tabular
displays can make them easier to understand, as illustrated in
Table~\vref{dfact}.

Here is how a portion of that table was written in \LaTeX{}, with
row-shading support from the \texttt{colortbl} package that is loaded
in the document preamble.

\begin{Verbatim}[formatcom = \color{brown4}]
\begin{table}[t]
    \definecolor {HeadingColor} {rgb} {0.60, 0.60, 0.90}
    \definecolor {ShadingColor} {rgb} {0.90, 0.90, 0.99}
    \newcommand{\R}{\rowcolor{ShadingColor}}
    \begin{center}
        \caption{Table of double factorials.}
        \label{dfact}
        \medskip
        % Generate table data and replace _ by \, (thin space)
        % hocd128> load("dfact")
        % hocd128> for (k = -1 ; k < 41; ++k) { m = k % 10; \
        %                if ((0 < m) && (m < 6)) s = "\\R" else s = "  " ; \
        %                printf("%2s %5d & %35.0..3f \\\\\n", s, k, dfact(k)) }
        \begin{tabular}{rr}
            \hline
            \rowcolor{HeadingColor}
            \boldmath $n$ & \multicolumn{1}{c}{\boldmath $n!!$} \\
            \hline
                  -1 &                                   1 \\
                   0 &                                   1 \\
            \R     1 &                                   1 \\
            \R     2 &                                   2 \\
            \R     3 &                                   3 \\
            \R     4 &                                   8 \\
            \R     5 &                                  15 \\
                   6 &                                  48 \\
                   7 &                                 105 \\
                   8 &                                 384 \\
                   9 &                                 945 \\
                  10 &                              3\,840 \\
            ...
        \end{tabular}
    \end{center}
\end{table}
\end{Verbatim}

\begin{table}[t]
    \definecolor {HeadingColor} {rgb} {0.60, 0.60, 0.90}
    \definecolor {ShadingColor} {rgb} {0.90, 0.90, 0.99}
    \newcommand{\R}{\rowcolor{ShadingColor}}
    \begin{center}
        \caption{Table of double factorials.}
        \label{dfact}
        \medskip
        % Generate table data and replace _ by \, (thin space)
        % hocd128> load("dfact")
        % hocd128> for (k = -1 ; k < 41; ++k) { m = k % 10; \
        %                if ((0 < m) && (m < 6)) s = "\\R" else s = "  " ; \
        %                printf("%2s %5d & %35.0..3f \\\\\n", s, k, dfact(k)) }
        \begin{tabular}{rr}
            \hline
            \rowcolor{HeadingColor}
            \boldmath $n$ & \multicolumn{1}{c}{\boldmath $n!!$} \\
            \hline
                  -1 &                                   1 \\
                   0 &                                   1 \\
            \R     1 &                                   1 \\
            \R     2 &                                   2 \\
            \R     3 &                                   3 \\
            \R     4 &                                   8 \\
            \R     5 &                                  15 \\
                   6 &                                  48 \\
                   7 &                                 105 \\
                   8 &                                 384 \\
                   9 &                                 945 \\
                  10 &                               3\,840 \\
            \R    11 &                              10\,395 \\
            \R    12 &                              46\,080 \\
            \R    13 &                             135\,135 \\
            \R    14 &                             645\,120 \\
            \R    15 &                           2\,027\,025 \\
                  16 &                          10\,321\,920 \\
                  17 &                          34\,459\,425 \\
                  18 &                         185\,794\,560 \\
                  19 &                         654\,729\,075 \\
                  20 &                       3\,715\,891\,200 \\
            \R    21 &                      13\,749\,310\,575 \\
            \R    22 &                      81\,749\,606\,400 \\
            \R    23 &                     316\,234\,143\,225 \\
            \R    24 &                   1\,961\,990\,553\,600 \\
            \R    25 &                   7\,905\,853\,580\,625 \\
                  26 &                  51\,011\,754\,393\,600 \\
                  27 &                 213\,458\,046\,676\,875 \\
                  28 &               1\,428\,329\,123\,020\,800 \\
                  29 &               6\,190\,283\,353\,629\,375 \\
                  30 &              42\,849\,873\,690\,624\,000 \\
            \R    31 &             191\,898\,783\,962\,510\,625 \\
            \R    32 &           1\,371\,195\,958\,099\,968\,000 \\
            \R    33 &           6\,332\,659\,870\,762\,850\,625 \\
            \R    34 &          46\,620\,662\,575\,398\,912\,000 \\
            \R    35 &         221\,643\,095\,476\,699\,771\,875 \\
                  36 &       1\,678\,343\,852\,714\,360\,832\,000 \\
                  37 &       8\,200\,794\,532\,637\,891\,559\,375 \\
                  38 &      63\,777\,066\,403\,145\,711\,616\,000 \\
                  39 &     319\,830\,986\,772\,877\,770\,815\,625 \\
                  40 &   2\,551\,082\,656\,125\,828\,464\,640\,000 \\
            \hline
        \end{tabular}
    \end{center}
\end{table}


The distinction between computer input and output is often improved by
showing them in different colors, as in the following indented examples.

\begin{quotation}
\begin{singlespace}
This is human input to a computer:
\definecoloralias {verbatimcolor} {utahred}
\begin{verbatim}
    % cal 1 2018
\end{verbatim}
%
Here is the computer output from that command:
%
\definecoloralias {verbatimcolor} {darkblue}
\begin{verbatim}
        January 2018      
    Su Mo Tu We Th Fr Sa  
        1  2  3  4  5  6  
     7  8  9 10 11 12 13  
    14 15 16 17 18 19 20  
    21 22 23 24 25 26 27  
    28 29 30 31           
\end{verbatim}
\end{singlespace}
\end{quotation}
%
Here is how those paragraphs were written in \LaTeX{}:
%
\begin{Verbatim}[formatcom = \color{brown4}]
\begin{quotation}
\begin{singlespace}
This is human input to a computer:
\definecoloralias {verbatimcolor} {utahred}
\begin{verbatim}
    % cal 1 2018
\end{verbatim}
%
Here is the computer output from that command:
%
\definecoloralias {verbatimcolor} {darkblue}
\begin{verbatim}
        January 2018      
    Su Mo Tu We Th Fr Sa  
        1  2  3  4  5  6  
     7  8  9 10 11 12 13  
    14 15 16 17 18 19 20  
    21 22 23 24 25 26 27  
    28 29 30 31           
\end{verbatim}
\end{singlespace}
\end{quotation}
\end{Verbatim}

\definecoloralias {verbatimcolor} {brown4}

Because verbatim displays are terminated by the first matching
\verb=\end= command, the illustrated input fragment had to be wrapped
in a second kind of verbatim supplied by the \texttt{fancyvrb} package
that is included in the \LaTeX{} preamble of this document.
%
\begin{verbatim}
\begin{Verbatim}[formatcom = \color{brown4}]
...
\end{Verbatim}
\end{verbatim}
%

Color reproduction fidelity is device dependent, and care must be
taken to avoid use of colors that should be distinct, but are not, on
some output devices, or for those humans who have some degree of color
blindness.  For many such people, red and green are just similar
shades of brown.

You should also ensure sufficient contrast between foreground and
background colors: yellow text on a white background, or blue text on
a black background, both pose severe reading difficulties.

For dissertations and theses, it is of particular concern that color
documents output on black-and-white printers should not suffer serious
loss of readability, such as a light color being printed in a
hard-to-see faint grey.
}
%%%
%%% just before the \uumaintext macro that begins the body of the thesis,
%%% and starts chapter numbering.
%%%
%%% [28-Jan-2018]
%%% ====================================================================

Color is ubiquitous in the commercial publishing industry, color
computer displays are now standard, and color printers are commonly
available on university campuses and in home offices.

The human visual system is acutely aware of color, and when used
carefully, color can make documents more attractive, and easier to
read.  Complex figures are often best presented in color to help
distinguish features, and color background shading in long tabular
displays can make them easier to understand, as illustrated in
Table~\vref{dfact}.

Here is how a portion of that table was written in \LaTeX{}, with
row-shading support from the \texttt{colortbl} package that is loaded
in the document preamble.

\begin{Verbatim}[formatcom = \color{brown4}]
\begin{table}[t]
    \definecolor {HeadingColor} {rgb} {0.60, 0.60, 0.90}
    \definecolor {ShadingColor} {rgb} {0.90, 0.90, 0.99}
    \newcommand{\R}{\rowcolor{ShadingColor}}
    \begin{center}
        \caption{Table of double factorials.}
        \label{dfact}
        \medskip
        % Generate table data and replace _ by \, (thin space)
        % hocd128> load("dfact")
        % hocd128> for (k = -1 ; k < 41; ++k) { m = k % 10; \
        %                if ((0 < m) && (m < 6)) s = "\\R" else s = "  " ; \
        %                printf("%2s %5d & %35.0..3f \\\\\n", s, k, dfact(k)) }
        \begin{tabular}{rr}
            \hline
            \rowcolor{HeadingColor}
            \boldmath $n$ & \multicolumn{1}{c}{\boldmath $n!!$} \\
            \hline
                  -1 &                                   1 \\
                   0 &                                   1 \\
            \R     1 &                                   1 \\
            \R     2 &                                   2 \\
            \R     3 &                                   3 \\
            \R     4 &                                   8 \\
            \R     5 &                                  15 \\
                   6 &                                  48 \\
                   7 &                                 105 \\
                   8 &                                 384 \\
                   9 &                                 945 \\
                  10 &                              3\,840 \\
            ...
        \end{tabular}
    \end{center}
\end{table}
\end{Verbatim}

\begin{table}[t]
    \definecolor {HeadingColor} {rgb} {0.60, 0.60, 0.90}
    \definecolor {ShadingColor} {rgb} {0.90, 0.90, 0.99}
    \newcommand{\R}{\rowcolor{ShadingColor}}
    \begin{center}
        \caption{Table of double factorials.}
        \label{dfact}
        \medskip
        % Generate table data and replace _ by \, (thin space)
        % hocd128> load("dfact")
        % hocd128> for (k = -1 ; k < 41; ++k) { m = k % 10; \
        %                if ((0 < m) && (m < 6)) s = "\\R" else s = "  " ; \
        %                printf("%2s %5d & %35.0..3f \\\\\n", s, k, dfact(k)) }
        \begin{tabular}{rr}
            \hline
            \rowcolor{HeadingColor}
            \boldmath $n$ & \multicolumn{1}{c}{\boldmath $n!!$} \\
            \hline
                  -1 &                                   1 \\
                   0 &                                   1 \\
            \R     1 &                                   1 \\
            \R     2 &                                   2 \\
            \R     3 &                                   3 \\
            \R     4 &                                   8 \\
            \R     5 &                                  15 \\
                   6 &                                  48 \\
                   7 &                                 105 \\
                   8 &                                 384 \\
                   9 &                                 945 \\
                  10 &                               3\,840 \\
            \R    11 &                              10\,395 \\
            \R    12 &                              46\,080 \\
            \R    13 &                             135\,135 \\
            \R    14 &                             645\,120 \\
            \R    15 &                           2\,027\,025 \\
                  16 &                          10\,321\,920 \\
                  17 &                          34\,459\,425 \\
                  18 &                         185\,794\,560 \\
                  19 &                         654\,729\,075 \\
                  20 &                       3\,715\,891\,200 \\
            \R    21 &                      13\,749\,310\,575 \\
            \R    22 &                      81\,749\,606\,400 \\
            \R    23 &                     316\,234\,143\,225 \\
            \R    24 &                   1\,961\,990\,553\,600 \\
            \R    25 &                   7\,905\,853\,580\,625 \\
                  26 &                  51\,011\,754\,393\,600 \\
                  27 &                 213\,458\,046\,676\,875 \\
                  28 &               1\,428\,329\,123\,020\,800 \\
                  29 &               6\,190\,283\,353\,629\,375 \\
                  30 &              42\,849\,873\,690\,624\,000 \\
            \R    31 &             191\,898\,783\,962\,510\,625 \\
            \R    32 &           1\,371\,195\,958\,099\,968\,000 \\
            \R    33 &           6\,332\,659\,870\,762\,850\,625 \\
            \R    34 &          46\,620\,662\,575\,398\,912\,000 \\
            \R    35 &         221\,643\,095\,476\,699\,771\,875 \\
                  36 &       1\,678\,343\,852\,714\,360\,832\,000 \\
                  37 &       8\,200\,794\,532\,637\,891\,559\,375 \\
                  38 &      63\,777\,066\,403\,145\,711\,616\,000 \\
                  39 &     319\,830\,986\,772\,877\,770\,815\,625 \\
                  40 &   2\,551\,082\,656\,125\,828\,464\,640\,000 \\
            \hline
        \end{tabular}
    \end{center}
\end{table}


The distinction between computer input and output is often improved by
showing them in different colors, as in the following indented examples.

\begin{quotation}
\begin{singlespace}
This is human input to a computer:
\definecoloralias {verbatimcolor} {utahred}
\begin{verbatim}
    % cal 1 2018
\end{verbatim}
%
Here is the computer output from that command:
%
\definecoloralias {verbatimcolor} {darkblue}
\begin{verbatim}
        January 2018      
    Su Mo Tu We Th Fr Sa  
        1  2  3  4  5  6  
     7  8  9 10 11 12 13  
    14 15 16 17 18 19 20  
    21 22 23 24 25 26 27  
    28 29 30 31           
\end{verbatim}
\end{singlespace}
\end{quotation}
%
Here is how those paragraphs were written in \LaTeX{}:
%
\begin{Verbatim}[formatcom = \color{brown4}]
\begin{quotation}
\begin{singlespace}
This is human input to a computer:
\definecoloralias {verbatimcolor} {utahred}
\begin{verbatim}
    % cal 1 2018
\end{verbatim}
%
Here is the computer output from that command:
%
\definecoloralias {verbatimcolor} {darkblue}
\begin{verbatim}
        January 2018      
    Su Mo Tu We Th Fr Sa  
        1  2  3  4  5  6  
     7  8  9 10 11 12 13  
    14 15 16 17 18 19 20  
    21 22 23 24 25 26 27  
    28 29 30 31           
\end{verbatim}
\end{singlespace}
\end{quotation}
\end{Verbatim}

\definecoloralias {verbatimcolor} {brown4}

Because verbatim displays are terminated by the first matching
\verb=\end= command, the illustrated input fragment had to be wrapped
in a second kind of verbatim supplied by the \texttt{fancyvrb} package
that is included in the \LaTeX{} preamble of this document.
%
\begin{verbatim}
\begin{Verbatim}[formatcom = \color{brown4}]
...
\end{Verbatim}
\end{verbatim}
%

Color reproduction fidelity is device dependent, and care must be
taken to avoid use of colors that should be distinct, but are not, on
some output devices, or for those humans who have some degree of color
blindness.  For many such people, red and green are just similar
shades of brown.

You should also ensure sufficient contrast between foreground and
background colors: yellow text on a white background, or blue text on
a black background, both pose severe reading difficulties.

For dissertations and theses, it is of particular concern that color
documents output on black-and-white printers should not suffer serious
loss of readability, such as a light color being printed in a
hard-to-see faint grey.
}
%%%
%%% just before the \uumaintext macro that begins the body of the thesis,
%%% and starts chapter numbering.
%%%
%%% [28-Jan-2018]
%%% ====================================================================

Color is ubiquitous in the commercial publishing industry, color
computer displays are now standard, and color printers are commonly
available on university campuses and in home offices.

The human visual system is acutely aware of color, and when used
carefully, color can make documents more attractive, and easier to
read.  Complex figures are often best presented in color to help
distinguish features, and color background shading in long tabular
displays can make them easier to understand, as illustrated in
Table~\vref{dfact}.

Here is how a portion of that table was written in \LaTeX{}, with
row-shading support from the \texttt{colortbl} package that is loaded
in the document preamble.

\begin{Verbatim}[formatcom = \color{brown4}]
\begin{table}[t]
    \definecolor {HeadingColor} {rgb} {0.60, 0.60, 0.90}
    \definecolor {ShadingColor} {rgb} {0.90, 0.90, 0.99}
    \newcommand{\R}{\rowcolor{ShadingColor}}
    \begin{center}
        \caption{Table of double factorials.}
        \label{dfact}
        \medskip
        % Generate table data and replace _ by \, (thin space)
        % hocd128> load("dfact")
        % hocd128> for (k = -1 ; k < 41; ++k) { m = k % 10; \
        %                if ((0 < m) && (m < 6)) s = "\\R" else s = "  " ; \
        %                printf("%2s %5d & %35.0..3f \\\\\n", s, k, dfact(k)) }
        \begin{tabular}{rr}
            \hline
            \rowcolor{HeadingColor}
            \boldmath $n$ & \multicolumn{1}{c}{\boldmath $n!!$} \\
            \hline
                  -1 &                                   1 \\
                   0 &                                   1 \\
            \R     1 &                                   1 \\
            \R     2 &                                   2 \\
            \R     3 &                                   3 \\
            \R     4 &                                   8 \\
            \R     5 &                                  15 \\
                   6 &                                  48 \\
                   7 &                                 105 \\
                   8 &                                 384 \\
                   9 &                                 945 \\
                  10 &                              3\,840 \\
            ...
        \end{tabular}
    \end{center}
\end{table}
\end{Verbatim}

\begin{table}[t]
    \definecolor {HeadingColor} {rgb} {0.60, 0.60, 0.90}
    \definecolor {ShadingColor} {rgb} {0.90, 0.90, 0.99}
    \newcommand{\R}{\rowcolor{ShadingColor}}
    \begin{center}
        \caption{Table of double factorials.}
        \label{dfact}
        \medskip
        % Generate table data and replace _ by \, (thin space)
        % hocd128> load("dfact")
        % hocd128> for (k = -1 ; k < 41; ++k) { m = k % 10; \
        %                if ((0 < m) && (m < 6)) s = "\\R" else s = "  " ; \
        %                printf("%2s %5d & %35.0..3f \\\\\n", s, k, dfact(k)) }
        \begin{tabular}{rr}
            \hline
            \rowcolor{HeadingColor}
            \boldmath $n$ & \multicolumn{1}{c}{\boldmath $n!!$} \\
            \hline
                  -1 &                                   1 \\
                   0 &                                   1 \\
            \R     1 &                                   1 \\
            \R     2 &                                   2 \\
            \R     3 &                                   3 \\
            \R     4 &                                   8 \\
            \R     5 &                                  15 \\
                   6 &                                  48 \\
                   7 &                                 105 \\
                   8 &                                 384 \\
                   9 &                                 945 \\
                  10 &                               3\,840 \\
            \R    11 &                              10\,395 \\
            \R    12 &                              46\,080 \\
            \R    13 &                             135\,135 \\
            \R    14 &                             645\,120 \\
            \R    15 &                           2\,027\,025 \\
                  16 &                          10\,321\,920 \\
                  17 &                          34\,459\,425 \\
                  18 &                         185\,794\,560 \\
                  19 &                         654\,729\,075 \\
                  20 &                       3\,715\,891\,200 \\
            \R    21 &                      13\,749\,310\,575 \\
            \R    22 &                      81\,749\,606\,400 \\
            \R    23 &                     316\,234\,143\,225 \\
            \R    24 &                   1\,961\,990\,553\,600 \\
            \R    25 &                   7\,905\,853\,580\,625 \\
                  26 &                  51\,011\,754\,393\,600 \\
                  27 &                 213\,458\,046\,676\,875 \\
                  28 &               1\,428\,329\,123\,020\,800 \\
                  29 &               6\,190\,283\,353\,629\,375 \\
                  30 &              42\,849\,873\,690\,624\,000 \\
            \R    31 &             191\,898\,783\,962\,510\,625 \\
            \R    32 &           1\,371\,195\,958\,099\,968\,000 \\
            \R    33 &           6\,332\,659\,870\,762\,850\,625 \\
            \R    34 &          46\,620\,662\,575\,398\,912\,000 \\
            \R    35 &         221\,643\,095\,476\,699\,771\,875 \\
                  36 &       1\,678\,343\,852\,714\,360\,832\,000 \\
                  37 &       8\,200\,794\,532\,637\,891\,559\,375 \\
                  38 &      63\,777\,066\,403\,145\,711\,616\,000 \\
                  39 &     319\,830\,986\,772\,877\,770\,815\,625 \\
                  40 &   2\,551\,082\,656\,125\,828\,464\,640\,000 \\
            \hline
        \end{tabular}
    \end{center}
\end{table}


The distinction between computer input and output is often improved by
showing them in different colors, as in the following indented examples.

\begin{quotation}
\begin{singlespace}
This is human input to a computer:
\definecoloralias {verbatimcolor} {utahred}
\begin{verbatim}
    % cal 1 2018
\end{verbatim}
%
Here is the computer output from that command:
%
\definecoloralias {verbatimcolor} {darkblue}
\begin{verbatim}
        January 2018      
    Su Mo Tu We Th Fr Sa  
        1  2  3  4  5  6  
     7  8  9 10 11 12 13  
    14 15 16 17 18 19 20  
    21 22 23 24 25 26 27  
    28 29 30 31           
\end{verbatim}
\end{singlespace}
\end{quotation}
%
Here is how those paragraphs were written in \LaTeX{}:
%
\begin{Verbatim}[formatcom = \color{brown4}]
\begin{quotation}
\begin{singlespace}
This is human input to a computer:
\definecoloralias {verbatimcolor} {utahred}
\begin{verbatim}
    % cal 1 2018
\end{verbatim}
%
Here is the computer output from that command:
%
\definecoloralias {verbatimcolor} {darkblue}
\begin{verbatim}
        January 2018      
    Su Mo Tu We Th Fr Sa  
        1  2  3  4  5  6  
     7  8  9 10 11 12 13  
    14 15 16 17 18 19 20  
    21 22 23 24 25 26 27  
    28 29 30 31           
\end{verbatim}
\end{singlespace}
\end{quotation}
\end{Verbatim}

\definecoloralias {verbatimcolor} {brown4}

Because verbatim displays are terminated by the first matching
\verb=\end= command, the illustrated input fragment had to be wrapped
in a second kind of verbatim supplied by the \texttt{fancyvrb} package
that is included in the \LaTeX{} preamble of this document.
%
\begin{verbatim}
\begin{Verbatim}[formatcom = \color{brown4}]
...
\end{Verbatim}
\end{verbatim}
%

Color reproduction fidelity is device dependent, and care must be
taken to avoid use of colors that should be distinct, but are not, on
some output devices, or for those humans who have some degree of color
blindness.  For many such people, red and green are just similar
shades of brown.

You should also ensure sufficient contrast between foreground and
background colors: yellow text on a white background, or blue text on
a black background, both pose severe reading difficulties.

For dissertations and theses, it is of particular concern that color
documents output on black-and-white printers should not suffer serious
loss of readability, such as a light color being printed in a
hard-to-see faint grey.
