%%% -*-LaTeX-*-

\chapter{Jump-Velocity Model}
\label{cha:jump-velocity-model}

\setcounter{section}{-1}
\section{Outline}
\label{sec:outline}

\begin{enumerate}
\item Intro and Motivation
  \begin{enumerate}
  \item Fast parameter estimation
  \item Uses effective platelet binding parameters: more intuitive
    connection to the type of data collected, however a less intuitive
    relationship with biological parameters
  \item Describe the data collected from experiments, present some
    data justifying why we want to exclude the steps before the first
    pause, and after the last pause (platelets take much longer steps,
    and travel much faster).
  \end{enumerate}
\item Model and Methods
  \begin{enumerate}
  \item Describe the 4-state model
  \item Adiabatic reduction
  \item Adding an escape term to the 4-state model
  \item Numerics and Parameter Estimation
  \end{enumerate}
\item Results
  \begin{enumerate}
  \item Parameter estimates of un/binding rates
  \item Uncertainty analysis of these estimates
  \item Goodness-of-fit (velocities, step times, dwell times, number
    of dwells)
  \end{enumerate}
\item Discussion
\end{enumerate}

\section{The model}
\label{sec:jv-model}

While the previous model can be used for some explorations of parameter
space, it is not feasible to use it for rigorous parameter
estimation. {\color{red} (Estimate of time involved?)}

Suppose we model platelets as particles translating at a constant
velocity along a wall coated with an immobilized agonist (e.g. collagen
or vWF). Platelets have two types of receptors that can bind to these
immobilized agonists. One type are the receptors which are
constitutively active, and have fast binding and unbinding kinetics. The
other type are receptors which are constitutively inactive, and become
activated as the platelet activates. \note{This sentence should
  be reworked}

Initially, assume that unprimed platelets can exist in two states:
unbound ($\stateUnbound$), and bound to the wall by fast receptors
($\stateBoundF$). Assume the platelet cannot be bound by slow receptors,
because they are not active in unprimed platelets. Platelets in the
unbound state advect at a velocity $\jvVel$ in the fluid, and platelets
in the fast-bound state are stationary. Platelets in the fluid can bind
through fast receptors at a constant rate $\jvOnFast$, and platelets
bound through fast receptors can unbind at a constant rate
$\jvOffFast$. The rates $\jvOnFast$ and $\jvOffFast$ are effective on
and off rates for the entire platelet, and \emph{not} on/off rates for
individual receptors.

It is useful to track the unbound platelets that have never been bound
to the wall (which we'll call $\stateUnboundNew$) separately from the
platelets that are unbound, but have previously been bound in the
experiment (which we'll call $\stateUnboundOld$). This can be added to
the model described above with only a slight modification: platelets in
either the $\stateUnboundNew$ or $\stateUnboundOld$ states can enter the
$\stateBoundF$ state at the constant on rate $\jvOnFast$, but platelets
in the $\stateBoundF$ state can only enter the $\stateUnboundOld$ state
at the off rate $\jvOffFast$. The state diagram describing this
situation is shown in Figure \ref{fig:unprimed-states}.

For primed platelets, assume that they can exist in four possible
states: unbound ($\stateUnbound$), fast-bound ($\stateBoundF$),
slow-bound ($\stateBoundS$), and both fast- and slow-bound
($\stateBoundFS$). Again, split $\stateUnbound$ into the
$\stateUnboundNew$ and $\stateUnboundOld$ states. The $\stateUnbound$
and $\stateBoundF$ states are the same as for the unprimed platelets,
however in addition both the $\stateUnbound$ and $\stateBoundF$ states
can form bonds mediated by slow-acting receptors to transition to the
$\stateBoundS$ and $\stateBoundFS$ states (Figure
\ref{fig:primed-states}).

\begin{figure}
  \centering
  
  \schemestart
  $\stateUnboundOld$ \arrow(u1--vv){<=>[$\jvOnFast$][$\jvOffFast$]}
  $\stateBoundF$ \arrow(--u0){<-[*{0}$\jvOnFast$]}[90]
  $\stateUnboundNew$
  \schemestop
  
  \caption[Possible states with one receptor]{An unprimed platelet can
    exist in three distinct states: ($\stateUnboundNew$) platelets which
    haven't interacted with the surface, ($\stateUnboundOld$) platelets
    which have interacted with the surface and are advecting in the
    fluid, or ($\stateBoundF$) bound to vWF on the surface and
    unmoving. Transitions between these states occur at constant rates
    $\jvOnFast$ and $\jvOffFast$. Platelets can only transition out of
    the $\stateUnboundNew$ state.}
  \label{fig:unprimed-states}
\end{figure}

\begin{figure}
  \centering

  \schemestart
  $\stateUnboundOld$ \arrow(u1--vv){<=>[$\jvOnFast$][$\jvOffFast$]} $\stateBoundF$
  \arrow(@u1--ff){<=>[*{0}$\jvOnSlow$][*{0}$\jvOffSlow$]}[-90] $\stateBoundS$
  \arrow(--vf){<=>[$\jvOnFast$][$\jvOffFast$]} $\stateBoundFS$
  \arrow(@vv--@vf){<=>[*{0}$\jvOnSlow$][*{0}$\jvOffSlow$]}
  \arrow(@vv--u01){<-[$\jvOnFast$]} $\stateUnboundNew$
  \arrow(@ff--u02){<-[$\jvOnFast$]}[180] $\stateUnboundNew$
  \schemestop

  \caption[Possible states of primed platelets]{A primed platelet can
    exist in five states: ($\stateUnbound$) unbound from the surface and
    advecting in the fluid (further split into the two categories
    defined above), ($\stateBoundF$) bound to vWF on the surface,
    ($\stateBoundS$) bound to fibrinogen on the surface, or
    ($\stateBoundFS$) bound to both vWF and fibrinogen. In all three
    bound states, the platelet is immobilized on the surface.}
  \label{fig:primed-states}
\end{figure}

% Where and how do we want to address the connection between receptor
% on/off rates and effective on/off rates?

These models describe a jump-velocity process, where a particle is
transitioning randomly between discrete states, which each move with a
different deterministic motion.

% We want to derive the distributions of
% three quantities of interest from the model:
% \begin{enumerate}
% \item dwell time---the time that a platelet spends bound to the
%   surface before unbinding, 
% \item step size---the distance a platelet travels in between
%   successive binding events, and
% \item average velocity---the time-averaged velocity over an experiment
%   of a single platelet.
% \end{enumerate}

The Fokker-Planck equation for these processes are given by the
following system of linear advection equations:
\begin{multline}
  \label{eq:fp-system}
  \Pder{}{t}
  \underbrace{
    \begin{pmatrix}
      p_{\stateUnboundNew} \\ p_{\stateUnboundOld} \\ p_\stateBoundF \\
      p_\stateBoundS \\ p_\stateBoundFS
    \end{pmatrix}}_{\equiv \mathbf{p}}
  =
  -\Pder{}{x}
  \begin{pmatrix}
    \jvVel p_{\stateUnboundNew} \\ \jvVel p_{\stateUnboundOld} \\ 0 \\
    0 \\ 0
  \end{pmatrix}
  + \\
  \underbrace{
    \begin{pmatrix}
      -(\jvOnFast + \jvOnSlow) & 0 & 0 & 0 & 0 \\
      0 & -(\jvOnFast + \jvOnSlow) & \jvOffFast & \jvOffSlow & 0 \\
      \jvOnFast & \jvOnFast & -(\jvOffFast + \jvOnSlow) & 0 & \jvOffSlow \\
      \jvOnSlow & \jvOnSlow & 0 & -(\jvOnFast + \jvOffSlow) & \jvOffFast \\
      0 & 0 & \jvOnSlow & \jvOnFast & -(\jvOffFast + \jvOffSlow)
  \end{pmatrix}}_{\equiv A}
  \begin{pmatrix}
    p_{\stateUnboundNew} \\ p_{\stateUnboundOld} \\ p_\stateBoundF \\
    p_\stateBoundS \\ p_\stateBoundFS
  \end{pmatrix}
\end{multline}
where $p_i = p_i(x, t \mid x_0, j, 0)$ is the probability the platelet
is in state $i$ and position $x$ at time $t$ given it was previously in
position $z$ and state $j$ at time $0$. If the platelets are not primed,
then the only difference is $\jvOnSlow = \jvOffSlow = 0$.

The Fokker-Planck equation can be used to find the probability density
function of the average velocity across a segment of length
$L$. Assuming all platelets enter the domain unbouned, we take the
initial condition of the PDE system (\ref{eq:fp-system}) to be
$\mathbf{p}(x, 0) = (\delta(x), 0, 0, 0, 0)^T$. That is, all platelets
enter in never-bound state $\stateUnboundNew$. The probability density
of the time it takes a platelet to cross the interval $[0, L]$ is
$\jvVel (p_{\stateUnboundNew}(L, t) + p_{\stateUnboundOld}(L,t))$. The
average velocity associated with a crossing time $t^*$ is just
$v^* = L/t^*$, and the probability density function of $v^*$ is given by
$f(v^*) = (L/v^*)^2 p_U(L, L/v^*)$. \note{I can put in more
  justification here if necessary}

\subsection{Nondimensionalization}
\label{sec:nondim}

Define the nondimensional variables $s$ and $y$ so that $t = Ts$ and
$x = Xy$. Let's scale $x$ by the domain length, so $X = L$, and scale
$t$ by the velocity, so that $X/T = \jvVel \implies T = L/\jvVel$. That
is, $T$ is the shortest possible crossing time of a platelet. Finally,
define the nondimensional parameters
$\epsilon_1 = 1/(T(\jvOnFast + \jvOffFast))$ and
$\epsilon_2 = 1/(T(\jvOnSlow + \jvOffSlow))$. If the sum of the relevant
reaction rates is much larger than $1/T$, then $\epsilon$ is a small
parameter. After the nondimensionalization, equation
(\ref{eq:fp-system}) becomes
\begin{multline}
  \label{eq:nd-system}
  \Pder{\mathbf{q}}{s} = -\Pder{}{y}
  \begin{pmatrix}
    q_{\stateUnboundNew} \\ q_{\stateUnboundOld} \\ 0 \\ 0 \\ 0
  \end{pmatrix}
  + \left[\left(\frac{1}{\epsilon_1}\right) 
  \begin{pmatrix}
    - b & 0 & 0 & 0 & 0 \\
    0 & - b & a & 0 & 0 \\
    b & b & - a & 0 & 0 \\
    0 & 0 & 0 & - b & a \\
    0 & 0 & 0 & b & - a
  \end{pmatrix} \right.
  \\
  \left. + \left(\frac{1}{\epsilon_2}\right)
  \begin{pmatrix}
    - d & 0 & 0 & 0 & 0 \\
    0 & - d & 0 & c & 0 \\
    0 & 0 & - d & 0 & c \\
    d & d & 0 & - c & 0 \\
    0 & 0 & d & 0 & - c
  \end{pmatrix}\right]
  \mathbf{q},
\end{multline}
where $a = \jvOffFast/(\jvOffFast + \jvOnFast)$,
$b = \jvOnFast/(\jvOffFast + \jvOnFast)$,
$c = \jvOffSlow/(\jvOffSlow + \jvOnSlow)$, and
$d = \jvOnSlow/(\jvOffSlow + \jvOnSlow)$. Note that when
$\jvOnSlow = \jvOffSlow = 0$, the second matrix is the zero matrix, and
the system effectively reduces to a system of three equations.

In the nondimensional system, the probability density function for the
average velocity is given by
$f(v^*) = ({v^*})^{-2} q_{\stateUnboundOld} \left(1, 1/v^*\right)$

\subsection{Adiabatic Reduction}
\label{sec:adiabatic-reduction}

Biologically, platelet interaction with a surface in the bloodstream is
mediated by fast and slow bonds, which suggests there may be a
quasi-steady-state reduction in the Fokker-Planck system. As mentioned
above, the relevant time scales to consider are $T = L/v$ (the time it
takes an unbound platelet to cross the domain), and
$\jvOnFast + \jvOffFast$. \note{Comment on the validity of
  assuming $\epsilon_1$ is small}

If we define the first and second matrices from equation
(\ref{eq:nd-system}) as $\underline{\underline{A}}$ and
$\underline{\underline{B}}$ respectively, then we can rewrite equation
(\ref{eq:nd-system}) in the more compact form
\begin{equation}
  \label{eq:four-par-ad-fp}
  \Pder{\mathbf{q}}{s} = - \Pder{q_{U}}{y} \mathbf{e}_1 +
  \frac{1}{\epsilon_1} \underline{\underline{A}}\mathbf{q} +
  \frac{1}{\epsilon_2} \underline{\underline{B}}\mathbf{q}.
\end{equation}

The fast reactions are those captured in matrix
$\underline{\underline{A}}$, so this matrix defines the separation of
the system into fast and slow subsystems. In linear algebra terms, the
left 0-eigenvectors of $\underline{\underline{A}}$ define the projection
onto the slow manifold, and the orthogonal projection projects onto the
fast manifold, but I think it makes more sense to refer to the reaction
diagram to see the fast-slow separation.

\begin{center}
  \schemestart
  \stateUnbound \arrow{<=>[$b/\epsilon_1$][$a/\epsilon_1$]}
  \stateBoundF \arrow{<=>[*{0}$d/\epsilon_2$][*{0}$c/\epsilon_2$]}[270]
  \stateBoundFS \arrow{<=>[$b/\epsilon_1$][$a/\epsilon_1$]}[180]
  \stateBoundS \arrow{<=>[*{0}$d/\epsilon_2$][*{0}$c/\epsilon_2$]}[90]
  \schemestop\par
\end{center}

The fast reactions are represented by the two horizontal reaction lines
in the diagram, and so the U and V states define one subsystem where the
reactions within that system are fast, and transitions into or out of
that system are slow, and F and VF define the other such subsystem. Thus
we define two slow variables: $v_1 = q_\stateUnbound + q_\stateBoundF$
and $v_2 = q_\stateBoundS + q_\stateBoundFS$.

The two fast variables are harder to intuit, but they end up being
$w_1 = b q_\stateUnbound - a q_\stateBoundF$ and
$w_2 = b q_\stateBoundS - a q_\stateBoundFS$. These two fast variables
basically represent the net formation of fast bonds (or the net
transition rate of U to V and F to VF) within the two fast subsystems.

Then after changing variables in equation (\ref{eq:4-par-ad-fp}), we
get
\begin{align}
  \Pder{v_1}{t} &= -\Pder{}{y}(a v_1 + w_1) - \frac{d}{\epsilon_2} v_1
                  + \frac{c}{\epsilon_2} v_2 \label{eq:v1} \\
  \Pder{v_2}{t} &= \frac{d}{\epsilon_2} v_1 - \frac{c}{\epsilon_2} v_2
  \label{eq:v2} \\
  \Pder{w_1}{t} &= -\frac{w_1}{\epsilon_1} - b\Pder{}{y} (a v_1 + w_1)
                  - \frac{d}{\epsilon_2} w_1 + \frac{c}{\epsilon_2}
                  w_2 \label{eq:w1} \\
  \Pder{w_2}{t} &= -\frac{w_2}{\epsilon_1} + \frac{d}{\epsilon_2} w_1
                  - \frac{c}{\epsilon_2} w_2. \label{eq:w2}
\end{align}

Because we are assuming the system is in quasi-steady-state,
$w_1 = \mathcal{O}(\epsilon_1)$ and $w_2 =
\mathcal{O}(\epsilon_1)$. Then multiplying equation (\ref{eq:w1}) and
equation (\ref{eq:w2}) by $\epsilon_1$ and excluding higher order terms,
we get the following QSS equations for $w_1$ and $w_2$:
\begin{align}
  \label{eq:w1-reduced}
  w_1 &= -\epsilon_1 a b \Pder{v_1}{y} + \mathcal{O}(\epsilon_1^2) \\
  \label{eq:w2-reduced}
  w_2 &= 0 + \mathcal{O}(\epsilon_1^2).
\end{align}

Thus at least to first order in $\epsilon_1$, the fast dynamics between
F and VF don't affect the average velocity of platelets. Then
substituting $w_1$ into equation (\ref{eq:v1}) gives us the following
system defining the evolution of the slow dynamics:
\begin{align}
  \label{eq:v1-reduced}
  \Pder{v_1}{t} &= -a \Pder{v_1}{y} + \epsilon_1 a b \frac{\partial^2
                  v_1}{\partial y^2} - \frac{d}{\epsilon_2} v_1 +
                  \frac{c}{\epsilon_2} v_2 \\
  \label{eq:v2-reduced}
  \Pder{v_2}{t} &= \frac{d}{\epsilon_2} v_1 - \frac{c}{\epsilon_2} v_2.
\end{align}

So the slow system evolves like its own jump-velocity process, except
that now the advecting quantity also has a small diffusion
component. This result is very similar to the result from the adiabatic
reduction of a two-state jump velocity process, where the same small
diffusive component in the evolution of the slow variable appears in the
limit of fast switching between states. The main difference in this
four-state system is presence of slow bond dynamics which remain
unchanged.

\subsection{Parameter estimation}
\label{sec:parameter-estimation}

\note{Not sure how much of this section to include. It gets pretty
  detailed}

We have a stochastic model of a jump-velocity process, which is used
to derive a Fokker-Planck equation for platelet position as a function
of time. From the Fokker-Planck equation we derive a probability
distribution for the average velocity of a platelet rolling across
some domain. Basically the average velocity of a platelet (or
equivalently, the time it takes a platelet to cross) is a random
variable. The Fokker-Planck equation defines a probability
distribution for this random variable as a function of model
parameters, and average velocities from individual platelet
trajectories are realizations of this random variable. We want to fit
the probability distribution defined by model parameters to the set of
realizations in the experimental data.

We chose to use a maximum-likelihood estimate for parameters in the
model. Let $\{v_i\}_{i=1}^N$ be a set of observations of average
velocities, and define $f(v;a, \epsilon)$ to be the probability
distribution of velocities given parameters $a$ and $\epsilon$ (for
simplicity, assume we're working with the two-state model). Define the
likelihood function
\begin{equation}
  \label{eq:likelihood}
  L(a, \epsilon) = \prod_{i=1}^N f(v_i; a, \epsilon),
\end{equation}
and then the maximum likelihood estimates for $a$ and $\epsilon$ are
those that maximize $L$. Define the log-likelihood function:
$\tilde{\ell}(a, \epsilon) = \log(L(a, \epsilon))$, and then
maximizing $\tilde{\ell}$ is equivalent to maximizing $L$.

We found $f(v; a, \epsilon)$ above in terms of the solution of the
Fokker-Planck equation: $f(v; a, \epsilon) = 1/v^2 q_{U^1}(1, 1/v; a,
\epsilon)$ where $q_{U^1}(x, t; a, \epsilon)$ is part of the solution
of the Fokker-Planck equation with parameters $a$ and $\epsilon$. Then
we can write $\tilde{\ell}$ in terms of $q_{U^1}$:
$\tilde{\ell} = \sum_{i=1}^N \log(f(v_i; a, \epsilon)) = \sum_{i=1}^N
\log(q_{U^1}(1, 1/v_i; a, \epsilon)) - 2\sum_{i=1}^N \log(v_i)$. The
second term is a constant factor with respect to $a$ and $\epsilon$,
and so can be excluded from the optimization. Therefore we can
maximize the modified log-likelihood function
\begin{equation}
  \label{eq:mod-log-like}
  \ell(a, \epsilon) = \sum_{i=1}^N \log(q_{U^1}(1, 1/v_i; a, \epsilon)).
\end{equation}

\subsubsection{Numerical Optimization}
\label{sec:numer-optim}

The modified log-likelihood function $\ell(a, \epsilon)$ must be
maximized iteratively. We haven't even found an analytical solution
for $q_{U^1}$. Therefore, the Fokker-Planck equation must be re-solved
each time the parameters $a$ and $\epsilon$ are changed. Fortunately
the PDE only needs to be solved once for each evaluation of
$\ell(a, \epsilon)$. We have to find $q_{U^1}(1, t)$ up to time
$t = 1/\min(v_i)$, but then to evaluate the probability density at
each time $1/v_i$, we only need to interpolate the solution.

Another consideration is that the model parameters $a$ and $\epsilon$
have bounds, in particular $a \in (0, 1)$ and $\epsilon > 0$. One
possiblility for staying within these constraints is to use a
numerical optimization algorithm that optimizes within a bounded
domain. However, near the boundaries of the domain $\ell(a, \epsilon)$
diverges and in some cases can cause the minimization to
fail. Instead, I have achieved good results in practice by
transforming the model parameters $a$ and $\epsilon$ to a pair of new
parameters $a'$ and $\epsilon'$ which can vary over all $\reals$. The
transformations for these parameters are:
\begin{align}
  \label{eq:a-fwd-trns}
  a' &= \frac{2a + 1}{2a(a + 1)} \\
  \label{eq:e-fwd-trns}
  \epsilon' &= \log \epsilon.
\end{align}
Note that $a'$ is defined so that $a'=0$ at $a=1/2$,
$a' \rightarrow -\infty$ as $a \rightarrow 0$, and
$a' \rightarrow \infty$ as $a \rightarrow 1$. With the fitting
parameters, the optimization problem becomes an unconstrained
optimization in $\reals^2$. To actually carry out the optimization I
use SciPy's \verb|minimize| function \cite{Virtanen2020}, which
implements the BFGS algorithm \note{citations}.

An iterative algorithm needs an intial guess, which at least in the
two-state model, can be chosen as an estimate of $a$ and $\epsilon$
from the adiabatic reduction. Equation (\ref{eq:vel-dens}) gives the
pdf of velocities from the adiabatic reduction, and can be used to
derive the mean $\mu(a, \epsilon)$ and variance $\sigma^2(a,
\epsilon)$ as functions of $a$ and $\epsilon$. These moments have
simple analytical expressions: 
\begin{align}
  \label{eq:mean-ar}
  \mu(a, \epsilon) &= a(1 + \epsilon - a \epsilon) \\
  \label{eq:var-ar}
  \sigma^2(a, \epsilon) &= a^2 \epsilon (1 - a) (2 + 5(1 - a) \epsilon).
\end{align}

Then we can equate $\mu$ and $\sigma^2$ to the first two sample
moments ($\bar{v}$ and $s^2$) of the data to estimate $a$ and
$\epsilon$. This is called the method of moments. We can find exact
analytical expressions for $a$ and $\epsilon$ in terms of $\mu$ and
$\sigma^2$, but the formulas are messy. Note that
$\sigma^2 = O(\epsilon)$ and $\epsilon \ll 1$, so we can use an
asymptotic approximation instead of the full formulas. The estimate is
only being used to initialize the optimization algorithm, so there is
no need to be exact. Therefore up to $O(\sigma^2)$, the method of
moments gives the following estimates for $a$ and $\epsilon$ based on
a data set with sample mean $\bar{v}$ and sample variance $s^2$:
\begin{align}
  \label{eq:mean-est}
  a &= \bar{v} - \frac{s^2}{2\bar{v}} \\
  \label{eq:var-est}
  \epsilon &= \frac{s^2}{2(1 - \bar{v})(\bar{v})^2}.
\end{align}

In figure \ref{fig:model-fits}, a histogram of 1000 sample velocities
is shown along with the PDEs generated by the reduced and full models
with maximum likelihood parameter estimates. The CDFs of the models
are also compared with the ECDF of the data. Visually these models fit
well (as they should), and provide parameter estimates with less than
10\% relative error.

\subsubsection{Bootstrapping}
\label{sec:bootstrapping}

The maximum likelihood estimator only gives a point estimate of the
parameters $a$ and $\epsilon$. In order to get confidence intervals,
we can use a bootstrapping approach. The basic idea is that for a set
of data $\{V_i\}_{i=1}^N$, a ``new'' data set can be generated by
randomly picking $N$ samples from the $\{V_i\}$ with replacement. Then
the MLE for the new data can be found, and in this way we get a bunch
of $\hat{a}$ and $\hat{\epsilon}$ estimates. After generating many of
these estimates, we can come up with an estimate of a 95\% confidence
interval for each parameter. If we assume that the ML estimates are
distributed sufficiently normally, then
$\hat{\theta} \pm 1.96 \sigma_{\hat{\theta}}$ approximates a 95\%
confidence interval well. But it seems like using the $2.5$th and
$97.5$th percentiles also provides a good estimate for the confidence
interval, and doesn't rely on assuming the parameter estimate is
distributed normally.

Figure \ref{fig:bootstraps} summarizes results from 64 bootstrap
trials on the sample data. In each case, the distribution of estimated
parameters looks symmetrical, and the two methods of estimating a 95\%
confidence interval described above give similar results (Table
\ref{tab:conf-int}).

While the MLE of $a$ looks unbiased (i.e. it is near the center of the
distribution of $a$s shown in figure \ref{fig:bootstraps}), the MLE of
$\epsilon$ is clearly biased to the upper end of the distribution of
$\epsilon$ estimates from the bootstrap procedure. Part of this may be
because I haven't filtered out platelets that didn't pause on the
surface, but it is still something to look out for. 

\subsubsection{Goodness-of-fit}
\label{sec:goodness-fit}

Finally, we want to find the goodness-of-fit of the model. Chapter 13
in \textit{Introduction to Probability and Mathematical Statistics} by
Bain and Engelhardt suggests two approaches. One approach is to group
the data into bins, and then compare the observed observations in each
bin to the expected number of observations in each bin. In particular,
\begin{equation}
  \label{eq:chi2}
  \chi^2 = \sum_{j=1}^c \frac{(o_j - \hat{e}_j)^2}{\hat{e}_j} \sim
  \chi^2(c - 1 - k)
\end{equation}
where $c$ is the number of bins, $k$ is the number of estimated
parameters, $o_j$ is the number of data points in bin $j$, and
$\hat{e}_j$ is the expected number of data points in bin $j$. Here
$\hat{e}_j = N \hat{p}_j$ where $N$ is the total number of data
points, and $\hat{p}_j$ is the model probability of picking a data
point from bin $j$. A weakness of this approach is that information is
lost by grouping the data, and there are other tests that depend
directly on the individual observations.

One of these tests is the Kolmogorov-Smirnov test, which essentially
uses the maximum difference between the predicted CDF and the
ECDF. The KS statistic is $D = \max(D^+, D^-)$ where
$D^+ = \max_i(i/N - F(x_{i:N}))$ and
$D^- = \max_i(F(x_{i:N}) - (i - 1)/N)$. A small value of $D$ indicates
a good fit (though precisely what value of $D$ means the fit is
``good'' depends on $N$). The weakness of this approach is that the
distribution of the KS statistic is derived assuming no parameters
have been estimated, which is obviously not true in our case. The
Kolmogorov-Smirnov test returned a $p$-value of 0.46 for the reduced
model, and 0.66 for the full model, indicating a good fit of these
models to the data (obviously, because the data were derived from the
jump-velocity model).

\section{Results}
\label{sec:results-jump-velocity}

\subsection{Step sizes and ``free velocities'' in the trajectory data}
\label{sec:step-sizes-free}

%% Quotation marks in a section title?

The adiabatic reduction of the 4-state model is motivated by an
observation that the ``free velocities'' of observed platelets is much
lower than expected free-flowing velocities. The experiments are run
with a wall shear rate of 100/s, and so the fluid velocity $1 \mu\tn{m}$
off the wall is about $100 \mu\tn{m}/s$. However the observed free
velocities (i.e. during a dwell) are much lower than this (see Figure
\ref{fig:ccp-vel-cmp}). We have also observed that the velocity of a
platelet before its first dwell, and after its last dwell, is much
higher than velocity in between dwells. Similarly, the initial and final
steps of a platelet (i.e. as it enters and leaves the field of view) are
much longer than the lengths of intermediate steps.

Boxplots of each of the populations of step sizes are shown in Figure
\ref{fig:ccp-step-cmp}. The steps that occur in between dwells are much
smaller than those at the beginning and end, therefore our assumption in
the jump-velocity model that the beginning and ending steps are the same
is violated. Similarly, the velocity of the platelet within a step is
different depending on whether the step is in between dwells, or at the
beginning or end of a trajectory (Figure \ref{fig:ccp-vel-cmp}). Another
thing to note in Figure \ref{fig:ccp-vel-cmp} is that velocities within
steps is not constant, or even particularly close to constant.

One possible explanation of this is that there are brief contacts
between the platelet and the wall which slow the platelet down to well
below the free-flowing velocity, but these binding dynamics occur too
quickly to be detected individually. These bonds are likely
GPVI-collagen bonds because they occur on a relatively fast time
scale. Then the observed pauses may be due to integrin-collagen binding.

An analysis of the platelet velocities in between steps provides one
possible check of this hypothesis. As shown in the adiabatic reduction
of the previous section, the sum of $\stateUnbound$ and $\stateBoundF$
platelets should advect and diffuse in a way that is well modeled by
equation (\ref{eq:v1-reduced}). I processed the experimental
trajectories by removing the observed dwells, and then comparing the
resulting distribution of average free velocities to the distribution of
average velocities predicted by the adiabatic reduction of the 2 state
model (equation (\ref{eq:vel-dens})) (Figure \ref{fig:avg-free-vel-col}).

\begin{figure}
  \centering
  \includegraphics[width=0.8\textwidth]{ccp-step-cmp.png}
  \caption{Comparison of intermediate step sizes with the first and
    last steps of the experiment. Data shown is from the
    collagen-collagen PRP experiment, but other experiments have
    qualitatively similar results.}
  \label{fig:ccp-step-cmp}
\end{figure}

\begin{figure}
  \centering
  \includegraphics[width=0.8\textwidth]{ccp-vel-cmp.png}
  \caption{Comparison of intermediate free velocities with the first
    and last steps of the experiment. Again, the data shown is from
    the collagen-collagen PRP experiment, but the others are similar.}
  \label{fig:ccp-vel-cmp}
\end{figure}

\begin{figure}
  \centering
  \includegraphics[width=\textwidth]{avg_free_vel_col.png}
  \caption{Comparison of the average free velocity in the four collagen
    experimental conditions. The effective binding parameters found by
    fitting the model to velocity data are $k_\tn{on}=34.6/s$ and
    $k_\tn{off} = 5.18/s$. An ANOVA test didn't find any significant
    difference in the average velocities among the 4 experiments, and so
    I fit the model to all the data simultaneously.}
  \label{fig:avg-free-vel-col}
\end{figure}

% Local Variables:
% TeX-master: "phd-thesis.ltx"
% End:
