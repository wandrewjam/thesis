%%% -*-LaTeX-*-

\chapter{Jump-Velocity Model}
\label{cha:jump-velocity-model}

\section{The model}
\label{sec:jv-model}

While the previous model can be used for some explorations of
parameter space, it is not feasible to use it for rigorous parameter
estimation. {\color{red} (Estimate of time involved?)}

Suppose we model platelets as particles translating at a constant
velocity along a wall coated with an immobilized agonist
(e.g. collagen or vWF). Platelets have two types of receptors that can
bind to these immobilized agonists. One type are the receptors which
are constitutively active, and have fast binding and unbinding
kinetics. The other type are receptors which are constitutively
inactive, and become activated as the platelet activates. {\color{red}
  This sentence should be reworked}

Initially, assume that unprimed platelets can exist in two states:
unbound ($\stateUnbound$), and bound to the wall by fast receptors
($\stateBoundF$). Assume the platelet cannot be bound by slow
receptors, because they are not active in unprimed
platelets. Platelets in the unbound state advect at a velocity
$\jvVel$ in the fluid, and platelets in the fast-bound state are
stationary. Platelets in the fluid can bind through fast receptors at
a constant rate $\jvOnFast$, and platelets bound through fast
receptors can unbind at a constant rate $\jvOffFast$. The rates
$\jvOnFast$ and $\jvOffFast$ are effective on and off rates for the
entire platelet, and \emph{not} on/off rates for individual receptors.

It is useful to track the unbound platelets that have never been bound
to the wall (which we'll call $\stateUnboundNew$) separately from the
platelets that are unbound, but have previously been bound in the
experiment (which we'll call $\stateUnboundOld$). This can be added to
the model described above with only a slight modification: platelets
in either the $\stateUnboundNew$ or $\stateUnboundOld$ states can enter
the $\stateBoundF$ state at the constant on rate $\jvOnFast$, but
platelets in the $\stateBoundF$ state can only enter the
$\stateUnboundOld$ state at the off rate $\jvOffFast$. The state diagram
describing this situation is shown in Figure
\ref{fig:unprimed-states}.

For primed platelets, assume that they can exist in four possible
states: unbound ($\stateUnbound$), fast-bound ($\stateBoundF$),
slow-bound ($\stateBoundS$), and both fast- and slow-bound
($\stateBoundFS$). Again, split $\stateUnbound$ into the
$\stateUnboundNew$ and $\stateUnboundOld$ states. The $\stateUnbound$
and $\stateBoundF$ states are the same as for the unprimed platelets,
however in addition both the $\stateUnbound$ and $\stateBoundF$ states
can form bonds mediated by slow-acting receptors to transition to the
$\stateBoundS$ and $\stateBoundFS$ states (Figure
\ref{fig:primed-states}). 

\begin{figure}
  \centering
  
  \schemestart
  $\stateUnboundOld$ \arrow(u1--vv){<=>[$\jvOnFast$][$\jvOffFast$]}
  $\stateBoundF$ \arrow(--u0){<-[*{0}$\jvOnFast$]}[90]
  $\stateUnboundNew$
  \schemestop
  
  \caption[Possible states with one receptor]{An unprimed platelet can
    exist in three distinct states: ($\stateUnboundNew$) platelets
    which haven't interacted with the surface, ($\stateUnboundOld$)
    platelets which have interacted with the surface and are advecting
    in the fluid, or ($\stateBoundF$) bound to vWF on the surface and
    unmoving. Transitions between these states occur at constant rates
    $\jvOnFast$ and $\jvOffFast$. Platelets can only transition out of
    the $\stateUnboundNew$ state.}
  \label{fig:unprimed-states}
\end{figure}

\begin{figure}
  \centering

  \schemestart
  $\stateUnboundOld$ \arrow(u1--vv){<=>[$\jvOnFast$][$\jvOffFast$]} $\stateBoundF$
  \arrow(@u1--ff){<=>[*{0}$\jvOnSlow$][*{0}$\jvOffSlow$]}[-90] $\stateBoundS$
  \arrow(--vf){<=>[$\jvOnFast$][$\jvOffFast$]} $\stateBoundFS$
  \arrow(@vv--@vf){<=>[*{0}$\jvOnSlow$][*{0}$\jvOffSlow$]}
  \arrow(@vv--u01){<-[$\jvOnFast$]} $\stateUnboundNew$
  \arrow(@ff--u02){<-[$\jvOnFast$]}[180] $\stateUnboundNew$
  \schemestop

  \caption[Possible states of primed platelets]{A primed platelet can
    exist in five states: ($\stateUnbound$) unbound from the surface
    and advecting in the fluid (further split into the two categories
    defined above), ($\stateBoundF$) bound to vWF on the surface,
    ($\stateBoundS$) bound to fibrinogen on the surface, or
    ($\stateBoundFS$) bound to both vWF and fibrinogen. In all three
    bound states, the platelet is immobilized on the surface.}
  \label{fig:primed-states}
\end{figure}

% Where and how do we want to address the connection between receptor
% on/off rates and effective on/off rates?

These models describe a jump-velocity process, where a particle is
transitioning randomly between discrete states, which each move with a
different deterministic motion.

% We want to derive the distributions of
% three quantities of interest from the model:
% \begin{enumerate}
% \item dwell time---the time that a platelet spends bound to the
%   surface before unbinding, 
% \item step size---the distance a platelet travels in between
%   successive binding events, and
% \item average velocity---the time-averaged velocity over an experiment
%   of a single platelet.
% \end{enumerate}

The Fokker-Planck equation for these processes are given by the
following system of linear advection equations:
\begin{equation}
  \label{eq:fp-system}
  \Pder{}{t}
  \underbrace{
    \begin{pmatrix}
      p_{\stateUnboundNew} \\ p_{\stateUnboundOld} \\ p_\stateBoundF
      \\ p_\stateBoundS \\ p_\stateBoundFS
    \end{pmatrix}}_{\equiv \mathbf{p}}
  =
  -\Pder{}{x}
  \begin{pmatrix}
    \jvVel p_{\stateUnboundNew} \\ \jvVel p_{\stateUnboundNew} \\ 0 \\
    0 \\ 0
  \end{pmatrix}
  +
  \underbrace{
    \begin{pmatrix}
      -(\jvOnFast + \jvOnSlow) & 0 & 0 & 0 & 0 \\
      0 & -(\jvOnFast + \jvOnSlow) & \jvOffFast & \jvOffSlow & 0 \\
      \jvOnFast & \jvOnFast & -(\jvOffFast + \jvOnSlow) & 0 & \jvOffSlow \\
      \jvOnSlow & \jvOnSlow & 0 & -(\jvOnFast + \jvOffSlow) & \jvOffFast \\
      0 & 0 & \jvOnSlow & \jvOnFast & -(\jvOffFast + \jvOffSlow)
  \end{pmatrix}}_{\equiv A}
  \begin{pmatrix}
    p_{\stateUnboundNew} \\ p_{\stateUnboundOld} \\ p_\stateBoundF \\
    p_\stateBoundS \\ p_\stateBoundFS
  \end{pmatrix}
\end{equation}
where $p_i = p_i(x, t \mid x_0, j, 0)$ is the probability the platelet
is in state $i$ and position $x$ at time $t$ given it was previously
in position $z$ and state $j$ at time $0$. If the platelets are
not primed, then the only difference is $\jvOnSlow = \jvOffSlow =
0$.

The goal is to find the probability density function of the average
velocity across a segment of length $L$, so take the initial condition
of the PDE system (\ref{eq:fp-system}) to be
$\mathbf{p}(x, 0) = (\delta(x), 0, 0, 0, 0)^T$. That is, all platelets
enter in never-bound state $\stateUnboundNew$. The probability
density of the time it takes a platelet to cross the interval $[0, L]$
is $\jvVel (p_{\stateUnboundNew}(L, t) +
p_{\stateUnboundOld}(L,t))$. The average velocity associated with a
crossing time $t^*$ is just $v^* = L/t^*$, and the probability density
function of $v^*$ is given by $f(v^*) = (L/v^*)^2 p_U(L, L/v^*)$

\subsection{Nondimensionalization}
\label{sec:nondim}

Define the nondimensional variables $s$ and $y$ so that $t = Ts$ and
$x = Xy$. Let's scale $x$ by the domain length, so $X = L$, and scale
$t$ by the velocity, so that $X/T = \jvVel \implies T =
L/\jvVel$. That is, $T$ is the shortest possible crossing time of a
platelet. Finally, define the nondimensional parameter
$\epsilon_1 = 1/(T(\jvOnFast + \jvOffFast))$ and
$\epsilon_2 = 1/(T(\jvOnSlow + \jvOffSlow))$. If the sum of the
relevant reaction rates is much larger than $1/T$, then $\epsilon$
is a small parameter. After the nondimensionalization, equation
(\ref{eq:fp-system}) becomes
\begin{equation}
  \label{eq:nd-system}
  \Pder{\mathbf{q}}{s} = -\Pder{}{y}
  \begin{pmatrix}
    q_{\stateUnboundNew} \\ q_{\stateUnboundOld} \\ 0 \\ 0 \\ 0
  \end{pmatrix}
  + \left[\left(\frac{1}{\epsilon_1}\right) 
  \begin{pmatrix}
    - b & 0 & 0 & 0 & 0 \\
    0 & - b & a & 0 & 0 \\
    b & b & - a & 0 & 0 \\
    0 & 0 & 0 & - b & a \\
    0 & 0 & 0 & b & - a
  \end{pmatrix}
  + \left(\frac{1}{\epsilon_2}\right)
  \begin{pmatrix}
    - d & 0 & 0 & 0 & 0 \\
    0 & - d & 0 & c & 0 \\
    0 & 0 & - d & 0 & c \\
    d & d & 0 & - c & 0 \\
    0 & 0 & d & 0 & - c
  \end{pmatrix}\right]
  \mathbf{q},
\end{equation}
where $a = \jvOffFast/(\jvOffFast + \jvOnFast)$,
$b = \jvOnFast/(\jvOffFast + \jvOnFast)$,
$c = \jvOffSlow/(\jvOffSlow + \jvOnSlow)$, and
$d = \jvOnSlow/(\jvOffSlow + \jvOnSlow)$. When
$\jvOnSlow = \jvOffSlow = 0$, this reduces to the example in
Dr. Keener's notes.

% Local Variables:
% TeX-master: "phd-thesis.ltx"
% End:
