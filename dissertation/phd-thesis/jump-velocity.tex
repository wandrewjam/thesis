%%% -*-LaTeX-*-

\chapter{Jump-Velocity Model}
\label{cha:jump-velocity-model}

\setcounter{section}{-1}
\section{Outline}
\label{sec:outline}

\begin{enumerate}
\item Intro and Motivation
  \begin{enumerate}
  \item Fast parameter estimation
  \item Uses effective platelet binding parameters: more intuitive
    connection to the type of data collected, however a less intuitive
    relationship with biological parameters
  \item Describe the data collected from experiments, present some
    data justifying why we want to exclude the steps before the first
    pause, and after the last pause (platelets take much longer steps,
    and travel much faster).
  \end{enumerate}
\item Model and Methods
  \begin{enumerate}
  \item Describe the 4-state model
  \item Adiabatic reduction
  \item Adding an escape term to the 4-state model
  \item Numerics and Parameter Estimation
  \end{enumerate}
\item Results
  \begin{enumerate}
  \item Parameter estimates of un/binding rates
  \item Uncertainty analysis of these estimates
  \item Goodness-of-fit (velocities, step times, dwell times, number
    of dwells)
  \end{enumerate}
\item Discussion
\end{enumerate}

\section{Introduction and Motivation}
\label{sec:intr-motiv}


\section{The model}
\label{sec:jv-model}

While the previous model can be used for some explorations of parameter
space, it is not feasible to use it for rigorous parameter
estimation. {\color{red} (Estimate of time involved?)}

Suppose we model platelets as particles translating at a constant
velocity along a wall coated with an immobilized agonist (e.g. collagen
or vWF). Platelets have two types of receptors that can bind to these
immobilized agonists. One type are the receptors which are
constitutively active, and have fast binding and unbinding kinetics. The
other type are receptors which are constitutively inactive, and become
activated as the platelet activates. \note{This sentence should
  be reworked}

Initially, assume that unprimed platelets can exist in two states:
unbound ($\stateUnbound$), and bound to the wall by fast receptors
($\stateBoundF$). Assume the platelet cannot be bound by slow receptors,
because they are not active in unprimed platelets. Platelets in the
unbound state advect at a velocity $\jvVel$ in the fluid, and platelets
in the fast-bound state are stationary. Platelets in the fluid can bind
through fast receptors at a constant rate $\jvOnFast$, and platelets
bound through fast receptors can unbind at a constant rate
$\jvOffFast$. The rates $\jvOnFast$ and $\jvOffFast$ are effective on
and off rates for the entire platelet, and \emph{not} on/off rates for
individual receptors (Figure \ref{fig:unprimed-states}).

% It is useful to track the unbound platelets that have never been bound
% to the wall (which we'll call $\stateUnboundNew$) separately from the
% platelets that are unbound, but have previously been bound in the
% experiment (which we'll call $\stateUnboundOld$). This can be added to
% the model described above with only a slight modification: platelets in
% either the $\stateUnboundNew$ or $\stateUnboundOld$ states can enter the
% $\stateBoundF$ state at the constant on rate $\jvOnFast$, but platelets
% in the $\stateBoundF$ state can only enter the $\stateUnboundOld$ state
% at the off rate $\jvOffFast$. The state diagram describing this
% situation is shown in Figure \ref{fig:unprimed-states}.

For primed platelets, assume that they can exist in four possible
states: unbound ($\stateUnbound$), fast-bound ($\stateBoundF$),
slow-bound ($\stateBoundS$), and both fast- and slow-bound
($\stateBoundFS$). The $\stateUnbound$ and $\stateBoundF$ states are
the same as for the unprimed platelets, however in addition both the
$\stateUnbound$ and $\stateBoundF$ states can form bonds mediated by
slow-acting receptors to transition to the $\stateBoundS$ and
$\stateBoundFS$ states (Figure \ref{fig:primed-states}).

\begin{figure}
  \centering
  
  \schemestart
  $\stateUnbound$ \arrow(uu--vv){<=>[$\jvOnFast$][$\jvOffFast$]}
  $\stateBoundF$
  \schemestop
  
  \caption[Possible states with one receptor]{An unprimed platelet can
    exist in two distinct states: ($\stateUnbound$) platelets which
    are unbound and advecting in the fluid, or ($\stateBoundF$) bound
    to vWF on the surface and unmoving. Transitions between these
    states occur at constant rates $\jvOnFast$ and $\jvOffFast$.}
  \label{fig:unprimed-states}
\end{figure}

\begin{figure}
  \centering

  \schemestart
  $\stateUnbound$ \arrow(uu--vv){<=>[$\jvOnFast$][$\jvOffFast$]} $\stateBoundF$
  \arrow(@uu--ff){<=>[*{0}$\jvOnSlow$][*{0}$\jvOffSlow$]}[-90] $\stateBoundS$
  \arrow(--vf){<=>[$\jvOnFast$][$\jvOffFast$]} $\stateBoundFS$
  \arrow(@vv--@vf){<=>[*{0}$\jvOnSlow$][*{0}$\jvOffSlow$]}
  \schemestop

  \caption[Possible states of primed platelets]{A primed platelet can
    exist in four states: ($\stateUnbound$) unbound from the surface
    and advecting in the fluid, ($\stateBoundF$) bound through
    fast-bonds to the surface, ($\stateBoundS$) bound to slow-bonds to
    the surface, or ($\stateBoundFS$) bound through both fast and slow
    bonds. In all three bound states, the platelet is immobilized on
    the surface.}
  \label{fig:primed-states}
\end{figure}

% Where and how do we want to address the connection between receptor
% on/off rates and effective on/off rates?

Both of these models describe a jump-velocity process, where a
particle is transitioning randomly between discrete states, which each
move with a different deterministic motion.

% We want to derive the distributions of
% three quantities of interest from the model:
% \begin{enumerate}
% \item dwell time---the time that a platelet spends bound to the
%   surface before unbinding, 
% \item step size---the distance a platelet travels in between
%   successive binding events, and
% \item average velocity---the time-averaged velocity over an experiment
%   of a single platelet.
% \end{enumerate}

The Fokker-Planck equation for these processes are given by the
following system of linear advection equations:
\begin{multline}
  \label{eq:fp-system}
  \Pder{}{t}
  \underbrace{
    \begin{pmatrix}
      p_{\stateUnbound} \\  p_\stateBoundF \\ p_\stateBoundS \\
      p_\stateBoundFS
    \end{pmatrix}}_{\equiv \mathbf{p}}
  =
  -\Pder{}{x}
  \begin{pmatrix}
    \jvVel p_{\stateUnbound} \\ 0 \\ 0 \\ 0
  \end{pmatrix}
  + \\
  \underbrace{
    \begin{pmatrix}
      -(\jvOnFast + \jvOnSlow) & \jvOffFast & \jvOffSlow & 0 \\
      \jvOnFast & -(\jvOffFast + \jvOnSlow) & 0 & \jvOffSlow \\
      \jvOnSlow & 0 & -(\jvOnFast + \jvOffSlow) & \jvOffFast \\
      0 & \jvOnSlow & \jvOnFast & -(\jvOffFast + \jvOffSlow)
  \end{pmatrix}}_{\equiv A}
  \begin{pmatrix}
    p_{\stateUnbound} \\ p_\stateBoundF \\ p_\stateBoundS \\
    p_\stateBoundFS
  \end{pmatrix}
\end{multline}
where $p_i = p_i(x, t \mid x_0, j, 0)$ is the probability the platelet
is in state $i$ and position $x$ at time $t$ given it was previously in
position $x_0$ and state $j$ at time $0$. If the platelets are not primed,
then the only difference is $\jvOnSlow = \jvOffSlow = 0$.

The Fokker-Planck equation can be used to find the probability density
function of the average velocity across a segment of length
$L$. Assuming all platelets enter the domain unbouned, we take the
initial condition of the PDE system (\ref{eq:fp-system}) to be
$\mathbf{p}(x, 0) = (\delta(x), 0, 0, 0)^T$. The probability density
of the time it takes a platelet to cross the interval $[0, L]$ is
$\jvVel (p_{\stateUnbound}(L,t))$ (because the platelet is immobile in
the other 3 states, it \emph{must} be in the $\stateUnbound$ state
when it leaves the domain). The average velocity associated with a
crossing time $t^*$ is just $v^* = L/t^*$, and the probability density
function of $v^*$ is given by $f(v^*) = (L/v^*)^2 p_U(L,
L/v^*)$. \note{I can put in more justification here if necessary}

\subsection{Nondimensionalization}
\label{sec:nondim}

Define the nondimensional variables $s$ and $y$ so that $t = Ts$ and
$x = Xy$. Let's scale $x$ by the domain length, so $X = L$, and scale
$t$ by the velocity, so that $X/T = \jvVel \implies T = L/\jvVel$. That
is, $T$ is the shortest possible crossing time of a platelet. Finally,
define the nondimensional parameters
$\epsilon_1 = 1/(T(\jvOnFast + \jvOffFast))$ and
$\epsilon_2 = 1/(T(\jvOnSlow + \jvOffSlow))$. If the sum of the relevant
reaction rates is much larger than $1/T$, then $\epsilon$ is a small
parameter. After the nondimensionalization, equation
(\ref{eq:fp-system}) becomes
\begin{multline}
  \label{eq:nd-system}
  \Pder{\mathbf{q}}{s} = -\Pder{}{y}
  \begin{pmatrix}
    q_{\stateUnbound} \\ 0 \\ 0 \\ 0
  \end{pmatrix}
  + \left[\left(\frac{1}{\epsilon_1}\right) 
  \begin{pmatrix}
    - b & a & 0 & 0 \\
    b & - a & 0 & 0 \\
    0 & 0 & - b & a \\
    0 & 0 & b & - a
  \end{pmatrix} \right.
  \\
  \left. + \left(\frac{1}{\epsilon_2}\right)
  \begin{pmatrix}
    - d & 0 & c & 0 \\
    0 & - d & 0 & c \\
    d & 0 & - c & 0 \\
    0 & d & 0 & - c
  \end{pmatrix}\right]
  \mathbf{q},
\end{multline}
where $a = \jvOffFast/(\jvOffFast + \jvOnFast)$,
$b = \jvOnFast/(\jvOffFast + \jvOnFast)$,
$c = \jvOffSlow/(\jvOffSlow + \jvOnSlow)$, and
$d = \jvOnSlow/(\jvOffSlow + \jvOnSlow)$. Note that when
$\jvOnSlow = \jvOffSlow = 0$, the second matrix is the zero matrix, and
the system effectively reduces to a system of two equations.

In the nondimensional system, the probability density function for the
average velocity across a domain of length $L$ is 
$f(v^*) = ({v^*})^{-2} q_{\stateUnbound} \left(1, 1/v^*\right)$

\subsection{Adiabatic Reduction}
\label{sec:adiabatic-reduction}

Biologically, platelet interaction with a surface in the bloodstream
is mediated by fast and slow bonds, which suggests there may be a fast
time scale in the Fokker-Planck system. As mentioned above, the
relevant time scales to consider are $T = L/v$ (the time it takes an
unbound platelet to cross the domain), and $\jvOnFast +
\jvOffFast$. \note{Comment on the validity of assuming $\epsilon_1$ is
  small.}

If we define the first and second matrices from equation
(\ref{eq:nd-system}) as $\underline{\underline{A}}$ and
$\underline{\underline{B}}$ respectively, then we can rewrite equation
(\ref{eq:nd-system}) in the more compact form
\begin{equation}
  \label{eq:four-par-ad-fp}
  \Pder{\mathbf{q}}{s} = - \Pder{q_{U}}{y} \mathbf{e}_1 +
  \frac{1}{\epsilon_1} \underline{\underline{A}}\mathbf{q} +
  \frac{1}{\epsilon_2} \underline{\underline{B}}\mathbf{q}.
\end{equation}

The fast reactions are those captured in matrix
$\underline{\underline{A}}$, so this matrix defines the separation of
the system into fast and slow subsystems. In linear algebra terms, the
left 0-eigenvectors of $\underline{\underline{A}}$ define the
projection onto the slow manifold, and the orthogonal projection
projects onto the fast manifold, but I think it makes more sense to
refer to the reaction diagram to see the fast-slow
separation.

\begin{figure}
  \centering
  
  \schemestart
  $\stateUnbound$ \arrow{<=>[$b/\epsilon_1$][$a/\epsilon_1$]}
  $\stateBoundF$
  \arrow{<=>[*{0}$d/\epsilon_2$][*{0}$c/\epsilon_2$]}[270]
  $\stateBoundFS$ \arrow{<=>[$b/\epsilon_1$][$a/\epsilon_1$]}[180]
  $\stateBoundS$
  \arrow{<=>[*{0}$d/\epsilon_2$][*{0}$c/\epsilon_2$]}[90]
  \schemestop
  
  \caption[Nondimensional reaction diagram]{Reaction diagram for the
    nondimensional, four-state jump-velocity system.}
  \label{fig:nd-primed-states}
\end{figure}

The fast reactions are represented by the two horizontal reaction
lines in the diagram, and so the $\stateUnbound$ and $\stateBoundF$
states define one subsystem where the reactions within that system are
fast, and transitions into or out of that system are slow, and
$\stateBoundS$ and $\stateBoundFS$ define the other such
subsystem. Thus we define two slow variables:
$v_1 = q_\stateUnbound + q_\stateBoundF$ and
$v_2 = q_\stateBoundS + q_\stateBoundFS$.

The two fast variables are harder to intuit, but we can define them in
the following way:
$w_1 = b q_\stateUnbound - a q_\stateBoundF$ and
$w_2 = b q_\stateBoundS - a q_\stateBoundFS$. These two fast variables
basically represent the net formation of fast bonds (or the net
transition rate of $\stateUnbound$ to $\stateBoundF$ and
$\stateBoundS$ to $\stateBoundFS$) within the two fast subsystems.

Then after changing variables in equation (\ref{eq:nd-system}), we
get
\begin{align}
  \Pder{v_1}{t} &= -\Pder{}{y}(a v_1 + w_1) - \frac{d}{\epsilon_2} v_1
                  + \frac{c}{\epsilon_2} v_2 \label{eq:v1} \\
  \Pder{v_2}{t} &= \frac{d}{\epsilon_2} v_1 - \frac{c}{\epsilon_2} v_2
  \label{eq:v2} \\
  \Pder{w_1}{t} &= -\frac{w_1}{\epsilon_1} - b\Pder{}{y} (a v_1 + w_1)
                  - \frac{d}{\epsilon_2} w_1 + \frac{c}{\epsilon_2}
                  w_2 \label{eq:w1} \\
  \Pder{w_2}{t} &= -\frac{w_2}{\epsilon_1} + \frac{d}{\epsilon_2} w_1
                  - \frac{c}{\epsilon_2} w_2. \label{eq:w2}
\end{align}

Because we are assuming the system is in quasi-steady-state,
$w_1 = \mathcal{O}(\epsilon_1)$ and $w_2 =
\mathcal{O}(\epsilon_1)$. Then multiplying equation (\ref{eq:w1}) and
equation (\ref{eq:w2}) by $\epsilon_1$ and excluding higher order terms,
we get the following QSS equations for $w_1$ and $w_2$:
\begin{align}
  \label{eq:w1-reduced}
  w_1 &= -\epsilon_1 a b \Pder{v_1}{y} + \mathcal{O}(\epsilon_1^2) \\
  \label{eq:w2-reduced}
  w_2 &= 0 + \mathcal{O}(\epsilon_1^2).
\end{align}

Thus at least to first order in $\epsilon_1$, the fast dynamics
between $\stateBoundS$ and $\stateBoundFS$ don't affect the average
velocity of platelets. Then substituting $w_1$ into equation
(\ref{eq:v1}) gives us the following system defining the evolution of
the slow dynamics:
\begin{align}
  \label{eq:v1-reduced}
  \Pder{v_1}{t} &= -a \Pder{v_1}{y} + \epsilon_1 a b \frac{\partial^2
                  v_1}{\partial y^2} - \frac{d}{\epsilon_2} v_1 +
                  \frac{c}{\epsilon_2} v_2 \\
  \label{eq:v2-reduced}
  \Pder{v_2}{t} &= \frac{d}{\epsilon_2} v_1 - \frac{c}{\epsilon_2} v_2.
\end{align}

% It makes sense that if $\jvOnSlow = \jvSlowOff = 0$ (that is, $1/\epsilon_2 =
% 0$), then equation (\ref{eq:v1-reduced}) is identical to the adiabatic
% reduction for a 2-state jump-velocity process.

So the slow system evolves like its own jump-velocity process, except
that now the advecting quantity also has a small diffusion
component. This result is very similar to the result from the
adiabatic reduction of a two-state jump-velocity process, where the
same small diffusive component in the evolution of the slow variable
appears in the limit of fast switching between states. The main
difference in this four-state system is presence of slow bond dynamics
which remain unchanged. In fact, if slow bond formation is turned off
(i.e. $1/\epsilon_2 = 0$), then equation (\ref{eq:v1-reduced}) becomes
exactly the adiabatic reduction for a two-state jump-velocity
process. Equivalently, when a platelet is not bound through a
slow-acting bond to the surface (i.e. it is on the slow $v_1$
manifold), it should behave like the adiabatic reduction of a
two-state jump-velocity system.

\subsubsection{Summary of the two-state adiabatic reduction}
\label{sec:summary-two-state}

We will use the adiabatic reduction of a two-state jump-velocity
system later, so I will summarize the reduction and quote the final
result here. Suppose we have a two-state system with fast switching
between the two states, as in Figure \ref{fig:two-state-jv}. The
parameters $a$ and $b$ are nondimensional reaction rates where $a + b
= 1$, and $\epsilon$ is a time scaling. We assume that switching
happens quickly, so $a/\epsilon$ and $b\epsilon$ are both $\gg 1$, and
therefore $\epsilon \ll 1$.

\begin{figure}
  \centering
  \schemestart
  $S_0$ \arrow(s0--s1){<=>[$b/\epsilon$][$a/\epsilon$]} $S_1$
  \schemestop
  \caption[Two-state jump-velocity process]{General two-state jump
    velocity process with fast switching.}
  \label{fig:two-state-jv}
\end{figure}

The Fokker-Planck equation for this system is given by the pair of
PDEs
\begin{align}
  \label{eq:two-state-eq1}
  \Pder{q_0}{s} &= -\Pder{q_0}{y} - \frac{1}{\epsilon} \left(b q_0 - a
                  q_1\right) \\
  \label{eq:two-state-eq2}
  \Pder{q_1}{s} &= \left(b q_0 - a q_1\right).
\end{align}

Define the slow variable to be $v = q_0 + q_1$ and the fast variable
to be $w = b q_0 - a q_1$. Then one can perform a change of variables
in Equations (\ref{eq:two-state-eq1}) and (\ref{eq:two-state-eq2}),
and after taking $w$ to be in quasi-steady state, the equation for $w$
is
\begin{equation}
  \label{eq:w}
  w = -\epsilon a b \Pder{v}{y}.
\end{equation}
This can then be used to derive an
advection-diffusion equation for $v$ that only depends on parameters
$a$, $b$, and $\epsilon$:
\begin{equation}
  \label{eq:v}
  \Pder{v}{s} = -a \Pder{v}{y} + \epsilon a b \frac{\partial^2
    v}{\partial y^2}.
\end{equation}

With the initial condition $v(y, 0) = \delta(y)$, equation
(\ref{eq:v}) can be solved analytically:
\begin{equation}
  \label{eq:v-soln}
  v(y, t) = \frac{1}{\sqrt{4 \pi \epsilon a b t}} \exp \left[ \frac{-(y
      - at)^2}{4 \epsilon a b t} \right],
\end{equation}
and then using equation (\ref{eq:w}) to find $w$,
\begin{equation}
  \label{eq:w-soln}
  w(y, t) = \frac{y - at}{4t\sqrt{\pi \epsilon a b t}} \exp \left[ \frac{-(y
      - at)^2}{4 \epsilon a b t} \right].
\end{equation}

A change of variables back to $q_0$ and $q_1$ gives explicit equations
for these two quantities, which can then be used to find the following
probability density function of average velocities for the two-state
jump-velocity process:
\begin{equation}
  \label{eq:1}
  f\left(v^*\right) = \sqrt{\frac{1}{4\pi \epsilon a b
      \left(v^*\right)^3}} \left(a + \frac{v^* - a}{2}\right)
  \exp\left[\frac{-\left(v^* - a\right)^2}{4\epsilon a b v^*}\right].
\end{equation}

\subsubsection{An examination of experimental trajectories}
\label{sec:an-exam-exper}

One of the key assumptions of the jump-velocity model is that
platelets in an unbound state advect at a constant velocity. This is
an assumption that can be checked directly against experimental
trajectory data. Specifically, I split the experimental platelet
trajectories into a segment leading up to the first pause events,
steps in between two pause events, and a segment following the last
pause (if there was one). \note{I should put in plots of Vlado's data;
  maybe one plot in this chapter as an example, and the others in an
  appendix} 

Boxplots of each of these populations of steps are shown in Figure
\ref{fig:ccp-step-cmp}. The steps that occur in between dwells are
much smaller than those at the beginning and end, therefore the
assumption in the jump-velocity model that the beginning and ending
steps are the same is violated. Similarly, the velocity of the
platelet within a step is different depending on whether the step is
in between dwells, or at the beginning or end of a trajectory (Figure
\ref{fig:ccp-vel-cmp}).

\note{Estimates of capture rates?}

\begin{figure}
  \centering
  \includegraphics[width=0.6\textwidth]{ccp-step-cmp.png}
  \caption{Comparison of intermediate step sizes with the first and
    last steps of the experiment. Data shown is from the
    collagen-collagen PRP experiment, but other experiments have
    qualitatively similar results.}
  \label{fig:ccp-step-cmp}
\end{figure}

\begin{figure}
  \centering
  \includegraphics[width=0.6\textwidth]{ccp-vel-cmp.png}
  \caption{Comparison of intermediate free velocities with the first
    and last steps of the experiment. Again, the data shown is from
    the collagen-collagen PRP experiment, but the others are similar.}
  \label{fig:ccp-vel-cmp}
\end{figure}

Furthermore the velocities are not constant, and in the case of the
velocities between pauses, the velocities are several times lower than
we would expect from a totally free-flowing platelet in a 100/s shear
flow. The velocities within the steps before the first pause, and
after the last pause, are much more consistent with the expected
velocity of a free-flowing platelet. Using the Goldman, Cox, and
Brenner approximations for the velocity of a free-flowing sphere in a
shear flow near a wall, we would expect the velocity of a platelet
that is separated from the wall by 50--100 nm to move at velocities in
the range of 67--84 $\mu$m/s \cite{Goldman1967a,Goldman1967b}. This is
still on the high end of the range of velocities observed in Figure
\ref{fig:ccp-vel-cmp}, but one of the assumptions in this
approximation is that platelets are spherical. In reality, platelets
are oblate spheroids, and in a shear flow undergo more complex motions
than spheres \cite{Mody2005}, and in some orientations will travel
more slowly than spheres.

There are a couple of possible explanations for the observation that
the velocities in steps between two pauses are much lower than those
before or after the final pauses. One possibility is that when a
platelet is released following a pause, it is released in an
orientation where the free-flowing velocity is much lower than it was
before the pause. In other words, there may be some hydrodynamic
reason the platelet is traveling more slowly. The physics in this
model is far too simplified to explore this hypothesis.

Another possible explanation is that there is fast binding and
unbinding ocurring on a scale too fast to observe the individual
events. This hypothesis can be tested and incorporated in the context
of the jump-velocity model by using the above quasi-steady state
reduction to fit the fast binding parameters to the inter-pause
velocities. To test this, we want to ignore the steps that
occurs before the first pause and after the last pause in the platelet
trajectories, because the platelet behavior is very different in these
steps compared to the behavior in steps between pauses.

\subsection{Adding an escape term}
\label{sec:adding-an-escape}

If we are only trying to model the behavior of platelets within the
portion of the trajectory between the first and last pause, then we
cannot assume that trajectories are some fixed length $L$. Instead, we
assume that there is some escape rate $\jvEscape$, which represents
the rate that platelets escape the rolling regime, and re-enter the
bulk flow. For convenience in the adiabatic reduction, assume that
platelets can escape when they are in either the unbound state
$\stateUnbound$ or the fast-bound state $\stateBoundF$ (Figure
\ref{fig:jv-model-escape}). This modification makes a small change to
equation (\ref{eq:v1-reduced}); there is an additional
$- \frac{\jvEscape}{T} v_1$ term in the equation. Everything else
remains the same.

\begin{figure}
  \centering
  \schemestart
  $\stateUnbound$ \arrow(u1--vv){<=>[$\jvOnFast$][$\jvOffFast$]}
  $\stateBoundF$ 
  \arrow(@u1--ff){<=>[*{0}$\jvOnSlow$][*{0}$\jvOffSlow$]}[-90]
  $\stateBoundS$ 
  \arrow(--vf){<=>[$\jvOnFast$][$\jvOffFast$]} $\stateBoundFS$
  \arrow(@vv--@vf){<=>[*{0}$\jvOnSlow$][*{0}$\jvOffSlow$]}
  \arrow(@u1--ww){->[*{0}$\jvEscape$]}[90]
  \arrow(@vv--vw){->[*{0}$\jvEscape$]}[90]
  \schemestop
  \caption[Jump velocity model with escape]{Jump velocity model with
    escape} 
  \label{fig:jv-model-escape}
\end{figure}

% AAAh stress. Does anyone care about any of this?? What is the point
% of this jump-velocity project?

% \begin{figure}
%   \centering
%   \includegraphics[width=\textwidth]{ccp-vel-interdwell.png}
%   \caption{Comparison of simulated velocities with observed average
%     velocities measured from the start of the first dwell to the end
%     of the last dwell. Data shown is for the collagen-collagen PRP
%     experiment.}
%   \label{fig:ccp-vel-interdwell}
% \end{figure}

\subsection{Parameter estimation}
\label{sec:parameter-estimation}

\note{Not sure how much of this section to include. It gets pretty
  detailed}

We have a stochastic model of a jump-velocity process, which is used
to derive a Fokker-Planck equation for platelet position as a function
of time. From the Fokker-Planck equation we derive a probability
distribution for the average velocity of a platelet rolling across
some domain. Basically the average velocity of a platelet (or
equivalently, the time it takes a platelet to cross) is a random
variable. The Fokker-Planck equation defines a probability
distribution for this random variable as a function of model
parameters, and average velocities from individual platelet
trajectories are realizations of this random variable. We want to fit
the probability distribution defined by model parameters to the set of
realizations in the experimental data.

We chose to use a maximum-likelihood estimate for parameters in the
model. Let $\{v_i\}_{i=1}^N$ be a set of observations of average
velocities, and define $f(v;a, \epsilon)$ to be the probability
distribution of velocities given parameters $a$ and $\epsilon$ (for
simplicity, assume we're working with the two-state model). Define the
likelihood function
\begin{equation}
  \label{eq:likelihood}
  L(a, \epsilon) = \prod_{i=1}^N f(v_i; a, \epsilon),
\end{equation}
and then the maximum likelihood estimates for $a$ and $\epsilon$ are
those that maximize $L$. Define the log-likelihood function:
$\tilde{\ell}(a, \epsilon) = \log(L(a, \epsilon))$, and then
maximizing $\tilde{\ell}$ is equivalent to maximizing $L$.

We found $f(v; a, \epsilon)$ above in terms of the solution of the
Fokker-Planck equation: $f(v; a, \epsilon) = 1/v^2 q_{U^1}(1, 1/v; a,
\epsilon)$ where $q_{U^1}(x, t; a, \epsilon)$ is part of the solution
of the Fokker-Planck equation with parameters $a$ and $\epsilon$. Then
we can write $\tilde{\ell}$ in terms of $q_{U^1}$:
$\tilde{\ell} = \sum_{i=1}^N \log(f(v_i; a, \epsilon)) = \sum_{i=1}^N
\log(q_{U^1}(1, 1/v_i; a, \epsilon)) - 2\sum_{i=1}^N \log(v_i)$. The
second term is a constant factor with respect to $a$ and $\epsilon$,
and so can be excluded from the optimization. Therefore we can
maximize the modified log-likelihood function
\begin{equation}
  \label{eq:mod-log-like}
  \ell(a, \epsilon) = \sum_{i=1}^N \log(q_{U^1}(1, 1/v_i; a, \epsilon)).
\end{equation}

\subsubsection{Numerical Optimization}
\label{sec:numer-optim}

The modified log-likelihood function $\ell(a, \epsilon)$ must be
maximized iteratively. We haven't even found an analytical solution
for $q_{U^1}$. Therefore, the Fokker-Planck equation must be re-solved
each time the parameters $a$ and $\epsilon$ are changed. Fortunately
the PDE only needs to be solved once for each evaluation of
$\ell(a, \epsilon)$. We have to find $q_{U^1}(1, t)$ up to time
$t = 1/\min(v_i)$, but then to evaluate the probability density at
each time $1/v_i$, we only need to interpolate the solution.

Another consideration is that the model parameters $a$ and $\epsilon$
have bounds, in particular $a \in (0, 1)$ and $\epsilon > 0$. One
possiblility for staying within these constraints is to use a
numerical optimization algorithm that optimizes within a bounded
domain. However, near the boundaries of the domain $\ell(a, \epsilon)$
diverges and in some cases can cause the minimization to
fail. Instead, I have achieved good results in practice by
transforming the model parameters $a$ and $\epsilon$ to a pair of new
parameters $a'$ and $\epsilon'$ which can vary over all $\reals$. The
transformations for these parameters are:
\begin{align}
  \label{eq:a-fwd-trns}
  a' &= \frac{2a + 1}{2a(a + 1)} \\
  \label{eq:e-fwd-trns}
  \epsilon' &= \log \epsilon.
\end{align}
Note that $a'$ is defined so that $a'=0$ at $a=1/2$,
$a' \rightarrow -\infty$ as $a \rightarrow 0$, and
$a' \rightarrow \infty$ as $a \rightarrow 1$. With the fitting
parameters, the optimization problem becomes an unconstrained
optimization in $\reals^2$. To actually carry out the optimization I
use SciPy's \cite{Virtanen2020} \verb|minimize| function, which
implements the BFGS algorithm \note{citations}.

An iterative algorithm needs an intial guess, which at least in the
two-state model, can be chosen as an estimate of $a$ and $\epsilon$
from the adiabatic reduction. Equation  gives the
pdf of velocities from the adiabatic reduction, and can be used to
derive the mean $\mu(a, \epsilon)$ and variance $\sigma^2(a,
\epsilon)$ as functions of $a$ and $\epsilon$. These moments have
simple analytical expressions: 
\begin{align}
  \label{eq:mean-ar}
  \mu(a, \epsilon) &= a(1 + \epsilon - a \epsilon) \\
  \label{eq:var-ar}
  \sigma^2(a, \epsilon) &= a^2 \epsilon (1 - a) (2 + 5(1 - a) \epsilon).
\end{align}

Then we can equate $\mu$ and $\sigma^2$ to the first two sample
moments ($\bar{v}$ and $s^2$) of the data to estimate $a$ and
$\epsilon$. This is called the method of moments. We can find exact
analytical expressions for $a$ and $\epsilon$ in terms of $\mu$ and
$\sigma^2$, but the formulas are messy. Note that
$\sigma^2 = O(\epsilon)$ and $\epsilon \ll 1$, so we can use an
asymptotic approximation instead of the full formulas. The estimate is
only being used to initialize the optimization algorithm, so there is
no need to be exact. Therefore up to $O(\sigma^2)$, the method of
moments gives the following estimates for $a$ and $\epsilon$ based on
a data set with sample mean $\bar{v}$ and sample variance $s^2$:
\begin{align}
  \label{eq:mean-est}
  a &= \bar{v} - \frac{s^2}{2\bar{v}} \\
  \label{eq:var-est}
  \epsilon &= \frac{s^2}{2(1 - \bar{v})(\bar{v})^2}.
\end{align}

In Figure \ref{fig:model-fits}, a histogram of 1000 sample velocities
is shown along with the PDEs generated by the reduced and full models
with maximum likelihood parameter estimates. The CDFs of the models
are also compared with the ECDF of the data. Visually these models fit
well (as they should), and provide parameter estimates with less than
10\% relative error.

\subsubsection{Bootstrapping}
\label{sec:bootstrapping}

The maximum likelihood estimator only gives a point estimate of the
parameters $a$ and $\epsilon$. In order to get confidence intervals,
we can use a bootstrapping approach. The basic idea is that for a set
of data $\{V_i\}_{i=1}^N$, a ``new'' data set can be generated by
randomly picking $N$ samples from the $\{V_i\}$ with replacement. Then
the MLE for the new data can be found, and in this way we get a bunch
of $\hat{a}$ and $\hat{\epsilon}$ estimates. After generating many of
these estimates, we can come up with an estimate of a 95\% confidence
interval for each parameter. If we assume that the ML estimates are
distributed sufficiently normally, then
$\hat{\theta} \pm 1.96 \sigma_{\hat{\theta}}$ approximates a 95\%
confidence interval well. But it seems like using the $2.5$th and
$97.5$th percentiles also provides a good estimate for the confidence
interval, and doesn't rely on assuming the parameter estimate is
distributed normally.

Figure \ref{fig:bootstraps} summarizes results from 64 bootstrap
trials on the sample data. In each case, the distribution of estimated
parameters looks symmetrical, and the two methods of estimating a 95\%
confidence interval described above give similar results (Table
\ref{tab:conf-int}).

While the MLE of $a$ looks unbiased (i.e. it is near the center of the
distribution of $a$s shown in figure \ref{fig:bootstraps}), the MLE of
$\epsilon$ is clearly biased to the upper end of the distribution of
$\epsilon$ estimates from the bootstrap procedure. Part of this may be
because I haven't filtered out platelets that didn't pause on the
surface, but it is still something to look out for. 

\subsubsection{Goodness-of-fit}
\label{sec:goodness-fit}

Finally, we want to find the goodness-of-fit of the model. Chapter 13
in \textit{Introduction to Probability and Mathematical Statistics} by
Bain and Engelhardt suggests two approaches. One approach is to group
the data into bins, and then compare the observed observations in each
bin to the expected number of observations in each bin. In particular,
\begin{equation}
  \label{eq:chi2}
  \chi^2 = \sum_{j=1}^c \frac{(o_j - \hat{e}_j)^2}{\hat{e}_j} \sim
  \chi^2(c - 1 - k)
\end{equation}
where $c$ is the number of bins, $k$ is the number of estimated
parameters, $o_j$ is the number of data points in bin $j$, and
$\hat{e}_j$ is the expected number of data points in bin $j$. Here
$\hat{e}_j = N \hat{p}_j$ where $N$ is the total number of data
points, and $\hat{p}_j$ is the model probability of picking a data
point from bin $j$. A weakness of this approach is that information is
lost by grouping the data, and there are other tests that depend
directly on the individual observations.

One of these tests is the Kolmogorov-Smirnov test, which essentially
uses the maximum difference between the predicted CDF and the
ECDF. The KS statistic is $D = \max(D^+, D^-)$ where
$D^+ = \max_i(i/N - F(x_{i:N}))$ and
$D^- = \max_i(F(x_{i:N}) - (i - 1)/N)$. A small value of $D$ indicates
a good fit (though precisely what value of $D$ means the fit is
``good'' depends on $N$). The weakness of this approach is that the
distribution of the KS statistic is derived assuming no parameters
have been estimated, which is obviously not true in our case. The
Kolmogorov-Smirnov test returned a $p$-value of 0.46 for the reduced
model, and 0.66 for the full model, indicating a good fit of these
models to the data (obviously, because the data were derived from the
jump-velocity model).

\section{Results}
\label{sec:results-jump-velocity}

\subsection{Estimating fast binding parameters by fitting the free
  velocities}
\label{sec:estim-fast-bind}

In order to fit the inter-pause velocities from the experimental data,
I processed the experimental trajectories by removing dwells from the
experiments and ``stacking'' the steps on top of each other. The
probability distributions of the average velocities of these processed
trajectories on collagen and fibrinogen are shown in Figure ?? and
Figure ??. The fast effective binding parameters predicted by each
model fit are given in the captions below each figure.

% What about fitting to individual dwells? This might make it easier
% to bootstrap uncertainty estimates of the parameters. This would
% probably also be a more convincing analysis that the free velocities
% don't change between experiments than an ANOVA

\subsection{Estimating slow binding parameters and escape}
\label{sec:estim-slow-bind}

One prediction from the jump-velocity model with escape is that the
number of dwells in a trajectory should be geometrically
distributed. When a platelet is in a step, one can think of the
platelet's next state as a Bernoulli trial. One possibile result of
the Bernoulli trial is the platelet pauses after the step, which
occurs with probability
$a = \frac{\jvOnSlow}{\jvOnSlow + \jvEscape}$. The other possible
result is the platelet escapes, and this occurs with probability $1 -
a = \frac{\jvEscape}{\jvOnSlow + \jvEscape}$.

Therefore the number of binding events that occur before the platelet
escapes is a geometric distribution, and the probability mass function
for the number of binding events $n$ is $p(n) = (1 - a) a^{n - 1}$,
where $n >= 1$. Therefore the mean number of pauses platelets have
before escaping can be used to find $a$: $\mu_\textnormal{dwell num.}
= (1 - a)\inv$. Figure \ref{fig:ndwells} shows the number of dwells per
trajectory in two experiments.

\begin{figure}
  \centering
  \includegraphics[width=\textwidth]{fbg_num_dwells}
  \caption{Number of dwells per trajectory in experiments and
    predicted by a geometric distribution.}
  \label{fig:ndwells}
\end{figure}

We need another estimate in order to uniquely find $\jvOnSlow$ and
$\jvEscape$. \note{Think about how to succinctly justify the
  following} It turns out that the distribution of step times is an
exponential distribution with rate $\jvTot$, and so we can estimate
$\jvTot$ with the average step time: $\mu_\tn{step time} = 1/\jvTot =
1/(\jvOnSlow + \jvEscape)$. With these estimates for $\jvOnSlow$ and
$\jvEscape$, along with the estimate of $\jvOffSlow$ from the pause
times, we get the following estimates for the primed and unprimed
experiments. \note{This thought is unfinished}

\begin{figure}
  \centering
  \begin{tikzpicture}
    \node (step) {Step};
    \node (escape) [right=of step] {Escape};
    \draw[-Stealth] (step.east)
    --node[above]{$\frac{e}{\jvTot}$} (escape.west); 
    %  {$\frac{e}{\jvTot}$}
    \draw[-Stealth] (step.north east)
    to[to path={
      .. controls +(60:1.5) and +(120:1.5) .. (\tikztotarget)
      \tikztonodes}
    ] node[above]{$\frac{d/\epsilon_2}{\jvTot}$} (step.north west); 
  \end{tikzpicture}
  \caption[Possible fates of an unbound platelet]{A platelet that is
    not bound to the surface has two possible fates: it can either
    re-bind with probability $a$ (in which case it must unbind at some
    point again and return to the same state), or it can escape with
    probability $1 - a$}
  \label{fig:unbd-plt-fates}
\end{figure}

% \subsection{Step sizes and ``free velocities'' in the trajectory data}
% \label{sec:step-sizes-free}

% %% Quotation marks in a section title?

% The adiabatic reduction of the 4-state model is motivated by an
% observation that the ``free velocities'' of observed platelets is much
% lower than expected free-flowing velocities. The experiments are run
% with a wall shear rate of 100/s, and so the fluid velocity $1 \mu\tn{m}$
% off the wall is about $100 \mu\tn{m}/s$. However the observed free
% velocities (i.e. during a dwell) are much lower than this (see Figure
% \ref{fig:ccp-vel-cmp}). We have also observed that the velocity of a
% platelet before its first dwell, and after its last dwell, is much
% higher than velocity in between dwells. Similarly, the initial and final
% steps of a platelet (i.e. as it enters and leaves the field of view) are
% much longer than the lengths of intermediate steps.

% Boxplots of each of the populations of step sizes are shown in Figure
% \ref{fig:ccp-step-cmp}. The steps that occur in between dwells are much
% smaller than those at the beginning and end, therefore our assumption in
% the jump-velocity model that the beginning and ending steps are the same
% is violated. Similarly, the velocity of the platelet within a step is
% different depending on whether the step is in between dwells, or at the
% beginning or end of a trajectory (Figure \ref{fig:ccp-vel-cmp}). Another
% thing to note in Figure \ref{fig:ccp-vel-cmp} is that velocities within
% steps is not constant, or even particularly close to constant.

% One possible explanation of this is that there are brief contacts
% between the platelet and the wall which slow the platelet down to well
% below the free-flowing velocity, but these binding dynamics occur too
% quickly to be detected individually. These bonds are likely
% GPVI-collagen bonds because they occur on a relatively fast time
% scale. Then the observed pauses may be due to integrin-collagen binding.

% An analysis of the platelet velocities in between steps provides one
% possible check of this hypothesis. As shown in the adiabatic reduction
% of the previous section, the sum of $\stateUnbound$ and $\stateBoundF$
% platelets should advect and diffuse in a way that is well modeled by
% equation (\ref{eq:v1-reduced}). I processed the experimental
% trajectories by removing the observed dwells, and then comparing the
% resulting distribution of average free velocities to the distribution of
% average velocities predicted by the adiabatic reduction of the 2 state
% model (equation (\ref{eq:vel-dens})) (Figure \ref{fig:avg-free-vel-col}).

% \begin{figure}
%   \centering
%   \includegraphics[width=0.8\textwidth]{ccp-step-cmp.png}
%   \caption{Comparison of intermediate step sizes with the first and
%     last steps of the experiment. Data shown is from the
%     collagen-collagen PRP experiment, but other experiments have
%     qualitatively similar results.}
%   \label{fig:ccp-step-cmp}
% \end{figure}

% \begin{figure}
%   \centering
%   \includegraphics[width=0.8\textwidth]{ccp-vel-cmp.png}
%   \caption{Comparison of intermediate free velocities with the first
%     and last steps of the experiment. Again, the data shown is from
%     the collagen-collagen PRP experiment, but the others are similar.}
%   \label{fig:ccp-vel-cmp}
% \end{figure}

% \begin{figure}
%   \centering
%   \includegraphics[width=\textwidth]{avg_free_vel_col.png}
%   \caption{Comparison of the average free velocity in the four collagen
%     experimental conditions. The effective binding parameters found by
%     fitting the model to velocity data are $k_\tn{on}=34.6/s$ and
%     $k_\tn{off} = 5.18/s$. An ANOVA test didn't find any significant
%     difference in the average velocities among the 4 experiments, and so
%     I fit the model to all the data simultaneously.}
%   \label{fig:avg-free-vel-col}
% \end{figure}

% Local Variables:
% TeX-master: "phd-thesis.ltx"
% End:
