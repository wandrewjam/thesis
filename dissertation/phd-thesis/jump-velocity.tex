%%% -*-LaTeX-*-

\chapter{Jump-Velocity Model}
\label{cha:jump-velocity-model}

\section{The model}
\label{sec:jv-model}

While the previous model can be used for some explorations of
parameter space, it is not feasible to use it for rigorous parameter
estimation. (Estimate of time involved?)

Suppose we model platelets as particles translating at a constant
velocity along a wall coated with an immobilized agonist
(e.g. collagen or vWF). Platelets have two types of receptors that can
bind to these immobilized agonists. One type are the receptors which
are constitutively active, and have fast binding and unbinding
kinetics. The other type are receptors which are constitutively
inactive, and become activated as the platelet activates. {\color{red}
  This sentence should be reworked}

Initially, assume that unprimed platelets can exist in two states:
unbound ($\stateUnbound$), and bound to the wall by fast receptors
($\stateBoundF$). Assume the platelet cannot be bound by slow
receptors, because they are not active in unprimed
platelets. Platelets in the unbound state advect at a velocity $v$ in
the fluid, and platelets in the fast-bound state are
stationary. Platelets in the fluid can bind through fast receptors at
a constant rate $\jvOnFast$, and platelets bound through fast
receptors can unbind at a constant rate $\jvOffFast$. The rates
$\jvOnFast$ and $\jvOffFast$ are effective on and off rates for the
entire platelet, and \emph{not} on/off rates for individual receptors.

For primed platelets, assume that they can exist in four possible
states: unbound ($\stateUnbound$), fast-bound ($\stateBoundF$),
slow-bound ($\stateBoundS$), and both fast- and slow-bound
($\stateBoundFS$).

\begin{figure}[h]
  \centering
  \schemestart
  $\stateUnboundOld$ \arrow(u1--vv){<=>[$\jvOnFast$][$\jvOffFast$]}
  $\stateBoundF$ \arrow(--u0){<-[*{0}$\jvOnFast$]}[90]
  $\stateUnboundNew$
  \schemestop 
  \caption[Possible states with one receptor]{An unprimed platelet can
    exist in three distinct states: ($\stateUnboundNew$) platelets
    which haven't interacted with the surface, ($\stateUnboundOld$)
    platelets which have interacted with the surface and are advecting
    in the fluid, or ($\stateBoundF$) bound to vWF on the surface and
    unmoving. Transitions between these states occur at constant rates
    $\jvOnFast$ and $\jvOffFast$. Platelets can only transition out of
    the $\stateUnboundNew$ state.}
  \label{fig:unprimed-states}
\end{figure}

% Where and how do we want to address the connection between receptor
% on/off rates and effective on/off rates?

% Local Variables:
% TeX-master: "phd-thesis.ltx"
% End:
