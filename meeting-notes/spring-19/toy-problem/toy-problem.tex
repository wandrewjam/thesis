\documentclass{article}

\newcommand{\ep}{\rule{.06in}{.1in}}
\textheight 9.5in

\usepackage{amssymb}
\usepackage{amsmath}
\usepackage{graphicx}
\usepackage{subcaption}
\usepackage{pgfplots}
\usepackage{textgreek}

\pgfplotstableset{col sep=comma}
\pgfkeys{/pgf/number format/.cd,fixed,precision=3}
\newcommand{\tn}{\textnormal}

\pagestyle{empty} 
\oddsidemargin -0.25in
\evensidemargin -0.25in 
\topmargin -0.75in 
\parindent 0pt
\parskip 12pt
\textwidth 7in
%\font\cj=msbm10 at 12pt
%\topmargin -.5in

\newcommand{\radius}{R}
\newcommand{\separation}{d}
\newcommand{\stiffness}{k_f}
\newcommand{\boltzmann}{k_B}
\newcommand{\temp}{T}
\newcommand{\onConst}{k_\text{on}}
\newcommand{\offConst}{k_\text{off}}
\newcommand{\refForce}{f_0}
\newcommand{\receptorDensity}{N_T}
\newcommand{\receptorNumber}{N_R}
\newcommand{\appliedRot}{\Omega_f}
\newcommand{\appliedVel}{V_f}
\newcommand{\velFriction}{\xi_V}
\newcommand{\rotFriction}{\xi_\omega}
\newcommand{\compliance}{\Gamma}
\newcommand{\width}{w}
\newcommand{\viscosity}{\mu}

\newcommand{\ndSeparation}{d'}
\newcommand{\ndAppliedRot}{\omega_f}
\newcommand{\ndAppliedVel}{v_f}
\newcommand{\ndOnConst}{\kappa}
\newcommand{\onForceScale}{\eta}
\newcommand{\offForceScale}{\delta}
\newcommand{\ndVelFriction}{\eta_v}
\newcommand{\ndRotFriction}{\eta_\omega}

\newcommand{\ITA}[1]{\textalpha\textsubscript{#1}}
\newcommand{\ITB}[1]{\textbeta\textsubscript{#1}}

% \graphicspath{{/Users/andrewwork/thesis/rolling-model/plots/}}

\begin{document}
\pagestyle{empty}

\begin{center}
{\Large Notes on a toy rolling problem}
\end{center}

\section{The rolling-only problem}
\label{sec:rolling-only-problem}

To simplify the platelet rolling model, assume that $d = 0$ and
$V = R \Omega$. That is, no slip is allowed between the platelet
surface and the wall. Then the bond density ($n$) evolution equation
simplifies to
\begin{equation}
  \label{eq:full-pde}
  \frac{\partial n}{\partial t} = R\Omega \frac{\partial n}{\partial
    x} + \Omega \frac{\partial n}{\partial \theta} + \alpha(x, \theta)
  \left(N_T - \int_{-\infty}^\infty n(x, \theta) dx \right) - \beta(x,
  \theta) n.
\end{equation}

We are looking to find steady state rolling velocities and bond
distributions as a function of the applied rolling velocity $V_f$, so
set $\frac{\partial n}{\partial t} = 0$:
\begin{equation}
  \label{eq:dim-ss}
  0 = R\Omega \frac{\partial n}{\partial x} + \Omega \frac{\partial
    n}{\partial \theta} + \alpha(x, \theta) \left(N_T -
    \int_{-\infty}^\infty n(x, \theta) dx \right) - \beta(x, \theta)n,
\end{equation}
or in nondimensional terms, 
\begin{equation}
  \label{eq:ndim-ss}
  0 = \omega \frac{\partial m}{\partial z} + \omega \frac{\partial
    m}{\partial \theta} + \alpha'(z, \theta) \left(1 -
    \int_{-\infty}^\infty m(z, \theta) dz \right) - \beta'(z,
  \theta)m.
\end{equation}

We can write evolution equations for the coordinates $z$ and $\theta$
using the rolling velocities:
\begin{equation}
  \label{eq:dz-dth}
  \frac{dz}{dt} = \frac{d\theta}{dt} = -\omega
\end{equation}

Therefore, $z(t) = -\omega t + z_0$ and $\theta(t) = -\omega t +
\theta_0$, then eliminating $t$ we get $\theta(z) = z + \theta_0 -
z_0$. Let's say we can define a density function $\hat{m}$ that only
varies as a function of $z$: $\hat{m}(z) = m(z, \theta(z))$. Then note
what happens when we take a derivative of $\hat{m}$:
\begin{equation}
  \frac{d\hat{m}}{dz} = \frac{\partial m}{\partial z} + \frac{\partial
    m}{\partial \theta} \frac{d \theta}{dz} = \frac{\partial m}{\partial
    z} + \frac{\partial m}{\partial\theta}.
\end{equation}

Multiplying through by $-\omega$, we get the following:
\begin{align}
  -\omega\frac{d\hat{m}}{dz} = -\omega \frac{\partial m}{\partial z} -
  \omega\frac{\partial m}{\partial\theta}
  &= \alpha'(z, \theta(z)) \left(1 - \int_{-\infty}^{\infty} m(z,
    \theta(z)) dz \right) - \beta'(z, \theta(z)) m \nonumber \\
  &= \alpha'(z, \theta(z)) \left(1 - \int_{-\infty}^{\infty}
    \hat{m}(z) dz \right) - \beta'(z, \theta(z))
    \hat{m}. \label{eq:ide}
\end{align}

If you believe the work above, we have simplified an
integro-differential equation in two variables to an IDE in one
variable.

\subsection{Solving the IDE for $\hat{m}$}
\label{sec:solving-ide}

From equation \eqref{eq:ide} above, we have
\begin{equation}
  -\omega\frac{d\hat{m}}{dz} = \hat{\alpha}'(z) \left(1 -
    \int_{-\infty}^{\infty} \hat{m}(z) dz \right) - \hat{\beta}'(z)
  \hat{m}.
\end{equation}

This can be solved with the following procedure:

\begin{enumerate}
\item Pick a value $I \in [0, 1)$
\item Solve the ODE $-\omega\frac{dm'}{dz} = \hat{\alpha}'(z) \left(1
    - I \right) - \hat{\beta}'(z) m'$ 
\item Define $I' = \int_{-\infty}^{\infty} m'(x) dx$ 
\item Then $\hat{m}(x) = m'(x)/(1 - I + I')$
\end{enumerate}

To see that $\hat{m}(z)$ solves the IDE, substitute $m'(z)/(1 - I +
I')$ into the IDE above:
\begin{equation}
  -\omega \frac{d\hat{m}}{dz} = -\frac{\omega}{1 - I +
    I'}\frac{dm'(z)}{dz} = \frac{\hat{\alpha}'}{1 - I + I'}(1 - I) -
  \hat{\beta}'\frac{m'}{1 - I + I'}. \label{eq:verify}
\end{equation}

Clearly the 2nd term on the RHS is just $\hat{\beta}'(z) \hat{m}$. The
1st term on the RHS simplifies to $\hat{\alpha}' \left(1 - \frac{I'}{1
    - I + I'}\right)$. Note that from the definition of $I'$,
$\frac{I'}{1 - I + I'} = \int_{-\infty}^{\infty} \hat{m}(z)
dz$. Therefore $\hat{m}(z)$ solves the IDE.

Dr. Keener's code solves the following ODE system (taking $I = 0$):
\begin{equation}
  \frac{d\mathbf{y}}{dz} = \mathbf{F}(z, \mathbf{y}(z)) \text{ where }
  \mathbf{y} = \begin{pmatrix} m' \\ M \\ T \end{pmatrix} \text{ and }
  \mathbf{F} = \begin{pmatrix} \hat{\alpha}'(z) - \hat{\beta}'(z) m'
    \\ m' \\ \tau \end{pmatrix}.
\end{equation}

$M$ and $T$ accumulate the bond number and torque, respectively, and
$\tau = -[(1 - \cos(z))\sin(z) + (\sin(z) - z)\cos(z)]m'(z)$. The
total bond number $\int m'(z) dz$ and total torque $\int \tau m'(z)
dz$ are just the endpoints of $M$ and $T$. Note: Dr. Keener's code
doesn't rescale $T$, and so it instead reports the total torque
generated by the steady state bond distribution of the modified
problem in step 2.

Figures \ref{fig:unscaled} \& \ref{fig:scaled} show results from
solving equation \eqref{eq:ide} with $\omega = 10$.

\begin{figure}
  \centering
  \begin{subfigure}{0.48\linewidth}
    \begin{tikzpicture}
      \begin{axis}[
        xlabel={$z$},
        legend entries={$m'(z)$, $\int^z m'(s) ds$,
          $\int^z \tau_s(s) ds$},
        legend pos=north
        west,
        grid=major,
        ]
        \addplot[no markers, smooth, blue, very thick] table[x=t,
        y=m_prime] {unscaled.dat};
        \addplot[no markers, smooth, red, very thick] table[x=t, y=M]
        {unscaled.dat};
        \addplot[no markers, smooth, black, very thick] table[x=t,
        y=torque] {unscaled.dat};
      \end{axis}
    \end{tikzpicture}
    \caption{Plots of $m'(z)$, $M$, and $T$}
    \label{fig:unscaled}
  \end{subfigure}
  \hfill
  \begin{subfigure}{0.48\linewidth}
    \begin{tikzpicture}
      \begin{axis}[
        xlabel={$z$},
        ylabel={Value of $\hat{m}(z)$},
        grid=major,
        yticklabel={$\pgfmathprintnumber{\tick}$},
        ]
        \addplot[no markers, smooth, blue, very thick] table
        {scaled.dat};
      \end{axis}
    \end{tikzpicture}
    \caption{Plot of $\hat{m}(z)$}
    \label{fig:scaled}
  \end{subfigure}
  \caption{Results from solving the reduced problem \eqref{eq:ide} for
  $\omega = 10$}
  \label{fig:reduced-prob}
\end{figure}

\subsection{Generating a bifurcation diagram}
\label{sec:bifurc-diagr}

We can use the above approach to generate a bifurcation diagram in the
parameter $\omega_f$. We choose a bunch of $\omega$ values, solve for
the steady state bond distribution, calculate the torque generated at
steady state, and finally solve for $\omega_f$ using $\omega_f =
\omega + \tau/\xi_\omega$. 

A plot of torque versus rolling velocity
$\omega$ is given in Figure \ref{fig:om-vs-tau} and a plot of steady
states in $(\omega_f, \omega)$ space are shown in Figure
\ref{fig:omf-vs-om}. The results in these plots match the results of
Dr. Keener's Matlab code (after applying the correct scaling).

\begin{figure}
  \centering
  \begin{subfigure}{0.48\linewidth}
    \begin{tikzpicture}
      \begin{axis}[
        xlabel={$\omega$},
        ylabel={$\tau$},
        grid=major,
        scaled ticks=false,
        ]
        \addplot[no markers, smooth, blue, very thick] table
        {om_vs_tau.dat};
      \end{axis}
    \end{tikzpicture}
    \caption{Torque versus rolling velocity $\omega$}
    \label{fig:om-vs-tau}
  \end{subfigure}
  \hfill
  \begin{subfigure}{0.48\linewidth}
    \begin{tikzpicture}
      \begin{axis}[
        xlabel={$\omega_f$},
        ylabel={$\omega$},
        grid=major
        ]
        \addplot[no markers, smooth, blue, very thick] table
        {omf_vs_om.dat};
        \addplot[no markers, black, very thick, dashed, domain=0:110]
        {x}; 
      \end{axis}
    \end{tikzpicture}
    \caption{Applied rolling velocity $\omega_f$ versus rolling
      velocity $\omega$}
    \label{fig:omf-vs-om}
  \end{subfigure}
  \caption{Results for many different rolling velocities $\omega$}
  \label{fig:bifurcation}
\end{figure}

\subsection{Comparison with 2D steady state solution}
\label{sec:comparison-with-2d}

In order to solve the full 2D problem (equation \eqref{eq:ndim-ss}),
restrict the numerical domain in $z$ to be $[-L, L]$. Then define $z_i
= -L + ih$ for $i = 0, \hdots, 2M$ and $h = L/M$, and $\theta_j =
-\pi/2 + j\nu$ for $j = 0, \hdots, N$ and $\nu = \pi/N$. Let $m_{ij}
\approx m(z_i, \theta_j)$, and define $\alpha_{ij}$ and $\beta_{ij}$
similarly. Also, define $D_z$ and $D_\theta$ be forward differences in
$z$ and $\theta$ respectively.

From boundary conditions, $m_{i,N+1} = m_{2M+1, j} = 0 \quad \forall
i,j$. Then we can define a numerical scheme:
\begin{equation}
  0 = \omega(D_z m_{ij} + D_\theta m_{ij}) + \alpha_{ij} \left(1 - h
    \sum_i a_i m_{ij} \right) - \beta_{ij} m_{ij} \text{ for } i=0,
  \hdots, 2M \text{ and } j=0, \hdots, N.
\end{equation}
This gives us a $(2M+1)(N+1) \times (2M+1)(N+1)$ linear system that we
can solve to find the $m_{ij}$s. From the steady state distribution,
it is straightforward to calculate the torque generated by the
distribution $\tau = h \nu \sum_j a_j \sum_i a_i \tau_{ij} m_{ij}$.

I repeated the bifurcation analysis using 100 $\omega$ values between
$0$ and $100$ with the 2D problem, and compared the results with the
reduced problem in Figure \ref{fig:bifurcation2d}. I used a first
order forward difference for $D_z$ and $D_\theta$, and I used a
trapezoid rule to approximate the integral (that is, $a_0 = a_{2M} =
1/2$ and $a_i = 1$ otherwise). In the 2D problem, the torque is less
sharply peaked as a function of $\omega$ than in the reduced problem,
and the peak has moved to the right. That is, the maximum torque is
achieved for a higher value of $\omega$ in the 2D problem.

\begin{figure}
  \centering
  \begin{subfigure}{0.48\linewidth}
    \begin{tikzpicture}
      \begin{axis}[
        xlabel={$\omega$},
        ylabel={$\tau$},
        grid=major,
        scaled ticks=false,
        legend entries={Reduced problem, 2D problem},
        legend pos=north east,
        ]
        \addplot[no markers, smooth, blue, very thick] table
        {om_vs_tau.dat};
        \addplot[no markers, smooth, red, very thick] table
        {om_vs_tau2d.dat};
      \end{axis}      
    \end{tikzpicture}
    \caption{Torque versus rolling velocity $\omega$}
    \label{fig:om-vs-tau2d}
  \end{subfigure}
  \hfill
  \begin{subfigure}{0.48\linewidth}
    \begin{tikzpicture}
      \begin{axis}[
        xlabel={$\omega_f$},
        ylabel={$\omega$},
        grid=major,
        legend entries={Reduced problem, 2D problem},
        legend pos=north west,
        ]
        \addplot[no markers, smooth, blue, very thick] table
        {omf_vs_om.dat};
        \addplot[no markers, smooth, red, very thick] table
        {omf_vs_om2d.dat}; 
        \addplot[no markers, black, very thick, dashed, domain=0:110]
        {x}; 
      \end{axis}
    \end{tikzpicture}
    \caption{Applied rolling velocity $\omega_f$ versus rolling
      velocity $\omega$ for the 2D steady state problem}
    \label{fig:omf-vs-om2d}
  \end{subfigure}

  \caption{Results from the 2D steady state problem with varying
    $\omega$}
  \label{fig:bifurcation2d}
\end{figure}

\subsection{Thoughts about the reduced problem}
\label{sec:thoughts}

If the reduced problem was equivalent to the 2D problem, we would
expect the curves in Figure \ref{fig:bifurcation2d} to be exactly the
same. There are a couple of reasons why I think the reduced problem is
not equivalent to the 2D problem:
\begin{enumerate}
\item In the 2D problem, $m$ is defined on the 2D domain $[-L, L]
  \times [-\pi/2, \pi/2]$. In order to reduce the problem to 1
  dimension, we took $z = \theta$, but this only represents a single
  path through the domain. In particular, the integral in equation
  \eqref{eq:ndim-ss} is integrating along a different curve in $(z,
  \theta)$ than the integral in equation \eqref{eq:ide}. The first
  integral is integrating along lines of constant $\theta$, whereas
  the second integral is integrating along the line $z = \theta$.
\item The 2D problem allows for bonds at a fixed $\theta$ to be bound
  to multiple different $z$s along the wall, and vice versa. The
  reduced problem does not allow this.
\item The quantities $m$ and $\hat{m}$ seem to have fundamentally
  different interpretations. Both of them are nondimensional, but
  $\hat{m}$ is a nondimensional density---you integrate it once to get
  a bond number---and $m$ is a nondimensional 2nd order (?)
  density---you integrate it twice to get a bond number.
\end{enumerate}

\section{Simplifications of the full problem}
\label{sec:simpl-full-probl}

Because the above reduction is not correct, and because it isn't clear
that there is a way to reduce the full model exactly, we need to
explore some simplifications of the problem that allow us to
search through parameter space rapidly. There are two simplifications
I have thought of so far:
\begin{enumerate}
\item Removing the saturation term (i.e. the integral term) in
  equation~\eqref{eq:ndim-ss}, and 
\item Approximating $\alpha(z, \theta)$ with a $\delta$-fn.
\end{enumerate}

\subsection{Model without saturation}
\label{sec:model-no-satur}

If we assume there are an excess of receptors relative to the steady
state number of bonds, then equation \eqref{eq:ndim-ss} is
approximately equivalent to
\begin{equation}
  \label{eq:reduced-model-sat}
  0 = \omega \left(\frac{\partial m}{\partial z} + \frac{\partial
      m}{\partial \theta} \right) + \alpha'(z, \theta) - \beta'(z, \theta)m.
\end{equation}

This simplification is a reasonable thing to try, because we are
looking for regions of parameter space where there are relatively few
bonds anyway, and so saturation may not be particularly relevant in
that parameter regime. (Caveat: we are looking for regions where $\pi
>> \int \int m dz d\theta$, which is not necessarily the same as $1 >>
\int m dz$ for all $\theta$).

This problem can be solved with the method of characteristics. Define
the following characteristics as functions of parameter $t$: $z(t, z_0) =
z_0 - \pi t$ and $\theta(t) = \pi/2 - \pi t$ where $0 \le t \le
1$. Then
\begin{equation}
  \label{eq:char}
  \frac{d m}{ds} (z(s, z_0), \theta(s)) = -\pi \left( \frac{\partial 
      m}{\partial z} + \frac{\partial m}{\partial \theta} \right) =
  -\frac{\pi}{\omega} \left[ \alpha'(z(s, z_0), \theta(s)) -
    \beta'(z(s, z_0), \theta(s)) m(z(s, z_0), \theta(s)) \right].
\end{equation}

Also note that with our change of variables, integrals over the domain
change:
\begin{equation}
  \label{eq:integral-cov}
  \iint f(z, \theta) dz d\theta = \pi \iint f(z(s, z_0), \theta(s)) ds dz_0.
\end{equation}

Then we can solve for $m$ as a function of $s$ and $z_0$ by
integrating equation \eqref{eq:char} and then find the total bond
number and total torque on the platelet at a given rolling velocity
$\omega$ by using equation \eqref{eq:integral-cov}.

The Radau method implemented in SciPy's \verb|solve_ivp| method was
too slow, so I just implemented a simple trapezoidal rule in Python to
solve this:
\begin{equation}
  \label{eq:trapezoidal-rule}
  m_{i+1} = m_i + \frac{h\pi}{2\omega} \left[ \alpha_i - \beta_i m_i +
    \alpha_{i+1} - \beta_{i+1} m_{i+1} \right],
\end{equation}
where the indices $i$ refer to a discretization in $s$. I chose the
trapezoidal rule because it is an A-stable method and second
order. The off-rate can change \emph{enormously} along a single
characteristic (e.g. take $z_0 = 0$, then at $s = 0$, $\beta = 1$ and
at $s = 1$, $\beta = \exp(\delta \pi) \approx 10^4$), and so the
obvious solution is to use an A-stable method so that you don't have
to choose punishingly short step sizes to ensure stability. In order
to integrate $m$ to find $M$ and $T$, I just use the trapezoidal rule
for integration.

\subsubsection{Redefining $\kappa$}
\label{sec:redefining-kappa}

In the work above, and in previous work, the on rate function was
defined to be $\alpha(L) = \kappa \exp(-\eta L^2/2)$. Clearly this is
a Gaussian function with $\kappa$ controlling the height and $\eta$
controlling the spread. While we may expect the biological
interpretation of these parameters to be that $\eta$ is a
nondimensional bond stiffness and $\kappa$ is a nondimensional rate
constant, this isn't quite correct. Imagine the overall bond formation
rate between a surface patch on the platelet $d\theta$ and a patch on
the wall $dz$. This overall formation rate depends not only on
$\kappa$, but on $\eta$ as well.

I think it makes more sense for the overall bond formation rate to
depend only on $\kappa$ (i.e. have $\kappa$ control the integral of
$\alpha$, not the maximum bond formation rate). I'm not entirely sure
this is obviously the better choice for the model as a whole, but it
seems better to me at least for the parameter exploration
phase. Therefore, in the work below, I redefine $\kappa$ so that
$\kappa_\tn{old} = \kappa_\tn{new} \sqrt{\eta/(2 \pi)}$, and then
\begin{equation}
  \alpha(L) = \kappa_\tn{new} \sqrt{\frac{\eta}{2\pi}}
  \exp\left(-\frac{\eta}{2}L^2\right).  
\end{equation}

\subsection{Results from the model without saturation}
\label{sec:results-from-model}

\begin{figure}
  \centering
  \begin{subfigure}{.48\linewidth}
    \begin{tikzpicture}
      \begin{axis}[
        xlabel={$\omega$},
        ylabel={$\tau$},
        xmin=0, ymin=0,
        grid=major,
        scaled ticks=false,
        ]
        \addplot[no markers, smooth, blue, very thick] table[x
        index=0, y index=2] {chars_om_M_T.dat};
      \end{axis}
    \end{tikzpicture}
    \caption{Torque versus rolling velocity}
    \label{fig:om-vs-T-M}
  \end{subfigure}
  \hfill
  \begin{subfigure}{.49\linewidth}
    \begin{tikzpicture}
      \begin{axis}[
        xlabel={$\omega_f$},
        ylabel={$\omega$},
        xmin=0, ymin=0,
        grid=major,
        scaled ticks=false, 
        ]
        \addplot[no markers, smooth, blue, very thick] table[x
        index=0, y index=1] {chars_omf_om_M.dat};
        \addplot[no markers, black, very thick, dashed, domain=0:120]
        {x};
      \end{axis}
    \end{tikzpicture}
    \caption{Applied rolling velocity $\omega_f$ versus rolling
      velocity $\omega$.}
    \label{fig:omf-om-M}
  \end{subfigure}
  \caption{Results for a range of rolling velocities $\omega$ in the
    simplified model without saturation.}
  \label{fig:char-results}
\end{figure}

\begin{figure}
  \centering
  \begin{subfigure}{.48\linewidth}
    \begin{tikzpicture}
      \begin{axis}[
        xlabel={$\omega$},
        ylabel={$M$},
        ymin=0, xmin=0,
        grid=major,
        scaled ticks=false,
        ]
        \addplot[no markers, smooth, blue, very thick] table[x
        index=0, y index=1] {chars_om_M_T.dat};
      \end{axis}
    \end{tikzpicture}
    \caption{Total bond number versus rolling velocity}
    \label{fig:om-vs-T-M}
  \end{subfigure}
  \hfill
  \begin{subfigure}{.49\linewidth}
    \begin{tikzpicture}
      \begin{axis}[
        xlabel={$\omega_f$},
        ylabel={$\omega$},
        xmin=0, ymin=0,
        grid=major,
        scaled ticks=false, 
        ]
        \addplot[no markers, smooth, blue, very thick] table[x
        index=0, y index=2] {chars_omf_om_M.dat};
      \end{axis}
    \end{tikzpicture}
    \caption{Applied rolling velocity $\omega_f$ versus total bond
      number $M$.}
    \label{fig:omf-om-M}
  \end{subfigure}
  \caption{Results for a range of rolling velocities $\omega$ in the
    simplified model without saturation.}
  \label{fig:char-results1}
\end{figure}

\subsection{Parameter sweeps}
\label{sec:parameter-sweeps}

We are interested in how parameters in the bond formation and breaking
rate functions affect the total bond number $M$. There are three
parameters that control these rate functions, $\kappa$ and $\eta$
control bond formation, and $\delta$ controls bond breaking. Based on
values in the literature, I previously estimated these parameters at
$\eta \approx 2 \times 10^4$ and $\delta \approx 17$, with $\kappa$
unknown. Previously I had experimented with $\kappa \approx $ 0.1--10
which seemed to give reasonable rolling behavior.

For a coarse parameter sweep, I used a log-scaled sweep in the
following parameter ranges $\kappa \in [10^{-3}, 10^{2}]$, $\eta \in
[10^2, 10^6]$, and $\delta \in [10^{-1}, 10^1]$. The scaling factor $N_T$
for nondimensional bond density is approximately $10^3$, so we are
looking for areas of parameter space where the nondimensional bond
number is approximately $10^{-3}$--$10^{-1}$. After calculating the total
bond number $M$ for each parameter and a range of $\omega$ values, I
tested whether each $M$ value was within the desired range. Then I
calculated slices of the parameters by averaging over 2 of the
parameters, and I plotted the averages as a function of $\omega$ and
each of the parameters.

\begin{figure}
  \centering
  \begin{subfigure}{0.48\linewidth}
    \includegraphics[width=\textwidth]{kappa_slice}
  \end{subfigure}
  \hfill
  \begin{subfigure}{0.48\linewidth}
    \includegraphics[width=\textwidth]{eta_slice}
  \end{subfigure}
  \begin{subfigure}{0.48\linewidth}
    \includegraphics[width=\textwidth]{delta_slice}
  \end{subfigure}
  \caption{Slices of the parameter sweeps}
  \label{fig:coarse_slices}
\end{figure}

The parameter that seems to matter the most is $\kappa$. So for a
refined sweep, I fixed $\delta = 10$, $\eta = 10^4$, and did a
linearly spaced sweep through $\kappa \in [0, 50]$.

\begin{figure}
  \centering
  \includegraphics[width=.75\textwidth]{refined_sweep}
  \caption{Refined sweep in $\kappa$}
  \label{fig:refined-sweep}
\end{figure}


% (It may be worth trying SciPy's higher-order BDF implementation, as it
% may be stable along the entire negative real axis).

% I think we should use a different approach to reduce the 2D problem to
% 1 dimension. One possibility that I haven't yet explored is to
% approximate the bond formation term $\alpha$ as a
% $\delta$-function. This way, every bond is uniquely determined by its
% attachment point on the platelet.

\bibliographystyle{plain}
\bibliography{/Users/andrewwork/Documents/grad-school/thesis/library}

\end{document}
