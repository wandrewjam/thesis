\documentclass{article}

\newcommand{\ep}{\rule{.06in}{.1in}}
\textheight 9.5in

\usepackage{amssymb}
\usepackage{amsmath}
\usepackage{graphicx}
\usepackage{subcaption}
\usepackage{pgfplots}
\usepackage{textgreek}

\newcommand{\tn}{\textnormal}

\pagestyle{empty} 
\oddsidemargin -0.25in
\evensidemargin -0.25in 
\topmargin -0.75in 
\parindent 0pt
\parskip 12pt
\textwidth 7in
%\font\cj=msbm10 at 12pt
%\topmargin -.5in 
\begin{document}
\pagestyle{empty}

%%% This is model-defs.tex
%%%
%%% Define symbols for model parameters here so that it is
%%% straightforward to change notation if necessary.

%% Coordinates
\newcommand{\wallDist}{x}
\newcommand{\ndWallDist}{z}
\newcommand{\recAngle}{\theta}
\newcommand{\dTime}{t}
\newcommand{\ndTime}{s}

%% Unknown Functions
\newcommand{\velocity}{V}
\newcommand{\rotation}{\Omega}
\newcommand{\ndVelocity}{v}
\newcommand{\ndRotation}{\omega}
\newcommand{\bondDensity}{n}
\newcommand{\ndBondDensity}{m}

%% Model Parameters (Dimensional)
\newcommand{\radius}{R}
\newcommand{\separation}{d}
\newcommand{\height}{h}
\newcommand{\length}{L}
\newcommand{\domLength}{A}
\newcommand{\shear}{\gamma}
\newcommand{\stiffness}{k_f}
\newcommand{\boltzmann}{k_B}
\newcommand{\temp}{T}
\newcommand{\onRate}{k_\tn{on}}
\newcommand{\offRate}{k_\tn{off}}
\newcommand{\onConst}{k_\tn{on}^0}
\newcommand{\offConst}{k_\tn{off}^0}
\newcommand{\refForce}{f_0}
\newcommand{\receptorDensity}{N_T}
\newcommand{\receptorNumber}{N_R}
\newcommand{\appliedVel}{V_f}
\newcommand{\appliedRot}{\Omega_f}
\newcommand{\velFriction}{\xi_V}
\newcommand{\rotFriction}{\xi_\Omega}
\newcommand{\compliance}{\Gamma}
\newcommand{\width}{w}
\newcommand{\viscosity}{\mu}

%% Force and Torque Functions
\newcommand{\horzForce}{f_h}
\newcommand{\torque}{\tau_s}
\newcommand{\horzTotalForce}{F_h}
\newcommand{\totalTorque}{\tau}

%% Force, Velocity, and Resistance Tensors
\newcommand{\forceVec}{\mathbf{F}}
\newcommand{\velVec}{\mathbf{U}}
\newcommand{\resMatrix}{\underline{\underline{R}}}

%% Model Parameters (Nondimensional)
\newcommand{\ndSeparation}{d'}
\newcommand{\ndLength}{\ell}
\newcommand{\ndAppliedRot}{\omega_f}
\newcommand{\ndAppliedVel}{v_f}
\newcommand{\ndOnConst}{\kappa}
\newcommand{\newOnConst}{\kappa_\textnormal{new}}
\newcommand{\onForceScale}{\eta}
\newcommand{\offForceScale}{\delta}
\newcommand{\ndVelFriction}{\eta_v}
\newcommand{\ndRotFriction}{\eta_\omega}

%% Nondimensional Force and Torque Functions
\newcommand{\ndHorzForce}{f_h'}
\newcommand{\ndTorque}{\tau_s'}
\newcommand{\ndHorzTotalForce}{F_h'}
\newcommand{\ndTotalTorque}{\tau'}

%% Shorthands for Chemical Species
\newcommand{\ITA}[1]{\textalpha\textsubscript{#1}}
\newcommand{\ITB}[1]{\textbeta\textsubscript{#1}}
\newcommand{\Ca}{$\tn{Ca}^{++}$}

%% Reynolds Number
\newcommand{\Reynolds}{\mathrm{Re}}

%% Bin Midpoint
\newcommand{\binMidpoint}[1]{\theta^*_{#1}}



\begin{center}
{\Large Meeting Notes for February 21st, 2019}
\end{center}

This document contains notes on estimating $\onConst$ from our meeting
on February 14th, and expands on ways to estimate and verify
$\onConst$ and $\onForceScale$.

\section{Estimating $\onConst$}
\label{sec:estimating-onconst}

In Fitzgibbon et. al. \cite{Fitzgibbon2014}, they estimate bond
formation by using a constant $k_\tn{on}^F$ that is a per-area rate
of platelet binding to a surface. Therefore the rate of bond
formation between the platelet and the surface is given by
$k_\tn{on}^F A$ where $A$ is the area of the platelet close enough to
the surface to bind.

\begin{figure}
  \centering
  \begin{subfigure}{0.48\textwidth}
    \includegraphics[width=\textwidth]{fitz-binding}
    \caption{Model of binding in \cite{Fitzgibbon2014}}
    \label{fig:fitz-binding}
  \end{subfigure}
  \hfill
  \begin{subfigure}{0.48\textwidth}
    \includegraphics[width=\textwidth]{wats-binding}
    \caption{Our model of binding}
    \label{fig:wats-binding}
  \end{subfigure}
  \caption{Fitzgibbon vs. Watson models of binding}
  \label{fig:binding}
\end{figure}

In our model we have a distance-dependent bond formation rate that
gives the rate of bond formation between a single receptor and a
single point on the vessel wall $\onConst \exp \left(-\eta/2 \ell(z,
  \theta)^2 \right)$. Figure \ref{fig:binding} sketches the two
different approaches to modeling binding. In order to use the
Fitzgibbon estimate, we have to find the total rate of bond formation
on our model between the platelet surface and the vessel wall and
equate that to $k_\tn{on}^F A$. Therefore we need to enforce
\begin{equation}
  \label{eq:fm-rate}
  k_\tn{on}^F A = \onConst \sum_j \left( \int_{-L}^{L} \exp
    \left(-\frac{\eta}{2} \ell(z, \theta_j)^2\right) dz \right)
  n_j^\tn{avail}.
\end{equation}

Now $n_j^\tn{avail}$ changes throughout the simulation, but it must
always be less than $n^\tn{max}$ where $n^\tn{max}$ is the total
number of receptors a bin (the receptors are distributed uniformly and
so $n^\tn{max}$ is the same for every bin). Furthermore platelet
receptors do not saturate in the Fitzgibbon model, which supports the
choice to use $n^\tn{max}$ instead of $n_j^\tn{avail}$. Therefore with
a given value for $\eta$, we can find $\onConst$ by rearranging
equation \eqref{eq:fm-rate}:

\begin{equation}
  \label{eq:kon-calc}
  \onConst = k_\tn{on}^F A \bigg/ \left( n^\tn{max} \sum_j \int_{-L}^{L}
    \exp \left(-\frac{\eta}{2} \ell(z, \theta_j)^2 \right) dz \right).
\end{equation}

The location of the $\theta_j$s also changes in each time step of the
simulation, however because we only allow bonds to form from receptors
on the lower semicircle they must always satisfy $-\pi/2 \le \theta_j
\le \pi/2$ for all $j$. The $\theta_j$s are also uniformly spaced, so
let $\Delta \theta \equiv \theta_j - \theta_{j-1}$ (equivalent to the
bin width) and rewrite equation \eqref{eq:kon-calc}:

\begin{equation}
  \label{eq:kon-calc-reimann}
  \onConst = k_\tn{on}^F A \bigg/ \left(\frac{n^\tn{max}}{\Delta
      \theta} \sum_j \Delta \theta \int_{-L}^{L}
    \exp\left(-\frac{\eta}{2} \ell(z, \theta_j)^2 \right) dz \right).
\end{equation}

Note that
\begin{enumerate}
\item $n^\tn{max} \equiv N_T \Delta \theta$ and,
\item the sum over $j$ is a Riemann sum over $[-\pi/2, \pi/2]$
  (approximately). 
\end{enumerate}
With these observations, equation \eqref{eq:kon-calc-reimann} becomes
\begin{equation}
  \label{eq:kon-calc-int}
  \onConst \approx k_\tn{on}^F A \bigg/ \left(N_T
    \int_{-\pi/2}^{\pi/2} \int_{-L}^{L} \exp \left(-\frac{\eta}{2}
      \ell(z, \theta)^2 \right) dz d\theta \right).
\end{equation}

\begin{figure}
  \centering
  \begin{tikzpicture}
    \begin{semilogxaxis}[
      xlabel={$d$},
      minor y tick num=5,
      grid=major,
    ]
      \addplot[blue] table[col sep=comma] {integral.dat};
    \end{semilogxaxis}
  \end{tikzpicture}
  \caption{Values of the integral in equation \eqref{eq:kon-calc-int}
    for a range of $d$ values.}
  \label{fig:int-vals}
\end{figure}

This approximation is now independent of the mesh size on the surface
of the platelet. For $\eta = 2.34 \times 10^4$ (estimated in the
parameters write-up), $N_T = 5,000$, $d = 0.01$, and $k_\tn{on}^F A =
0.4 \tn{s}^{-1}$ from \cite{Fitzgibbon2014}, we get $\onConst \approx .4/3.76
\approx 0.106 \tn{s}^{-1}$. It is worth noting that I also computed
$\onConst$ using equation \eqref{eq:kon-calc} with $N=128$
$\theta$-bins, and the result was accurate to within 3 decimal places
of the $\onConst = 0.106 \tn{s}^{-1}$ estimate from equation
\eqref{eq:kon-calc-int}. This estimate for $\onConst$ is similar to
what we estimated last week, which ultimately resulted in only a tiny
fraction of the platelets forming bonds with the surface at all.

I think this observation is consistent with our understanding of the
Fitzgibbon model \cite{Fitzgibbon2014}. They report a \emph{total}
bond formation rate of 0.4 s$^{-1}$ between a platelet and the
surface, and a bond breaking rate of 5 s$^{-1}$. This means that it
takes 2.5 s on average for one bond to form, and that bond lasts an
average of 0.2 s before breaking. With these parameters, I don't see
how they can get platelets translocating along the surface for several
seconds. I would think they should see most platelets releasing within
a fraction of a second.

One possibility is that 0.4 bonds/s is not a good estimate of the
formation rate of bonds on a platelet which already has a bond with
the surface. It may be possible that the fluid flattens the platelet
against the wall within the lifespan of the existing bond, exposing a
much greater area of the platelet to binding than the average over
platelets which are moving freely in the F-L layer. If this is the
case, then 0.4 s$^{-1}$ is too low of an estimate for the bond
formation rate on a platelet which is already bound to the wall.

\section{Comparing binding behavior in our model with other rolling models}
\label{sec:comparing-binding}

In order to confirm that our model and parameter choices are
generating realistic rolling behavior, I think it will be useful to
catalog some binding characteristics from previous rolling models and
compare our results with those already published.

Here is a list of quantities I think will be useful to track in our
binding models and are often reported in published rolling papers:
\begin{enumerate}
\item Total number of bonds between the cell and surface
\item Average length of bonds
\item Average force per bond
\item Average bond lifetimes
\item Average distance from cell to surface
\item Values of $k_\tn{on}(L)$ for a range of $L$ values.
\end{enumerate}

It is probably also important to note which specific bonds they are
modeling, the shear rates used, and the size of the cell used.

\subsection{Number of bonds}
\label{sec:number-bonds}

\begin{figure}
  \centering
  \includegraphics[width=.6\textwidth]{hammer-fig3}
  \caption{Figure 3 from Hammer \& Apte, 1992 \cite{Hammer1992}. Instantaneous
    translational velocities and the number of bonds are
    plotted as a function of time for a sample run. The number of
    bonds are shown with a dotted line.}
  \label{fig:hammer-apte}
\end{figure}

In Hammer \& Apte, 1992 (\cite{Hammer1992}), they see about 10--50
bonds throughout the course of a simulation (Figure
\ref{fig:hammer-apte}). In their simulations they use shear rates in
the range 50--400 s$^{-1}$, and their cell radius is 4.5
{\textmugreek}m.

In Wang et. al. \cite{Wang2013} they model a platelet translocating
along vWF at a shear rate of 1250 s$^{-1}$. In their simulations a
single platelet had between 0 and 10 bonds at a time, and this handful
of bonds was sufficient to bring the platelet to a halt (see Figure
\ref{fig:wang}).

In T\"{o}zeren and Ley \cite{Tozeren1992}, they find that leukocyte
rolling is mediated by between 12 and 21 bonds in their model.
\begin{figure}
  \centering
  \includegraphics[width=.5\textwidth]{wang-fig2}
  \caption{Figure 2 from Wang, Mody, \& King, 2013.}
  \label{fig:wang}
\end{figure}

In our model with toy parameter values we've been using, $\sim$11.7\% of
available receptors are bound at steady state. For 1,000s or 10,000s
of receptors, this predicts a higher bond count than other models by
an order of magnitude or two. For example, Figure \ref{fig:toy-number}
shows hundreds of bonds have formed between the platelet and wall, but
only slow the platelet down by $\sim 20\%$.

\begin{figure}
  \centering
  \includegraphics[width=.9\textwidth]{{M512_N512_tsteps16000_initfree_trials1000_bmax10_cflux1_vf22_omf20_kappa1_eta0.1_d0.1_delta3_on1_off1_sat1_xiv0.01_xiom0.01_L2.5_T1_sbh0}.png}
  \caption{Translational and angular velocities, and number of bonds
    for $\kappa = 1$, $\eta = 0.1$, $N_T = 1630$ receptors/radian}
  \label{fig:toy-number}
\end{figure}

In order to see the effects of different $\onConst$ values on the
number of bonds and rolling dynamics, I ran simulations for 3
different values of $\ndOnConst \equiv \onConst/\offConst$:
$\ndOnConst = 0.1, 1, 10$. The other parameter values are estimated
from the literature as outlined in the parameter estimation document:
$\onForceScale = 2.3 \times 10^4$, $\ndSeparation = 0.01$,
$\offForceScale = 1$, and $\shear = 200 \tn{ s}^{-1} \implies
\shear/\offConst = 40$. Results for these three cases are shown in
Figures \ref{fig:small-kap}--~\ref{fig:large-kap}.

\begin{figure}
  \centering
  \includegraphics[width=.9\textwidth]{{M128_N128_tsteps30720_initfree_trials16_bmax100_cflux0_vf40.4_omf20_kappa0.1_eta23000_d0.01_delta1_on1_off1_sat1_xiv1e-06_xiom1e-06_L2.5_T3_sbh0}.png}
  \caption{Results of simulations with $\kappa = 0.1$}
  \label{fig:small-kap}
\end{figure}

\begin{figure}
  \centering
  \includegraphics[width=.9\textwidth]{{M128_N128_tsteps30720_initfree_trials16_bmax100_cflux0_vf40.4_omf20_kappa1_eta23000_d0.01_delta1_on1_off1_sat1_xiv1e-06_xiom1e-06_L2.5_T3_sbh0}.png}
  \caption{Results of simulations with $\kappa = 1$}
  \label{fig:medium-kap}
\end{figure}

\begin{figure}
  \centering
  \includegraphics[width=.9\textwidth]{{M128_N128_tsteps30720_initfree_trials16_bmax100_cflux0_vf40.4_omf20_kappa10_eta23000_d0.01_delta1_on1_off1_sat1_xiv1e-06_xiom1e-06_L2.5_T3_sbh0}.png}
  \caption{Results of simulations with $\kappa = 10$}
  \label{fig:large-kap}
\end{figure}

\subsection{Length, force, and bond lifetime}
\label{sec:length-force-lifetime}

\subsection{Distance between surfaces}
\label{sec:sep-distance}

\subsection{On rate as a function of $L$}
\label{sec:on-rate-fn}

In Hammer \& Apte \cite{Hammer1992}, they use the Dembo models of
binding and unbinding with parameter values $\sigma_\tn{ts} \approx
1 \frac{\tn{pN}}{\tn{nm}}$, $\sigma \approx 1.5
\frac{\tn{pN}}{\tn{nm}}$, $\lambda = 20 \tn{nm}$:
\begin{equation}
  \label{eq:bell-on}
  k_\tn{on}^\tn{Bell}(L) = 
\end{equation}
In the current
version of our rolling model, we are using $k_\tn{on} = \onConst \exp
\left(-\frac{\stiffness}{2\boltzmann\temp} \length^2\right)$. 

\section{A catalog of model parameters in other cell rolling models}
\label{sec:catal-model-param}

\subsection{Hammer \& Apte, 1992 \cite{Hammer1992}}
\label{sec:hammer-apte}

\begin{table}[h]
  \centering
  \begin{tabular}{ll} \hline
    Parameter & Value \\ \hline
    $\sigma$ & 1--2 dyn/cm \\
    $\sigma_\tn{s}$ & 0--2 dyn/cm \\
    $k_f^0$ & $10^{-10}$ cm$^2$/s $\implies$ 1--10$^4$ 1/s \\
    $\lambda$ & 20 nm \\
    $k_r^0$ & $10^{-5}--1$ 1/s \\
    $\gamma$ & 100 1/s \\ \hline
  \end{tabular}
  \caption{Binding parameters for selectin--LECAM-1 binding}
  \label{tab:hammer-apte-pars}
\end{table}

They use the Dembo model of binding and unbinding to get
length-dependent on-rates.

I wasn't able to find information on bond lifetimes or height of the
cell.

\subsection{Bhatia et. al., 2003 \cite{Bhatia2003}}
\label{sec:bhatia}

\begin{table}[h]
  \centering
  \begin{tabular}{lll}
    &&
  \end{tabular}
  \caption{Binding parameters for selectin--sLe\textsuperscript{x} and
    integrin--ICAM-1}
  \label{tab:bhatia-pars}
\end{table}
%% What about other rolling models? (Numbers of bonds)
%% There are a few bonds present in the PAD multiscale model from Wang
%% et. al.
%% Take a look at Dan Hammer's review (2014)
%% What about shear rates? Distances from the wall?

%% I want to compare our estimate of the bond formation function to
%% those used in other rolling models.

%% How many bonds are there in the Fitzgibbon paper?
%% How many receptors are available to bind in the Fitzgibbon model?

%% I want to look at average bond lengths
%% Also, I should plot numbers of bonds, instead of whatever weird
%% nondimensional quantity I'm using.

\bibliographystyle{plain}
\bibliography{../../library}

\end{document}
