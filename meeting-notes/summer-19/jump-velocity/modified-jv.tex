\documentclass{article}

\newcommand{\ep}{\rule{.06in}{.1in}}
\textheight 9.5in

\usepackage{amssymb}
\usepackage{amsmath}
\usepackage{amsthm}
\usepackage{graphicx, subcaption, algorithmic}
\graphicspath{{/Users/andrewwork/thesis/jump-velocity/plots/}}

\usepackage{tikz, pgfplots, chemfig}
\usepgfplotslibrary{colorbrewer, statistics}
\pgfplotsset{
  exact axis/.style={grid=major, minor tick num=4, xlabel=$v^*$,
    legend entries={PDF, CDF},},
  every axis plot post/.append style={thick},
  table/search
  path={/Users/andrewwork/thesis/jump-velocity/dat-files},
  colormap/YlGnBu,
  cycle list/Set1-5,
  legend style={legend cell align=left,},
}
\usepgfplotslibrary{external}
\tikzexternalize

\renewcommand{\arraystretch}{1.2}
\pagestyle{empty} 
\oddsidemargin -0.25in
\evensidemargin -0.25in 
\topmargin -0.75in 
\parindent 0pt
\parskip 12pt
\textwidth 7in
%\font\cj=msbm10 at 12pt

\newcommand{\tn}{\textnormal}
\newcommand{\stiff}{\frac{k_f}{\gamma}}
\newcommand{\dd}{d}
\newcommand{\Der}[2]{\frac{\dd #1}{\dd #2}}
\newcommand{\Pder}[2]{\frac{\partial #1}{\partial #2}}
\newcommand{\Integral}[4]{\int_{#3}^{#4} {#1} \dd #2}
\DeclareMathOperator{\Exp}{Exp}

% Text width is 7 inches

\def\R{\mathbb{R}}
\def\N{\mathbb{N}}
\def\C{\mathbb{C}}
\def\Z{\mathbb{Z}}
\def\Q{\mathbb{Q}}
\def\H{\mathbb{H}}
\def\B{\mathcal{B}} 
%\topmargin -.5in 

\setcounter{secnumdepth}{2}
\begin{document}
\pagestyle{plain}

\begin{center}
  {\Large Notes on a Modification to the Jump-Velocity Rolling Model
    (\today)}
\end{center}

\section{A modification to the single bond model of rolling}
\label{sec:modif-single-bond}

Experimental step size data does not fit a simple one-step model of
binding and unbinding. We want to test the hypothesis that the
distribution of step sizes can be explained by a mixture of two types
of steps: (1) steps where the platelet comes completely unbound from
the surface, and (2) steps where a platelet is multiply bound to the
vessel wall, and the rearmost load-bearing bond breaks, causing the
platelet to lurch forward.

We want to test this with a model that approximates this behavior,
without the expense of tracking an arbitrary number of bonds that can
bind in arbitrary locations along the wall. Therefore, we will use a
model that allows bonds to form in two positions: on the front of the
platelet and on the back of the platelet (see Figure
\ref{fig:four-states}).

\begin{figure}
  \centering
  \includegraphics[width=.7\textwidth]{double-binding-model.png}
  \caption{Four states in the modified jump-velocity model}
  \label{fig:four-states}
\end{figure}

In the unbound state $S_U$, the platelet translates forward at the
free-flowing velocity $V^*$. A bond can form on the back of the
platelet at a constant binding rate $k_\tn{on}$ (or rather, a bond
forms \emph{somewhere} between the platelet and wall, but the platelet
keeps translating forward until the force applied by the bond balances
the drag force on the platelet). Let's call this state $S_b$. This
bond can then break at a rate $k_\tn{off}$, returning the platelet to
the unbound state.

An $S_b$ platelet can form a second bond with the surface on the front
(downstream) end of the platelet at rate $k_\tn{on, f}$, and we'll
call this the $S_{bf}$ state. Then, either the back or the front bonds
of a $S_{bf}$ platelet can break. The back bond breaks at a rate
$k_\tn{off, b}$ and the front bond breaks at a rate $k_\tn{off,
  f}$. If the front bond breaks, the platelet returns to the $S_b$
state, and if the back bond breaks, the platelet is temporarily left
with a single front bond. However, because of the flow, the platelet
is pushed downstream until the front bond becomes a back bond, and the
bond force balances with the drag force on the platelet. We'll call
this state the transition state---$S_T$.

Now we have to make a choice to model how long platelets remain in the
$S_T$ state. One possible choice is to say that every platelet stays
in the $S_T$ state for the exact same length of time and travels the
same distance. While one could argue this agrees geometrically with
the series of pictures drawn in Figure \ref{fig:four-states}, this
isn't a realistic representation of small step sizes in the more
complicated case that bonds can form continuously along the vessel
wall. If we instead say that platelets transition out of state $S_T$
with some rate $\kappa$, that is equivalent (I think) to assuming
that the small step sizes are distributed exponentially with mean step
length $V^*/\kappa$.

How can we choose a reasonable transition rate $\kappa$ out of the
$S_T$ state? One possibility is to define a characteristic step length
$\ell$, and then assert that some fraction $q$ of the small steps must
be smaller than the characteristic step length. To choose a $\kappa$
based on some given $\ell$ and $q$, we can use the CDF of small step
times. For a constant transition rate $\kappa$, the small step times
are distributed as an exponential distribution with mean
$1/\kappa$. Therefore, the CDF of step times is
$F(t) = 1 - \exp(-\kappa t)$. Then $q = 1 - \exp(-\kappa \ell/V^*)$
enforces the condition that the fraction of steps which are less than
the characteristic step size is $q$. Solving this condition for
$\kappa$ gives us
\begin{equation}
  \label{eq:kappa-defn}
  \kappa = -\frac{\ln(1 - q)}{\ell/V^*}.
\end{equation}

\subsection{Fokker-Planck equation}
\label{sec:fokk-planck-equat}

The resulting Fokker-Planck equation of this process is an
advection-reaction system similar to that of the ordinary jump
velocity process. One major difference is that there are 2 states
which translate in the flow, not just one.

Clearly the Fokker-Planck equation will depend on the assumption made
about the distribution of transition times (or small step sizes), but
for the case where transition times are distributed exponentially, the
Fokker-Planck equation is a simple linear advection-reaction PDE:
\begin{align}
  \Pder{p_U}{t} &= -V^* \Pder{p_U}{x} - k_\tn{on} p_U + k_\tn{off} p_b
  \\
  \Pder{p_b}{t} &= k_\tn{on} p_U - (k_\tn{off} + k_\tn{on, f}) p_b +
                  k_\tn{off, f} p_{bf} + \kappa p_T \\
  \Pder{p_{bf}}{t} &= k_\tn{on, f} p_b - (k_\tn{off, f} + k_\tn{off,
                     b}) p_{bf} \\
  \Pder{p_T}{t} &= -V^* \Pder{p_T}{x} + k_\tn{off} p_{bf} - \kappa p_T.
\end{align}



% \bibliographystyle{plain}
% \bibliography{/Users/andrewwork/Documents/grad-school/thesis/library}

\end{document}




