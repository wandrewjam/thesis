\documentclass{article}

\newcommand{\ep}{\rule{.06in}{.1in}}
\textheight 9.5in

\usepackage{amssymb}
\usepackage{amsmath}
\usepackage{amsthm}
\usepackage{graphicx, subcaption}
\graphicspath{{/Users/andrewwork/thesis/jump-velocity/plots/}}

% \usepackage{tikz, pgfplots}
% \usepgfplotslibrary{colorbrewer}
% \pgfplotsset{
%   table/search path = {/Users/andrewwork/thesis/first-binding-model/},
%   table/col sep = comma,
%   colormap/YlGnBu,
% }

\pagestyle{empty} 
\oddsidemargin -0.25in
\evensidemargin -0.25in 
\topmargin -0.75in 
\parindent 0pt
\parskip 12pt
\textwidth 7in
%\font\cj=msbm10 at 12pt

\newcommand{\tn}{\textnormal}
\newcommand{\stiff}{\frac{k_f}{\gamma}}
\newcommand{\dd}{d}
\newcommand{\Der}[2]{\frac{\dd #1}{\dd #2}}
\newcommand{\Pder}[2]{\frac{\partial #1}{\partial #2}}
\newcommand{\Integral}[4]{\int_{#3}^{#4} {#1} \dd #2}

% Text width is 7 inches

\def\R{\mathbb{R}}
\def\N{\mathbb{N}}
\def\C{\mathbb{C}}
\def\Z{\mathbb{Z}}
\def\Q{\mathbb{Q}}
\def\H{\mathbb{H}}
\def\B{\mathcal{B}} 
%\topmargin -.5in 

\setcounter{secnumdepth}{2}
\begin{document}
\pagestyle{plain}

\begin{center}
  {\Large Notes on a Jump-Velocity Model of Platelet Rolling (6/27/19)}
\end{center}

\section{Description of the Jump-Velocity Model}
\label{sec:jump-vel}

Suppose we model platelets as particles in a moving fluid. Assume that
unprimed platelets can exist in two states: unbound (U), and bound to
vWF (V) (Figure \ref{fig:unprimed-states}). Platelets in the unbound
state advect at velocity $v$ in the fluid, and platelets in the
vWF-bound state are stationary. Platelets in the fluid can bind to vWF
at a constant rate $k_\tn{on}$, and platelets bound to vWF can come
unbound at constant rate $k_\tn{off}$.

\emph{Modification}: We can split the U state into platelets that have
not bound to the surface (call these U\textsuperscript{0} platelets)
and those that have bound to a surface and then unbound
(U\textsuperscript{1} platelets).

\begin{figure}[h]
  \centering
  \includegraphics[width=.3\textwidth]{unprimed-states.png}
  \caption[Possible states of unprimed platelets]{An unprimed platelet
    can exist in three distinct states: (U\textsuperscript{0})
    platelets which haven't interacted with the surface,
    (U\textsuperscript{0}) platelets which have interacted with the
    surface and are advecting in the fluid, or (V) bound to vWF on the
    surface and unmoving. Transitions between these states occur at
    constant rates $k_\tn{on}$ and $k_\tn{off}$. Platelets can only
    transition out of the U\textsuperscript{0} state.}
  \label{fig:unprimed-states}
\end{figure}

For primed platelets, assume that they can exist in 4 possible states:
unbound (U), vWF-bound (V), fibrinogen-bound (F), and vWF- and
fibrinogen-bound (VF) (Figure \ref{fig:primed-states}). The U and V
states are the same as for the unprimed platelets, however in addition
both the U and V states can bind with fibrinogen to transition to the
F and VF states. 

\begin{figure}[h]
  \centering
  \includegraphics[width=.3\textwidth]{primed-states.png}
  \caption[Possible states of primed platelets]{A primed platelet can
    exist in five states: (U) unbound from the surface and advecting
    in the fluid (further split into the two categories defined
    above), (V) bound to vWF on the surface, (F) bound to fibrinogen
    on the surface, or (VF) bound to both vWF and fibrinogen. In all
    three bound states, the platelet is immobilized on the surface.}
  \label{fig:primed-states}
\end{figure}

These models describe a jump-velocity process, where a particle is
transitioning randomly between discrete states, which each move with a
different deterministic motion. The Fokker-Planck equation for these
processes are given by the following system of linear advection
equations:
\begin{equation}
  \label{eq:fp-system}
  \Pder{}{t}
  \underbrace{
    \begin{pmatrix}
      p_{\tn{U}^0} \\ p_{\tn{U}^1} \\ p_\tn{V} \\ p_\tn{F} \\ p_\tn{VF}
    \end{pmatrix}}_{\equiv \mathbf{p}}
  =
  -\Pder{}{x}
  \begin{pmatrix}
    v p_{\tn{U}^0} \\ v p_{\tn{U}^0} \\ 0 \\ 0 \\ 0
  \end{pmatrix}
  +
  \underbrace{
    \begin{pmatrix}
      -(k_\tn{on} + k_\tn{on}^F) & 0 & 0 & 0 & 0 \\
      0 & -(k_\tn{on} + k_\tn{on}^F) & k_\tn{off} & k_\tn{off}^F & 0 \\
      k_\tn{on} & k_\tn{on} & -(k_\tn{off} + k_\tn{on}^F) & 0 & k_\tn{off}^F \\
      k_\tn{on}^F & k_\tn{on}^F & 0 & -(k_\tn{on} + k_\tn{off}^F) & k_\tn{off} \\
      0 & 0 & k_\tn{on}^F & k_\tn{on} & -(k_\tn{off} + k_\tn{off}^F)
  \end{pmatrix}}_{\equiv A}
  \begin{pmatrix}
    p_{\tn{U}^0} \\ p_{\tn{U}^1} \\ p_\tn{V} \\ p_\tn{F} \\ p_\tn{VF}
  \end{pmatrix}
\end{equation}
where $p_i = p_i(x, t \mid x_0, j, 0)$ is the probability the platelet
is in state $i$ and position $x$ at time $t$ given it was previously
in position $z$ and state $j$ at time $0$. If the platelets are
unprimed, then the only difference is $k_\tn{on}^F = k_\tn{off}^F =
0$.

The goal is to find the probability density function of the average
velocity across a segment of length $L$, so take the initial condition
of the PDE system (\ref{eq:fp-system}) to be
$\mathbf{p}(x, 0) = (\delta(x), 0, 0, 0, 0)^T$. That is, all platelets
enter in never-bound state U\textsuperscript{0}. The probability
density of the time it takes a platelet to cross the interval $[0, L]$
is $v(p_{\tn{U}^0}(L, t) + p_{\tn{U}^1}(L,t))$. The average velocity
associated with a crossing time $t^*$ is just $v^* = L/t^*$, and the
probability density function of $v^*$ is given by
$f(v^*) = (L/v^*)^2 p_U(L, L/v^*)$

\subsection{Nondimensionalization}
\label{sec:nondim}

Define the nondimensional variables $s$ and $y$ so that $t = Ts$ and
$x = Xy$. Let's scale $x$ by the domain length, so $X = L$, and scale
$t$ by the velocity, so that $X/T = v \implies T = L/v$. That is, $T$
is the shortest possible crossing time of a
platelet. Finally---motivated by the adiabatic reduction in example
3.6.1 in Dr. Keener's notes---define the nondimensional parameter
$\epsilon_1 = 1/(T(k_\tn{on} + k_\tn{off}))$ and
$\epsilon_2 = 1/(T(k_\tn{on}^F + k_\tn{off}^F))$. If the sum of the
(relevant) reaction rates is much larger than $1/T$, then $\epsilon$
is a small parameter. After the nondimensionalization, equation
(\ref{eq:fp-system}) becomes
\begin{equation}
  \label{eq:nd-system}
  \Pder{\mathbf{q}}{s} = -\Pder{}{y}
  \begin{pmatrix}
    q_{\tn{U}^0} \\ q_{\tn{U}^1} \\ 0 \\ 0 \\ 0
  \end{pmatrix}
  + \left(\frac{1}{\epsilon_1} + \frac{1}{\epsilon_2}\right)
  \begin{pmatrix}
    - kb - kd & 0 & 0 & 0 & 0 \\
    0 & - kb - kd & ka & kc & 0 \\
    kb & kb & - ka - kd & 0 & kc \\
    kd & kd & 0 & - kb - kc & ka \\
    0 & 0 & kd & kb & - ka - kc
  \end{pmatrix}
  \mathbf{q},
\end{equation}
where $a = k_\tn{off}/(k_\tn{off} + k_\tn{on})$,
$b = k_\tn{on}/(k_\tn{off} + k_\tn{on})$,
$c = k_\tn{off}^F/(k_\tn{off}^F + k_\tn{on}^F)$, and
$d = k_\tn{on}^F/(k_\tn{off}^F + k_\tn{on}^F)$. Then
$ka = a/(1 + \epsilon_1/\epsilon_2)$,
$kb = b/(1 + \epsilon_1/\epsilon_2)$,
$kc = c/(1 + \epsilon_2/\epsilon_1)$, and
$kd = d/(1 + \epsilon_2/\epsilon_1)$. When
$k_\tn{on}^F = k_\tn{off}^F = 0$, this reduces to the example in
Dr. Keener's notes.

\subsection{Numerics and results for unprimed platelets}
\label{sec:res-unpr}

The system of equations for unprimed platelets reduces to
\begin{align}
  \label{eq:unprimed-1}
  \Pder{q_\tn{U}}{s} &= -\Pder{q_\tn{U}}{y} - \frac{1}{\epsilon}
                       \left(b q_\tn{U} + a q_\tn{V}\right) \\
  \label{eq:unprimed-2}
  \Pder{q_\tn{V}}{s} &= \frac{1}{\epsilon} \left(b q_\tn{U} - a
                       q_\tn{V}\right).
\end{align}

I use a first order upwind scheme to solve this system. The nodes of
the mesh are given by $y_i = ih$ for $i = 0, \hdots, N$ where
$h = 1/N$ and $s_j = jk$ for $j = 0, \hdots, M$ where $k =
s_\tn{max}/M$. In practice, I take $h = k$ so that the upwind scheme
can solve the advection part exactly.
% This results in the
% semi-discretized system
% \begin{align}
%   \Der{q_\tn{U}^i}{s} &= \frac{1}{h} \left(q_\tn{U}^{i-1} - q_\tn{U}^i
%                         \right) + \frac{1}{\epsilon} \left(-b
%                         q_\tn{U}^i + a q_\tn{V}^i \right) \quad
%                         \tn{for} \quad i = 1, \hdots, N
%   \\
%   \Der{q_\tn{V}^i}{s} &= \frac{1}{\epsilon} \left(b q_\tn{U}^i - a
%                         q_\tn{V}^i \right) \quad \tn{for} \quad i = 1,
%                         \hdots, N
% \end{align}
% which I solve with SciPy's implementation of the RK45 scheme.

The initial condition $q_{\tn{U}^0}(y, 0) = \delta(y)$ must be
approximated, so I use Peskin's approximate $\delta$-function
\cite{Peskin2002}:
\begin{equation}
  \label{eq:delta-h}
  \delta_h(y) =
  \begin{cases}
    \frac{1}{4h} \left( 1 + \cos\left(\frac{\pi y}{2h}\right) \right)
    & \tn{if} \quad \left|\frac{y}{2h}\right| \le 1 \\
    0 & \tn{otherwise.}
  \end{cases}
\end{equation}
Therefore $q_\tn{U}^i(0) = \delta_h(y_i)$ and $q_\tn{V}^i(0) =
0$. Half of $\delta_h$ is outside of the domain, so the other half
must be advected into the domain as an inflow boundary condition:
$q_\tn{U}^0(s) = \delta_h(-s)$.

Because $a + b = 1$, there are only 2 nondimensional parameters to
choose: $\epsilon$ and $a$. The parameter $\epsilon$ gives the ratio
of the minimum crossing time to a characteristic reaction time, and
$a$ gives the ratio of $k_\tn{off}$ to the sum of reaction
rates. Below are some sample results for average velocity
distributions $f(v^*) = q_U^N(1/v^*)$. 

\begin{figure}
  \centering
  \begin{subfigure}{0.48\textwidth}
    \includegraphics[width=\textwidth]{{twostate-smalleps-smalla}.png}
    \caption{$\epsilon = 0.1$, $a = 0.2$: Reactions are fast, and
      $k_\tn{on} > k_\tn{off}$.}
  \end{subfigure}
  \hfill
  \begin{subfigure}{0.48\textwidth}
    \includegraphics[width=\textwidth]{{twostate-largeeps-smalla}.png}
    \caption{$\epsilon = 1$, $a = 0.2$: Reactions and advection are
      the same order, and $k_\tn{on} > k_\tn{off}$.}
  \end{subfigure}
  \\
  \begin{subfigure}{0.48\textwidth}
    \includegraphics[width=\textwidth]{{twostate-smalleps-meda}.png}
    \caption{$\epsilon = 0.1$, $a = 0.5$: Reactions are fast, and
      $k_\tn{on} = k_\tn{off}$.}
  \end{subfigure}
  \hfill
  \begin{subfigure}{0.48\textwidth}
    \includegraphics[width=\textwidth]{{twostate-largeeps-meda}.png}
    \caption{$\epsilon = 1$, $a = 0.5$: Reactions and advection are
      the same order, and $k_\tn{on} = k_\tn{off}$.}
  \end{subfigure}
  \\
  \begin{subfigure}{0.48\textwidth}
    \includegraphics[width=\textwidth]{{twostate-smalleps-largea}.png}
    \caption{$\epsilon = 0.1$, $a = 0.8$: Reactions are fast, and
      $k_\tn{on} < k_\tn{off}$.}
  \end{subfigure}
  \hfill
  \begin{subfigure}{0.48\textwidth}
    \includegraphics[width=\textwidth]{{twostate-largeeps-largea}.png}
    \caption{$\epsilon = 1$, $a = 0.8$: Reactions and advection are
      the same order, and $k_\tn{on} < k_\tn{off}$.}
  \end{subfigure}
  \caption[Probability density functions]{Probability density
    functions for 6 different $(\epsilon, a)$ pairs}
  \label{fig:prob-dens}
\end{figure}

Observations:
\begin{itemize}
\item There is some nonzero probability mass that a platelet crosses
  the domain without ever binding. Specifically, this probability is
  $\exp(-b/\epsilon)$. Therefore the exact probability density
  function will have a $\delta$-function like spike at $v^* = 1$ that
  integrates to $\exp(-b/\epsilon)$.
\item As the off rate $a$ increases, the distribution of average
  velocities shifts to the right, as expected.
\end{itemize}

\subsection{Adiabatic reduction}
\label{sec:adb-red}

The adiabatic reduction is solved in example 3.6.1 of Dr. Keener's
stochastics notes. Starting with equations (\ref{eq:unprimed-1}) and
(\ref{eq:unprimed-2}), assume that $\epsilon \ll 1$ and define $v =
q_\tn{U} + q_\tn{V}$ to be the slow variable and $w = b q_\tn{U} - a
q_\tn{V}$ to be the fast variable. Then $v$ evolves as an
advection-diffusion equation:
\begin{equation}
  \label{eq:v}
  \Pder{v}{s} = -a \Pder{v}{y} + \epsilon a b \frac{\partial^2
    v}{\partial y^2}
\end{equation}
and $w$ satisfies the equation
\begin{equation}
  \label{eq:w}
  w = -\epsilon a b \Pder{v}{y}.
\end{equation}

With the initial condition $v(y, 0) = \delta(y)$, equation
(\ref{eq:v}) can be solved analytically:
\begin{equation}
  \label{eq:v-soln}
  v(y, t) = \frac{1}{\sqrt{4 \pi \epsilon a b t}} \exp \left[ \frac{-(y
      - at)^2}{4 \epsilon a b t} \right],
\end{equation}
and then using equation (\ref{eq:w}) to find $w$,
\begin{equation}
  \label{eq:w-soln}
  w(y, t) = \frac{y - at}{4t\sqrt{\pi \epsilon a b t}} \exp \left[ \frac{-(y
      - at)^2}{4 \epsilon a b t} \right].
\end{equation}

Then reversing the change of variables, $q_\tn{U} = av + w$ and
$q_\tn{V} = bv - w$, resulting in the following solutions for
$q_\tn{U}$ and $q_\tn{V}$:
\begin{align}
  \label{eq:qu-soln}
  q_\tn{U} &= \frac{1}{\sqrt{4 \pi \epsilon a b t}} \left(a + \frac{y
             - at}{2t} \right) \exp \left[ \frac{-(y - at)^2}{4
             \epsilon a b t} \right], \\
  \label{eq:qv-soln}
  q_\tn{V} &= \frac{1}{\sqrt{4 \pi \epsilon a b t}} \left(b - \frac{y
             - at}{2t} \right) \exp \left[ \frac{-(y - at)^2}{4
             \epsilon a b t} \right].
\end{align}

The probability density function of the average velocity is then
\begin{equation}
  \label{eq:vel-dens}
  f(v^*) = (v^*)^{-2} q_\tn{U}(1, 1/v^*) = \sqrt{\frac{1}{4 \pi
      \epsilon a b (v^*)^3}}
  \left(a + \frac{v^* - a}{2} \right) \exp\left[ \frac{-(v^* -
      a)^2}{4\epsilon a b v^*} \right].
\end{equation}



\bibliographystyle{plain}
\bibliography{/Users/andrewwork/Documents/grad-school/thesis/library}

\end{document}




