\documentclass{article}

\newcommand{\ep}{\rule{.06in}{.1in}}
\textheight 9.5in

\usepackage{amssymb}
\usepackage{amsmath}
\usepackage{amsthm}
\usepackage{graphicx, subcaption}
\graphicspath{{/Users/andrewwork/thesis/}}

% \usepackage{tikz, pgfplots}
% \usepgfplotslibrary{colorbrewer}
% \pgfplotsset{
%   table/search path = {/Users/andrewwork/thesis/first-binding-model/},
%   table/col sep = comma,
%   colormap/YlGnBu,
% }

\pagestyle{empty} 
\oddsidemargin -0.25in
\evensidemargin -0.25in 
\topmargin -0.75in 
\parindent 0pt
\parskip 12pt
\textwidth 7in
%\font\cj=msbm10 at 12pt

\newcommand{\tn}{\textnormal}
\newcommand{\stiff}{\frac{k_f}{\gamma}}
\newcommand{\dd}{d}
\newcommand{\Der}[2]{\frac{\dd #1}{\dd #2}}
\newcommand{\Pder}[2]{\frac{\partial #1}{\partial #2}}
\newcommand{\Integral}[4]{\int_{#3}^{#4} {#1} \dd #2}

% Text width is 7 inches

\def\R{\mathbb{R}}
\def\N{\mathbb{N}}
\def\C{\mathbb{C}}
\def\Z{\mathbb{Z}}
\def\Q{\mathbb{Q}}
\def\H{\mathbb{H}}
\def\B{\mathcal{B}} 
%\topmargin -.5in 

\setcounter{secnumdepth}{2}
\begin{document}
\pagestyle{plain}

\begin{center}
  {\Large Notes on a Jump-Velocity Model of Platelet Rolling (6/17/19)}
\end{center}

\section{Description of the Jump-Velocity Model}
\label{sec:jump-vel}

Suppose we model platelets as particles in a moving fluid. Assume that
unprimed platelets can exist in two states: unbound (U), and bound to
vWF (V) (Figure \ref{fig:unprimed-states}). Platelets in the unbound
state advect at velocity $v$ in the fluid, and platelets in the
vWF-bound state are stationary. Platelets in the fluid can bind to vWF
at a constant rate $k_\tn{on}$, and platelets bound to vWF can come
unbound at constant rate $k_\tn{off}$.

\begin{figure}
  \centering
  \includegraphics[width=.3\textwidth]{unprimed-states.png}
  \caption[Possible states of unprimed platelets]{An unprimed platelet
    can exist in two distinct states: (U) unbound from the surface and
    advecting in the fluid, or (V) bound to vWF on the surface and
    unmoving. Transitions between these states occur at constant rates
    $k_\tn{on}$ and $k_\tn{off}$.}
  \label{fig:unprimed-states}
\end{figure}

For primed platelets, assume that they can exist in 4 possible states:
unbound (U), vWF-bound (V), fibrinogen-bound (F), and vWF- and
fibrinogen-bound (VF) (Figure \ref{fig:primed-states}). The U and V
states are the same as for the unprimed platelets, however in addition
both the U and V states can bind with fibrinogen to transition to the
F and VF states.

\begin{figure}
  \centering
  \includegraphics[width=.3\textwidth]{primed-states.png}
  \caption[Possible states of primed platelets]{A primed platelet can
    exist in four states: (U) unbound from the surface and advecting
    in the fluid, (V) bound to vWF on the surface, (F) bound to
    fibrinogen on the surface, or (VF) bound to both vWF and
    fibrinogen. In all three bound states, the platelet is immobilized
    on the surface.}
  \label{fig:primed-states}
\end{figure}

These models describe a jump-velocity process, where a particle is
transitioning randomly between discrete states, which each move with a
different deterministic motion. The Fokker-Planck equation for these
processes are given by the following system of linear advection
equations:
\begin{equation}
  \Pder{}{t}
  \begin{pmatrix}
    p_\tn{U} \\ p_\tn{V} \\ p_\tn{F} \\ p_\tn{VF}
  \end{pmatrix}
  =
  \Pder{}{y}
  \begin{pmatrix}
    v p_\tn{U} \\ 0 \\ 0 \\ 0
  \end{pmatrix}
  +
  \begin{pmatrix}
    -(k_\tn{on} + k_\tn{on}^F) & k_\tn{off} & k_\tn{off}^F & 0 \\
    k_\tn{on} & -(k_\tn{off} + k_\tn{on}^F) & 0 & k_\tn{off}^F \\
    k_\tn{on}^F & 0 & -(k_\tn{on} + k_\tn{off}^F) & k_\tn{off} \\
    0 & k_\tn{on}^F & k_\tn{on} & -(k_\tn{off} + k_\tn{off}^F)
  \end{pmatrix}
  \begin{pmatrix}
    p_\tn{U} \\ p_\tn{V} \\ p_\tn{F} \\ p_\tn{VF}
  \end{pmatrix}
\end{equation}
where $p_i = p_i(y, t \mid z, 0)$ is the probability the platelet is
in state $i$ and position $y$ at time $t$ given it was 

\bibliographystyle{plain}
\bibliography{/Users/andrewwork/Documents/grad-school/thesis/library}

\end{document}




