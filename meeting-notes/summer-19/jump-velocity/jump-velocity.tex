\documentclass{article}

\newcommand{\ep}{\rule{.06in}{.1in}}
\textheight 9.5in

\usepackage{amssymb}
\usepackage{amsmath}
\usepackage{amsthm}
\usepackage{graphicx, subcaption, algorithmic}
\graphicspath{{/Users/andrewwork/thesis/jump-velocity/plots/}}

\usepackage{tikz, pgfplots, chemfig}
\usepgfplotslibrary{colorbrewer, statistics}
\pgfplotsset{
  exact axis/.style={grid=major, minor tick num=4, xlabel=$v^*$,
    legend entries={PDF, CDF},},
  every axis plot post/.append style={thick},
  table/search
  path={/Users/andrewwork/thesis/jump-velocity/dat-files},
  colormap/YlGnBu,
  cycle list/Set1-5,
  legend style={legend cell align=left,},
}
\usepgfplotslibrary{external}
\tikzexternalize

\renewcommand{\arraystretch}{1.2}
\pagestyle{empty} 
\oddsidemargin -0.25in
\evensidemargin -0.25in 
\topmargin -0.75in 
\parindent 0pt
\parskip 12pt
\textwidth 7in
%\font\cj=msbm10 at 12pt

\newcommand{\tn}{\textnormal}
\newcommand{\stiff}{\frac{k_f}{\gamma}}
\newcommand{\dd}{d}
\newcommand{\Der}[2]{\frac{\dd #1}{\dd #2}}
\newcommand{\Pder}[2]{\frac{\partial #1}{\partial #2}}
\newcommand{\Integral}[4]{\int_{#3}^{#4} {#1} \dd #2}
\DeclareMathOperator{\Exp}{Exp}

% Text width is 7 inches

\def\R{\mathbb{R}}
\def\N{\mathbb{N}}
\def\C{\mathbb{C}}
\def\Z{\mathbb{Z}}
\def\Q{\mathbb{Q}}
\def\H{\mathbb{H}}
\def\B{\mathcal{B}} 
%\topmargin -.5in 

\setcounter{secnumdepth}{2}
\begin{document}
\pagestyle{plain}

\begin{center}
  {\Large Notes on a Jump-Velocity Model of Platelet Rolling (\today)}
\end{center}

\section{Description of the Jump-Velocity Model}
\label{sec:jump-vel}

Suppose we model platelets as particles in a moving fluid. Assume that
unprimed platelets can exist in two states: unbound (U), and bound to
vWF (V) (Figure \ref{fig:unprimed-states}). Platelets in the unbound
state advect at velocity $v$ in the fluid, and platelets in the
vWF-bound state are stationary. Platelets in the fluid can bind to vWF
at a constant rate $k_\tn{on}$, and platelets bound to vWF can come
unbound at constant rate $k_\tn{off}$.

\emph{Modification}: We can split the U state into platelets that have
not bound to the surface (call these U\textsuperscript{0} platelets)
and those that have bound to a surface and then unbound
(U\textsuperscript{1} platelets).

\begin{figure}[h]
  \centering
  \includegraphics[width=.3\textwidth]{unprimed-states.png}
  \caption[Possible states of unprimed platelets]{An unprimed platelet
    can exist in three distinct states: (U\textsuperscript{0})
    platelets which haven't interacted with the surface,
    (U\textsuperscript{0}) platelets which have interacted with the
    surface and are advecting in the fluid, or (V) bound to vWF on the
    surface and unmoving. Transitions between these states occur at
    constant rates $k_\tn{on}$ and $k_\tn{off}$. Platelets can only
    transition out of the U\textsuperscript{0} state.}
  \label{fig:unprimed-states}
\end{figure}

For primed platelets, assume that they can exist in 4 possible states:
unbound (U), vWF-bound (V), fibrinogen-bound (F), and vWF- and
fibrinogen-bound (VF) (Figure \ref{fig:primed-states}). The U and V
states are the same as for the unprimed platelets, however in addition
both the U and V states can bind with fibrinogen to transition to the
F and VF states. 

\begin{figure}[h]
  \centering
  \includegraphics[width=.3\textwidth]{primed-states.png}
  \caption[Possible states of primed platelets]{A primed platelet can
    exist in five states: (U) unbound from the surface and advecting
    in the fluid (further split into the two categories defined
    above), (V) bound to vWF on the surface, (F) bound to fibrinogen
    on the surface, or (VF) bound to both vWF and fibrinogen. In all
    three bound states, the platelet is immobilized on the surface.}
  \label{fig:primed-states}
\end{figure}

These models describe a jump-velocity process, where a particle is
transitioning randomly between discrete states, which each move with a
different deterministic motion. The Fokker-Planck equation for these
processes are given by the following system of linear advection
equations:
\begin{equation}
  \label{eq:fp-system}
  \Pder{}{t}
  \underbrace{
    \begin{pmatrix}
      p_{\tn{U}^0} \\ p_{\tn{U}^1} \\ p_\tn{V} \\ p_\tn{F} \\ p_\tn{VF}
    \end{pmatrix}}_{\equiv \mathbf{p}}
  =
  -\Pder{}{x}
  \begin{pmatrix}
    v p_{\tn{U}^0} \\ v p_{\tn{U}^0} \\ 0 \\ 0 \\ 0
  \end{pmatrix}
  +
  \underbrace{
    \begin{pmatrix}
      -(k_\tn{on} + k_\tn{on}^F) & 0 & 0 & 0 & 0 \\
      0 & -(k_\tn{on} + k_\tn{on}^F) & k_\tn{off} & k_\tn{off}^F & 0 \\
      k_\tn{on} & k_\tn{on} & -(k_\tn{off} + k_\tn{on}^F) & 0 & k_\tn{off}^F \\
      k_\tn{on}^F & k_\tn{on}^F & 0 & -(k_\tn{on} + k_\tn{off}^F) & k_\tn{off} \\
      0 & 0 & k_\tn{on}^F & k_\tn{on} & -(k_\tn{off} + k_\tn{off}^F)
  \end{pmatrix}}_{\equiv A}
  \begin{pmatrix}
    p_{\tn{U}^0} \\ p_{\tn{U}^1} \\ p_\tn{V} \\ p_\tn{F} \\ p_\tn{VF}
  \end{pmatrix}
\end{equation}
where $p_i = p_i(x, t \mid x_0, j, 0)$ is the probability the platelet
is in state $i$ and position $x$ at time $t$ given it was previously
in position $z$ and state $j$ at time $0$. If the platelets are
unprimed, then the only difference is $k_\tn{on}^F = k_\tn{off}^F =
0$.

The goal is to find the probability density function of the average
velocity across a segment of length $L$, so take the initial condition
of the PDE system (\ref{eq:fp-system}) to be
$\mathbf{p}(x, 0) = (\delta(x), 0, 0, 0, 0)^T$. That is, all platelets
enter in never-bound state U\textsuperscript{0}. The probability
density of the time it takes a platelet to cross the interval $[0, L]$
is $v(p_{\tn{U}^0}(L, t) + p_{\tn{U}^1}(L,t))$. The average velocity
associated with a crossing time $t^*$ is just $v^* = L/t^*$, and the
probability density function of $v^*$ is given by
$f(v^*) = (L/v^*)^2 p_U(L, L/v^*)$

\subsection{Nondimensionalization}
\label{sec:nondim}

Define the nondimensional variables $s$ and $y$ so that $t = Ts$ and
$x = Xy$. Let's scale $x$ by the domain length, so $X = L$, and scale
$t$ by the velocity, so that $X/T = v \implies T = L/v$. That is, $T$
is the shortest possible crossing time of a
platelet. Finally---motivated by the adiabatic reduction in example
3.6.1 in Dr. Keener's notes---define the nondimensional parameter
$\epsilon_1 = 1/(T(k_\tn{on} + k_\tn{off}))$ and
$\epsilon_2 = 1/(T(k_\tn{on}^F + k_\tn{off}^F))$. If the sum of the
(relevant) reaction rates is much larger than $1/T$, then $\epsilon$
is a small parameter. After the nondimensionalization, equation
(\ref{eq:fp-system}) becomes
\begin{equation}
  \label{eq:nd-system}
  \Pder{\mathbf{q}}{s} = -\Pder{}{y}
  \begin{pmatrix}
    q_{\tn{U}^0} \\ q_{\tn{U}^1} \\ 0 \\ 0 \\ 0
  \end{pmatrix}
  + \left(\frac{1}{\epsilon_1} + \frac{1}{\epsilon_2}\right)
  \begin{pmatrix}
    - kb - kd & 0 & 0 & 0 & 0 \\
    0 & - kb - kd & ka & kc & 0 \\
    kb & kb & - ka - kd & 0 & kc \\
    kd & kd & 0 & - kb - kc & ka \\
    0 & 0 & kd & kb & - ka - kc
  \end{pmatrix}
  \mathbf{q},
\end{equation}
where $a = k_\tn{off}/(k_\tn{off} + k_\tn{on})$,
$b = k_\tn{on}/(k_\tn{off} + k_\tn{on})$,
$c = k_\tn{off}^F/(k_\tn{off}^F + k_\tn{on}^F)$, and
$d = k_\tn{on}^F/(k_\tn{off}^F + k_\tn{on}^F)$. Then
$ka = a/(1 + \epsilon_1/\epsilon_2)$,
$kb = b/(1 + \epsilon_1/\epsilon_2)$,
$kc = c/(1 + \epsilon_2/\epsilon_1)$, and
$kd = d/(1 + \epsilon_2/\epsilon_1)$. When
$k_\tn{on}^F = k_\tn{off}^F = 0$, this reduces to the example in
Dr. Keener's notes.

\section{Numerics and results for unprimed platelets (the two-state model)}
\label{sec:res-unpr}

The system of equations for unprimed platelets reduces to
\begin{align}
  \label{eq:unprimed-1}
  \Pder{q_\tn{U}}{s} &= -\Pder{q_\tn{U}}{y} - \frac{1}{\epsilon}
                       \left(b q_\tn{U} + a q_\tn{V}\right) \\
  \label{eq:unprimed-2}
  \Pder{q_\tn{V}}{s} &= \frac{1}{\epsilon} \left(b q_\tn{U} - a
                       q_\tn{V}\right).
\end{align}

I use a first order upwind scheme to solve this system. The nodes of
the mesh are given by $y_i = ih$ for $i = 0, \hdots, N$ where
$h = 1/N$ and $s_j = jk$ for $j = 0, \hdots, M$ where $k =
s_\tn{max}/M$. In practice, I take $h = k$ so that the upwind scheme
can solve the advection part exactly.
% This results in the
% semi-discretized system
% \begin{align}
%   \Der{q_\tn{U}^i}{s} &= \frac{1}{h} \left(q_\tn{U}^{i-1} - q_\tn{U}^i
%                         \right) + \frac{1}{\epsilon} \left(-b
%                         q_\tn{U}^i + a q_\tn{V}^i \right) \quad
%                         \tn{for} \quad i = 1, \hdots, N
%   \\
%   \Der{q_\tn{V}^i}{s} &= \frac{1}{\epsilon} \left(b q_\tn{U}^i - a
%                         q_\tn{V}^i \right) \quad \tn{for} \quad i = 1,
%                         \hdots, N
% \end{align}
% which I solve with SciPy's implementation of the RK45 scheme.

The initial condition $q_{\tn{U}^0}(y, 0) = \delta(y)$ must be
approximated, so I use Peskin's approximate $\delta$-function:
%\cite{Peskin2002}:
\begin{equation}
  \label{eq:delta-h}
  \delta_h(y) =
  \begin{cases}
    \frac{1}{4h} \left( 1 + \cos\left(\frac{\pi y}{2h}\right) \right)
    & \tn{if} \quad \left|\frac{y}{2h}\right| \le 1 \\
    0 & \tn{otherwise.}
  \end{cases}
\end{equation}
Therefore $q_\tn{U}^i(0) = \delta_h(y_i)$ and $q_\tn{V}^i(0) =
0$. Half of $\delta_h$ is outside of the domain, so the other half
must be advected into the domain as an inflow boundary condition:
$q_\tn{U}^0(s) = \delta_h(-s)$.

Because $a + b = 1$, there are only 2 nondimensional parameters to
choose: $\epsilon$ and $a$. The parameter $\epsilon$ gives the ratio
of the minimum crossing time to a characteristic reaction time, and
$a$ gives the ratio of $k_\tn{off}$ to the sum of reaction
rates. Below are some sample results for average velocity
distributions $f(v^*) = 1/{v^*}^2 q_U^N(1/v^*)$.

\begin{figure}
  \centering
  \begin{subfigure}{0.48\textwidth}
    % \includegraphics[width=\textwidth]{{twostate-smalleps-smalla}.png}
    \begin{tikzpicture}
      \begin{axis}[
        exact axis,
        ]
        \addplot table[x index=0, y index=1]
        {distributions/twostate-smalleps-smalla-dst.dat}; 
        \addplot table[x index=0, y index=2]
        {distributions/twostate-smalleps-smalla-dst.dat}; 
        \draw[thick] ({axis cs:1, 0}|-{rel axis cs:0, 0})
        -- ({axis cs:1, 0}|-{rel axis cs:0, 1});
      \end{axis}
    \end{tikzpicture}
    \caption{$\epsilon = 0.1$, $a = 0.2$: Reactions are fast, and
      $k_\tn{on} > k_\tn{off}$.}
  \end{subfigure}
  \hfill
  \begin{subfigure}{0.48\textwidth}
    % \includegraphics[width=\textwidth]{{twostate-largeeps-smalla}.png}
    \begin{tikzpicture}
      \begin{axis}[
        exact axis,
        ]
        \addplot table[x index=0, y index=1]
        {distributions/twostate-largeeps-smalla-dst.dat}; 
        \addplot table[x index=0, y index=2]
        {distributions/twostate-largeeps-smalla-dst.dat}; 
        \draw[thick] ({axis cs:1, 0}|-{rel axis cs:0, 0})
        -- ({axis cs:1, 0}|-{rel axis cs:0, 1});
      \end{axis}
    \end{tikzpicture}
    \caption{$\epsilon = 1$, $a = 0.2$: Reactions and advection are
      the same order, and $k_\tn{on} > k_\tn{off}$.}
  \end{subfigure}
  \\
  \begin{subfigure}{0.48\textwidth}
    % \includegraphics[width=\textwidth]{{twostate-smalleps-meda}.png}
    \begin{tikzpicture}
      \begin{axis}[
        exact axis,
        ]
        \addplot table[x index=0, y index=1]
        {distributions/twostate-smalleps-meda-dst.dat}; 
        \addplot table[x index=0, y index=2]
        {distributions/twostate-smalleps-meda-dst.dat}; 
        \draw[thick] ({axis cs:1, 0}|-{rel axis cs:0, 0})
        -- ({axis cs:1, 0}|-{rel axis cs:0, 1});
      \end{axis}
    \end{tikzpicture}
    \caption{$\epsilon = 0.1$, $a = 0.5$: Reactions are fast, and
      $k_\tn{on} = k_\tn{off}$.}
  \end{subfigure}
  \hfill
  \begin{subfigure}{0.48\textwidth}
    % \includegraphics[width=\textwidth]{{twostate-largeeps-meda}.png}
    \begin{tikzpicture}
      \begin{axis}[
        exact axis,
        ]
        \addplot table[x index=0, y index=1]
        {distributions/twostate-largeeps-meda-dst.dat}; 
        \addplot table[x index=0, y index=2]
        {distributions/twostate-largeeps-meda-dst.dat}; 
        \draw[thick] ({axis cs:1, 0}|-{rel axis cs:0, 0})
        -- ({axis cs:1, 0}|-{rel axis cs:0, 1});
      \end{axis}
    \end{tikzpicture}    
    \caption{$\epsilon = 1$, $a = 0.5$: Reactions and advection are
      the same order, and $k_\tn{on} = k_\tn{off}$.}
  \end{subfigure}
  \\
  \begin{subfigure}{0.48\textwidth}
    % \includegraphics[width=\textwidth]{{twostate-smalleps-largea}.png}
    \begin{tikzpicture}
      \begin{axis}[
        exact axis,
        ]
        \addplot table[x index=0, y index=1]
        {distributions/twostate-smalleps-largea-dst.dat}; 
        \addplot table[x index=0, y index=2]
        {distributions/twostate-smalleps-largea-dst.dat}; 
        \draw[thick] ({axis cs:1, 0}|-{rel axis cs:0, 0})
        -- ({axis cs:1, 0}|-{rel axis cs:0, 1});
      \end{axis}
    \end{tikzpicture}    
    \caption{$\epsilon = 0.1$, $a = 0.8$: Reactions are fast, and
      $k_\tn{on} < k_\tn{off}$.}
  \end{subfigure}
  \hfill
  \begin{subfigure}{0.48\textwidth}
    % \includegraphics[width=\textwidth]{{twostate-largeeps-largea}.png}
    \begin{tikzpicture}
      \begin{axis}[
        exact axis,
        ]
        \addplot table[x index=0, y index=1]
        {distributions/twostate-largeeps-largea-dst.dat}; 
        \addplot table[x index=0, y index=2]
        {distributions/twostate-largeeps-largea-dst.dat}; 
        \draw[thick] ({axis cs:1, 0}|-{rel axis cs:0, 0})
        -- ({axis cs:1, 0}|-{rel axis cs:0, 1});
      \end{axis}
    \end{tikzpicture}    
    \caption{$\epsilon = 1$, $a = 0.8$: Reactions and advection are
      the same order, and $k_\tn{on} < k_\tn{off}$.}
  \end{subfigure}
  \caption[Probability density functions]{Probability density
    functions for 6 different $(\epsilon, a)$ pairs in the two-state
    jump-velocity model.}
  \label{fig:prob-dens}
\end{figure}

Observations:
\begin{itemize}
\item There is some nonzero probability mass that a platelet crosses
  the domain without ever binding. Specifically, this probability is
  $\exp(-b/\epsilon)$. Therefore the exact probability density
  function will have a $\delta$-function like spike at $v^* = 1$ that
  integrates to $\exp(-b/\epsilon)$.
\item As the off rate $a$ increases, the distribution of average
  velocities shifts to the right, as expected.
\end{itemize}

\subsection{Adiabiatic reduction}
\label{sec:adiabiatic-reduction}

The adiabatic reduction for the two-state model is solved in example
3.6.1 of Dr. Keener's stochastics notes. Starting with equations
(\ref{eq:unprimed-1}) and (\ref{eq:unprimed-2}), assume that
$\epsilon \ll 1$ and define $v = q_\tn{U} + q_\tn{V}$ to be the slow
variable and $w = b q_\tn{U} - a q_\tn{V}$ to be the fast
variable. Then $v$ evolves as an advection-diffusion equation:
\begin{equation}
  \label{eq:v}
  \Pder{v}{s} = -a \Pder{v}{y} + \epsilon a b \frac{\partial^2
    v}{\partial y^2}
\end{equation}
and $w$ satisfies the equation
\begin{equation}
  \label{eq:w}
  w = -\epsilon a b \Pder{v}{y}.
\end{equation}

With the initial condition $v(y, 0) = \delta(y)$, equation
(\ref{eq:v}) can be solved analytically:
\begin{equation}
  \label{eq:v-soln}
  v(y, t) = \frac{1}{\sqrt{4 \pi \epsilon a b t}} \exp \left[ \frac{-(y
      - at)^2}{4 \epsilon a b t} \right],
\end{equation}
and then using equation (\ref{eq:w}) to find $w$,
\begin{equation}
  \label{eq:w-soln}
  w(y, t) = \frac{y - at}{4t\sqrt{\pi \epsilon a b t}} \exp \left[ \frac{-(y
      - at)^2}{4 \epsilon a b t} \right].
\end{equation}

Then reversing the change of variables, $q_\tn{U} = av + w$ and
$q_\tn{V} = bv - w$, resulting in the following solutions for
$q_\tn{U}$ and $q_\tn{V}$:
\begin{align}
  \label{eq:qu-soln}
  q_\tn{U} &= \frac{1}{\sqrt{4 \pi \epsilon a b t}} \left(a + \frac{y
             - at}{2t} \right) \exp \left[ \frac{-(y - at)^2}{4
             \epsilon a b t} \right], \\
  \label{eq:qv-soln}
  q_\tn{V} &= \frac{1}{\sqrt{4 \pi \epsilon a b t}} \left(b - \frac{y
             - at}{2t} \right) \exp \left[ \frac{-(y - at)^2}{4
             \epsilon a b t} \right].
\end{align}

The probability density function of the average velocity is then
\begin{equation}
  \label{eq:vel-dens}
  f(v^*) = (v^*)^{-2} q_\tn{U}(1, 1/v^*) = \sqrt{\frac{1}{4 \pi
      \epsilon a b (v^*)^3}}
  \left(a + \frac{v^* - a}{2} \right) \exp\left[ \frac{-(v^* -
      a)^2}{4\epsilon a b v^*} \right].
\end{equation}

\subsection{Parameter Estimation}
\label{sec:parameter-estimation}

Chapter 15 from \textit{Numerical Recipes in Fortran} describes
methods for fitting proposed models to data. In the situation they
describe, data is collected at several values of some independent
variable (say $x$) with some measurement error $\sigma$ that may vary
with $x$, and the goal is to fit a deterministic model to that
data. They develop a maximum-likelihood estimate for model parameters
based on the data collected $(x_i, y_i)$ and measurement errors
$\sigma_i$. 

In our jump-velocity model, the situation is a little different. We
have a stochastic model of a jump-velocity process, which is used to
derive a Fokker-Planck equation for platelet position as a function of
time. From the Fokker-Planck equation we derive a probability
distribution for the average velocity of a platelet rolling across
some domain. Basically the average velocity of a platelet (or
equivalently, the time it takes a platelet to cross) is a random
variable. The Fokker-Planck equation defines a probability
distribution for this random variable as a function of model
parameters, and the data Vlado's group are collecting on average
velocity are realizations of this random variable. Then we want to fit
the probability distribution defined by model parameters to the
realizations collected by Vlado's group.

One way to fit the model to average velocity data is to use a
maximum-likelihood estimate. This is analogous to the procedure
described in Chapter 15 of \textit{Numerical Recipes}. Let
$\{v_i\}_{i=1}^N$ be a set of observations of average velocities, and
define $f(v;a, \epsilon)$ to be the probability distribution of
velocities given parameters $a$ and $\epsilon$ (for simplicity, assume
we're working with the two-state model). Define the likelihood
function
\begin{equation}
  \label{eq:likelihood}
  L(a, \epsilon) = \prod_{i=1}^N f(v_i; a, \epsilon),
\end{equation}
and then the maximum likelihood estimates for $a$ and $\epsilon$ are
``simply'' those that maximize $L$. As usual, define the
log-likelihood function: $\tilde{\ell}(a, \epsilon) = \log(L(a, \epsilon))$,
and then maximizing $\tilde{\ell}$ is equivalent to maximizing $L$.

We found $f(v; a, \epsilon)$ above in terms of the solution of the
Fokker-Planck equation: $f(v; a, \epsilon) = 1/v^2 q_{U^1}(1, 1/v; a,
\epsilon)$ where $q_{U^1}(x, t; a, \epsilon)$ is part of the solution
of the Fokker-Planck equation with parameters $a$ and $\epsilon$. Then
we can write $\tilde{\ell}$ in terms of $q_{U^1}$:
$\tilde{\ell} = \sum_{i=1}^N \log(f(v_i; a, \epsilon)) = \sum_{i=1}^N
\log(q_{U^1}(1, 1/v_i; a, \epsilon)) - 2\sum_{i=1}^N \log(v_i)$. The
second term is a constant factor with respect to $a$ and $\epsilon$,
and so can be excluded from the optimization. Therefore we can
maximize the modified log-likelihood function
\begin{equation}
  \label{eq:mod-log-like}
  \ell(a, \epsilon) = \sum_{i=1}^N \log(q_{U^1}(1, 1/v_i; a, \epsilon)).
\end{equation}

\subsubsection{Numerical Optimization}
\label{sec:numer-optim}

Clearly $\ell(a, \epsilon)$ must be maximized iteratively. We haven't
even found an analytical expression for $q_{U^1}$. Unless we do, the
Fokker-Planck equation must be re-solved each time the parameters $a$
and $\epsilon$ are changed. Fortunately the PDE only needs to be
solved once for each evaluation of $\ell(a, \epsilon)$. We have to
find $q_{U^1}(1, t)$ up to time $t = 1/\min(v_i)$, but then to
evaluate the probability density at each time $1/v_i$, we only need to
interpolate the solution.

Another important consideration is that the model parameters $a$ and
$\epsilon$ have bounds, in particular $a \in (0, 1)$ and
$\epsilon > 0$. One option to deal with this is to choose a numerical
method that optimizes within a bounded domain, however when I tried
this the optimization function I was using ran into overflow errors
when it got too close to the edge of the domain. Restricting the
parameters to even tighter domains fixed this problem, but the bounds
were chosen arbitrarily, and with prior knowledge of the true
parameters of the ``data''. I have achieved good results in my
experiments by transforming the model parameters $a$ and $\epsilon$ to
``fitting'' parameters $a'$ and $\epsilon'$ which can vary over all
$\R$. The transformations I have chosen are:
\begin{align}
  \label{eq:a-fwd-trns}
  a' &= \frac{2a + 1}{2a(a + 1)} \\
  \label{eq:e-fwd-trns}
  \epsilon' &= \log \epsilon.
\end{align}
$a'$ is defined so that $a'=0$ at $a=1/2$, $a' \rightarrow -\infty$ as
$a \rightarrow 0$, and $a' \rightarrow \infty$ as $a \rightarrow
1$. With the fitting parameters, the optimization problem becomes an
unconstrained optimization in $\R^2$. To actually carry out the
optimization I use SciPy's \verb|minimize| function, which implements
the BFGS algorithm.

An iterative algorithm needs an intial guess, which at least in the
two-state model, can be chosen as an estimate of $a$ and $\epsilon$
from the adiabatic reduction. Equation (\ref{eq:vel-dens}) gives the
pdf of velocities from the adiabatic reduction, and can be used to
derive the mean $\mu(a, \epsilon)$ and variance $\sigma^2(a,
\epsilon)$ as functions of $a$ and $\epsilon$. These moments actually
have simple analytical expressions (according to Mathematica):
\begin{align}
  \label{eq:mean-ar}
  \mu(a, \epsilon) &= a(1 + \epsilon - a \epsilon) \\
  \label{eq:var-ar}
  \sigma^2(a, \epsilon) &= a^2 \epsilon (1 - a) (2 + 5(1 - a) \epsilon).
\end{align}

Then we can equate $\mu$ and $\sigma^2$ to the first two sample
moments ($\bar{v}$ and $s^2$) of the data to estimate $a$ and
$\epsilon$ (this is called the method of moments). $a$ and $\epsilon$
can be expressed in terms of $\mu$ and $\sigma^2$ analytically, but
the formulas are messy. Note that $\sigma^2 = O(\epsilon)$ and
$\epsilon << 1$, so we can use an asymptotic approximation instead of
the full formulas. The estimate is only being used to initialize the
optimization algorithm, so there is no need to be exact. Therefore up
to $O(\sigma^2)$, the method of moments gives the following estimates
for $a$ and $\epsilon$ based on a data set with sample mean $\bar{v}$
and sample variance $s^2$:
\begin{align}
  \label{eq:mean-est}
  a &= \bar{v} - \frac{s^2}{2\bar{v}} \\
  \label{eq:var-est}
  \epsilon &= \frac{s^2}{2(1 - \bar{v})(\bar{v})^2}.
\end{align}

In figure \ref{fig:model-fits}, a histogram of 1000 sample velocities
is shown along with the PDEs generated by the reduced and full models
with maximum likelihood parameter estimates. The CDFs of the models
are also compared with the ECDF of the data. Visually these models fit
well (as they should), and provide parameter estimates with less than
10\% relative error.

\subsubsection{Bootstrapping}
\label{sec:bootstrapping}

The maximum likelihood estimator only gives a point estimate of the
parameters $a$ and $\epsilon$. In order to get confidence intervals,
we can use a bootstrapping approach. The basic idea is that for a set
of data $\{V_i\}_{i=1}^N$, a ``new'' data set can be generated by
randomly picking $N$ samples from the $\{V_i\}$ with replacement. Then
the MLE for the new data can be found, and in this way we get a bunch
of $\hat{a}$ and $\hat{\epsilon}$ estimates. After generating many of
these estimates, we can come up with an estimate of a 95\% confidence
interval for each parameter. If we assume that the ML estimates are
distributed sufficiently normally, then
$\hat{\theta} \pm 1.96 \sigma_{\hat{\theta}}$ approximates a 95\%
confidence interval well. But it seems like using the $2.5$th and
$97.5$th percentiles also provides a good estimate for the confidence
interval, and doesn't rely on assuming the parameter estimate is
distributed normally.

Figure \ref{fig:bootstraps} summarizes results from 64 bootstrap
trials on the sample data. In each case, the distribution of estimated
parameters looks symmetrical, and the two methods of estimating a 95\%
confidence interval described above give similar results (Table
\ref{tab:conf-int}).

While the MLE of $a$ looks unbiased (i.e. it is near the center of the
distribution of $a$s shown in figure \ref{fig:bootstraps}), the MLE of
$\epsilon$ is clearly biased to the upper end of the distribution of
$\epsilon$ estimates from the bootstrap procedure. Part of this may be
because I haven't filtered out platelets that didn't pause on the
surface, but it is still something to look out for. 

\begin{table}
  \centering
  \begin{tabular}{cccc}
    \hline
    Parameter & Model & Normal method & Quantile method \\
    \hline
    $a$ & Reduced & $[0.5216, 0.5392]$ & $[0.5225, 0.5408]$ \\
    $a$ & Full & $[0.4972, 0.5154]$ & $[0.4979, 0.5164]$ \\
    $\epsilon$ & Reduced & $[0.0936, 0.1090]$ & $[0.0948, 0.1095]$ \\
    $\epsilon$ & Full & $[0.0916, 0.1048]$ & $[0.0926, 0.1049]$\\
    \hline
  \end{tabular}
  \caption[Summary of confidence intervals]{Summary of 95\% confidence
    intervals of model parameters estimated by bootstrapping}
  \label{tab:conf-int}
\end{table}

\begin{figure}
  \centering
  \begin{subfigure}{0.48\textwidth}
    \begin{tikzpicture}
      \begin{axis}[
        legend entries={Simulated data, Reduced model, Full model},
        xlabel=$v^*$,
        ylabel={Probability density},
        ]
        \addplot+[hist=density, fill]
        table[y index=0] {simulations/test-sim.dat};
        \addplot table[x index=0, y index=1]
        {ml-estimates/test-est.dat};
        \addplot table[x index=0, y index=3]
        {ml-estimates/test-est.dat};
      \end{axis}
    \end{tikzpicture}
  \end{subfigure}
  \hfill
  \begin{subfigure}{0.48\textwidth}
    \begin{tikzpicture}
      \begin{axis}[
        legend entries={Simulated data, Reduced model, Full model},
        legend pos=south east,
        xlabel=$v^*$,
        ylabel={$P[V < v^*]$},
        ]
        \addplot+[const plot] table[x index=0, y expr=\coordindex/1024]
        {simulations/test-sim.dat};
        \addplot table[x index=0, y index=2]
        {ml-estimates/test-est.dat};
        \addplot table[x index=0, y index=4]
        {ml-estimates/test-est.dat};
      \end{axis}
    \end{tikzpicture}
  \end{subfigure}
  \caption[Model fits]{Results of the ML estimates for the reduced
    (i.e. adiabatic reduction) and full models. For this simulated
    data, the maximum likelihood parameters are $a = 0.5300$ and
    $\epsilon = 0.1014$ for the reduced model, and $a = 0.5059$ and
    $\epsilon = 0.0982$ for the full model. Total number of simulated
    data points: $N = 1024$}
  \label{fig:model-fits}
\end{figure}

\begin{figure}
  \centering
  \begin{subfigure}{0.48\textwidth}
    \begin{tikzpicture}[baseline]
      \begin{axis}[
        ytick={1, 2},
        yticklabels={Reduced model, Full model},
        xticklabel style={/pgf/number format/precision=3,
          /pgf/number format/fixed},
        title={Distributions of $a$ estimates},
        ]
        \draw[thin, dashed] ({axis cs:.5, 0}|-{rel axis cs:0, 0})
        -- ({axis cs:.5, 0}|-{rel axis cs:0, 1});
        \addplot+ [boxplot] table[y index=0]
        {bootstrap/test-boot.dat};
        \addplot+ [boxplot] table[y index=2]
        {bootstrap/test-boot.dat};
      \end{axis}
    \end{tikzpicture}
  \end{subfigure}
  \hfill
  \begin{subfigure}{0.48\textwidth}
    \begin{tikzpicture}[baseline]
      \begin{axis}[
        ytick={1, 2},
        yticklabels={},
        xticklabel style={/pgf/number format/precision=3,
          /pgf/number format/fixed},
        title={Distributions of $\epsilon$ estimates},
        ]
        \draw[thin, dashed] ({axis cs:.1, 0}|-{rel axis cs:0, 0})
        -- ({axis cs:.1, 0}|-{rel axis cs:0, 1});
        \addplot+ [boxplot] table[y index=1]
        {bootstrap/test-boot.dat};
        \addplot+ [boxplot] table[y index=3]
        {bootstrap/test-boot.dat};
      \end{axis}
    \end{tikzpicture}
  \end{subfigure}
  \caption[Boxplots of bootstap trials]{Boxplots of parameter
    estimates from 64 bootstap trials. The left plot summarizes estimates
    of $a$ and the right plot summarizes estimates of $\epsilon$. The
    true parameter values are shown with a dashed vertical
    line.} 
  \label{fig:bootstraps}
\end{figure}

\begin{figure}
  \centering
  \begin{tikzpicture}
    \begin{axis}[
      xticklabel style={/pgf/number format/precision=3,
        /pgf/number format/fixed},
      yticklabel style={/pgf/number format/precision=3,
        /pgf/number format/fixed},
      legend entries={Reduced model, Full model, ML estimates,
        True value}, 
      legend pos=outer north east,
      xlabel=$a$ estimate,
      ylabel=$\epsilon$ estimate,
      ]
      \addplot+ [only marks] table [x index=0, y index=1]
      {bootstrap/test-boot.dat};
      \addplot+ [only marks] table [x index=2, y index=3]
      {bootstrap/test-boot.dat};
      \addplot+ [only marks] coordinates {
        (0.529979, 0.101435)
        (0.505879, 0.0982116)
      };
      \addplot+ [only marks, black] coordinates {(0.5, 0.1)};
    \end{axis}
  \end{tikzpicture}
  \caption[$(a, \epsilon)$ scatter plot]{Scatter plot of
    $(a, \epsilon)$ estimates}
  \label{fig:scatter-plot}
\end{figure}

\begin{figure}
  \centering
  \begin{subfigure}{0.48\textwidth}
    \begin{tikzpicture}
      \begin{axis}[
        % legend entries={Simulated data, Reduced model, Full model},
        % legend pos=north west,
        xlabel=$v^*$,
        ylabel={Probability density},
        ]
        \addplot+[hist=density, fill]
        table[y index=0] {simulations/large-a-sim.dat};
        \addplot table[x index=0, y index=1]
        {ml-estimates/large-a-est.dat};
        \addplot table[x index=0, y index=3]
        {ml-estimates/large-a-est.dat};
      \end{axis}
    \end{tikzpicture}
  \end{subfigure}
  \hfill
  \begin{subfigure}{0.48\textwidth}
    \begin{tikzpicture}
      \begin{axis}[
        legend entries={Simulated data, Reduced model, Full model},
        legend pos=north west,
        xlabel=$v^*$,
        ylabel={$P[V < v^*]$},
        ]
        \addplot+[const plot] table[x index=0, y expr=\coordindex/1024]
        {simulations/large-a-sim.dat};
        \addplot table[x index=0, y index=2]
        {ml-estimates/large-a-est.dat};
        \addplot table[x index=0, y index=4]
        {ml-estimates/large-a-est.dat};
      \end{axis}
    \end{tikzpicture}
  \end{subfigure}
  \caption[Model fits]{Results of the ML estimates for the reduced
    (i.e. adiabatic reduction) and full models. For this simulated
    data, the maximum likelihood parameters are $a = 0.78$ and
    $\epsilon = 0.075$ for the reduced model, and $a = 0.77$ and
    $\epsilon = 0.066$ for the full model. Total number of simulated
    data points: $N = 1024$}
  \label{fig:model-fits-large-a}
\end{figure}

\begin{figure}
  \centering
  \begin{subfigure}{0.48\textwidth}
    \begin{tikzpicture}
      \begin{axis}[
        legend entries={Simulated data, Reduced model, Full model},
        xlabel=$v^*$,
        ylabel={Probability density},
        ]
        \addplot+[hist=density, fill]
        table[y index=0] {simulations/small-a-sim.dat};
        \addplot table[x index=0, y index=1]
        {ml-estimates/small-a-est.dat};
        \addplot table[x index=0, y index=3]
        {ml-estimates/small-a-est.dat};
      \end{axis}
    \end{tikzpicture}
  \end{subfigure}
  \hfill
  \begin{subfigure}{0.48\textwidth}
    \begin{tikzpicture}
      \begin{axis}[
        legend entries={Simulated data, Reduced model, Full model},
        legend pos=south east,
        xlabel=$v^*$,
        ylabel={$P[V < v^*]$},
        ]
        \addplot+[const plot] table[x index=0, y expr=\coordindex/1024]
        {simulations/small-a-sim.dat};
        \addplot table[x index=0, y index=2]
        {ml-estimates/small-a-est.dat};
        \addplot table[x index=0, y index=4]
        {ml-estimates/small-a-est.dat};
      \end{axis}
    \end{tikzpicture}
  \end{subfigure}
  \caption[Model fits]{Results of the ML estimates for the reduced
    (i.e. adiabatic reduction) and full models. For this simulated
    data, the maximum likelihood parameters are $a = 0.22$ and
    $\epsilon = 0.12$ for the reduced model, and $a = 0.20$ and
    $\epsilon = 0.10$ for the full model. Total number of simulated
    data points: $N = 1024$}
  \label{fig:model-fits-small-a}
\end{figure}

\subsubsection{Goodness-of-fit}
\label{sec:goodness-fit}

The final step in parameter estimation (at least according to
\textit{Numerical Recipes}) is to find the goodness-of-fit of the
model. In \textit{Numerical Recipes}, they use the $\chi^2$ statistic
to estimate the fit of the model to the data. Briefly, this arises
because they assume that their data $\{Y(x_i)\}$ are drawn from a
normal distribution. We have to use a different approach to estimate
the fit, because there isn't an obvious connection between the sample
probability distribution and a normal distribution.

Chapter 13 in \textit{Introduction to Probability and Mathematical
  Statistics} by Bain and Engelhardt suggests two approaches. One
approach is to group the data into bins, and then compare the observed
observations in each bin to the expected number of observations in
each bin. In particular,
\begin{equation}
  \label{eq:chi2}
  \chi^2 = \sum_{j=1}^c \frac{(o_j - \hat{e}_j)^2}{\hat{e}_j} \sim
  \chi^2(c - 1 - k)
\end{equation}
where $c$ is the number of bins, $k$ is the number of estimated
parameters, $o_j$ is the number of data points in bin $j$, and
$\hat{e}_j$ is the expected number of data points in bin $j$. Here
$\hat{e}_j = N \hat{p}_j$ where $N$ is the total number of data
points, and $\hat{p}_j$ is the model probability of picking a data
point from bin $j$. A weakness of this approach is that information is
lost by grouping the data, and there are other tests that depend
directly on the individual observations.

One of these tests is the Kolmogorov-Smirnov test, which essentially
uses the maximum difference between the predicted CDF and the
ECDF. The KS statistic is $D = \max(D^+, D^-)$ where
$D^+ = \max_i(i/N - F(x_{i:N}))$ and
$D^- = \max_i(F(x_{i:N}) - (i - 1)/N)$. A small value of $D$ indicates
a good fit (though precisely what value of $D$ means the fit is
``good'' depends on $N$). The weakness of this approach is that the
distribution of the KS statistic is derived assuming no parameters
have been estimated, which is obviously not true in our case. The
Kolmogorov-Smirnov test returned a $p$-value of 0.46 for the reduced
model, and 0.66 for the full model, indicating a good fit of these
models to the data (obviously, because the data were derived from the
jump-velocity model). Essentially the $p$-value is the probability
that $D$ could be as large as it is (that is, the probability that the
fit is as poor as it is) by chance alone.

\section{Numerics and results for primed platelets}
\label{sec:numer-results-prim}

For primed platelets, $k_\tn{on}^F > 0$ and the Fokker-Planck equation
for the platelets' positions is given in equation
(\ref{eq:nd-system}). The numerics for solving these equations are
similar to the two-state case, and so I won't go into detail on that
here. The idea behind including these two extra states is to model the
slow binding/unbinding dynamics of fibrinogen, and so in practice we
take $\epsilon_1 \ll \epsilon_2$. Figure \ref{fig:prob-dens} shows
velocity probability densities for a range of parameter values.

The fitting prodecure to find the maximum likelihood estimates for the
4 model parameters is exactly analogous to the 2-variable
case. However, the optimization is a much slower process because we
now have to optimize over a 4-dimensional space. Also, we don't yet
have an adiabatic reduction of the 4-dimensional model, and so I am
not sure how to find a good initial guess for the parameter
values. Right now, I am cheating and using the true parameter values
plus Gaussian white noise to mimic error in the initial guess.

In the test example I tried, the maximum likelihood estimate for the
parameters fit the data well. Figure \ref{fig:4-par-fit} shows the fit
of the model to the data. The bootstrap trials found 95\% confidence
intervals that included the true value of the parameters (Table
\ref{tab:conf-int} and Figure \ref{fig:4-par-boxplots}). It is worth
pointing out that there is a symmetry in the 4 parameter model,
because if you interchange $a$ and $c$, and $\epsilon_1$ and
$\epsilon_2$ then the model is identical. In a couple of the bootstrap
trials, the optimization function converged to the ``mirror image'' of
the parameter values instead of the true parameter values. I would
guess this comes down to the accuracy of the initial guess. In any
case, this problem is easy to spot and fix as long as the mirror
parameters aren't too similar.

\subsection{Adiabatic reduction in the 4-parameter model}
\label{sec:adiab-reduct-4}


If we only assume $\epsilon_1 \ll 1$, then the adiabatic reduction
follows the two-parameter reduction closely. It will be more
convenient to rewrite the Fokker-Planck equation of the system as:
\begin{equation}
  \label{eq:4-par-ad-fp}
  \Pder{\mathbf{q}}{s} = - \Pder{q_{U}}{y} \mathbf{e}_1 +
  \frac{1}{\epsilon_1} \underline{\underline{A}}\mathbf{q} +
  \frac{1}{\epsilon_2} \underline{\underline{B}}\mathbf{q},
\end{equation}
where
$\underline{\underline{A}} = \begin{pmatrix} -b & a & 0 & 0 \\ b & -a
  & 0 & 0 \\ 0 & 0 & -b & a \\ 0 & 0 & b & -a \end{pmatrix}$ and
$\underline{\underline{B}} = \begin{pmatrix} -d & 0 & c & 0 \\ 0 & -d
  & 0 & c \\ d & 0 & -c & 0 \\ 0 & d & 0 & -c \end{pmatrix}$. The fast
reactions are those captured in matrix $\underline{\underline{A}}$, so
this matrix defines the separation of the system into fast and slow
subsystems. In linear algebra terms, the left 0-eigenvectors of
$\underline{\underline{A}}$ define the projection onto the slow
manifold, and the orthogonal projection projects onto the fast
manifold, but I think it makes more sense to refer to the reaction
diagram to see the fast-slow separation.

\tikzexternaldisable
\begin{center}
  \schemestart
  U \arrow{<=>[$b/\epsilon_1$][$a/\epsilon_1$]}
  V \arrow{<=>[*{0}$d/\epsilon_2$][*{0}$c/\epsilon_2$]}[270]
  VF \arrow{<=>[$b/\epsilon_1$][$a/\epsilon_1$]}[180]
  F \arrow{<=>[*{0}$d/\epsilon_2$][*{0}$c/\epsilon_2$]}[90]
  \schemestop\par
\end{center}
\tikzexternalenable

The fast reactions are represented by the two horizontal reaction
lines in the diagram, and so the U and V states define one subsystem
where the reactions within that system are fast, and transitions into
or out of that system are slow, and F and VF define the other such
subsystem. Thus we define two slow variables: $v_1 = q_{U} + q_{V}$
and $v_2 = q_{F} + q_{VF}$.

The two fast variables are harder to intuit, but they end up being
$w_1 = b q_{U} - a q_{V}$ and $w_2 = b q_{F} - a q_{VF}$. These two
fast variables basically represent the net formation of fast bonds (or
the net transition rate of U to V and F to VF) within the two fast
subsystems.

Then after changing variables in equation (\ref{eq:4-par-ad-fp}), we
get
\begin{align}
  \Pder{v_1}{t} &= -\Pder{}{y}(a v_1 + w_1) - \frac{d}{\epsilon_2} v_1
                  + \frac{c}{\epsilon_2} v_2 \label{eq:v1} \\
  \Pder{v_2}{t} &= \frac{d}{\epsilon_2} v_1 - \frac{c}{\epsilon_2} v_2
  \label{eq:v2} \\
  \Pder{w_1}{t} &= -\frac{w_1}{\epsilon_1} - b\Pder{}{y} (a v_1 + w_1)
                  - \frac{d}{\epsilon_2} w_1 + \frac{c}{\epsilon_2}
                  w_2 \label{eq:w1} \\
  \Pder{w_2}{t} &= -\frac{w_2}{\epsilon_1} + \frac{d}{\epsilon_2} w_1
                  - \frac{c}{\epsilon_2} w_2. \label{eq:w2}
\end{align}

Because we are assuming the system is in quasi-steady-state, $w_1 =
\mathcal{O}(\epsilon_1)$ and $w_2 = \mathcal{O}(\epsilon_1)$. Then
multiplying equation (\ref{eq:w1}) and equation (\ref{eq:w2}) by
$\epsilon_1$ and excluding higher order terms, we get the following
QSS equations for $w_1$ and $w_2$:
\begin{align}
  w_1 &= -\epsilon_1 a b \Pder{v_1}{y} + \mathcal{O}(\epsilon_1^2) \\
  w_2 &= 0 + \mathcal{O}(\epsilon_1^2).
\end{align}

Thus at least to first order in $\epsilon_1$, the fast dynamics
between F and VF don't affect the average velocity of platelets. Then
substituting $w_1$ into equation (\ref{eq:v1}) gives us the following
system defining the evolution of the slow dynamics:
\begin{align}
  \Pder{v_1}{t} &= -a \Pder{v_1}{y} + \epsilon_1 a b \frac{\partial^2
                  v_1}{\partial y^2} - \frac{d}{\epsilon_2} v_1 +
                  \frac{c}{\epsilon_2} v_2 \\
  \Pder{v_2}{t} &= \frac{d}{\epsilon_2} v_1 - \frac{c}{\epsilon_2} v_2.
\end{align}

So the slow system evolves like its own jump-velocity process, except
that now the advecting quantity also has a small diffusion
component. In view of the adiabatic reduction for the two-parameter
model, this solution makes perfect sense. In the two-parameter system,
fast bond dynamics introduced a small diffusion component in the slow
variable, and in the four-parameter system the fast bonds introduced
exactly the same diffusion component, and the slow bond dynamics
remain unchanged.

Now this reduced model isn't particularly easier to solve numerically
than the full model (it doesn't seem like we've reduced the asymptotic
complexity at all, and we've traded stiff reaction terms for a
diffusion term), so we may want to try reducing on $\epsilon_2$ as
well. I'm not sure what the best way to proceed from here is. I tried
doing another adiabatic reduction by assuming that $1 \gg \epsilon_2
\gg \epsilon_1$, but then you run into the problem that only taking
the first order in $\epsilon_2$ eliminates all terms containing
$\epsilon_1$.

I think we need to relate $\epsilon_2$ to $\epsilon_1$ somehow,
possibly taking $\epsilon_2^\alpha = \epsilon_1$ and then re-doing the
asymptotic analysis on $\epsilon_2$, and keeping some higher-order terms.

\begin{figure}
  \centering
  \begin{subfigure}{0.48\textwidth}
    \begin{tikzpicture}
      \begin{axis}[
        exact axis,
        ]
        \addplot table[x index=0, y index=1]
        {distributions/fourstate-smalla-dst.dat}; 
        \addplot table[x index=0, y index=2]
        {distributions/fourstate-smalla-dst.dat}; 
        \draw[thick] ({axis cs:1, 0}|-{rel axis cs:0, 0})
        -- ({axis cs:1, 0}|-{rel axis cs:0, 1});
      \end{axis}
    \end{tikzpicture}
    \caption{$\epsilon_1 = 0.1$, $\epsilon_2 = 1$, $a = 0.2$,
      $c = 0.5$: $k_\tn{on} > k_\tn{off}$ and
      $k_\tn{on}^F = k_\tn{off}^F$.}
  \end{subfigure}
  \hfill
  \begin{subfigure}{0.48\textwidth}
    \begin{tikzpicture}
      \begin{axis}[
        exact axis,
        ]
        \addplot table[x index=0, y index=1]
        {distributions/fourstate-meda-dst.dat}; 
        \addplot table[x index=0, y index=2]
        {distributions/fourstate-meda-dst.dat}; 
        \draw[thick] ({axis cs:1, 0}|-{rel axis cs:0, 0})
        -- ({axis cs:1, 0}|-{rel axis cs:0, 1});
      \end{axis}
    \end{tikzpicture}
    \caption{$\epsilon_1 = .1$, $\epsilon_2 = 1$, $a = 0.5$,
      $c = 0.5$: $k_\tn{on} = k_\tn{off}$ and
      $k_\tn{on}^F = k_\tn{off}^F$.}
  \end{subfigure}
  \\
  \begin{subfigure}{0.48\textwidth}
    % \includegraphics[width=\textwidth]{{twostate-smalleps-meda}.png}
    \begin{tikzpicture}
      \begin{axis}[
        exact axis,
        ]
        \addplot table[x index=0, y index=1]
        {distributions/fourstate-largea-dst.dat}; 
        \addplot table[x index=0, y index=2]
        {distributions/fourstate-largea-dst.dat}; 
        \draw[thick] ({axis cs:1, 0}|-{rel axis cs:0, 0})
        -- ({axis cs:1, 0}|-{rel axis cs:0, 1});
      \end{axis}
    \end{tikzpicture}
    \caption{$\epsilon_1 = 0.1$, $\epsilon_2 = 1$, $a = 0.8$,
      $c = 0.5$: $k_\tn{on} < k_\tn{off}$ and
      $k_\tn{on}^F = k_\tn{off}^F$.}
  \end{subfigure}
  \hfill
  \begin{subfigure}{0.48\textwidth}
    % \includegraphics[width=\textwidth]{{twostate-largeeps-meda}.png}
    \begin{tikzpicture}
      \begin{axis}[
        exact axis,
        ]
        \addplot table[x index=0, y index=1]
        {distributions/fourstate-smallc-dst.dat}; 
        \addplot table[x index=0, y index=2]
        {distributions/fourstate-smallc-dst.dat}; 
        \draw[thick] ({axis cs:1, 0}|-{rel axis cs:0, 0})
        -- ({axis cs:1, 0}|-{rel axis cs:0, 1});
      \end{axis}
    \end{tikzpicture}    
    \caption{$\epsilon_1 = .1$, $\epsilon_2 = 1$, $a = 0.5$,
      $c = 0.2$: $k_\tn{on} = k_\tn{off}$ and
      $k_\tn{on}^F > k_\tn{off}^F$.}
  \end{subfigure}
  \\
  \begin{subfigure}{0.48\textwidth}
    % \includegraphics[width=\textwidth]{{twostate-smalleps-largea}.png}
    \begin{tikzpicture}
      \begin{axis}[
        exact axis,
        ]
        \addplot table[x index=0, y index=1]
        {distributions/fourstate-vsmalleps-smallc-dst.dat}; 
        \addplot table[x index=0, y index=2]
        {distributions/fourstate-vsmalleps-smallc-dst.dat}; 
        \draw[thick] ({axis cs:1, 0}|-{rel axis cs:0, 0})
        -- ({axis cs:1, 0}|-{rel axis cs:0, 1});
      \end{axis}
    \end{tikzpicture}    
    \caption{$\epsilon_1 = 0.01$, $\epsilon_2 = 1$, $a = 0.5$,
      $c = 0.2$: $k_\tn{on} = k_\tn{off}$ and
      $k_\tn{on}^F > k_\tn{off}^F$.}
  \end{subfigure}
  \caption[Probability density functions]{Probability density
    functions for selected $(\epsilon_1, \epsilon_2, a, c)$ quadruples
    in the four-state jump-velocity model.}
  \label{fig:prob-dens}
\end{figure}

\begin{table}
  \centering
  \begin{tabular}{ccc}
    \hline
    Parameter & Normal method & Quantile method \\
    \hline
    $a$ & $[0.469, 0.568]$ & $[0.471, 0.564]$ \\
    $c$ & $[0.175, 0.219]$ & $[0.179, 0.225]$ \\
    $\epsilon_1$ & $[0.068, 0.102]$ & $[0.070, 0.099]$ \\
    $\epsilon_2$ & $[0.851, 1.150]$ & $[0.869, 1.164]$\\
    \hline
  \end{tabular}
  \caption[Summary of confidence intervals]{Summary of 95\% confidence
    intervals of model parameters estimated by bootstrapping}
  \label{tab:conf-int}
  % I'm not worried that the true value of eps_1 is outside the
  % quantile confidence interval, it probably isn't a good method when
  % the number of bootstrap trials is less than 100
\end{table}

\begin{figure}
  \centering
  \begin{subfigure}{0.48\textwidth}
    \begin{tikzpicture}
      \begin{axis}[
        legend entries={Simulated data, Full model},
        xlabel=$v^*$,
        ylabel={Probability density},
        ]
        \addplot+[hist=density, fill]
        table[y index=0] {simulations/test4-sim.dat};
        \addplot table[x index=0, y index=1]
        {ml-estimates/test4-est.dat}; 
      \end{axis}
    \end{tikzpicture}
  \end{subfigure}
  \hfill
  \begin{subfigure}{0.48\textwidth}
    \begin{tikzpicture}
      \begin{axis}[
        legend entries={Simulated data, Full model},
        legend pos=south east,
        xlabel=$v^*$,
        ylabel={$P[V < v^*]$},
        ]
        \addplot+[const plot] table[x index=0, y
        expr=\coordindex/1024] {simulations/test4-sim.dat};
        \addplot table[x index=0, y index=2]
        {ml-estimates/test4-est.dat};
      \end{axis}
    \end{tikzpicture}
  \end{subfigure}
  \caption[4-parameter ML estimates]{Results of the ML estimates for
    the 4-parameter model. For this simulated data, the maximum
    likelihood parameters are $a = 0.5107$, $\epsilon_1 = 0.0886$,
    $c = 0.1995$, and $\epsilon_2 = 1.0142$. Total number of simulated
    data points: $N = 1024$.}
  \label{fig:4-par-fit}
\end{figure}

\begin{figure}
  \centering
  \begin{tikzpicture}
    \begin{axis}[
      ytick={1, 2, 3, 4},
      yticklabels={$a$, $\epsilon_1$, $c$, $\epsilon_2$},
      xlabel=Relative error,
      xmin=-.4, xmax=.3,
      ]
      \addplot+ [boxplot] table[y expr=(\thisrowno{0} - 0.5)/0.5]
      {bootstrap/test4-boot.dat};
      \addplot+ [boxplot] table[y expr=(\thisrowno{1} - 0.1)/0.1]
      {bootstrap/test4-boot.dat};
      \addplot+ [boxplot] table[y expr=(\thisrowno{2} - 0.2)/0.2]
      {bootstrap/test4-boot.dat};
      \addplot+ [boxplot] table[y expr=(\thisrowno{3} - 1)]
      {bootstrap/test4-boot.dat}; 
    \end{axis}
  \end{tikzpicture}
  \caption[Boxplots of 4-parameter bootstraps]{Boxplots of parameter
    estimates in the 4-parameter model from 64 bootstrap trials.}
  \label{fig:4-par-boxplots}
\end{figure}

\begin{figure}
  \centering
  \begin{subfigure}{0.48\textwidth}
    \begin{tikzpicture}
      \begin{axis}[
        legend entries={Data, Model fit},
        xlabel = $v^*$,
        ylabel = Probability density,
        ]
        \addplot+[hist=density, fill] table [y index=0]
        {simulations/CCC-vels-sim.dat};
        \addplot table [x index=0, y index=1]
        {ml-estimates/CCC-vels-est.dat}; 
      \end{axis}
    \end{tikzpicture}
  \end{subfigure}
  \hfill
  \begin{subfigure}{0.48\textwidth}
    \begin{tikzpicture}
      \begin{axis}[
        legend entries={Data, Model fit},
        legend pos = south east,
        xlabel=$v^*$,
        ylabel={$P[V < v^*]$},
        ]
        \addplot+[const plot] table[x index=0, y expr=\coordindex/42]
        {simulations/CCC-vels-sim.dat};
        \addplot table[x index=0, y index=2]
        {ml-estimates/CCC-vels-est.dat}; `
      \end{axis}
    \end{tikzpicture}
  \end{subfigure}
  \caption{Model fit to CCC platelets. Parameter estimates are $a =
    0.16$, $\epsilon_1 = 0.13$, $c = 0.84$, $\epsilon_2 = 13.3$.}
  \label{fig:ccc}
\end{figure}

\newpage

\section{Step size in the jump-velocity model}
\label{sec:step-size-jump}

The jump-velocity model predicts that step sizes should be
exponentially distributed in both the two-parameter and four-parameter
models. In the two-parameter model, $\tn{step size} \sim
\Exp\left(\lambda = k_\tn{on}/V\right)$ and in the four-parameter
model $\tn{step size} \sim \Exp\left(\lambda = (k_\tn{on} +
  k_\tn{on}^F)/V \right)$ where $\lambda$ is the rate parameter of the
exponential distribution. The maximum likelihood estimate for the
parameter $\lambda$ is easy, it is just the sample mean of the
observations.

In order to test the hypothesis that the step sizes are exponentially
distributed, I fit exponential curves to the observed step sizes in
four sets of experiments: unprimed platelets in PRP (albumin in
priming region, collagen in rolling region), unprimed platlets in
whole blood, primed platelets in PRP (collagen in both regions), and
primed platelets in whole blood. Then I used the Anderson-Darling
goodness-of-fit test for an exponential distribution.

Plots of the step size distribution at the maximum-likelihood
parameter value are shown in Figures \ref{fig:hc-step-fit} and
\ref{fig:cc-step-fit}. The Anderson-Darling test rejects the
hypothesis that the step sizes are exponentially distributed for both
experiments with $p < 0.01$ for the HC experiment and $p < 0.025$ for
the CC experiment. That is, the test is more than 95\% sure the step
sizes are \emph{not} distributed exponentially in either experiment.

\begin{figure}
  \centering
  \begin{subfigure}{0.48\textwidth}
    \begin{tikzpicture}
      \begin{axis}[
        legend entries={Data, Model fit},
        xlabel = Step size ($\mu$m),
        ylabel = Probability density,
        ]
        \addplot+[hist=density, fill] table [y index=0]
        {simulations/HC-step.dat};
        \addplot table [x index=0, y index=1]
        {ml-estimates/HC-exp.dat}; 
      \end{axis}
    \end{tikzpicture}
  \end{subfigure}
  \hfill
  \begin{subfigure}{0.48\textwidth}
    \begin{tikzpicture}
      \begin{axis}[
        legend entries={Data, Model fit},
        legend pos = south east,
        xlabel = Step size ($\mu$m),
        ylabel = Cumulative Probability,
        ]
        \addplot+[const plot] table[x index=0, y expr=(\coordindex+1)/21]
        {simulations/HC-step.dat};
        \addplot table[x index=0, y index=2]
        {ml-estimates/HC-exp.dat};
        \draw[thick] ({axis cs:1, 0}|-{rel axis cs:0, 0})
        -- ({axis cs:1, 0}|-{rel axis cs:0, 1});
      \end{axis}
    \end{tikzpicture}
  \end{subfigure}
  \caption[HC step size fit]{Data and fit of the step size
    distribution model to HC step size data. ($N = 21$) The vertical
    black line is located at 1 $\mu$m, or approximately one platelet
    radius. $\hat{\mu} = 2.38 \mu\tn{m}$}
  \label{fig:hc-step-fit}
\end{figure}

\begin{figure}
  \centering
  \begin{subfigure}{0.48\textwidth}
    \begin{tikzpicture}
      \begin{axis}[
        legend entries={Data, Model fit},
        xlabel = Step size ($\mu$m),
        ylabel = Probability density,
        ]
        \addplot+[hist=density, fill] table [y index=0]
        {simulations/CC-step.dat};
        \addplot table [x index=0, y index=1]
        {ml-estimates/CC-exp.dat}; 
      \end{axis}
    \end{tikzpicture}
  \end{subfigure}
  \hfill
  \begin{subfigure}{0.48\textwidth}
    \begin{tikzpicture}
      \begin{axis}[
        legend entries={Data, Model fit},
        legend pos = south east,
        xlabel = Step size ($\mu$m),
        ylabel = Cumulative Probability,
        ]
        \addplot+[const plot] table[x index=0, y expr=(\coordindex+1)/40]
        {simulations/CC-step.dat};
        \addplot table[x index=0, y index=2]
        {ml-estimates/CC-exp.dat};
        \draw[thick] ({axis cs:1, 0}|-{rel axis cs:0, 0})
        -- ({axis cs:1, 0}|-{rel axis cs:0, 1});
      \end{axis}
    \end{tikzpicture}
  \end{subfigure}
  \caption[CC step size fit]{Data and fit of the step size
    distribution model to CC step size data. ($N = 40$) The vertical
    black line is located at 1 $\mu$m, or approximately one platelet
    radius. $\hat{\mu} = 2.48 \mu\tn{m}$}
  \label{fig:cc-step-fit}
\end{figure}

\begin{figure}
  \centering
  \begin{subfigure}{0.48\textwidth}
    \begin{tikzpicture}
      \begin{axis}[
        legend entries={Data, Model fit},
        xlabel = Step size ($\mu$m),
        ylabel = Probability density,
        ]
        \addplot+[hist=density, fill] table [y index=0]
        {simulations/HCwhole-step.dat};
        \addplot table [x index=0, y index=1]
        {ml-estimates/HCwhole-exp.dat}; 
      \end{axis}
    \end{tikzpicture}
  \end{subfigure}
  \hfill
  \begin{subfigure}{0.48\textwidth}
    \begin{tikzpicture}
      \begin{axis}[
        legend entries={Data, Model fit},
        legend pos = south east,
        xlabel = Step size ($\mu$m),
        ylabel = Cumulative Probability,
        ]
        \addplot+[const plot] table[x index=0, y expr=(\coordindex+1)/11]
        {simulations/HCwhole-step.dat};
        \addplot table[x index=0, y index=2]
        {ml-estimates/HCwhole-exp.dat};
        \draw[thick] ({axis cs:1, 0}|-{rel axis cs:0, 0})
        -- ({axis cs:1, 0}|-{rel axis cs:0, 1});
      \end{axis}
    \end{tikzpicture}
  \end{subfigure}
  \caption[HC whole blood step size]{Data and fit of the step size
    distribution model to HC whole blood step size data. ($N = 11$)
    The vertical black line is located at 1 $\mu$m, or approximately
    one platelet radius. $\hat{\mu} = 1.98 \mu\tn{m}$}
  \label{fig:hc-step-fit}
\end{figure}

\begin{figure}
  \centering
  \begin{subfigure}{0.48\textwidth}
    \begin{tikzpicture}
      \begin{axis}[
        legend entries={Data, Model fit},
        xlabel = Step size ($\mu$m),
        ylabel = Probability density,
        ]
        \addplot+[hist=density, fill] table [y index=0]
        {simulations/CCwhole-step.dat};
        \addplot table [x index=0, y index=1]
        {ml-estimates/CCwhole-exp.dat}; 
      \end{axis}
    \end{tikzpicture}
  \end{subfigure}
  \hfill
  \begin{subfigure}{0.48\textwidth}
    \begin{tikzpicture}
      \begin{axis}[
        legend entries={Data, Model fit},
        legend pos = south east,
        xlabel = Step size ($\mu$m),
        ylabel = Cumulative Probability,
        ]
        \addplot+[const plot] table[x index=0, y expr=(\coordindex+1)/76]
        {simulations/CCwhole-step.dat};
        \addplot table[x index=0, y index=2]
        {ml-estimates/CCwhole-exp.dat};
        \draw[thick] ({axis cs:1, 0}|-{rel axis cs:0, 0})
        -- ({axis cs:1, 0}|-{rel axis cs:0, 1});
      \end{axis}
    \end{tikzpicture}
  \end{subfigure}
  \caption[CC whole blood step size]{Data and fit of the step size
    distribution model to CC whole blood step size data. ($N = 76$)
    The vertical black line is located at 1 $\mu$m, or approximately
    one platelet radius. $\hat{\mu} = 2.37 \mu\tn{m}$.}
  \label{fig:cc-step-fit}
\end{figure}

\begin{figure}
  \centering
  \begin{subfigure}{0.48\textwidth}
    \begin{tikzpicture}
      \begin{semilogyaxis}[
        legend entries={Data, Model fit},
        legend pos = north east,
        xlabel = Step size ($\mu$m),
        ylabel = 1 - CDF,
        ]
        \addplot+[const plot] table[x index=0, y expr=1 -
        (\coordindex+1)/21] {simulations/HC-step.dat};
        \addplot table[x index=0, y expr=exp(-1/2.385*\thisrowno{0})]
        {simulations/HC-step.dat}; 
      \end{semilogyaxis}
    \end{tikzpicture}
    \caption{HC PRP experiment}
    \label{fig:log-cdf-hc}
  \end{subfigure}
  \hfill
  \begin{subfigure}{0.48\textwidth}
    \begin{tikzpicture}
      \begin{semilogyaxis}[
        legend entries={Data, Model fit},
        legend pos = north east,
        xlabel = Step size ($\mu$m),
        ylabel = 1 - CDF,
        ]
        \addplot+[const plot] table[x index=0, y expr=1 -
        (\coordindex+1)/40] {simulations/CC-step.dat};
        \addplot table[x index=0, y expr=exp(-1/2.476*\thisrowno{0}]
        {simulations/CC-step.dat};
      \end{semilogyaxis}
    \end{tikzpicture}
    \caption{CC PRP experiment}
    \label{fig:log-cdf-cc}
  \end{subfigure}
  \\
  \begin{subfigure}{0.48\textwidth}
    \begin{tikzpicture}
      \begin{semilogyaxis}[
        legend entries={Data, Model fit},
        legend pos = north east,
        xlabel = Step size ($\mu$m),
        ylabel = 1 - CDF,
        ]
        \addplot+[const plot] table[x index=0, y expr=1 -
        (\coordindex+1)/11] {simulations/HCwhole-step.dat};
        \addplot table[x index=0, y expr=exp(-1/1.981*\thisrowno{0})]
        {simulations/HCwhole-step.dat}; 
      \end{semilogyaxis}
    \end{tikzpicture}
    \caption{HC whole blood experiment}
    \label{fig:log-cdf-hcwhole}
  \end{subfigure}
  \hfill
  \begin{subfigure}{0.48\textwidth}
    \begin{tikzpicture}
      \begin{semilogyaxis}[
        legend entries={Data, Model fit},
        legend pos = north east,
        xlabel = Step size ($\mu$m),
        ylabel = 1 - CDF,
        ]
        \addplot+[const plot] table[x index=0, y expr=1 -
        (\coordindex+1)/76] {simulations/CCwhole-step.dat};
        \addplot table[x index=0, y expr=exp(-1/2.375*\thisrowno{0})]
        {simulations/CCwhole-step.dat};
      \end{semilogyaxis}
    \end{tikzpicture}
    \caption{CC whole blood experiment}
    \label{fig:log-cdf-ccwhole}
  \end{subfigure}
  \caption{Log plots of $1 - \tn{CDF}$ for each of the 4 experiments,
    with a simple exponential fit.}
  \label{fig:semilogy-cdf}
\end{figure}

\begin{figure}
  \centering
  \begin{subfigure}{0.48\textwidth}
    \begin{tikzpicture}
      \begin{axis}[
        legend entries={Data, Model fit},
        xlabel = Step size ($\mu$m),
        ylabel = Probability density,
        ]
        \addplot+[hist=density, fill] table [y index=0]
        {simulations/HC-step.dat};
        \addplot table [x index=0, y index=1]
        {ml-estimates/HC-dexp.dat}; 
      \end{axis}
    \end{tikzpicture}
  \end{subfigure}
  \hfill
  \begin{subfigure}{0.48\textwidth}
    \begin{tikzpicture}
      \begin{axis}[
        legend entries={Data, Model fit},
        legend pos = south east,
        xlabel = Step size ($\mu$m),
        ylabel = Cumulative Probability,
        ]
        \addplot+[const plot] table[x index=0, y expr=(\coordindex+1)/21]
        {simulations/HC-step.dat};
        \addplot table[x index=0, y index=2]
        {ml-estimates/HC-dexp.dat};
        \draw[thick] ({axis cs:1, 0}|-{rel axis cs:0, 0})
        -- ({axis cs:1, 0}|-{rel axis cs:0, 1});
      \end{axis}
    \end{tikzpicture}
  \end{subfigure}
  \caption[HC step size fit]{Data and fit of the step size
    distribution model to HC step size data. ($N = 21$) The vertical
    black line is located at 1 $\mu$m, or approximately one platelet
    radius. $\alpha = 0.764$,
    $\hat{\lambda_1} = 1.462 \mu\tn{m}^{-1}$,
    $\hat{\lambda_2} = 0.127 \mu\tn{m}^{-1}$.}
  \label{fig:hc-step-fit}
\end{figure}

\begin{figure}
  \centering
  \begin{subfigure}{0.48\textwidth}
    \begin{tikzpicture}
      \begin{axis}[
        legend entries={Data, Model fit},
        xlabel = Step size ($\mu$m),
        ylabel = Probability density,
        ]
        \addplot+[hist=density, fill] table [y index=0]
        {simulations/CC-step.dat};
        \addplot table [x index=0, y index=1]
        {ml-estimates/CC-dexp.dat}; 
      \end{axis}
    \end{tikzpicture}
  \end{subfigure}
  \hfill
  \begin{subfigure}{0.48\textwidth}
    \begin{tikzpicture}
      \begin{axis}[
        legend entries={Data, Model fit},
        legend pos = south east,
        xlabel = Step size ($\mu$m),
        ylabel = Cumulative Probability,
        ]
        \addplot+[const plot] table[x index=0, y expr=(\coordindex+1)/40]
        {simulations/CC-step.dat};
        \addplot table[x index=0, y index=2]
        {ml-estimates/CC-dexp.dat};
        \draw[thick] ({axis cs:1, 0}|-{rel axis cs:0, 0})
        -- ({axis cs:1, 0}|-{rel axis cs:0, 1});
      \end{axis}
    \end{tikzpicture}
  \end{subfigure}
  \caption[CC step size fit]{Data and fit of the step size
    distribution model to CC step size data. ($N = 40$) The vertical
    black line is located at 1 $\mu$m, or approximately one platelet
    radius. $\alpha = 0.781$,
    $\hat{\lambda_1} = 0.756 \mu\tn{m}^{-1}$,
    $\hat{\lambda_2} = 0.152 \mu\tn{m}^{-1}$.}
  \label{fig:cc-step-fit}
\end{figure}

\begin{figure}
  \centering
  \begin{subfigure}{0.48\textwidth}
    \begin{tikzpicture}
      \begin{axis}[
        legend entries={Data, Model fit},
        xlabel = Step size ($\mu$m),
        ylabel = Probability density,
        ]
        \addplot+[hist=density, fill] table [y index=0]
        {simulations/HCwhole-step.dat};
        \addplot table [x index=0, y index=1]
        {ml-estimates/HCwhole-dexp.dat}; 
      \end{axis}
    \end{tikzpicture}
  \end{subfigure}
  \hfill
  \begin{subfigure}{0.48\textwidth}
    \begin{tikzpicture}
      \begin{axis}[
        legend entries={Data, Model fit},
        legend pos = south east,
        xlabel = Step size ($\mu$m),
        ylabel = Cumulative Probability,
        ]
        \addplot+[const plot] table[x index=0, y expr=(\coordindex+1)/11]
        {simulations/HCwhole-step.dat};
        \addplot table[x index=0, y index=2]
        {ml-estimates/HCwhole-dexp.dat};
        \draw[thick] ({axis cs:1, 0}|-{rel axis cs:0, 0})
        -- ({axis cs:1, 0}|-{rel axis cs:0, 1});
      \end{axis}
    \end{tikzpicture}
  \end{subfigure}
  \caption[HC whole blood step size]{Data and fit of the step size
    distribution model to HC whole blood step size data. ($N = 11$)
    The vertical black line is located at 1 $\mu$m, or approximately
    one platelet radius. $\alpha = 0.865$,
    $\hat{\lambda_1} = 1.849 \mu\tn{m}^{-1}$,
    $\hat{\lambda_2} = 0.0676 \mu\tn{m}^{-1}$.}
  \label{fig:hc-step-fit}
\end{figure}

\begin{figure}
  \centering
  \begin{subfigure}{0.48\textwidth}
    \begin{tikzpicture}
      \begin{axis}[
        legend entries={Data, Model fit},
        xlabel = Step size ($\mu$m),
        ylabel = Probability density,
        ]
        \addplot+[hist=density, fill] table [y index=0]
        {simulations/CCwhole-step.dat};
        \addplot table [x index=0, y index=1]
        {ml-estimates/CCwhole-dexp.dat}; 
      \end{axis}
    \end{tikzpicture}
  \end{subfigure}
  \hfill
  \begin{subfigure}{0.48\textwidth}
    \begin{tikzpicture}
      \begin{axis}[
        legend entries={Data, Model fit},
        legend pos = south east,
        xlabel = Step size ($\mu$m),
        ylabel = Cumulative Probability,
        ]
        \addplot+[const plot] table[x index=0, y expr=(\coordindex+1)/76]
        {simulations/CCwhole-step.dat};
        \addplot table[x index=0, y index=2]
        {ml-estimates/CCwhole-dexp.dat};
        \draw[thick] ({axis cs:1, 0}|-{rel axis cs:0, 0})
        -- ({axis cs:1, 0}|-{rel axis cs:0, 1});
      \end{axis}
    \end{tikzpicture}
  \end{subfigure}
  \caption[CC whole blood step size]{Data and fit of the step size
    distribution model to CC whole blood step size data. ($N = 76$)
    The vertical black line is located at 1 $\mu$m, or approximately
    one platelet radius. $\alpha = 0.620$,
    $\hat{\lambda_1} = 1.049 \mu\tn{m}^{-1}$,
    $\hat{\lambda_2} = 0.213 \mu\tn{m}^{-1}$..}
  \label{fig:cc-step-fit}
\end{figure}

\begin{figure}
  \centering
  \begin{subfigure}{0.48\textwidth}
    \begin{tikzpicture}
      \begin{semilogyaxis}[
        legend entries={Data, Model fit},
        legend pos = north east,
        xlabel = Step size ($\mu$m),
        ylabel = 1 - CDF,
        ]
        \addplot+[const plot] table[x index=0, y expr=1 -
        (\coordindex+1)/21] {simulations/HC-step.dat};
        \addplot table[x index=0, y
        expr=.764*exp(-1.462*\thisrowno{0}) +
        .236*exp(-.127*\thisrowno{0})] {ml-estimates/HC-dexp.dat};
      \end{semilogyaxis}
    \end{tikzpicture}
    \caption{HC PRP experiment}
    \label{fig:log-cdf-hc}
  \end{subfigure}
  \hfill
  \begin{subfigure}{0.48\textwidth}
    \begin{tikzpicture}
      \begin{semilogyaxis}[
        legend entries={Data, Model fit},
        legend pos = north east,
        xlabel = Step size ($\mu$m),
        ylabel = 1 - CDF,
        ]
        \addplot+[const plot] table[x index=0, y expr=1 -
        (\coordindex+1)/40] {simulations/CC-step.dat};
        \addplot table[x index=0, y
        expr=.781*exp(-0.756*\thisrowno{0}) +
        .219*exp(-.152*\thisrowno{0})]
        {ml-estimates/CC-dexp.dat};
      \end{semilogyaxis}
    \end{tikzpicture}
    \caption{CC PRP experiment}
    \label{fig:log-cdf-cc}
  \end{subfigure}
  \\
  \begin{subfigure}{0.48\textwidth}
    \begin{tikzpicture}
      \begin{semilogyaxis}[
        legend entries={Data, Model fit},
        legend pos = north east,
        xlabel = Step size ($\mu$m),
        ylabel = 1 - CDF,
        ]
        \addplot+[const plot] table[x index=0, y expr=1 -
        (\coordindex+1)/11] {simulations/HCwhole-step.dat};
        \addplot table[x index=0, y
        expr=.865*exp(-1.849*\thisrowno{0}) +
        .135*exp(-.0676*\thisrowno{0}]
        {ml-estimates/HCwhole-dexp.dat}; 
      \end{semilogyaxis}
    \end{tikzpicture}
    \caption{HC whole blood experiment}
    \label{fig:log-cdf-hcwhole}
  \end{subfigure}
  \hfill
  \begin{subfigure}{0.48\textwidth}
    \begin{tikzpicture}
      \begin{semilogyaxis}[
        legend entries={Data, Model fit},
        legend pos = north east,
        xlabel = Step size ($\mu$m),
        ylabel = 1 - CDF,
        ]
        \addplot+[const plot] table[x index=0, y expr=1 -
        (\coordindex+1)/76] {simulations/CCwhole-step.dat};
        \addplot table[x index=0, y
        expr=.62*exp(-1.049*\thisrowno{0}) +
        .38*exp(-.213*\thisrowno{0}]
        {ml-estimates/CCwhole-dexp.dat};
      \end{semilogyaxis}
    \end{tikzpicture}
    \caption{CC whole blood experiment}
    \label{fig:log-cdf-ccwhole}
  \end{subfigure}
  \caption{Log plots of $1 - \tn{CDF}$ for each of the 4 experiments,
    with a double exponential fit.}
  \label{fig:semilogy-cdf}
\end{figure}

What are potential reasons for this discrepancy?
\begin{enumerate}
\item The velocity of free-flowing platelets is not constant. In
  reality, the velocity of a platelet can vary depending on its
  distance from the vessel wall, collisions with other cells, and
  probably other factors I'm not thinking of right now.
\item In reality, multiple bonds can form between the platelet and the
  wall, and small steps can occur even when a platelet doesn't fully
  detach from the wall. Based on this, we would expect to find a
  higher proportion of small steps ($< 1 \mu\tn{m}$) than predicted by
  an exponential distribution.
\item If the platelets can be in either a primed or an unprimed state,
  the step size distribution will no longer be an exponential
  distribution. Instead it will be something like a sum of
  exponentials.
\end{enumerate}

Out of curiosity, I made a log plot of $1 - \tn{CDF}$ for the CC whole
blood experiment (simply because it has the most data) in Figure
\ref{fig:log-cdf}. It is hard to say, but it \emph{might} be fit well
by a piecewise linear function, which would suggest the step size
distribution is a sum of exponentials.

\begin{figure}
  \centering
  \begin{subfigure}{0.48\textwidth}
    \includegraphics[width=\textwidth]{CCwhole-velfit.png}
    \caption{Velocity data}
  \end{subfigure}
  \hfill
  \begin{subfigure}{0.48\textwidth}
    \includegraphics[width=\textwidth]{CCwhole-stepfit.png}
    \caption{Step data}
  \end{subfigure}
  \\
  \begin{subfigure}{0.48\textwidth}
    \includegraphics[width=\textwidth]{CCwhole-dwellfit.png}
    \caption{Pause data}
  \end{subfigure}
  \caption{Fit of velocity model to CC whole blood data and compared
    with step size and dwell time distribution. Fit parameters:
    $\hat{a} = 0.126$, $\hat{\epsilon_1} = 0.11$, $\hat{c} = 0.635$,
    $\hat{\epsilon_2} = 37.5$.}
  \label{fig:ccwhole-velfit}
\end{figure}

\begin{figure}
  \centering
  \begin{subfigure}{0.48\textwidth}
    \includegraphics[width=\textwidth]{HC-velfit.png}
    \caption{Velocity data}
  \end{subfigure}
  \hfill
  \begin{subfigure}{0.48\textwidth}
    \includegraphics[width=\textwidth]{HC-dwellfit.png}
    \caption{Pause data}
  \end{subfigure}
  \caption{Fit of velocity model to HC PRP data and compared
    with step size and dwell time distribution}
  \label{fig:hc-velfit}
\end{figure}

\section{Dwell time in the jump-velocity model}
\label{sec:dwell-time-jump}

I haven't fit any of the dwell time data yet, but I think I have
figured out the theoretical dwell time distribution for the
four-parameter model. Obviously in the two-parameter model, $\tn{dwell
  time} \sim \Exp\left(\lambda = k_\tn{off}\right)$.

For the four-parameter model, finding the dwell time distribution
requires solving an ODE that models transitions between the three
bound states. That is, the dwell time distribution is found by solving
the system:
\begin{align}
  \Der{p_V}{t} &= -(k_\tn{off} + k_\tn{on}^F)p_V + k_\tn{off}^F p_{VF}
  \\
  \Der{p_F}{t} &= -(k_\tn{on} + k_\tn{off}^F)p_F + k_\tn{off} p_{VF}
  \\
  \Der{p_{VF}}{t} &= k_\tn{on}^F p_V + k_\tn{on} p_F - (k_\tn{off} +
                    k_\tn{off}^F) p_{VF},
\end{align}
and the dwell time distribution is given by
$k_\tn{off} p_V(t) + k_\tn{off}^F p_F(t)$. This ODE is solved with the
initial condition $p_V(0) = P[\tn{the V bond forms first}]$,
$p_F(0) = P[\tn{the F bond forms first}]$, and $p_{VF}(0) = 0$.

% \bibliographystyle{plain}
% \bibliography{/Users/andrewwork/Documents/grad-school/thesis/library}

\end{document}




