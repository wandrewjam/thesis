\documentclass{article}

\newcommand{\ep}{\rule{.06in}{.1in}}
\textheight 9.5in

\usepackage{amssymb}
\usepackage{amsmath}
\usepackage{amsthm}
\usepackage{graphicx}

\pagestyle{empty} 
\oddsidemargin -0.25in
\evensidemargin -0.25in 
\topmargin -0.75in 
\parindent 0pt
\parskip 12pt
\textwidth 7in
%\font\cj=msbm10 at 12pt

\newcommand{\tn}{\textnormal}
\newcommand{\stiff}{\frac{k_\tn{f}}{\gamma}}
\newcommand{\dd}{d}
\newcommand{\Der}[2]{\frac{\dd #1}{\dd #2}}
\newcommand{\Pder}[2]{\frac{\partial #1}{\partial #2}}
\newcommand{\Integral}[4]{\int_{#3}^{#4} {#1} \dd #2}

\newcommand{\xdiff}{\frac{\partial^2}{\partial x^2}}
\newcommand{\zdiff}{\frac{\partial^2}{\partial z^2}}

\def\R{\mathbb{R}}
\def\N{\mathbb{N}}
\def\C{\mathbb{C}}
\def\Z{\mathbb{Z}}
\def\Q{\mathbb{Q}}
\def\H{\mathbb{H}}
\def\B{\mathcal{B}} 
%\topmargin -.5in 

\begin{document}
\pagestyle{empty}


\begin{center}
  {\Large Notes on Mean Binding Time of a Diffusing Cell}
\end{center}

\section{Langevin and Fokker-Planck Equations}

\begin{figure}
  \centering
  \includegraphics[width=.4\textwidth]{binding-schematic}
  \caption{A diffusing cell with a diffusing receptor head}
  \label{fig:binding-schematic}
\end{figure}

Our goal in these notes is to find the mean first binding time of a
diffusing cell with a receptor with a diffusing head. The problem
setup is sketched in Figure \ref{fig:binding-schematic}. The cell is a
rectangle with a single receptor attached to the bottom face. The
orientation of the cell is fixed, and the variable $x$ is the distance
from the bottom face of the cell to the wall. We assume the cell
diffuses perpendicular to the wall with a diffusion coefficient of
$D_\tn{cell} \equiv D_c$. Therefore $x(t)$ is a continuous time random
variable described by the Langevin equation
\begin{equation}
  dx(t) = \sqrt{2D_c}dW.
  \label{eq:lang-x}
\end{equation}

We also assume the head of the receptor is diffusing perpendicular to
the wall and is attached to a linear spring with a rest length of
$\lambda$, so that the restoring force on the receptor is proportional
to the deviation from its rest length. Let $z(t)$ denote the length of
the spring, then $z$ is a random variable described by the Langevin
equation
\begin{equation}
  dz(t) = -\stiff (z(t) - \lambda) dt + \sqrt{2D_s} dW.
  \label{eq:lang-z}
\end{equation}

\subsection{Boundary conditions}

$x(t)$ and $z(t)$ can be any positive quantity, so we have to define
boundary conditions at $x = 0$ and $z = 0$. Both of these boundaries
we assume are reflecting boundaries. We also assume the curve $x = z$
is an absorbing boundary, that is as soon as the receptor head
contacts the wall it binds.

\subsection{Fokker-Planck equation}

With the Langevin equations and boundary conditions specified above,
we get a Fokker-Planck equation for the PDF of the system state $(x,
z)$ at time $t$, given a previous state $(x_0, z_0)$ at previous time
$t_0$:
\begin{equation}
  \Pder{}{t}p(x, z, t \mid x_0, z_0, t_0) = \left[D_c \xdiff + D_s \zdiff -
    \stiff \Pder{}{z} (z - \lambda)\right] p(x, z, t \mid x_0, z_0, t_0).
  \label{eq:fokker-planck}
\end{equation}

Equation (\ref{eq:fokker-planck}) is solved on the domain $\Omega =
\{(x, z) \in \R^2  \mid  z \ge 0 \tn{ and } x \ge z\}$ for times $t >
t_0$. The boundary conditions are:
\begin{align}
  \label{eq:x-noflux}
  0 &= \Pder{}{x}p(x, z, t \mid x_0, z_0, t_0)_{|x = 0} \\
  \label{eq:z-noflux}
  0 &= \left[D_s \Pder{}{z} - \stiff (z - \lambda)\right]p(x, z,
  t \mid x_0, z_0, t_0) _{|z = 0} \\
  \label{eq:xz-absorb}
  0 &= p(x, x, t \mid x_0, z_0, t_0) \quad \tn{for} \quad x > 0.
\end{align}

The Fokker-Planck equation is solved for the initial condition $p(x,
z, t_0 \mid x_0, z_0, t_0) = \delta(x - x_0, z - z_0)$.

Again, our goal is to find the mean first time the head of the
receptor binds to the wall. However in order to solve this problem, we
first need to know the \emph{backward Kolmogorov equation}, that is
the equation that defines the conditional probability $p(x, z, t \mid x_0,
z_0, t_0)$ as functions of $x_0$ and $z_0$.

\section{The Backward Equation}

Following Dr. Keener's notes, the backward equation is derived using
the Chapman-Kolmogorov equation:
\begin{equation}
  p(x, z, t \mid x_0, z_0, t_0) = \Integral{\Integral{p(x, z, t \mid x', z', t')
      p(x', z', t' \mid x_0, z_0, t_0)}{x'}{z'}{\infty}}{z'}{0}{\infty}.
  \label{eq:chap-kolm}
\end{equation}

This equation just follows from the Markov property of the stochastic
process. Differentiating both sides of equation (\ref{eq:chap-kolm})
with respect to $t'$ gives
\begin{equation}
  0 = \Integral{\Integral{\left(
      \Der{}{t'}p(x, z, t \mid x', z', t') p(x', z', t' \mid x_0, z_0, t_0)
      + p(x, z, t \mid x', z', t') \Der{}{t'}p(x', z', t' \mid x_0, z_0,
      t_0)\right)
    }{x'}{z'}{\infty}}{z'}{0}{\infty}.
\end{equation}

For convenience, define the linear partial differential operator
$L_{x, z}$:
\begin{equation}
  L_{x, z} \equiv \left[D_c \xdiff + D_s \zdiff - \stiff \Pder{}{z} (z
    - \lambda)\right].
  \label{eq:L-defn}
\end{equation}

Then applying equation (\ref{eq:fokker-planck}) to the second term gives 
\begin{equation}
  0 = \Integral{\Integral{\left(\Der{}{t'}p(x, z, t \mid x', z', t') p(x',
      z', t' \mid x_0, z_0, t_0) + p(x, z, t \mid x', z', t') L_{x', z'} p(x',
      z', t' \mid x_0, z_0, t_0)\right)}{x'}{z'}{\infty}}{z'}{0}{\infty}.
\label{eq:back-deriv1}
\end{equation}

Then integrate by parts the second term in equation
(\ref{eq:back-deriv1}):
\begin{equation}
  0 = \Integral{\Integral{p(x', z', t' \mid x_0, z_0, t_0) \left[\Der{}{t'}
        + L^\dag_{x', z'}\right] p(x, z, t \mid x', z',
      t')}{x'}{z'}{\infty}}{z'}{0}{\infty}
  + \text{boundary terms}.
  \label{eq:back-deriv2}
\end{equation}
where $L^\dag_{x', z'} = D_c \frac{\partial^2}{\partial x'^2} + D_s
\frac{\partial^2}{\partial z'^2} + \stiff (z' - \lambda) \Pder{}{z'}$.

The boundary terms that appear as a result of integrating by parts
are:
\begin{multline}
  \Integral{D_c \left[p(x, z, t \mid x', z', t') \Pder{}{x'} p(x', z',
      t' \mid x_0, z_0, t_0) - \Pder{}{x'} p(x, z, t \mid x', z', t') p(x_0,
      z_0, t_0 \mid x', z', t')\right]_{x' = z'}^\infty}{z'}{0}{\infty} \\
  + \Integral{\left[p(x, z, t \mid x', z', t') \left(D_s \Pder{}{z'} p(x',
      z', t' \mid x_0, z_0, t_0) - \stiff(z' - \lambda)p(x', z', t' \mid x_0,
      z_0, t_0)\right)\right]_{z' = 0}^{x'}}{x'}{0}{\infty} \\
  - \Integral{\left[\Pder{}{z'}p(x, z, t \mid x', z', t') p(x', z', t' \mid x_0,
      z_0, t_0)\right]_{z' = 0}^{x'}}{x'}{0}{\infty}.
\end{multline}

This can be simplified by applying known boundary conditions. In
particular we know that $p(x', z', t' \mid x_0, z_0, t_0)_{|x' = z'} = 0$,
$\left(D_s \Pder{}{z'} p(x', z', t' \mid x_0, z_0, t_0) - \stiff (z' -
\lambda) p(x', z', t' \mid x_0, z_0, t_0)\right)_{|z' = 0}$, and $p(x', z',
t' \mid x_0, z_0, t_0) \rightarrow 0$ and $\Pder{}{z'}p(x', z', t' \mid x_0,
z_0, t_0) \rightarrow 0$ as $z' \rightarrow \infty$. After applying
these and setting the boundary terms to 0:
\begin{multline}
  \label{eq:bdy-terms}
  0 = \Integral{D_c \left(p(x, z, t \mid x', z', t') \Pder{}{x'}p(x',
        z', t' \mid x_0, z_0, t_0)\right)_{|x' = z'}}{z'}{0}{\infty}
  \\
  + \Integral{\left(\Pder{}{z'}p(x, z, t \mid x', z', t') p(x', z',
        t' \mid x_0, z_0, t_0)\right)_{|z'=0}}{x'}{0}{\infty}.
\end{multline}

Because the choices of $x'$, $z'$, $t'$ are arbitrary, equation
(\ref{eq:bdy-terms}) implies the following boundary conditions on the
backward equation:
\begin{align}
  \label{eq:bck-absorb}
  0 &= p(x, z, t \mid x', z', t)_{|x' = z'} \\
  \label{eq:bck-reflect}
  0 &= \Pder{}{z'} p(x, z, t \mid x', z', t')_{|z' = 0}.
\end{align}

After cancelling the boundary terms in equation
(\ref{eq:back-deriv2}), and again because $x', z', t'$ are all
arbitrary, we get the \emph{backward equation}
\begin{equation}
  \Pder{}{t'} p(x, z, t \mid x', z', t') = -L^\dag_{x', z'} p(x, z, t \mid x',
  z', t')
  \label{eq:backward}
\end{equation}
for boundary conditions (\ref{eq:bck-absorb}) and
(\ref{eq:bck-reflect}), and for $t' < t$ and $(x', z') \in \Omega$.

\section{Mean First Binding Time}

Now define the function $G(x', z', t)$ to be the probability that the
cell is still unbound at time $t$ given that it was initially in state
$(x, z)$ at time $0$:
\begin{equation}
  G(x', z', t) \equiv \iint_\Omega p(x, z, t \mid x', z', 0) dx dz.
  \label{eq:G-defn}
\end{equation}

If $T(x', z')$ is the random variable for the first binding time, then
the complement of the CDF is
\begin{equation}
  P(T(x', z') > t) = G(x', z', t) = -\Integral{\Pder{}{s} G(x', z',
    s)}{s}{t}{\infty}
  \label{eq:cdf-complement}
\end{equation}
and the CDF is 
\begin{equation}
  P(T(x', z') < t) = -\Integral{\Pder{}{s} G(x', z', s)}{s}{0}{t}.
  \label{eq:cdf}
\end{equation}

Therefore the PDF of $T(x', z')$ is $-\Pder{}{t} G(x', z', t)$ and the
mean first binding time is given by
\begin{equation}
  E(T(x', z')) = -\Integral{t \Pder{}{t} G(x', z', t)}{t}{0}{\infty}
  \label{eq:expect}
\end{equation}
is the expected value of $T(x', z')$. Then because it is a
time-autonomous process, we can shift time arbitrarily. In particular,
$p(x, z, t \mid x', z', 0) = p(x, z, 0 \mid x', z', -t)$. Then
\begin{align}
  \Pder{}{t} G(x', z', t) &= \iint_\Omega \Pder{}{t} p(x, z, 0 \mid x', z',
  -t) dx dz \\
  &= -\iint_\Omega L^\dag_{x', z'} p(x, z, 0 \mid x', z', -t) dx dz \\
  &= L^\dag_{x', z'} G(x', z', t)
  \label{eq:mbt-deriv}
\end{align}



Define $S(x', z') = E(T(x', z'))$ to be the mean first binding time,
and integrate equation (\ref{eq:mbt-deriv}) with respect to $t$. Then
\begin{equation}
  -1 = \Integral{\Pder{}{t}G(x', z', t)}{t}{0}{\infty}
  = \Integral{L^\dag_{x', z'} G(x', z', t)}{t}{0}{\infty}
  = L^\dag_{x', z'} \Integral{G(x', z', t)}{t}{0}{\infty}
  = L^\dag_{x', z'} S(x', z').
  \label{eq:mfbt}
\end{equation}

Then the mean first binding time for a cell starting in position $(x',
z')$ can be found by integrating equation (\ref{eq:mfbt}). There are
still boundary conditions to work out.

\end{document}




