\documentclass{article}

\newcommand{\ep}{\rule{.06in}{.1in}}
\textheight 9.5in

\usepackage{amssymb}
\usepackage{amsmath}
\usepackage{graphicx}
\usepackage{subcaption}
\usepackage{pgfplots}
\usepackage{textgreek}

\newcommand{\tn}{\textnormal}

\pagestyle{empty} 
\oddsidemargin -0.25in
\evensidemargin -0.25in 
\topmargin -0.75in 
\parindent 0pt
\parskip 12pt
\textwidth 7in
%\font\cj=msbm10 at 12pt
%\topmargin -.5in

\usepackage{textgreek}
%%% This is model-defs.tex
%%%
%%% Define symbols for model parameters here so that it is
%%% straightforward to change notation if necessary.

%% Coordinates
\newcommand{\wallDist}{x}
\newcommand{\ndWallDist}{z}
\newcommand{\recAngle}{\theta}
\newcommand{\dTime}{t}
\newcommand{\ndTime}{s}

%% Unknown Functions
\newcommand{\velocity}{V}
\newcommand{\rotation}{\Omega}
\newcommand{\ndVelocity}{v}
\newcommand{\ndRotation}{\omega}
\newcommand{\bondDensity}{n}
\newcommand{\ndBondDensity}{m}

%% Model Parameters (Dimensional)
\newcommand{\radius}{R}
\newcommand{\separation}{d}
\newcommand{\height}{h}
\newcommand{\length}{L}
\newcommand{\domLength}{A}
\newcommand{\shear}{\gamma}
\newcommand{\stiffness}{k_f}
\newcommand{\boltzmann}{k_B}
\newcommand{\temp}{T}
\newcommand{\onRate}{k_\tn{on}}
\newcommand{\offRate}{k_\tn{off}}
\newcommand{\onConst}{k_\tn{on}^0}
\newcommand{\offConst}{k_\tn{off}^0}
\newcommand{\refForce}{f_0}
\newcommand{\receptorDensity}{N_T}
\newcommand{\receptorNumber}{N_R}
\newcommand{\appliedVel}{V_f}
\newcommand{\appliedRot}{\Omega_f}
\newcommand{\velFriction}{\xi_V}
\newcommand{\rotFriction}{\xi_\Omega}
\newcommand{\compliance}{\Gamma}
\newcommand{\width}{w}
\newcommand{\viscosity}{\mu}

%% Force and Torque Functions
\newcommand{\horzForce}{f_h}
\newcommand{\torque}{\tau_s}
\newcommand{\horzTotalForce}{F_h}
\newcommand{\totalTorque}{\tau}

%% Force, Velocity, and Resistance Tensors
\newcommand{\forceVec}{\mathbf{F}}
\newcommand{\velVec}{\mathbf{U}}
\newcommand{\resMatrix}{\underline{\underline{R}}}

%% Model Parameters (Nondimensional)
\newcommand{\ndSeparation}{d'}
\newcommand{\ndLength}{\ell}
\newcommand{\ndAppliedRot}{\omega_f}
\newcommand{\ndAppliedVel}{v_f}
\newcommand{\ndOnConst}{\kappa}
\newcommand{\newOnConst}{\kappa_\textnormal{new}}
\newcommand{\onForceScale}{\eta}
\newcommand{\offForceScale}{\delta}
\newcommand{\ndVelFriction}{\eta_v}
\newcommand{\ndRotFriction}{\eta_\omega}

%% Nondimensional Force and Torque Functions
\newcommand{\ndHorzForce}{f_h'}
\newcommand{\ndTorque}{\tau_s'}
\newcommand{\ndHorzTotalForce}{F_h'}
\newcommand{\ndTotalTorque}{\tau'}

%% Shorthands for Chemical Species
\newcommand{\ITA}[1]{\textalpha\textsubscript{#1}}
\newcommand{\ITB}[1]{\textbeta\textsubscript{#1}}
\newcommand{\Ca}{$\tn{Ca}^{++}$}

%% Reynolds Number
\newcommand{\Reynolds}{\mathrm{Re}}

%% Bin Midpoint
\newcommand{\binMidpoint}[1]{\theta^*_{#1}}

\graphicspath{{/Users/andrewwork/thesis/rolling-model/plots/}}

\begin{document}
\pagestyle{empty}

\begin{center}
{\Large Meeting Notes for February 28th, 2019}
\end{center}

These notes contain results from experiments with an initially-bound
platelet, as well as an exploration of anomalous behavior in the
simulations with a large on rate constant.

\begin{table}
  \centering
  \begin{tabular}{lll} \hline
    Name & Value & Source \\ \hline
    $\onConst$ & 5--250 s$^{-1}$ & \\
    $\offConst$ & 5 s$^{-1}$ & \cite{Mody2008b,Fitzgibbon2014} \\
    $\compliance$ & 0.04 nm & \cite{Bhatia2003,Doggett2002} \\
    $\stiffness$ & 100 {\textmugreek}m/nm & \cite{Fitzgibbon2014} \\
    $\receptorDensity$ & 1280/$\pi$ receptors/radian & \cite{Wang2013}
    \\
    \hline
  \end{tabular}
  \caption{Dimensional binding parameters}
  \label{tab:dim-parameters}
\end{table}

The biological binding parameters (Table \ref{tab:dim-parameters})
are estimates of the kinetics of GP1b-vWF bond formation.

I realized my plots from last week were not generated with the correct
estimate of the nondimensional friction coefficients, and so while I
changed the number of receptors in the stochastic model from previous
simulations I did not properly account for this in the friction
coefficients. In particular, the vertical axes on the bond count plots
were incorrect.

I repeated the unbound experiments and those plots are shown below
(Figures \ref{fig:kappa-1-unbound}---~\ref{fig:kappa-50-bound}),
and then also ran the same simulations where the platelet was
initialized with a single bond to the surface (with endpoints $z = 0$
and $\theta = 0$). The deterministic simulations were initialized with
a regularized $\delta$ function:
\begin{equation*}
  m(z, \theta, 0) = \frac{1}{\pi N_T \epsilon^2} \exp\left(
    -\frac{(z^2 + \theta^2)} {\epsilon^2}\right)
\end{equation*}
with $\epsilon = 0.1$.

In both cases where $\kappa = 50$ (Figures \ref{fig:kappa-50-unbound}
\& \ref{fig:kappa-50-bound}), there are enormous nonphysical
oscillations in the velocities. Figure \ref{fig:oscillations} shows
the bond positions at 4 consecutive time points. I suspect these
oscillations are due to the fact that the friction coefficients are
very small: $\mathcal{O}(10^{-6})$. Therefore, even though the total
bond forces on the platelet are ``only'' $\mathcal{O}(10^{-3})$, the
force and torque balance equations yield velocities
$\mathcal{O}(10^3)$. I suspect a finer time discretization will
eliminate this error, but I haven't tested it.

There is also strange behavior in the deterministic solution for the
initially bound case with $\kappa = 1$ (Figure
\ref{fig:kappa-1-bound}), but I don't know what this could be due to.

\begin{figure}[h]
  \centering
  \includegraphics[width=0.9\textwidth]{{M128_N128_tsteps25600_initfree_trials1000_bmax10_cflux0_vf40.4_omf20_kappa1_eta23000_d0.01_delta1_on1_off1_sat1_xiv7.36e-06_xiom9.82e-06_L2.5_T5_sbh0}.png}
  \caption{$\kappa = 1$, initially unbound}
  \label{fig:kappa-1-unbound}
\end{figure}

\begin{figure}[h]
  \centering
  \includegraphics[width=0.9\textwidth]{{M128_N128_tsteps25600_initfree_trials1000_bmax10_cflux0_vf40.4_omf20_kappa10_eta23000_d0.01_delta1_on1_off1_sat1_xiv7.36e-06_xiom9.82e-06_L2.5_T5_sbh0}.png}
  \caption{$\kappa = 10$, initially unbound}
  \label{fig:kappa-10-unbound}
\end{figure}

\begin{figure}[h]
  \centering
  \includegraphics[width=0.9\textwidth]{{M128_N128_tsteps25600_initfree_trials1000_bmax10_cflux0_vf40.4_omf20_kappa50_eta23000_d0.01_delta1_on1_off1_sat1_xiv7.36e-06_xiom9.82e-06_L2.5_T5_sbh0}.png}
  \caption{$\kappa = 50$, initially unbound}
  \label{fig:kappa-50-unbound}
\end{figure}

\begin{figure}[h]
  \centering
  \includegraphics[width=0.9\textwidth]{{M128_N128_tsteps25600_inittbound_trials1000_bmax10_cflux0_vf40.4_omf20_kappa1_eta23000_d0.01_delta1_on1_off1_sat1_xiv7.36e-06_xiom9.82e-06_L2.5_T5_sbh0}.png}
  \caption{$\kappa = 1$, initially bound}
  \label{fig:kappa-1-bound}
\end{figure}

\begin{figure}[h]
  \centering
  \includegraphics[width=0.9\textwidth]{{M128_N128_tsteps25600_inittbound_trials1000_bmax10_cflux0_vf40.4_omf20_kappa10_eta23000_d0.01_delta1_on1_off1_sat1_xiv7.36e-06_xiom9.82e-06_L2.5_T5_sbh0}.png}
  \caption{$\kappa = 10$, initially bound}
  \label{fig:kappa-10-bound}
\end{figure}

\begin{figure}[h]
  \centering
  \includegraphics[width=0.9\textwidth]{{M128_N128_tsteps25600_inittbound_trials1000_bmax10_cflux0_vf40.4_omf20_kappa50_eta23000_d0.01_delta1_on1_off1_sat1_xiv7.36e-06_xiom9.82e-06_L2.5_T5_sbh0}.png}
  \caption{$\kappa = 50$, initially bound}
  \label{fig:kappa-50-bound}
\end{figure}

\begin{figure}[h]
  \centering
  \includegraphics[width=0.5\textwidth]{{oscillations_kappa10_M128_bmax40}.png}
  \caption{A sequence of bond positions during the nonphysical
    oscillations. Each point in the $(z, \theta)$ space represents a
    single bond. The different colors represent 4 consecutive time
    steps in the order: blue, orange, green, red. The black line is $z
    = \sin\theta$; that is, a bond that is oriented pointing straight
    down from the platelet will lie along that line. It is clear from
    the picture that the oscillations are growing in time. (Since the
    axes aren't labeled, the horizontal axis is $z$, and the vertical
    axis is $\sin\theta$.)}
  \label{fig:oscillations}
\end{figure}

\bibliographystyle{plain}
\bibliography{../../library}

\end{document}
