\documentclass{article}
% \let\bb\enumerate
% \let\ee\endenumerate
% \let\bd\description
% \let\ed\enddescription
% \let\ii\item
\let\ds\displaystyle
\newcommand{\ep}{\rule{.06in}{.1in}}
\textheight 9.5in
\usepackage{amssymb}
\usepackage{amsmath}
\usepackage{amsthm}
\pagestyle{empty} 
\oddsidemargin -0.25in
\evensidemargin -0.25in 
\topmargin -0.75in 
\parindent 0pt
\parskip 12pt
\textwidth 7in
%\font\cj=msbm10 at 12pt
\def\R{\mathbb{R}}
\def\N{\mathbb{N}}
\def\C{\mathbb{C}}
\def\Z{\mathbb{Z}}
\def\Q{\mathbb{Q}}
\def\H{\mathbb{H}}
\def\B{\mathcal{B}} 
\def\and{\wedge}
%\def\ior{\vee} %this line can cause problems (in tables, nested lists, others?), comment out if necessary
%\topmargin -.5in 
\begin{document}
\pagestyle{empty}


\begin{center}
{\Large Meeting Notes}

from March 29, 2018
\end{center}

\textbf{Papers Discussed:} Doggett et. al. 2003
and Tokarev et. al. 2011

\textbf{Doggett et. al. 2003:}
\begin{itemize}
\item Why should $q$ be 0? This seems odd, does it imply that
  essentially no normal GP1b receptors are expressed in platelets from
  patients with the PT-vWD? 
\item Go back and look at the concentrations of proteins used. Because
  they reject hypothesis 3, are they suggesting that only single bonds
  form in physiological conditions? How fast do the platelets move away from
  the wall? They say there are about 30 A1 sites$/\mu
  m^2$. \textbf{Answer:} In the ``Tethering frequency and detachment
  assays for microspheres'' section of Materials and Methods in
  Doggett et. al., 2002, they state that ``the coating concentrations
  of beads were chosen that only supported transient adhesive
  events.'' It may also be important that beads were flowing over
  immobilized platelets, and so the forces on the bond are greater
  than they would be in a physiological situation at the same shear
  rate. Also, in these experiments platelets have been treated with
  all kinds of inactivating chemicals and so it isn't clear that their
  finding that only a single bond formed in their experiments has any
  physiological relevance.
\end{itemize}

\textbf{Tokarev et. al. 2011:}
\begin{itemize}
\item What is the relevance of this paper to priming? Sasha talked
  about ``stickiness,'' which seems to have more to do with on and off
  rates of the receptors. 
\item How do kinetic parameters change with priming? One possibility
  is that the number of active integrins increases with activation
\item Maybe Stefanini et. al. has numbers?
\end{itemize}

\end{document}




