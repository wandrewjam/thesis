\documentclass{article}

\newcommand{\ep}{\rule{.06in}{.1in}}
\textheight 9.5in

\usepackage{amssymb}
\usepackage{amsmath}
\usepackage{graphicx}
\usepackage{subcaption}
\usepackage{pgfplots}

\newcommand{\tn}{\textnormal}

\pagestyle{empty} 
\oddsidemargin -0.25in
\evensidemargin -0.25in 
\topmargin -0.75in 
\parindent 0pt
\parskip 12pt
\textwidth 7in
%\font\cj=msbm10 at 12pt
%\topmargin -.5in 
\begin{document}
\pagestyle{empty}

\newcommand{\radius}{R}
\newcommand{\separation}{d}
\newcommand{\stiffness}{k_f}
\newcommand{\boltzmann}{k_B}
\newcommand{\temp}{T}
\newcommand{\onConst}{k_\text{on}}
\newcommand{\offConst}{k_\text{off}}
\newcommand{\refForce}{f_0}
\newcommand{\receptorDensity}{N_T}
\newcommand{\receptorNumber}{N_R}
\newcommand{\appliedRot}{\Omega_f}
\newcommand{\appliedVel}{V_f}
\newcommand{\velFriction}{\xi_V}
\newcommand{\rotFriction}{\xi_\omega}
\newcommand{\compliance}{\Gamma}
\newcommand{\width}{w}
\newcommand{\viscosity}{\mu}

\newcommand{\ndSeparation}{d'}
\newcommand{\ndAppliedRot}{\omega_f}
\newcommand{\ndAppliedVel}{v_f}
\newcommand{\ndOnConst}{\kappa}
\newcommand{\onForceScale}{\eta}
\newcommand{\offForceScale}{\delta}
\newcommand{\ndVelFriction}{\eta_v}
\newcommand{\ndRotFriction}{\eta_\omega}

\newcommand{\ITA}[1]{\textalpha\textsubscript{#1}}
\newcommand{\ITB}[1]{\textbeta\textsubscript{#1}}



\begin{center}
{\Large Meeting Notes for February 21st, 2019}
\end{center}

This document contains notes on estimating $\onConst$ from our meeting
on February 14th, and expands on ways to estimate and verify
$\onConst$ and $\onForceScale$.

\section{Estimating $\onConst$}
\label{sec:estimating-onconst}

In Fitzgibbon et. al. \cite{Fitzgibbon2014}, they estimate bond
formation by using a constant $k_\tn{on}^F$ that is a per-area rate
of platelet binding to a surface. Therefore the rate of bond
formation between the platelet and the surface is given by
$k_\tn{on}^F A$ where $A$ is the area of the platelet close enough to
the surface to bind.

\begin{figure}
  \centering
  \begin{subfigure}{0.48\textwidth}
    \includegraphics[width=\textwidth]{fitz-binding}
    \caption{Model of binding in \cite{Fitzgibbon2014}}
    \label{fig:fitz-binding}
  \end{subfigure}
  \hfill
  \begin{subfigure}{0.48\textwidth}
    \includegraphics[width=\textwidth]{wats-binding}
    \caption{Our model of binding}
    \label{fig:wats-binding}
  \end{subfigure}
  \caption{Fitzgibbon vs. Watson models of binding}
  \label{fig:binding}
\end{figure}

In our model we have a distance-dependent bond formation rate that
gives the rate of bond formation between a single receptor and a
single point on the vessel wall $\onConst \exp \left(-\eta/2 \ell(z,
  \theta)^2 \right)$. Figure \ref{fig:binding} sketches the two
different approaches to modeling binding. In order to use the
Fitzgibbon estimate, we have to find the total rate of bond formation
on our model between the platelet surface and the vessel wall and
equate that to $k_\tn{on}^F A$. Therefore we need to enforce
\begin{equation}
  \label{eq:fm-rate}
  k_\tn{on}^F A = \onConst \sum_j \left( \int_{-L}^{L} \exp
    \left(-\frac{\eta}{2} \ell(z, \theta_j)^2\right) dz \right)
  n_j^\tn{avail}.
\end{equation}

Now $n_j^\tn{avail}$ changes throughout the simulation, but it must
always be less than $n^\tn{max}$ where $n^\tn{max}$ is the total
number of receptors a bin (the receptors are distributed uniformly and
so $n^\tn{max}$ is the same for every bin). Furthermore platelet
receptors do not saturate in the Fitzgibbon model, which supports the
choice to use $n^\tn{max}$ instead of $n_j^\tn{avail}$. Therefore with
a given value for $\eta$, we can find $\onConst$ by rearranging
equation \eqref{eq:fm-rate}:

\begin{equation}
  \label{eq:kon-calc}
  \onConst = k_\tn{on}^F A \bigg/ \left( n^\tn{max} \sum_j \int_{-L}^{L}
    \exp \left(-\frac{\eta}{2} \ell(z, \theta_j)^2 \right) dz \right).
\end{equation}

The location of the $\theta_j$s also changes in each time step of the
simulation, however because we only allow bonds to form from receptors
on the lower semicircle they must always satisfy $-\pi/2 \le \theta_j
\le \pi/2$ for all $j$. The $\theta_j$s are also uniformly spaced, so
let $\Delta \theta \equiv \theta_j - \theta_{j-1}$ (equivalent to the
bin width) and rewrite equation \eqref{eq:kon-calc}:

\begin{equation}
  \label{eq:kon-calc-reimann}
  \onConst = k_\tn{on}^F A \bigg/ \left(\frac{n^\tn{max}}{\Delta
      \theta} \sum_j \Delta \theta \int_{-L}^{L}
    \exp\left(-\frac{\eta}{2} \ell(z, \theta_j)^2 \right) dz \right).
\end{equation}

Note that
\begin{enumerate}
\item $n^\tn{max} \equiv N_T \Delta \theta$ and,
\item the sum over $j$ is a Riemann sum over $[-\pi/2, \pi/2]$
  (approximately). 
\end{enumerate}
With these observations, equation \eqref{eq:kon-calc-reimann} becomes
\begin{equation}
  \label{eq:kon-calc-int}
  \onConst \approx k_\tn{on}^F A \bigg/ \left(N_T
    \int_{-\pi/2}^{\pi/2} \int_{-L}^{L} \exp \left(-\frac{\eta}{2}
      \ell(z, \theta)^2 \right) dz d\theta \right).
\end{equation}

This approximation is now independent of the mesh size on the surface
of the platelet. For $\eta = 2.34 \times 10^4$ (estimated in the
parameters write-up), $N_T = 5,000$, $d = 0.01$, and $k_\tn{on}^F A =
0.4 \tn{s}^{-1}$ from \cite{Fitzgibbon2014}, we get $\onConst \approx .4/3.76
\approx 0.106 \tn{s}^{-1}$. It is worth noting that I also computed
$\onConst$ using equation \eqref{eq:kon-calc} with $N=128$
$\theta$-bins, and the result was accurate to within 3 decimal places
of the $\onConst = 0.106 \tn{s}^{-1}$ estimate from equation
\eqref{eq:kon-calc-int}.


\begin{figure}
  \centering
  \begin{tikzpicture}
    \begin{semilogxaxis}[
      xlabel={$d$},
      minor y tick num=5,
      grid=major,
    ]
      \addplot[blue] table[col sep=comma] {integral.dat};
    \end{semilogxaxis}
  \end{tikzpicture}
  \caption{Values of the integral in equation \eqref{eq:kon-calc-int}
    for a range of $d$ values.}
  \label{fig:int-vals}
\end{figure}

\section{Comparing binding behavior in our model with other rolling models}
\label{sec:comparing-binding}

In order to confirm that our model and parameter choices are
generating realistic rolling behavior, I think it will be useful to
catalog some binding characteristics from previous rolling models and
compare our results with those already published.

Here is a list of quantities I think will be useful to track in our
binding models and are often reported in published rolling papers:
\begin{enumerate}
\item Total number of bonds between the cell and surface
\item Average length of bonds
\item Average force per bond
\item Average bond lifetimes
\item Average distance from cell to surface
\item Values of $k_\tn{on}(L)$ for a range of $L$ values.
\end{enumerate}

\subsection{Number of bonds}
\label{sec:number-bonds}

\subsection{Length, force, and bond lifetime}
\label{sec:length-force-lifetime}

\subsection{Distance between surfaces}
\label{sec:sep-distance}

\subsection{On rate as a function of $L$}
\label{sec:on-rate-fn}


In the current version of our rolling model, we are using $k_\tn{on} =
\onConst \exp \left(-\frac{\stiffness}{2\boltzmann\temp}
  \length^2\right)$. 

\begin{figure}
  \centering
  \includegraphics[width=.6\textwidth]{hammer-fig3}
  \caption{Figure 3 from Hammer \& Apte, 1992 \cite{Hammer1992}. Instantaneous
    translational velocities and the number of bonds are
    plotted as a function of time for a sample run. The number of
    bonds are shown with a dotted line.}
  \label{fig:hammer-apte}
\end{figure}

%% What about other rolling models? (Numbers of bonds)
%% There are a few bonds present in the PAD multiscale model from Wang
%% et. al.
%% Take a look at Dan Hammer's review (2014)
%% What about shear rates? Distances from the wall?

%% I want to compare our estimate of the bond formation function to
%% those used in other rolling models.

%% How many bonds are there in the Fitzgibbon paper?
%% How many receptors are available to bind in the Fitzgibbon model?

%% I want to look at average bond lengths
%% Also, I should plot numbers of bonds, instead of whatever weird
%% nondimensional quantity I'm using.

\bibliographystyle{plain}
\bibliography{../../library}

\end{document}




