\documentclass{article}

\newcommand{\ep}{\rule{.06in}{.1in}}
\textheight 9.5in

\usepackage{amssymb}
\usepackage{amsmath}
\usepackage{graphicx}
\usepackage{subcaption}

\newcommand{\tn}{\textnormal}

\pagestyle{empty} 
\oddsidemargin -0.25in
\evensidemargin -0.25in 
\topmargin -0.75in 
\parindent 0pt
\parskip 12pt
\textwidth 7in
%\font\cj=msbm10 at 12pt
%\topmargin -.5in 
\begin{document}
\pagestyle{empty}

%%% This is model-defs.tex
%%%
%%% Define symbols for model parameters here so that it is
%%% straightforward to change notation if necessary.

%% Coordinates
\newcommand{\wallDist}{x}
\newcommand{\ndWallDist}{z}
\newcommand{\recAngle}{\theta}
\newcommand{\dTime}{t}
\newcommand{\ndTime}{s}

%% Unknown Functions
\newcommand{\velocity}{V}
\newcommand{\rotation}{\Omega}
\newcommand{\ndVelocity}{v}
\newcommand{\ndRotation}{\omega}
\newcommand{\bondDensity}{n}
\newcommand{\ndBondDensity}{m}

%% Model Parameters (Dimensional)
\newcommand{\radius}{R}
\newcommand{\separation}{d}
\newcommand{\height}{h}
\newcommand{\length}{L}
\newcommand{\domLength}{A}
\newcommand{\shear}{\gamma}
\newcommand{\stiffness}{k_f}
\newcommand{\boltzmann}{k_B}
\newcommand{\temp}{T}
\newcommand{\onRate}{k_\tn{on}}
\newcommand{\offRate}{k_\tn{off}}
\newcommand{\onConst}{k_\tn{on}^0}
\newcommand{\offConst}{k_\tn{off}^0}
\newcommand{\refForce}{f_0}
\newcommand{\receptorDensity}{N_T}
\newcommand{\receptorNumber}{N_R}
\newcommand{\appliedVel}{V_f}
\newcommand{\appliedRot}{\Omega_f}
\newcommand{\velFriction}{\xi_V}
\newcommand{\rotFriction}{\xi_\Omega}
\newcommand{\compliance}{\Gamma}
\newcommand{\width}{w}
\newcommand{\viscosity}{\mu}

%% Force and Torque Functions
\newcommand{\horzForce}{f_h}
\newcommand{\torque}{\tau_s}
\newcommand{\horzTotalForce}{F_h}
\newcommand{\totalTorque}{\tau}

%% Force, Velocity, and Resistance Tensors
\newcommand{\forceVec}{\mathbf{F}}
\newcommand{\velVec}{\mathbf{U}}
\newcommand{\resMatrix}{\underline{\underline{R}}}

%% Model Parameters (Nondimensional)
\newcommand{\ndSeparation}{d'}
\newcommand{\ndLength}{\ell}
\newcommand{\ndAppliedRot}{\omega_f}
\newcommand{\ndAppliedVel}{v_f}
\newcommand{\ndOnConst}{\kappa}
\newcommand{\newOnConst}{\kappa_\textnormal{new}}
\newcommand{\onForceScale}{\eta}
\newcommand{\offForceScale}{\delta}
\newcommand{\ndVelFriction}{\eta_v}
\newcommand{\ndRotFriction}{\eta_\omega}

%% Nondimensional Force and Torque Functions
\newcommand{\ndHorzForce}{f_h'}
\newcommand{\ndTorque}{\tau_s'}
\newcommand{\ndHorzTotalForce}{F_h'}
\newcommand{\ndTotalTorque}{\tau'}

%% Shorthands for Chemical Species
\newcommand{\ITA}[1]{\textalpha\textsubscript{#1}}
\newcommand{\ITB}[1]{\textbeta\textsubscript{#1}}
\newcommand{\Ca}{$\tn{Ca}^{++}$}

%% Reynolds Number
\newcommand{\Reynolds}{\mathrm{Re}}

%% Bin Midpoint
\newcommand{\binMidpoint}[1]{\theta^*_{#1}}



\begin{center}
{\Large Meeting Notes (February 14th, 2019)}
\end{center}

\section{Estimating $\onConst$}
\label{sec:estimating-onconst}

In Fitzgibbon et. al. \cite{Fitzgibbon2014}, they estimate bond
formation by using a constant $k_\tn{on}^F$ that is a per-area rate
of platelet binding to a surface. Therefore the rate of bond
formation between the platelet and the surface is given by
$k_\tn{on}^F A$ where $A$ is the area of the platelet close enough to
the surface to bind.

\begin{figure}
  \centering
  \begin{subfigure}{0.48\textwidth}
    \includegraphics[width=\textwidth]{fitz-binding}
    \caption{Model of binding in \cite{Fitzgibbon2014}}
    \label{fig:fitz-binding}
  \end{subfigure}
  \hfill
  \begin{subfigure}{0.48\textwidth}
    \includegraphics[width=\textwidth]{wats-binding}
    \caption{Our model of binding}
    \label{fig:wats-binding}
  \end{subfigure}
  \caption{Fitzgibbon vs. Watson models of binding}
  \label{fig:binding}
\end{figure}

In our model we have a distance-dependent bond formation rate that
gives the rate of bond formation between a single receptor and a
single point on the vessel wall $k_\tn{on}^0 \exp \left(-\eta/2
  \ell(z, \theta)^2 \right)$. Figure \ref{fig:binding} sketches the
two different approaches to modeling binding. In order to use the
Fitzgibbon estimate, we have to find the total rate of bond formation
on our model between the platelet surface and the vessel wall and
equate that to $k_\tn{on}^F A$. Therefore we need to enforce
\begin{equation}
  \label{eq:fm-rate}
  k_\tn{on}^F A = k_\tn{on}^0 \sum_j \left( \int_{-L}^{L} \exp
    \left(-\frac{\eta}{2} \ell(z, \theta_j)^2\right) dz \right)
  n_j^\tn{avail}.
\end{equation}

Now $n_j^\tn{avail}$ changes throughout the simulation, but it must
always be less than $n^\tn{max}$ where $n^\tn{max}$ is the total
number of receptors a bin (the receptors are distributed uniformly and
so $n^\tn{max}$ is the same for every bin). Furthermore platelet
receptors do not saturate in the Fitzgibbon model, which supports the
choice to use $n^\tn{max}$ instead of $n_j^\tn{avail}$. Therefore with
a given value for $\eta$, we can find $k_\tn{on}^0$ by rearranging
equation \eqref{eq:fm-rate}:

\begin{equation}
  \label{eq:kon-calc}
  k_\tn{on}^0 = k_\tn{on}^F A \bigg/ \left( n^\tn{max} \sum_j \int_{-L}^{L}
    \exp \left(-\frac{\eta}{2} \ell(z, \theta_j)^2 \right) dz \right).
\end{equation}

The location of the $\theta_j$s also changes in each time step of the
simulation, however because we only allow bonds to form from receptors
on the lower semicircle they must always satisfy $-\pi/2 \le \theta_j
\le \pi/2$ for all $j$. The $\theta_j$s are also uniformly spaced, so
let $\Delta \theta \equiv \theta_j - \theta_{j-1}$ (equivalent to the
bin width) and rewrite equation \eqref{eq:kon-calc}:

\begin{equation}
  \label{eq:kon-calc-reimann}
  k_\tn{on}^0 = k_\tn{on}^F A \bigg/ \left(\frac{n^\tn{max}}{\Delta
      \theta} \sum_j \Delta \theta \int_{-L}^{L}
    \exp\left(-\frac{\eta}{2} \ell(z, \theta_j)^2 \right) dz \right).
\end{equation}

Note that
\begin{enumerate}
\item $n^\tn{max} \equiv N_T \Delta \theta$ and,
\item the sum over $j$ is a Riemann sum over $[-\pi/2, \pi/2]$
  (approximately). 
\end{enumerate}
With these observations, equation \eqref{eq:kon-calc-reimann} becomes
\begin{equation}
  \label{eq:kon-calc-int}
  k_\tn{on}^0 \approx k_\tn{on}^F A \bigg/ \left(N_T
    \int_{-\pi/2}{\pi/2} \int_{-L}^{L} \exp \left(-\frac{\eta}{2}
      \ell(z, \theta)^2 \right) dz d\theta \right).
\end{equation}

This approximation is now independent of the mesh size on the surface
of the platelet. For $\eta = 2.34 \times 10^4$ (estimated in the
parameters write-up), $N_T = 5,000$, $d = 0.1$, and $k_\tn{on}^F A = 0.4$ from
\cite{Fitzgibbon2014}, 

\bibliographystyle{plain}
\bibliography{../../library}

\end{document}




