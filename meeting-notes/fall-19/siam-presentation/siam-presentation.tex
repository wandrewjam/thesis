% I USED PDFLATEX TO COMPILE THIS PRESENTATION BECAUSE OF THE
%   GRAPHIC FILE TYPES I HAD.
% IF YOU WOULD LIKE TO COMPILE IT YOURSELF, COMMENT OUT
%   ALL OF THE FIGURES AND MOVIE AND YOU WILL BE ABLE TO
%   COMPILE WITH BOTH PDFLATEX AND LATEX+DVIPS.

\documentclass{beamer}
% \usepackage{pgfpages}
% \setbeameroption{show only notes}
\usetheme{Frankfurt}
\usecolortheme{beaver}

\usepackage{time}             % date and time
\usepackage{graphicx,epsfig,subcaption}
% \usepackage[T1]{fontenc}      % european characters
\usepackage{amssymb,amsmath}  % use mathematical symbols
\usepackage{textgreek}
% \usepackage{palatino}         % use palatino as the default font
% \usepackage{multimedia}
% \usepackage{mathrsfs}

% CREATES SHADED INSTEAD OF HIDDEN OVERLAYS
\setbeamercovered{transparent}

% package for customizing figures and backgrounds
\usepackage{tikz, chemfig}
\usetikzlibrary{decorations.pathmorphing}
\newcommand{\tn}{\textnormal}
\newcommand{\dd}{d}
\newcommand{\Der}[2]{\frac{\dd #1}{\dd #2}}
\newcommand{\Pder}[2]{\frac{\partial #1}{\partial #2}}
\newcommand{\Int}[4]{\int_{#3}^{#4} #1 \, \dd #2}

\newcommand{\DimReceptorSaturation}{\Int
  {\bondDensity\left(\wallDist,\recAngle,\dTime\right)} {\wallDist}
  {-\infty} {\infty}}
\newcommand{\NDReceptorSaturation}{\Int
  {\ndBondDensity\left(\ndWallDist,\recAngle,\ndTime\right)}{\ndWallDist}
  {-\infty} {\infty}}
\DeclareMathOperator{\Exp}{Exp}
\newcommand{\radius}{R}
\newcommand{\separation}{d}
\newcommand{\stiffness}{k_f}
\newcommand{\boltzmann}{k_B}
\newcommand{\temp}{T}
\newcommand{\onConst}{k_\text{on}}
\newcommand{\offConst}{k_\text{off}}
\newcommand{\refForce}{f_0}
\newcommand{\receptorDensity}{N_T}
\newcommand{\receptorNumber}{N_R}
\newcommand{\appliedRot}{\Omega_f}
\newcommand{\appliedVel}{V_f}
\newcommand{\velFriction}{\xi_V}
\newcommand{\rotFriction}{\xi_\omega}
\newcommand{\compliance}{\Gamma}
\newcommand{\width}{w}
\newcommand{\viscosity}{\mu}

\newcommand{\ndSeparation}{d'}
\newcommand{\ndAppliedRot}{\omega_f}
\newcommand{\ndAppliedVel}{v_f}
\newcommand{\ndOnConst}{\kappa}
\newcommand{\onForceScale}{\eta}
\newcommand{\offForceScale}{\delta}
\newcommand{\ndVelFriction}{\eta_v}
\newcommand{\ndRotFriction}{\eta_\omega}

\newcommand{\ITA}[1]{\textalpha\textsubscript{#1}}
\newcommand{\ITB}[1]{\textbeta\textsubscript{#1}}



\title{A Jump-Velocity Model of Platelet Rolling}
% \subtitle{An Introduction}
\author{Andrew Watson}
\institute{University of Utah} % COMMAND UNIQUE TO BEAMER
\date{\today}

\begin{document}

\begin{frame}
  \titlepage
  \note[item]{Platelet rolling---important first step in the formation
    of blood clots}
  \note[item]{Talk about a simpler approach to modeling rolling that
    Aaron and I have been working on over the summer}
\end{frame}


%%%%%%%%%%%%%%%%%%%
\section[Outline]{}
%%%%%%%%%%%%%%%%%%%
\begin{frame}
  \tableofcontents

  \note{I want to start by talking about the biology behind platelet
    rolling, and describe some rolling experiments that are being
    conducted by a biomedical engineering group on campus. Then I'll
    describe some previous mathematical models of cell rolling, talk
    about our current model, and finally where we want to go with this
    model}
\end{frame}

%%%%%%%%%%%%%%%%%%%
\section{Background}
%%%%%%%%%%%%%%%%%%%

\begin{frame}
  \frametitle{Platelet activation}
  \centering
  \includegraphics[width=.8\textwidth]{platelet-overview.png}
  \note[item]{Platelets have to be activated to do all the things
    they do to facilitate blood clot formation}
  \note[item]{Agonists---chemicals that cause platelets to activate}
  \note[item]{Soluble agonists released into the bloodstream}
  \note[item]{Immobilized agonists are bound to the vessel wall or
    to platelets. Because these chemical signals don't diffuse in
    the blood, platelets have to be close enough to a surface to
    bind.}
  \note[item]{}
  \note[item]{Immobilized agonists are exposed on damaged vessels;
    not present on healthy vessels}
\end{frame}

\begin{frame}
  \frametitle{Activation through rolling}
  \begin{columns}
    \begin{column}{.49\textwidth}
      \begin{itemize}
      \item Fast receptors---transient binding \& rolling
        \note[item]{``Fast receptors''---receptors with fast binding
          and unbinding kinetics; form bonds quickly while the
          platelet is moving past the wall, but are short-lived}
      \item Agonist-receptor bonds trigger activation signals
        \note[item]{When the receptor is bound to an agonist, it
          triggers intracellular activation signals}
        \note[item]{Enough binding events with the agonist causes the
          platelet to activate}
      \item Slow receptors---firm adhesion
        \note[item]{Platelets also have slow receptors. Part of
          platelet activation involves activation of these slow
          receptors, which mediate stable adhesion.} 
      % \item Examples: fast receptor---GP1b, slow
      %   receptor---\ITA{IIb}\ITB{3}, agonist---vWF
      \end{itemize}
      \note[item]{Be sure to distinguish between transient binding
        and firm adhesion}
    \end{column}
    \begin{column}{.49\textwidth}
      \begin{figure}
        \centering
        \includegraphics[width=\textwidth]{transient-binding}
        \caption{An example of a platelet binding to an immobilized
          agonist}
        \label{fig:transient-binding}
      \end{figure}
      \note[item]{The figure on right shows an example of a
        platelet transiently binding to an immobilized agonist
        vWF. The unbound platelet has fast receptors (GP1b) and
        inactive slow receptors (\ITA{IIb}\ITB{3}). The GP1b can bind
        to the vWF, and trigger platelet activation, causing the slow
        \ITA{IIb}\ITB{3} receptors to activate.} 
    \end{column}
  \end{columns}
\end{frame}

\begin{frame}
  \frametitle{Rolling experiments}
  \begin{figure}
    \centering
    \includegraphics[width=.75\textwidth]{expt-sideview}
    \caption{Side view of the priming microfluidic
      chambers. Eichinger, Ph.D. dissertation, 2016.}
    \label{fig:flow-chamber}
  \end{figure}

  \begin{itemize}
    \note[item]{Activation of platelets through rolling is an
      important concern in biomedical engineering as well}
    \note[item]{Artificial devices implanted in the blood must be
      engineered to not trigger immune or clotting responses}
  % \item Nonlocal effects---experiments from
  %   Proteins-Polymers-Interfaces group led by Dr. Hlady
  % \item Basic idea: platelets can transiently bind to agonists without
  %   firmly adhering, and be primed for full activation downstream
  \item Question: how does priming (i.e. previous exposure to agonist)
    affect rolling behavior?
  \item Setup:
    $\tn{Control} = \tn{Unprimed (inert material in priming region)}$
    vs $\tn{Experimental} = \tn{Primed (agonist in priming region)}$
  \item Data extracted: platelet velocities, step sizes, pause
    times 
  \end{itemize}
  \note[item]{This general setup underpins several different
    experiments they are carrying out. Essentially there are two
    regions printed with agonist, and platelets get primed in the
    upstream region, and then are more likely to adhere to the surface
    downstream in the capture region.}
  \note[item]{They have a set of experiments looking at how priming
    state affects rolling behavior.}
  \note[item]{Extract the following data...}
\end{frame}

\begin{frame}
  \frametitle{Example platelet trajectories}
  \begin{figure}
    \centering \includegraphics[width=.75\textwidth]{plt-trajectories}
    \caption{Examples of platelet trajectories from a single
      experiment. Resolution: ($\Delta t = 80 \tn{ms}$, $\Delta x
      \approx 150\tn{nm} \tn{, maybe?}$)}
    \label{fig:plt-trajectory}
  \end{figure}
  \note[item]{Our goal: simulate platelet trajectories, get some mechanistic
    insight into the rolling process}
\end{frame}

%%%%%%%%%%%%%%%%%%%
\section{Model}
%%%%%%%%%%%%%%%%%%%

\begin{frame}
  \frametitle{A simple model of rolling}
  \begin{figure}
    \centering
    \includegraphics[width=.75\textwidth]{jump-velocity-sketch}
    \caption{A simple stochastic binding model}
    \label{fig:jump-velocity-sketch}
  \end{figure}

  \begin{figure}
    \centering
    \schemestart
    U \arrow{<=>[$k_\tn{on}$][$k_\tn{off}$]} V
    \schemestop\par
    \caption{Two-state reaction diagram for the binding model}
    \label{fig:two-state-scheme}
  \end{figure}
\end{frame}

\begin{frame}
  \frametitle{The word story}
  \begin{center}
    \schemestart
    U \arrow{<=>[$k_\tn{on}$][$k_\tn{off}$]} V
    \schemestop\par    
  \end{center}
  \begin{itemize}
  \item Stochastic switching between states U and V with constant
    rates $k_\tn{on}$ and $k_\tn{off}$
  \item When the platelet is in state U, it translates with
    $\tn{velocity} = V^*$, while platelets in the V state remain stationary
  \item Defines a stochastic process called a jump-velocity process
  \item Easy to define quantities analogous to the experimental data:
    average velocity (of a single trajectory), step sizes, pause times
  \end{itemize}
\end{frame}

\begin{frame}
  \frametitle{A mathematical description}
  \begin{equation}
    \frac{dy}{dt} = X(t), \quad \tn{where } X =
    \begin{cases}
      0 & \tn{when state} = V \\
      1 & \tn{when state} = U 
    \end{cases}
  \end{equation}
  
  \begin{itemize}
  \item Nondimensionalize the model, scaling out $L$ and $V^*$
  \item Nondimensional parameters: $a = \frac{k_\tn{off}}{k_\tn{off} +
      k_\tn{on}}$ and $\epsilon = \frac{V^*/L}{k_\tn{off} + k_\tn{on}}$
  \item $X$ switches between $0$ and $1$ at rates $a/\epsilon$ and
    $b/\epsilon$ where $b = 1 - a$
  \item $y(t)$---continuous, piecewise linear stochastic process
  \item $T^*$---nondimensional crossing time, $y\left(T^*\right) = 1$
  \item $1/T^*$---average velocity
  \end{itemize}
\end{frame}

\begin{frame}
  \frametitle{Four state model}
  \begin{itemize}
  \item Motivation: rolling is mediated by two types of receptors,
    fast receptors and slow receptors
  \item The two state model may be a good model of unprimed platelets,
    because the slow receptors aren't activated
  \item The four state model may be a better choice for primed
    platelets
  \end{itemize}
  \begin{center}
    \schemestart
    U \arrow{<=>[$k_\tn{on}$][$k_\tn{off}$]}
    V \arrow{<=>[*{0}$k_\tn{on}^F$][*{0}$k_\tn{off}^F$]}[270]
    VF \arrow{<=>[$k_\tn{on}$][$k_\tn{off}$]}[180]
    F \arrow{<=>[*{0}$k_\tn{on}^F$][*{0}$k_\tn{off}^F$]}[90]
    \schemestop\par
  \end{center}
\end{frame}

\section{Analysis and Improved Model}
\label{sec:analys-impr-model}

\begin{frame}
  \frametitle{Fokker-Planck equation}
  \begin{itemize}
  \item $p_i(x, t)$---probability that at time $t$, a platelet is in
    position $x$ and state $i$
  \item The Fokker-Planck equation for the two-state model:
    \begin{align}
      \Pder{p_U}{t} &= -\Pder{p_U}{y} - \frac{b}{\epsilon} p_U +
                      \frac{a}{\epsilon} p_V \\ 
      \Pder{p_V}{t} &= \frac{b}{\epsilon} p_U - \frac{a}{\epsilon} p_V
    \end{align}
  \item Initial condition: $p_U(y, 0) = \delta(y)$, $p_V(y, 0) = 0$
  \item $\tn{PDF of crossing times} = p_U(1, t)$
  \end{itemize}
\end{frame}

\begin{frame}
  \begin{figure}
    \centering
    \includegraphics[width=.75\textwidth]{jump-velocity-figure6.pdf}
    \caption{Sample fit of average velocity to simulated data}
    \label{fig:avg-vel-fit}
  \end{figure}
\end{frame}

\begin{frame}
  \frametitle{Step size}
  \begin{itemize}
  \item Step size---information about on rates
  \item In both models, we expect the step sizes to be exponentially
    distributed
  \item Define $\mathcal{S}$ to be the step size random variable, then
    \begin{itemize}
    \item $\mathcal{S} \sim \Exp(b/\epsilon)$ in the two-state model, and
    \item $\mathcal{S} \sim \Exp(b/\epsilon_1 + d/\epsilon_2)$ in the four-state model.
    \end{itemize}
  \item Fit experimental step size data to an exponential distribution
  \end{itemize}
\end{frame}

\begin{frame}
  \begin{figure}
    \centering
    \begin{subfigure}{0.48\textwidth}
      \includegraphics[width=\textwidth]{jump-velocity-figure31.pdf}
    \end{subfigure}
    \hfill
    \begin{subfigure}{0.48\textwidth}
      \includegraphics[width=\textwidth]{jump-velocity-figure32.pdf}
    \end{subfigure}
    \caption{Maximum likelihood fit of an exponential model to step
      size data of primed platelets ($N=76$). The Anderson-Darling
      goodness-of-fit test rejects the hypothesis that the steps are
      exponentially distributed ($p < 0.01$). The plots for the other
      available data are similar (but have fewer data points).}
    \label{fig:step-fit}
  \end{figure}
\end{frame}

\begin{frame}
  \begin{figure}
    \centering
    \includegraphics[width=.75\textwidth]{jump-velocity-figure33.pdf}
    \caption{Log plot of $1 - \tn{CDF}$}
    \label{fig:log-cdf}
  \end{figure}
\end{frame}

\begin{frame}
  \frametitle{Mixture of exponential processes}
  \begin{itemize}
  \item Steps are not exponentially distributed; could be a mixture of
    exponentials: $f(s) = \alpha\lambda_1\exp(-\lambda_1 s) + (1 -
    \alpha)\lambda_2\exp(-\lambda_2 s)$
  \item Source of the mixture?
  \item One possibility: some cells in a primed state, others unprimed
    \begin{itemize}
    \item Not consistent with results of fitting to an exponential
      mixture model
    \end{itemize}
  \item Another possibility: platelets can form multiple bonds, and
    step without unbinding
  \end{itemize}
\end{frame}

\begin{frame}
  \begin{figure}
    \centering
    \begin{subfigure}{0.48\textwidth}
      \includegraphics[width=\textwidth]{jump-velocity-figure43.pdf}
    \end{subfigure}
    \hfill
    \begin{subfigure}{0.48\textwidth}
      \includegraphics[width=\textwidth]{jump-velocity-figure44.pdf}
    \end{subfigure}
    \caption{Maximum likelihood fit of a mixture of two exponentials
      to step size data of primed platelets ($N = 76$). }
    \label{fig:step-fit}
  \end{figure}
\end{frame}

\begin{frame}
  \frametitle{A new model}
  \begin{itemize}
  \item Extend the two-state model to allow bonds to form in different
    positions---fake space
  \end{itemize}

  \begin{figure}
    \centering
    \includegraphics[width=0.7\textwidth]{double-binding-model.png}
    \caption{Schematic of the two-state, two-position model}
    \label{fig:two-state-two-position}
  \end{figure}
\end{frame}

\begin{frame}
  \frametitle{Next steps}
  \begin{itemize}
  \item Implement the two-state, two-position model: stochastic and
    deterministic versions
  \item Fit the new model to experimental data
  \item Long-term/Speculative:
    \begin{itemize}
    \item Include formation of multiple bonds of a single
      type---model platelets that remain bound throughout an
      experiment
    \item Integrate findings into more detailed model of rolling
    \end{itemize}
  \end{itemize}
\end{frame}

\begin{frame}
  \frametitle{Summary}
  \begin{itemize}
  \item Platelet contacts with an agonist-coated surface are important
    in initial stages of blood clot formation
  \item Platelet rolling behavior cannot be described by simple
    first-order binding and unbinding with the surface
  \item Hypothesis: overrepresentation of small steps (relative to an
    exponential distribution) is due to stepping without full
    unbinding
  \end{itemize}
\end{frame}

\section{}
\begin{frame}
  \frametitle{Acknowledgements}
  \begin{itemize}
  \item My advisor and my committee
  \item Biofluids research group
  \item Proteins-Polymers-Interfaces group
  \item Funding---NIH Grant 1R01HL126864 'Upstream Priming of
    Platelets for Adhesion to Biomaterials'
  \end{itemize}
\end{frame}

\end{document}
