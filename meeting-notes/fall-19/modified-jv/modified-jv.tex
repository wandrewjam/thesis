\documentclass{article}

\newcommand{\ep}{\rule{.06in}{.1in}}
\textheight 9.5in

\usepackage{amssymb}
\usepackage{amsmath}
\usepackage{amsthm}
\usepackage{graphicx, subcaption, algorithmic}
\graphicspath{{/Users/andrewwork/thesis/jump-velocity/plots/}}

\usepackage{tikz, pgfplots, chemfig}
\usepgfplotslibrary{colorbrewer, statistics}
\pgfplotsset{
  exact axis/.style={grid=major, minor tick num=4, xlabel=$v^*$,
    legend entries={PDF, CDF},},
  every axis plot post/.append style={thick},
  table/search
  path={/Users/andrewwork/thesis/jump-velocity/dat-files},
  colormap/YlGnBu,
  cycle list/Set1-5,
  legend style={legend cell align=left,},
}
\usepgfplotslibrary{external}
\tikzexternalize

\renewcommand{\arraystretch}{1.2}
\pagestyle{empty} 
\oddsidemargin -0.25in
\evensidemargin -0.25in 
\topmargin -0.75in 
\parindent 0pt
\parskip 12pt
\textwidth 7in
%\font\cj=msbm10 at 12pt

\newcommand{\tn}{\textnormal}
\newcommand{\stiff}{\frac{k_f}{\gamma}}
\newcommand{\dd}{d}
\newcommand{\Der}[2]{\frac{\dd #1}{\dd #2}}
\newcommand{\Pder}[2]{\frac{\partial #1}{\partial #2}}
\newcommand{\Integral}[4]{\int_{#3}^{#4} {#1} \dd #2}
\DeclareMathOperator{\Exp}{Exp}

% Text width is 7 inches

\def\R{\mathbb{R}}
\def\N{\mathbb{N}}
\def\C{\mathbb{C}}
\def\Z{\mathbb{Z}}
\def\Q{\mathbb{Q}}
\def\H{\mathbb{H}}
\def\B{\mathcal{B}} 
%\topmargin -.5in 

\setcounter{secnumdepth}{2}
\begin{document}
\pagestyle{plain}

\begin{center}
  {\Large Notes on a Modification to the Jump-Velocity Rolling Model
    (\today)}
\end{center}

\section{A modification to the single bond model of rolling}
\label{sec:modif-single-bond}

Experimental step size data does not fit a simple one-step model of
binding and unbinding. We want to test the hypothesis that the
distribution of step sizes can be explained by a mixture of two types
of steps: (1) steps where the platelet comes completely unbound from
the surface, and (2) steps where a platelet is multiply bound to the
vessel wall, and the rearmost load-bearing bond breaks, causing the
platelet to lurch forward.

We want to test this with a model that approximates this behavior,
without the expense of tracking an arbitrary number of bonds that can
bind in arbitrary locations along the wall. Therefore, we will use a
model that allows bonds to form in two positions: on the front of the
platelet and on the back of the platelet (see Figure
\ref{fig:four-states}).

\begin{figure}
  \centering
  \includegraphics[width=.7\textwidth]{double-binding-model.png}
  \caption{Four states in the modified jump-velocity model}
  \label{fig:four-states}
\end{figure}

In the unbound state $S_U$, the platelet translates forward at the
free-flowing velocity $V^*$. A bond can form on the back of the
platelet at a constant binding rate $k_\tn{on}$ (or rather, a bond
forms \emph{somewhere} between the platelet and wall, but the platelet
keeps translating forward until the force applied by the bond balances
the drag force on the platelet). Let's call this state $S_b$. This
bond can then break at a rate $k_\tn{off}$, returning the platelet to
the unbound state.

An $S_b$ platelet can form a second bond with the surface on the front
(downstream) end of the platelet at rate $k_\tn{on, f}$, and we'll
call this the $S_{bf}$ state. Then, either the back or the front bonds
of a $S_{bf}$ platelet can break. The back bond breaks at a rate
$k_\tn{off, b}$ and the front bond breaks at a rate $k_\tn{off,
  f}$. If the front bond breaks, the platelet returns to the $S_b$
state, and if the back bond breaks, the platelet is temporarily left
with a single front bond. However, because of the flow, the platelet
is pushed downstream until the front bond becomes a back bond, and the
bond force balances with the drag force on the platelet. We'll call
this state the transition state---$S_T$.

Now we have to make a choice to model how long platelets remain in the
$S_T$ state. One possible choice is to say that every platelet stays
in the $S_T$ state for the exact same length of time and travels the
same distance. While one could argue this agrees geometrically with
the series of pictures drawn in Figure \ref{fig:four-states}, this
isn't a realistic representation of small step sizes in the more
complicated case that bonds can form continuously along the vessel
wall. If we instead say that platelets transition out of state $S_T$
with some rate $\kappa$, that is equivalent (I think) to assuming
that the small step sizes are distributed exponentially with mean step
length $V^*/\kappa$.

How can we choose a reasonable transition rate $\kappa$ out of the
$S_T$ state? One possibility is to define a characteristic step length
$\ell$, and then assert that some fraction $q$ of the small steps must
be smaller than the characteristic step length. To choose a $\kappa$
based on some given $\ell$ and $q$, we can use the CDF of small step
times. For a constant transition rate $\kappa$, the small step times
are distributed as an exponential distribution with mean
$1/\kappa$. Therefore, the CDF of step times is
$F(t) = 1 - \exp(-\kappa t)$. Then $q = 1 - \exp(-\kappa \ell/V^*)$
enforces the condition that the fraction of steps which are less than
the characteristic step size is $q$. Solving this condition for
$\kappa$ gives us
\begin{equation}
  \label{eq:kappa-defn}
  \kappa = -\frac{\ln(1 - q)}{\ell/V^*}.
\end{equation}

\subsection{Fokker-Planck equation}
\label{sec:fokk-planck-equat}

The resulting Fokker-Planck equation of this process is an
advection-reaction system similar to that of the ordinary jump
velocity process. One major difference is that there are 2 states
which translate in the flow, not just one.

Clearly the Fokker-Planck equation will depend on the assumption made
about the distribution of transition times (or small step sizes), but
for the case where transition times are distributed exponentially, the
Fokker-Planck equation is a simple linear advection-reaction PDE:
\begin{align}
  \Pder{p_U}{t} &= -V^* \Pder{p_U}{x} - k_\tn{on} p_U + k_\tn{off} p_b
  \\
  \Pder{p_b}{t} &= k_\tn{on} p_U - (k_\tn{off} + k_\tn{on, f}) p_b +
                  k_\tn{off, f} p_{bf} + \kappa p_T \\
  \Pder{p_{bf}}{t} &= k_\tn{on, f} p_b - (k_\tn{off, f} + k_\tn{off,
                     b}) p_{bf} \\
  \Pder{p_T}{t} &= -V^* \Pder{p_T}{x} + k_\tn{off, b} p_{bf} - \kappa
                  p_T.
\end{align}

If we choose the same nondimensionalization as the first jump-velocity
model, i.e. scaling space by the chamber length $L$ and time by the
time it takes to cross the chamber $T \equiv L/V$, then we end up with the
following nondimensional set of equations:
\begin{align}
  \Pder{p_U}{t} &= - \Pder{p_U}{x} - \beta p_U + \alpha
                  p_b \label{eq:nd-U} \\
  \Pder{p_b}{t} &= \beta p_U - (\alpha + \delta) p_b +
                  \gamma p_{bf} + \lambda p_T \label{eq:nd-B} \\
  \Pder{p_{bf}}{t} &= \delta p_b - (\gamma + \eta)
                     p_{bf} \label{eq:nd-BF} \\
  \Pder{p_T}{t} &= -\Pder{p_T}{x} + \eta p_{bf} - \lambda
                  p_T. \label{eq:nd-T}
\end{align}

There are 6 nondimensional rate constants in this system of equations:
$\alpha = k_\tn{off}T$, $\beta = k_\tn{on}T$, $\gamma = k_\tn{off,
  f}T$, $\delta = k_\tn{on, f}T$, $\eta = k_\tn{off, b} T$, $\lambda =
\kappa T$. We can make further simplifying assumptions to reduce the
number of rate constants if that is required, but this is the general
form.

\subsubsection{Other distributions of small steps}
\label{sec:other-distr-small}

What if we don't make the assumption that the small step sizes are
distributed exponentially? How can we implement other distributions of
small step sizes? One possibility is to use a distributed
delay. Briefly, distributed delay differential equations extend the
concept of delay differential equations by allowing for a probability
distribution of delays, instead of specifying a single (or multiple)
discrete, deterministic delay(s). In the following section, I develop
the system of equations that I think should describe the rolling model
with a general probability distribution $f(x)$ of small step sizes.

First, let's consider the case where every platelet spends the exact
same time $\tau$  in the transition state, that is the small steps are
deterministic and length $\tau$ (in the nondimensional model). The
Fokker-Planck equation remains the same, except for the $\lambda p_T$
terms in equations (\ref{eq:nd-B}) and (\ref{eq:nd-T}) which give the
transition of platelets out of the $S_T$ state. This term is replaced
by $\eta p_{bf}(x - \tau, t - \tau)$ in each of these equations, giving
us a delay partial differential equation.

Now we can think of this single discrete delay as a delay that is
distributed as a $\delta$-function centered at $\tau$. We can also
rewrite the delay term to emphasize this point:
$\eta p_{bf}(x - \tau, t - \tau) = \eta \Integral{p_{bf}(x - t', t - t')
  \delta(t' - \tau)}{t'}{0}{\infty}$. Written in this form, we can see
that we can generalize the delays to a distribution of possible delays
$f(t)$ by replacing $\delta(t' - \tau)$ with $f(t')$ in the integral,
resulting in a delay term $\eta \Integral{p_{bf}(x - t', t - t')
  f(t')}{t'}{0}{\infty}$. Therefore with an arbitrary distribution of
small steps, where $f(x)$ is the distribution of small step sizes, the
Fokker-Planck equation is a system of partial integro-differential
equations:
\begin{align}
  \Pder{p_U}{t} &= - \Pder{p_U}{x} - \beta p_U + \alpha
                  p_b \label{eq:nd-U-dst} \\
  \Pder{p_b}{t} &= \beta p_U - (\alpha + \delta) p_b +
                  \gamma p_{bf} + \eta \Integral{p_{bf}(x - t', t -
                  t') f(t')} {t'} {0} {\infty} \label{eq:nd-B-dst} \\
  \Pder{p_{bf}}{t} &= \delta p_b - (\gamma + \eta)
                     p_{bf} \label{eq:nd-BF-dst} \\
  \Pder{p_T}{t} &= -\Pder{p_T}{x} + \eta p_{bf} - \eta
                  \Integral{p_{bf}(x - t', t - t') f(t')} {t'} {0}
                  {\infty}. \label{eq:nd-T-dst}
\end{align}

\section{Simulations of the modified jump velocity model}
\label{sec:simul-modif-jump}

I wrote code to run stochastic trials of the modified jump-velocity
process (with exponentially distributed small steps), and compared the
results to the distribution of average velocities predicted by the
Fokker-Planck equation for a couple different choices of
parameters. Figure \ref{fig:exact-cmp-equal-rates} shows the case
where all off and on rates are equal, figure
\ref{fig:exact-cmp-small-a} shows the case where the on rates are
larger than the off rates, and figure \ref{fig:exact-cmp-large-a}
shows the case where the on rates are smaller than the off rates.

\begin{figure}
  \centering
  \begin{subfigure}{0.48\textwidth}
    \begin{tikzpicture}
      \begin{axis}[
        legend entries={Data, Model},
        xlabel = $v^*$,
        ylabel = Probability density,
        ]
        \addplot+[hist=density, fill] table[y index=0]
        {simulations/testmod-sim.dat};
        \addplot table[x index=0, y expr=\thisrowno{1}/(1 - exp(-10))]
        {distributions/testmod-dst.dat};
      \end{axis}
    \end{tikzpicture}
  \end{subfigure}
  \hfill
  \begin{subfigure}{0.48\textwidth}
    \begin{tikzpicture}
      \begin{axis}[
        legend entries = {Data, Model fit},
        legend pos = south east,
        xlabel = $v^*$,
        ylabel = {$P[V < v^*]$},
        ]
        \addplot+[const plot] table[x index=0, y
        expr=\coordindex/1024] {simulations/testmod-sim.dat};
        \addplot table[x index=0, y index=2]
        {distributions/testmod-dst.dat}; 
      \end{axis}
    \end{tikzpicture}
  \end{subfigure}
  \caption{PDF and CDF of average velocities with $\alpha = \beta =
    \gamma = \delta = \eta = 10$, $\lambda = 100$, and
    $N_\tn{simulations} = 1024$}
  \label{fig:exact-cmp-equal-rates}
\end{figure}

\begin{figure}
  \centering
  \begin{subfigure}{0.48\textwidth}
    \begin{tikzpicture}
      \begin{axis}[
        legend entries={Data, Model},
        xlabel = $v^*$,
        ylabel = Probability density,
        ]
        \addplot+[hist=density, fill] table[y index=0]
        {simulations/testmod1-sim.dat};
        \addplot table[x index=0, y expr=\thisrowno{1}/(1 - exp(-10))]
        {distributions/testmod1-dst.dat};
      \end{axis}
    \end{tikzpicture}
  \end{subfigure}
  \hfill
  \begin{subfigure}{0.48\textwidth}
    \begin{tikzpicture}
      \begin{axis}[
        legend entries = {Data, Model fit},
        legend pos = south east,
        xlabel = $v^*$,
        ylabel = {$P[V < v^*]$},
        ]
        \addplot+[const plot] table[x index=0, y
        expr=\coordindex/1024] {simulations/testmod1-sim.dat};
        \addplot table[x index=0, y index=2]
        {distributions/testmod1-dst.dat}; 
      \end{axis}
    \end{tikzpicture}
  \end{subfigure}
  \caption{PDF and CDF of average velocities with $\alpha = \gamma =
    5$, $\beta = \delta = \eta = 15$, $\lambda = 100$, and
    $N_\tn{simulations} = 1024$}
  \label{fig:exact-cmp-small-a}
\end{figure}

\begin{figure}
  \centering
  \begin{subfigure}{0.48\textwidth}
    \begin{tikzpicture}
      \begin{axis}[
        legend entries={Data, Model},
        xlabel = $v^*$,
        ylabel = Probability density,
        ]
        \addplot+[hist=density, fill] table[y index=0]
        {simulations/testmod2-sim.dat};
        \addplot table[x index=0, y expr=\thisrowno{1}/(1 - exp(-10))]
        {distributions/testmod2-dst.dat};
      \end{axis}
    \end{tikzpicture}
  \end{subfigure}
  \hfill
  \begin{subfigure}{0.48\textwidth}
    \begin{tikzpicture}
      \begin{axis}[
        legend entries = {Data, Model fit},
        legend pos = south east,
        xlabel = $v^*$,
        ylabel = {$P[V < v^*]$},
        ]
        \addplot+[const plot] table[x index=0, y
        expr=\coordindex/1024] {simulations/testmod2-sim.dat};
        \addplot table[x index=0, y index=2]
        {distributions/testmod2-dst.dat}; 
      \end{axis}
    \end{tikzpicture}
  \end{subfigure}
  \caption{PDF and CDF of average velocities with $\alpha = \gamma =
    15$, $\beta = \delta = \eta = 5$, $\lambda = 100$, and
    $N_\tn{simulations} = 1024$}
  \label{fig:exact-cmp-large-a}  
\end{figure}

\section{Step sizes}
\label{sec:step-sizes}

The distribution of step sizes in this model should be a mixture of
exponential distributions. There are two separate random variables
representing the two different types of steps: $S_1 \sim \Exp(\beta)$
is the random variable representing a complete unbinding and
reattachment of the platelet, and $S_2 \sim \Exp(\lambda)$ is the
random variable representing a step without complete unbinding (which
I am calling ``small'' steps).

Then the PDF of the mixture of these two random variables should be
\begin{equation}
  g(x) = \beta \exp(-\beta t) \cdot P[\tn{step is }S_1] + \lambda
  \exp(-\lambda t) \cdot P[\tn{step is }
  S_2]. \label{eq:partially-specified-pdf}
\end{equation}
To fully specify $g(x)$, we need to figure out the probability that a
particular step is a step of type 1.

In order to find the probability of each type of step, and to find the
distribution of dwell times later, we need to consider a subsystem of
the platelet state transition diagram. In particular, we want to model
the evolution of platelet states during a single pause. The reaction
diagram for this subsystem is given in Figure
\ref{fig:pause-subsystem}. Clearly all 4 platelet states are still
present, but not all the reactions from the full system are
present. In this subsystem, state U and state T are absorbing states
because there are no transitions out of these states. In the model
this corresponds to the platelet transitioning from a paused state to
a moving state.

\tikzexternaldisable
\begin{figure}
  \centering
  \schemestart
  U \arrow{<-[$\alpha$]}[-135]
  B \arrow{<=>[*{0}$\delta$][*{0}$\gamma$]}[-45]
  BF \arrow{->[*{0}$\eta$]}[45] T
  \schemestop  
  \caption[Pause subsystem]{Subsystem describing transitions within a
    single pause}
  \label{fig:pause-subsystem}
\end{figure}
\tikzexternalenable

Then the splitting probability of this system gives the probability of
a platelet taking a ``large'' step versus a ``small'' step. Also, the
distribution of first exit times gives the distribution of dwell times
in the modified model.

If we define $q_i(t)$ to be the probability that a platelet is in
state $i$ at time $t$ after a binding event, then the evolution of the
$q_i$s is given by the ODE system:
\begin{align}
  \Der{q_U}{t} &= \alpha q_b \label{eq:qu-nd} \\
  \Der{q_T}{t} &= \eta q_{bf} \label{eq:qt-nd} \\
  \Der{q_b}{t} &= -(\alpha + \delta) q_b + \gamma
                 q_{bf} \label{eq:qb-nd} \\
  \Der{q_{bf}}{t} &= \delta q_b - (\gamma + \eta)
                    q_{bf}. \label{eq:qbf-nd}
\end{align}

Following Dr. Keener's stochastics notes, we can group the absorbing
and non-absorbing states separately by defining
$Q(t) \equiv (q_U(t), q_T(t))^T$ to be the absorbing states and
$P(t) \equiv (q_b(t), q_{bf}(t))^T$ to be the non-absorbing
states. Then we can write the system of equations
(\ref{eq:qu-nd})---(\ref{eq:qbf-nd}) as
\begin{equation}
  \label{eq:abs-nabs-sys}
  \Der{}{t}
  \begin{pmatrix}
    Q \\ P
  \end{pmatrix}
  =
  \begin{pmatrix}
    0 & B \\
    0 & A
  \end{pmatrix}
  \begin{pmatrix}
    Q \\ P
  \end{pmatrix},
\end{equation}
where $B = \begin{pmatrix} \alpha & 0 \\ 0 & \eta \end{pmatrix}$, and
$A = \begin{pmatrix} -(\alpha + \delta) & \gamma \\ \delta & -(\gamma
  + \eta) \end{pmatrix}$.

The splitting probability $Q^\infty$ is the vector of probabilities
that the platelet is in one of the absorbing states as $t \rightarrow
\infty$. This can be found by integrating $\Der{Q}{t}$ from $0$ to
$\infty$, which gives us:
\begin{equation*}
  Q^\infty = \Integral{\Der{Q}{t}}{t}{0}{\infty} = \Integral{B
    P}{t}{0}{\infty} = \Integral{B A^{-1} P}{t}{0}{\infty} = B A^{-1}
  \left(P(\infty) - P(0) \right) = - B A^{-1} P(0).
\end{equation*}

The splitting probability (and the first exit time) depends on the
initial state $P(0)$. I think the appropriate initial state is
$q_b(0) = 1$ and $q_{bf}(0) = 0$, because every pause begins in the
$B$ state, with a single bond between the platelet and the
surface. Therefore, the splitting probability is
$Q^\infty = (q_U^\infty, q_T^\infty)^T = \frac{1}{\delta \eta +
  \alpha(\gamma + \eta)} \left( \alpha (\gamma + \eta), \delta \eta
\right)^T$. In the context of equation
(\ref{eq:partially-specified-pdf}),
$P[\tn{step is } S_1] = q_U^\infty$ and
$P[\tn{step is } S_2] = q_T^\infty$. Therefore, the fully-specified
PDF of step sizes is the mixture model:
\begin{equation}
  \label{eq:fully-specified-pdf}
  g(x) = \beta \frac{\alpha (\gamma + \eta)}{\alpha (\gamma + \eta) +
    \delta\eta} \exp(-\beta x) + \delta \frac{\delta \eta}{\alpha
    (\gamma + \eta) + \delta \eta} \exp(-\delta x).
\end{equation}



To do:
\begin{itemize}
\item Test the step size distribution
\item Write code to fit the step size distribution to data
\item Research goodness-of-fit tests for mixture models
\end{itemize}

% \bibliographystyle{plain}
% \bibliography{/Users/andrewwork/Documents/grad-school/thesis/library}

\end{document}




