\documentclass{article}

\newcommand{\ep}{\rule{.06in}{.1in}}
\textheight 9.5in

\usepackage{amssymb}
\usepackage{amsmath}
\usepackage{amsthm}
\usepackage{graphicx, subcaption, algorithmic}
\graphicspath{{/Users/andrewwork/thesis/jump-velocity/plots/}}

\usepackage{tikz, pgfplots, chemfig}
\usepgfplotslibrary{colorbrewer, statistics}
\pgfplotsset{
  exact axis/.style={grid=major, minor tick num=4, xlabel=$v^*$,
    legend entries={PDF, CDF},},
  every axis plot post/.append style={thick},
  table/search
  path={/Users/andrewwork/thesis/jump-velocity/dat-files},
  colormap/YlGnBu,
  cycle list/Set1-5,
  legend style={legend cell align=left,},
}
\usepgfplotslibrary{external}
\tikzexternalize

\renewcommand{\arraystretch}{1.2}
\pagestyle{empty} 
\oddsidemargin -0.25in
\evensidemargin -0.25in 
\topmargin -0.75in 
\parindent 0pt
\parskip 12pt
\textwidth 7in
%\font\cj=msbm10 at 12pt

\newcommand{\tn}{\textnormal}
\newcommand{\stiff}{\frac{k_f}{\gamma}}
\newcommand{\dd}{d}
\newcommand{\Der}[2]{\frac{\dd #1}{\dd #2}}
\newcommand{\Pder}[2]{\frac{\partial #1}{\partial #2}}
\newcommand{\Integral}[4]{\int_{#3}^{#4} {#1} \dd #2}
\DeclareMathOperator{\Exp}{Exp}

% Text width is 7 inches

\def\R{\mathbb{R}}
\def\N{\mathbb{N}}
\def\C{\mathbb{C}}
\def\Z{\mathbb{Z}}
\def\Q{\mathbb{Q}}
\def\H{\mathbb{H}}
\def\B{\mathcal{B}} 
%\topmargin -.5in 

\setcounter{secnumdepth}{2}
\begin{document}
\pagestyle{plain}

\begin{center}
  {\Large Notes on a Modification to the Jump-Velocity Rolling Model
    (\today)}
\end{center}

\section{A modification to the single bond model of rolling}
\label{sec:modif-single-bond}

Experimental step size data does not fit a simple one-step model of
binding and unbinding. We want to test the hypothesis that the
distribution of step sizes can be explained by a mixture of two types
of steps: (1) steps where the platelet comes completely unbound from
the surface, and (2) steps where a platelet is multiply bound to the
vessel wall, and the rearmost load-bearing bond breaks, causing the
platelet to lurch forward.

We want to test this with a model that approximates this behavior,
without the expense of tracking an arbitrary number of bonds that can
bind in arbitrary locations along the wall. Therefore, we will use a
model that allows bonds to form in two positions: on the front of the
platelet and on the back of the platelet (see Figure
\ref{fig:four-states}).

\begin{figure}
  \centering
  \includegraphics[width=.7\textwidth]{double-binding-model.png}
  \caption{Four states in the modified jump-velocity model}
  \label{fig:four-states}
\end{figure}

In the unbound state $S_U$, the platelet translates forward at the
free-flowing velocity $V^*$. A bond can form on the back of the
platelet at a constant binding rate $k_\tn{on}$ (or rather, a bond
forms \emph{somewhere} between the platelet and wall, but the platelet
keeps translating forward until the force applied by the bond balances
the drag force on the platelet). Let's call this state $S_b$. This
bond can then break at a rate $k_\tn{off}$, returning the platelet to
the unbound state.

An $S_b$ platelet can form a second bond with the surface on the front
(downstream) end of the platelet at rate $k_\tn{on, f}$, and we'll
call this the $S_{bf}$ state. Then, either the back or the front bonds
of a $S_{bf}$ platelet can break. The back bond breaks at a rate
$k_\tn{off, b}$ and the front bond breaks at a rate $k_\tn{off,
  f}$. If the front bond breaks, the platelet returns to the $S_b$
state, and if the back bond breaks, the platelet is temporarily left
with a single front bond. However, because of the flow, the platelet
is pushed downstream until the front bond becomes a back bond, and the
bond force balances with the drag force on the platelet. We'll call
this state the transition state---$S_T$.

Now we have to make a choice to model how long platelets remain in the
$S_T$ state. One possible choice is to say that every platelet stays
in the $S_T$ state for the exact same length of time and travels the
same distance. While one could argue this agrees geometrically with
the series of pictures drawn in Figure \ref{fig:four-states}, this
isn't a realistic representation of small step sizes in the more
complicated case that bonds can form continuously along the vessel
wall. If we instead say that platelets transition out of state $S_T$
with some rate $\kappa$, that is equivalent (I think) to assuming
that the small step sizes are distributed exponentially with mean step
length $V^*/\kappa$.

How can we choose a reasonable transition rate $\kappa$ out of the
$S_T$ state? One possibility is to define a characteristic step length
$\ell$, and then assert that some fraction $q$ of the small steps must
be smaller than the characteristic step length. To choose a $\kappa$
based on some given $\ell$ and $q$, we can use the CDF of small step
times. For a constant transition rate $\kappa$, the small step times
are distributed as an exponential distribution with mean
$1/\kappa$. Therefore, the CDF of step times is
$F(t) = 1 - \exp(-\kappa t)$. Then $q = 1 - \exp(-\kappa \ell/V^*)$
enforces the condition that the fraction of steps which are less than
the characteristic step size is $q$. Solving this condition for
$\kappa$ gives us
\begin{equation}
  \label{eq:kappa-defn}
  \kappa = -\frac{\ln(1 - q)}{\ell/V^*}.
\end{equation}

\subsection{Fokker-Planck equation}
\label{sec:fokk-planck-equat}

The resulting Fokker-Planck equation of this process is an
advection-reaction system similar to that of the ordinary jump
velocity process. One major difference is that there are 2 states
which translate in the flow, not just one.

Clearly the Fokker-Planck equation will depend on the assumption made
about the distribution of transition times (or small step sizes), but
for the case where transition times are distributed exponentially, the
Fokker-Planck equation is a simple linear advection-reaction PDE:
\begin{align}
  \Pder{p_U}{t} &= -V^* \Pder{p_U}{x} - k_\tn{on} p_U + k_\tn{off} p_b
  \\
  \Pder{p_b}{t} &= k_\tn{on} p_U - (k_\tn{off} + k_\tn{on, f}) p_b +
                  k_\tn{off, f} p_{bf} + \kappa p_T \\
  \Pder{p_{bf}}{t} &= k_\tn{on, f} p_b - (k_\tn{off, f} + k_\tn{off,
                     b}) p_{bf} \\
  \Pder{p_T}{t} &= -V^* \Pder{p_T}{x} + k_\tn{off, b} p_{bf} - \kappa
                  p_T.
\end{align}

If we choose the same nondimensionalization as the first jump-velocity
model, i.e. scaling space by the chamber length $L$ and time by the
time it takes to cross the chamber $T \equiv L/V$, then we end up with the
following nondimensional set of equations:
\begin{align}
  \Pder{p_U}{t} &= - \Pder{p_U}{x} - \beta p_U + \alpha
                  p_b \label{eq:nd-U} \\
  \Pder{p_b}{t} &= \beta p_U - (\alpha + \delta) p_b +
                  \gamma p_{bf} + \lambda p_T \label{eq:nd-B} \\
  \Pder{p_{bf}}{t} &= \delta p_b - (\gamma + \eta)
                     p_{bf} \label{eq:nd-BF} \\
  \Pder{p_T}{t} &= -\Pder{p_T}{x} + \eta p_{bf} - \lambda
                  p_T. \label{eq:nd-T}
\end{align}

There are 6 nondimensional rate constants in this system of equations:
$\alpha = k_\tn{off}T$, $\beta = k_\tn{on}T$, $\gamma = k_\tn{off,
  f}T$, $\delta = k_\tn{on, f}T$, $\eta = k_\tn{off, b} T$, $\lambda =
\kappa T$. We can make further simplifying assumptions to reduce the
number of rate constants if that is required, but this is the general
form.

\subsubsection{Other distributions of small steps}
\label{sec:other-distr-small}

What if we don't make the assumption that the small step sizes are
distributed exponentially? How can we implement other distributions of
small step sizes? One possibility is to use a distributed
delay. Briefly, distributed delay differential equations extend the
concept of delay differential equations by allowing for a probability
distribution of delays, instead of specifying a single (or multiple)
discrete, deterministic delay(s). In the following section, I develop
the system of equations that I think should describe the rolling model
with a general probability distribution $f(x)$ of small step sizes.

First, let's consider the case where every platelet spends the exact
same time $\tau$  in the transition state, that is the small steps are
deterministic and length $\tau$ (in the nondimensional model). The
Fokker-Planck equation remains the same, except for the $\lambda p_T$
terms in equations (\ref{eq:nd-B}) and (\ref{eq:nd-T}) which give the
transition of platelets out of the $S_T$ state. This term is replaced
by $\eta p_{bf}(x - \tau, t - \tau)$ in each of these equations, giving
us a delay partial differential equation.

Now we can think of this single discrete delay as a delay that is
distributed as a $\delta$-function centered at $\tau$. We can also
rewrite the delay term to emphasize this point:
$\eta p_{bf}(x - \tau, t - \tau) = \eta \Integral{p_{bf}(x - t', t - t')
  \delta(t' - \tau)}{t'}{0}{\infty}$. Written in this form, we can see
that we can generalize the delays to a distribution of possible delays
$f(t)$ by replacing $\delta(t' - \tau)$ with $f(t')$ in the integral,
resulting in a delay term $\eta \Integral{p_{bf}(x - t', t - t')
  f(t')}{t'}{0}{\infty}$. Therefore with an arbitrary distribution of
small steps, where $f(x)$ is the distribution of small step sizes, the
Fokker-Planck equation is a system of partial integro-differential
equations:
\begin{align}
  \Pder{p_U}{t} &= - \Pder{p_U}{x} - \beta p_U + \alpha
                  p_b \label{eq:nd-U-dst} \\
  \Pder{p_b}{t} &= \beta p_U - (\alpha + \delta) p_b +
                  \gamma p_{bf} + \eta \Integral{p_{bf}(x - t', t -
                  t') f(t')} {t'} {0} {\infty} \label{eq:nd-B-dst} \\
  \Pder{p_{bf}}{t} &= \delta p_b - (\gamma + \eta)
                     p_{bf} \label{eq:nd-BF-dst} \\
  \Pder{p_T}{t} &= -\Pder{p_T}{x} + \eta p_{bf} - \eta
                  \Integral{p_{bf}(x - t', t - t') f(t')} {t'} {0}
                  {\infty}. \label{eq:nd-T-dst}
\end{align}

\section{Simulations of the modified jump velocity model}
\label{sec:simul-modif-jump}

I wrote code to run stochastic trials of the modified jump-velocity
process (with exponentially distributed small steps), and compared the
results to the distribution of average velocities predicted by the
Fokker-Planck equation for a couple different choices of
parameters. Figure \ref{fig:exact-cmp-equal-rates} shows the case
where all off and on rates are equal, figure
\ref{fig:exact-cmp-small-a} shows the case where the on rates are
larger than the off rates, and figure \ref{fig:exact-cmp-large-a}
shows the case where the on rates are smaller than the off rates.

\begin{figure}
  \centering
  \begin{subfigure}{0.48\textwidth}
    \begin{tikzpicture}
      \begin{axis}[
        legend entries={Data, Model},
        xlabel = $v^*$,
        ylabel = Probability density,
        ]
        \addplot+[hist=density, fill] table[y index=0]
        {simulations/testmod-sim.dat};
        \addplot table[x index=0, y expr=\thisrowno{1}/(1 - exp(-10))]
        {distributions/testmod-dst.dat};
      \end{axis}
    \end{tikzpicture}
  \end{subfigure}
  \hfill
  \begin{subfigure}{0.48\textwidth}
    \begin{tikzpicture}
      \begin{axis}[
        legend entries = {Data, Model fit},
        legend pos = south east,
        xlabel = $v^*$,
        ylabel = {$P[V < v^*]$},
        ]
        \addplot+[const plot] table[x index=0, y
        expr=\coordindex/1024] {simulations/testmod-sim.dat};
        \addplot table[x index=0, y index=2]
        {distributions/testmod-dst.dat}; 
      \end{axis}
    \end{tikzpicture}
  \end{subfigure}
  \caption{PDF and CDF of average velocities with $\alpha = \beta =
    \gamma = \delta = \eta = 10$, $\lambda = 100$, and
    $N_\tn{simulations} = 1024$}
  \label{fig:exact-cmp-equal-rates}
\end{figure}

\begin{figure}
  \centering
  \begin{subfigure}{0.48\textwidth}
    \begin{tikzpicture}
      \begin{axis}[
        legend entries={Data, Model},
        xlabel = $v^*$,
        ylabel = Probability density,
        ]
        \addplot+[hist=density, fill] table[y index=0]
        {simulations/testmod1-sim.dat};
        \addplot table[x index=0, y expr=\thisrowno{1}/(1 - exp(-10))]
        {distributions/testmod1-dst.dat};
      \end{axis}
    \end{tikzpicture}
  \end{subfigure}
  \hfill
  \begin{subfigure}{0.48\textwidth}
    \begin{tikzpicture}
      \begin{axis}[
        legend entries = {Data, Model fit},
        legend pos = south east,
        xlabel = $v^*$,
        ylabel = {$P[V < v^*]$},
        ]
        \addplot+[const plot] table[x index=0, y
        expr=\coordindex/1024] {simulations/testmod1-sim.dat};
        \addplot table[x index=0, y index=2]
        {distributions/testmod1-dst.dat}; 
      \end{axis}
    \end{tikzpicture}
  \end{subfigure}
  \caption{PDF and CDF of average velocities with $\alpha = \gamma =
    5$, $\beta = \delta = \eta = 15$, $\lambda = 100$, and
    $N_\tn{simulations} = 1024$}
  \label{fig:exact-cmp-small-a}
\end{figure}

\begin{figure}
  \centering
  \begin{subfigure}{0.48\textwidth}
    \begin{tikzpicture}
      \begin{axis}[
        legend entries={Data, Model},
        xlabel = $v^*$,
        ylabel = Probability density,
        ]
        \addplot+[hist=density, fill] table[y index=0]
        {simulations/testmod2-sim.dat};
        \addplot table[x index=0, y expr=\thisrowno{1}/(1 - exp(-10))]
        {distributions/testmod2-dst.dat};
      \end{axis}
    \end{tikzpicture}
  \end{subfigure}
  \hfill
  \begin{subfigure}{0.48\textwidth}
    \begin{tikzpicture}
      \begin{axis}[
        legend entries = {Data, Model fit},
        legend pos = south east,
        xlabel = $v^*$,
        ylabel = {$P[V < v^*]$},
        ]
        \addplot+[const plot] table[x index=0, y
        expr=\coordindex/1024] {simulations/testmod2-sim.dat};
        \addplot table[x index=0, y index=2]
        {distributions/testmod2-dst.dat}; 
      \end{axis}
    \end{tikzpicture}
  \end{subfigure}
  \caption{PDF and CDF of average velocities with $\alpha = \gamma =
    15$, $\beta = \delta = \eta = 5$, $\lambda = 100$, and
    $N_\tn{simulations} = 1024$}
  \label{fig:exact-cmp-large-a}  
\end{figure}

\section{Step sizes}
\label{sec:step-sizes}

The distribution of step sizes in this model should be a mixture of
exponential distributions. There are two separate random variables
representing the two different types of steps: $S_1 \sim \Exp(\beta)$
is the random variable representing a complete unbinding and
reattachment of the platelet, and $S_2 \sim \Exp(\lambda)$ is the
random variable representing a step without complete unbinding (which
I am calling ``small'' steps).

Then the PDF of the mixture of these two random variables should be
\begin{equation}
  g(x) = \beta \exp(-\beta t) \cdot P[\tn{step is }S_1] + \lambda
  \exp(-\lambda t) \cdot P[\tn{step is }
  S_2]. \label{eq:partially-specified-pdf}
\end{equation}
To fully specify $g(x)$, we need to figure out the probability that a
particular step is a step of type 1.

In order to find the probability of each type of step, and to find the
distribution of dwell times later, we need to consider a subsystem of
the platelet state transition diagram. In particular, we want to model
the evolution of platelet states during a single pause. The reaction
diagram for this subsystem is given in Figure
\ref{fig:pause-subsystem}. Clearly all 4 platelet states are still
present, but not all the reactions from the full system are
present. In this subsystem, state U and state T are absorbing states
because there are no transitions out of these states. In the model
this corresponds to the platelet transitioning from a paused state to
a moving state.

\tikzexternaldisable
\begin{figure}
  \centering
  \schemestart
  U \arrow{<-[$\alpha$]}[-135]
  B \arrow{<=>[*{0}$\delta$][*{0}$\gamma$]}[-45]
  BF \arrow{->[*{0}$\eta$]}[45] T
  \schemestop  
  \caption[Pause subsystem]{Subsystem describing transitions within a
    single pause}
  \label{fig:pause-subsystem}
\end{figure}
\tikzexternalenable

Then the splitting probability of this system gives the probability of
a platelet taking a ``large'' step versus a ``small'' step. Also, the
distribution of first exit times gives the distribution of dwell times
in the modified model.

If we define $q_i(t)$ to be the probability that a platelet is in
state $i$ at time $t$ after a binding event, then the evolution of the
$q_i$s is given by the ODE system:
\begin{align}
  \Der{q_U}{t} &= \alpha q_b \label{eq:qu-nd} \\
  \Der{q_T}{t} &= \eta q_{bf} \label{eq:qt-nd} \\
  \Der{q_b}{t} &= -(\alpha + \delta) q_b + \gamma
                 q_{bf} \label{eq:qb-nd} \\
  \Der{q_{bf}}{t} &= \delta q_b - (\gamma + \eta)
                    q_{bf}. \label{eq:qbf-nd}
\end{align}

Following Dr. Keener's stochastics notes, we can group the absorbing
and non-absorbing states separately by defining
$Q(t) \equiv (q_U(t), q_T(t))^T$ to be the absorbing states and
$P(t) \equiv (q_b(t), q_{bf}(t))^T$ to be the non-absorbing
states. Then we can write the system of equations
(\ref{eq:qu-nd})---(\ref{eq:qbf-nd}) as
\begin{equation}
  \label{eq:abs-nabs-sys}
  \Der{}{t}
  \begin{pmatrix}
    Q \\ P
  \end{pmatrix}
  =
  \begin{pmatrix}
    0 & B \\
    0 & A
  \end{pmatrix}
  \begin{pmatrix}
    Q \\ P
  \end{pmatrix},
\end{equation}
where $B = \begin{pmatrix} \alpha & 0 \\ 0 & \eta \end{pmatrix}$, and
$A = \begin{pmatrix} -(\alpha + \delta) & \gamma \\ \delta & -(\gamma
  + \eta) \end{pmatrix}$.

The splitting probability $Q^\infty$ is the vector of probabilities
that the platelet is in one of the absorbing states as $t \rightarrow
\infty$. This can be found by integrating $\Der{Q}{t}$ from $0$ to
$\infty$, which gives us:
\begin{equation*}
  Q^\infty = \Integral{\Der{Q}{t}}{t}{0}{\infty} = \Integral{B
    P}{t}{0}{\infty} = \Integral{B A^{-1} P}{t}{0}{\infty} = B A^{-1}
  \left(P(\infty) - P(0) \right) = - B A^{-1} P(0).
\end{equation*}

The splitting probability (and the first exit time) depends on the
initial state $P(0)$. I think the appropriate initial state is
$q_b(0) = 1$ and $q_{bf}(0) = 0$, because every pause begins in the
$B$ state, with a single bond between the platelet and the
surface. Therefore, the splitting probability is
$Q^\infty = (q_U^\infty, q_T^\infty)^T = \frac{1}{\delta \eta +
  \alpha(\gamma + \eta)} \left( \alpha (\gamma + \eta), \delta \eta
\right)^T$. In the context of equation
(\ref{eq:partially-specified-pdf}),
$P[\tn{step is } S_1] = q_U^\infty$ and
$P[\tn{step is } S_2] = q_T^\infty$. Therefore, the fully-specified
PDF of step sizes is the mixture model:
\begin{equation}
  \label{eq:fully-specified-pdf}
  g(x) = \beta \frac{\alpha (\gamma + \eta)}{\alpha (\gamma + \eta) +
    \delta\eta} \exp(-\beta x) + \lambda \frac{\delta \eta}{\alpha
    (\gamma + \eta) + \delta \eta} \exp(-\lambda x).
\end{equation}

There are really three identifiable(?) parameters in this PDF:
$\beta$, $\lambda$, and the convexity parameter. If we call the
convexity parameter $\chi$ (because we're running out of letters),
then
$\chi = \frac{\alpha (\gamma + \eta)}{\alpha (\gamma + \eta) +
  \delta\eta}$ and we can rewrite equation
(\ref{eq:fully-specified-pdf}) as
\begin{equation}
  \label{eq:pdf-three-par}
  g(x; \chi, \beta, \lambda) = \chi \beta \exp(-\beta x) + (1 - \chi)
  \lambda \exp(-\lambda x).
\end{equation}

Figures \ref{fig:exact-step-cmp-equal-rates},
\ref{fig:exact-step-cmp-small-a}, and \ref{fig:exact-step-cmp-large-a}
compare equation (\ref{eq:fully-specified-pdf}) with data from
stochastic simulation. The step data is extracted from a single long
experiment ($L = 1024$) in order eliminate the filtering effect that
occurs when an experiment is terminated at length $L = 1$ (that is,
steps greater than length $L$ cannot be observed). For the parameters
chosen, the PDF (\ref{eq:fully-specified-pdf}) and its CDF agree well
with the simulated data, and this was confirmed by running
Kolmogorov-Smirnov tests for each case.

\begin{figure}
  \centering
  \begin{subfigure}{0.48\textwidth}
    \begin{tikzpicture}
      \begin{axis}[
        legend entries={Data, Exact PDF},
        xlabel = Step size,
        ylabel = Probability density,
        ]
        \addplot+[hist=density, fill] table[y index=0]
        {simulations/testmod-step.dat};
        \addplot table[x index=0, y index=1]
        {distributions/testmod-step-dst.dat};
      \end{axis}
    \end{tikzpicture}
  \end{subfigure}
  \hfill
  \begin{subfigure}{0.48\textwidth}
    \begin{tikzpicture}
      \begin{axis}[
        legend entries = {Data, Exact CDF},
        legend pos = south east,
        xlabel = Step size,
        ylabel = {$P[S < s^*]$},
        ]
        \addplot table[x index=0, y index=1]
        {simulations/testmod-step-cdf.dat};
        \addplot+[dashed] table[x index=0, y index=2]
        {distributions/testmod-step-dst.dat}; 
      \end{axis}
    \end{tikzpicture}
  \end{subfigure}
  \caption{PDF and CDF of step sizes with $\alpha = \beta = \gamma =
    \delta = \eta = 10$, $\lambda = 100$, and $N_\tn{steps} =
    14\,556$}
  \label{fig:exact-step-cmp-equal-rates}
\end{figure}

\begin{figure}
  \centering
  \begin{subfigure}{0.48\textwidth}
    \begin{tikzpicture}
      \begin{axis}[
        legend entries={Data, Exact PDF},
        xlabel = Step size,
        ylabel = Probability density,
        ]
        \addplot+[hist=density, fill] table[y index=0]
        {simulations/testmod1-step.dat};
        \addplot table[x index=0, y index=1]
        {distributions/testmod1-step-dst.dat};
      \end{axis}
    \end{tikzpicture}
  \end{subfigure}
  \hfill
  \begin{subfigure}{0.48\textwidth}
    \begin{tikzpicture}
      \begin{axis}[
        legend entries = {Data, Exact CDF},
        legend pos = south east,
        xlabel = Step size,
        ylabel = {$P[S < s^*]$},
        ]
        \addplot table[x index=0, y index=1]
        {simulations/testmod1-step-cdf.dat}; 
        \addplot+[dashed] table[x index=0, y index=2]
        {distributions/testmod1-step-dst.dat}; 
      \end{axis}
    \end{tikzpicture}
  \end{subfigure}
  \caption{PDF and CDF of step sizes with $\alpha = \gamma =
    5$, $\beta = \delta = \eta = 15$, $\lambda = 100$, and
    $N_\tn{steps} = 37\,782$}
  \label{fig:exact-step-cmp-small-a}
\end{figure}

\begin{figure}
  \centering
  \begin{subfigure}{0.48\textwidth}
    \begin{tikzpicture}
      \begin{axis}[
        legend entries={Data, Exact PDF},
        xlabel = Step size,
        ylabel = Probability density,
        ]
        \addplot+[hist=density, fill] table[y index=0]
        {simulations/testmod2-step.dat};
        \addplot table[x index=0, y index=1]
        {distributions/testmod2-step-dst.dat};
      \end{axis}
    \end{tikzpicture}
  \end{subfigure}
  \hfill
  \begin{subfigure}{0.48\textwidth}
    \begin{tikzpicture}
      \begin{axis}[
        legend entries = {Data, Exact CDF},
        legend pos = south east,
        xlabel = Step size,
        ylabel = {$P[S < s^*]$},
        ]
        \addplot table[x index=0, y index=1]
        {simulations/testmod2-step-cdf.dat};
        \addplot+[dashed] table[x index=0, y index=2]
        {distributions/testmod2-step-dst.dat}; 
      \end{axis}
    \end{tikzpicture}
  \end{subfigure}
  \caption{PDF and CDF of step sizes with $\alpha = \gamma = 15$,
    $\beta = \delta = \eta = 5$, $\lambda = 100$, and $N_\tn{steps} =
    5604$}
  \label{fig:exact-step-cmp-large-a}  
\end{figure}

Next, I estimated the three parameters in equation
(\ref{eq:pdf-three-par}) from the simulated data sets by numerically
maximizing the log-likelihood function
\begin{equation*}
  \ell(\chi, \beta, \lambda) = \sum_{i=1}^{N_\tn{steps}} \ln(g(x_i;
  \chi, \beta, \lambda)).
\end{equation*}
The optimization process terminated quickly ($< 1\,\tn{s}$), but the
estimation could be made faster if necessary. In particular, we can at
least write down a set of algebraic equations that we can solve to
find the values of parameters that maximize $\ell$. However, for these
test cases, the optimization algorithm converged quickly and gave
accurate estimates to the true parameter values (results shown in
Figures \ref{fig:fit-step-cmp-equal-rates},
\ref{fig:fit-step-cmp-small-a}, and \ref{fig:fit-cmp-large-a}).

\begin{figure}
  \centering
  \begin{subfigure}{0.48\textwidth}
    \begin{tikzpicture}
      \begin{axis}[
        legend entries={Data, Fit PDF},
        xlabel = Step size,
        ylabel = Probability density,
        ]
        \addplot+[hist=density, fill] table[y index=0]
        {simulations/testmod-step.dat};
        \addplot table[x index=0, y index=1]
        {ml-estimates/testmod-dexp.dat};
      \end{axis}
    \end{tikzpicture}
  \end{subfigure}
  \hfill
  \begin{subfigure}{0.48\textwidth}
    \begin{tikzpicture}
      \begin{axis}[
        legend entries = {Data, Fit CDF},
        legend pos = south east,
        xlabel = Step size,
        ylabel = {$P[S < s^*]$},
        ]
        \addplot table[x index=0, y index=1]
        {simulations/testmod-step-cdf.dat};
        \addplot+[dashed] table[x index=0, y index=2]
        {ml-estimates/testmod-dexp.dat}; 
      \end{axis}
    \end{tikzpicture}
  \end{subfigure}
  \caption{PDF and CDF of step sizes with ML parameter
    estimates. $\alpha = \beta = \gamma = \delta = \eta = 10$,
    $\lambda = 100$, and $N_\tn{steps} = 14\,556$. For these parameter
    values, $\chi = 2/3$. Estimated parameters are
    $\hat{\chi} = 0.6683$, $\hat{\beta} = 9.972$,
    $\hat{\lambda} = 99.96$, and the $\chi^2$ goodness-of-fit test
    gave $p = 0.86$.}
  \label{fig:fit-step-cmp-equal-rates}
\end{figure}

\begin{figure}
  \centering
  \begin{subfigure}{0.48\textwidth}
    \begin{tikzpicture}
      \begin{axis}[
        legend entries={Data, Fit PDF},
        xlabel = Step size,
        ylabel = Probability density,
        ]
        \addplot+[hist=density, fill] table[y index=0]
        {simulations/testmod1-step.dat};
        \addplot table[x index=0, y index=1]
        {ml-estimates/testmod1-dexp.dat};
      \end{axis}
    \end{tikzpicture}
  \end{subfigure}
  \hfill
  \begin{subfigure}{0.48\textwidth}
    \begin{tikzpicture}
      \begin{axis}[
        legend entries = {Data, Fit CDF},
        legend pos = south east,
        xlabel = Step size,
        ylabel = {$P[S < s^*]$},
        ]
        \addplot table[x index=0, y index=1]
        {simulations/testmod1-step-cdf.dat}; 
        \addplot+[dashed] table[x index=0, y index=2]
        {ml-estimates/testmod1-dexp.dat}; 
      \end{axis}
    \end{tikzpicture}
  \end{subfigure}
  \caption{PDF and CDF of step sizes with ML parameter
    estimates. $\alpha = \gamma = 5$, $\beta = \delta = \eta = 15$,
    $\lambda = 100$, and $N_\tn{steps} = 37\,782$. For these parameter
    values, $\chi = 4/13 \approx 0.3077$. Estimated parameters are
    $\hat{\chi} = 0.3078$, $\hat{\beta} = 15.23$,
    $\hat{\lambda} = 100.2$, and the $\chi^2$ goodness-of-fit test
    gave $p = 0.26$.}
  \label{fig:fit-step-cmp-small-a}
\end{figure}

\begin{figure}
  \centering
  \begin{subfigure}{0.48\textwidth}
    \begin{tikzpicture}
      \begin{axis}[
        legend entries={Data, Fit PDF},
        xlabel = Step size,
        ylabel = Probability density,
        ]
        \addplot+[hist=density, fill] table[y index=0]
        {simulations/testmod2-step.dat};
        \addplot table[x index=0, y index=1]
        {ml-estimates/testmod2-dexp.dat};
      \end{axis}
    \end{tikzpicture}
  \end{subfigure}
  \hfill
  \begin{subfigure}{0.48\textwidth}
    \begin{tikzpicture}
      \begin{axis}[
        legend entries = {Data, Fit CDF},
        legend pos = south east,
        xlabel = Step size,
        ylabel = {$P[S < s^*]$},
        ]
        \addplot table[x index=0, y index=1]
        {simulations/testmod2-step-cdf.dat};
        \addplot+[dashed] table[x index=0, y index=2]
        {ml-estimates/testmod2-dexp.dat}; 
      \end{axis}
    \end{tikzpicture}
  \end{subfigure}
  \caption{PDF and CDF of step sizes with ML parameter
    estimates. $\alpha = \gamma = 15$, $\beta = \delta = \eta = 5$,
    $\lambda = 100$, and $N_\tn{steps} = 5604$. For these parameter
    values, $\chi = 12/13 \approx 0.9231$. Estimated parameters are
    $\hat{\chi} = 0.9225$, $\hat{\beta} = 5.073$,
    $\hat{\lambda} = 97.65$, and the $\chi^2$ goodness-of-fit test
    gave $p = 0.68$.}
  \label{fig:fit-cmp-large-a}  
\end{figure}

\subsection{Experimental step size distributions}
\label{sec:exper-step-size}

Figures \ref{fig:hc-prp-step-fit}---\ref{fig:cc-whole-step-fit} show
the distribution of small steps (steps less than $2 \mu\tn{m}$ in
length) in four experiments: HC PRP, CC PRP, HC whole blood, and CC
whole blood. I also fit 4 different models to these small steps:
exponential, uniform, normal, and gamma distributions. The uniform
distribution wasn't \emph{fit} to the data, rather the upper and lower
bounds were chosen to be 2 and 0, respectively, because that is how we
defined the small steps.

After fitting each of these models, I calculated the Akaike
Information Criterion (AIC) which assesses \emph{relative} model
quality among a set of models. The AIC is defined as:
\begin{equation}
  \tn{AIC} = 2k - 2\ln(\hat{L}),
\end{equation}
where $\hat{L}$ is the maximum of the likelihood function and $k$ is
the number of estimated parameters in the model \cite{Akaike1974}. For
each of our models, $k_\tn{exp} = 1$,
$k_\tn{norm} = k_\tn{gamma} = 2$, and $k_\tn{unif} = 0$.

% I also ran simulations of the stochastic model with these three
% distributions, as well as a deterministic step and plotted histograms
% of small steps in Figure \ref{fig:exact-unif-step-cmp-equal-rates}.

\begin{figure}
  \centering
  \begin{subfigure}{0.48\textwidth}
    \begin{tikzpicture}
      \begin{axis}[
        legend entries={Data, Exponential, Uniform, Normal, Gamma},
        xmin = 0, xmax = 2,
        xlabel = Step size ($\mu$m),
        ylabel = Probability density,
        ]
        \addplot+[hist=density, fill] table [y index=0]
        {simulations/HC-small-step.dat};
        \addplot table [x index=0, y index=1]
        {distributions/HC-small-step.dat};
        \addplot table [x index=0, y index=3]
        {distributions/HC-small-step.dat};
        \addplot table [x index=0, y index=5]
        {distributions/HC-small-step.dat};
        \addplot table [x index=0, y index=7]
        {distributions/HC-small-step.dat};
      \end{axis}
    \end{tikzpicture}
  \end{subfigure}
  \hfill
  \begin{subfigure}{0.48\textwidth}
    \begin{tikzpicture}
      \begin{axis}[
        legend entries={Data, Exponential, Uniform, Normal, Gamma},
        legend pos = south east,
        xmin=0, xmax=2,
        xlabel = Step size ($\mu$m),
        ylabel = Cumulative Probability,
        ]
        \addplot+[const plot] table[x index=0, y expr=(\coordindex+1)/17]
        {simulations/HC-small-step.dat};
        \addplot table [x index=0, y index=2]
        {distributions/HC-small-step.dat};
        \addplot table [x index=0, y index=4]
        {distributions/HC-small-step.dat};
        \addplot table [x index=0, y index=6]
        {distributions/HC-small-step.dat};
        \addplot table [x index=0, y index=8]
        {distributions/HC-small-step.dat};
      \end{axis}
    \end{tikzpicture}
  \end{subfigure}
  \caption[HC step size fit]{Data and fit of several distributions to
    small step size data in the HC PRP experiment ($N =
    17$). Parameters of the distributions are: $\lambda = 0.66$,
    $(\mu, \sigma) = (0.66, 0.30)$, and $(\theta, k) = (0.15,
    4.36)$. $\tn{AIC}_\tn{exp} = 66.1$, $\tn{AIC}_\tn{unif} = 23.6$,
    $\tn{AIC}_\tn{norm} = 11.8$, $\tn{AIC}_\tn{gamma} = 10.3$.}
  \label{fig:hc-prp-step-fit}
\end{figure}

\begin{figure}
  \centering
  \begin{subfigure}{0.48\textwidth}
    \begin{tikzpicture}
      \begin{axis}[
        legend entries={Data, Exponential, Uniform, Normal, Gamma},
        xmin = 0, xmax = 2,
        xlabel = Step size ($\mu$m),
        ylabel = Probability density,
        ]
        \addplot+[hist=density, fill] table [y index=0]
        {simulations/CC-small-step.dat};
        \addplot table [x index=0, y index=1]
        {distributions/CC-small-step.dat};
        \addplot table [x index=0, y index=3]
        {distributions/CC-small-step.dat};
        \addplot table [x index=0, y index=5]
        {distributions/CC-small-step.dat};
        \addplot table [x index=0, y index=7]
        {distributions/CC-small-step.dat};        
      \end{axis}
    \end{tikzpicture}
  \end{subfigure}
  \hfill
  \begin{subfigure}{0.48\textwidth}
    \begin{tikzpicture}
      \begin{axis}[
        legend entries={Data, Exponential, Uniform, Normal, Gamma},
        legend pos = south east,
        xmin = 0, xmax = 2,
        xlabel = Step size ($\mu$m),
        ylabel = Cumulative Probability,
        ]
        \addplot+[const plot] table[x index=0, y expr=(\coordindex+1)/28]
        {simulations/CC-small-step.dat};
        \addplot table [x index=0, y index=2]
        {distributions/CC-small-step.dat};
        \addplot table [x index=0, y index=4]
        {distributions/CC-small-step.dat};
        \addplot table [x index=0, y index=6]
        {distributions/CC-small-step.dat};
        \addplot table [x index=0, y index=8]
        {distributions/CC-small-step.dat};        
      \end{axis}
    \end{tikzpicture}
  \end{subfigure}
  \caption[CC step size fit]{Data and fit of several distributions to
    small step size data in the CC PRP experiment ($N =
    28$). Parameters of the distributions are: $\lambda = 0.91$,
    $(\mu, \sigma) = (0.91, 0.53)$, and $(\theta, k) = (0.33,
    2.78)$. $\tn{AIC}_\tn{exp} = 64.2$, $\tn{AIC}_\tn{unif} = 38.8$,
    $\tn{AIC}_\tn{norm} = 47.3$, $\tn{AIC}_\tn{gamma} = 42.3$.}
  \label{fig:cc-prp-step-fit}
\end{figure}

\begin{figure}
  \centering
  \begin{subfigure}{0.48\textwidth}
    \begin{tikzpicture}
      \begin{axis}[
        legend entries={Data, Exponential, Uniform, Normal, Gamma},
        xmin = 0, xmax = 2,
        xlabel = Step size ($\mu$m),
        ylabel = Probability density,
        ]
        \addplot+[hist=density, fill] table [y index=0]
        {simulations/HCwhole-small-step.dat};
        \addplot table [x index=0, y index=1]
        {distributions/HCwhole-small-step.dat};
        \addplot table [x index=0, y index=3]
        {distributions/HCwhole-small-step.dat};
        \addplot table [x index=0, y index=5]
        {distributions/HCwhole-small-step.dat};
        \addplot table [x index=0, y index=7]
        {distributions/HCwhole-small-step.dat};        
      \end{axis}
    \end{tikzpicture}
  \end{subfigure}
  \hfill
  \begin{subfigure}{0.48\textwidth}
    \begin{tikzpicture}
      \begin{axis}[
        legend entries={Data, Exponential, Uniform, Normal, Gamma},
        legend pos = south east,
        xmin = 0, xmax = 2,
        xlabel = Step size ($\mu$m),
        ylabel = Cumulative Probability,
        ]
        \addplot+[const plot] table[x index=0, y expr=(\coordindex+1)/10]
        {simulations/HCwhole-small-step.dat};
        \addplot table [x index=0, y index=2]
        {distributions/HCwhole-small-step.dat};
        \addplot table [x index=0, y index=4]
        {distributions/HCwhole-small-step.dat};
        \addplot table [x index=0, y index=6]
        {distributions/HCwhole-small-step.dat};
        \addplot table [x index=0, y index=8]
        {distributions/HCwhole-small-step.dat};        
      \end{axis}
    \end{tikzpicture}
  \end{subfigure}
  \caption[HC whole blood step size]{Data and fit of several
    distributions to small step size data in the HC whole blood
    experiment ($N = 10$). Parameters of the distributions are:
    $\lambda = 0.64$, $(\mu, \sigma) = (0.64, 0.41)$, and
    $(\theta, k) = (0.31, 2.04)$. $\tn{AIC}_\tn{exp} = 42.1$,
    $\tn{AIC}_\tn{unif} = 13.9$, $\tn{AIC}_\tn{norm} = 14.7$,
    $\tn{AIC}_\tn{gamma} = 12.6$.}
  \label{fig:hc-whole-step-fit}
\end{figure}

\begin{figure}
  \centering
  \begin{subfigure}{0.48\textwidth}
    \begin{tikzpicture}
      \begin{axis}[
        legend entries={Data, Exponential, Uniform, Normal, Gamma},
        xmin = 0, xmax = 2,
        xlabel = Step size ($\mu$m),
        ylabel = Probability density,
        ]
        \addplot+[hist=density, fill] table [y index=0]
        {simulations/CCwhole-small-step.dat};
        \addplot table [x index=0, y index=1]
        {distributions/CCwhole-small-step.dat};
        \addplot table [x index=0, y index=3]
        {distributions/CCwhole-small-step.dat};
        \addplot table [x index=0, y index=5]
        {distributions/CCwhole-small-step.dat};
        \addplot table [x index=0, y index=7]
        {distributions/CCwhole-small-step.dat};        
      \end{axis}
    \end{tikzpicture}
  \end{subfigure}
  \hfill
  \begin{subfigure}{0.48\textwidth}
    \begin{tikzpicture}
      \begin{axis}[
        legend entries={Data, Exponential, Uniform, Normal, Gamma},
        legend pos = south east,
        xmin = 0, xmax = 2,
        xlabel = Step size ($\mu$m),
        ylabel = Cumulative Probability,
        ]
        \addplot+[const plot] table[x index=0, y expr=(\coordindex+1)/54]
        {simulations/CCwhole-small-step.dat};
        \addplot table [x index=0, y index=2]
        {distributions/CCwhole-small-step.dat};
        \addplot table [x index=0, y index=4]
        {distributions/CCwhole-small-step.dat};
        \addplot table [x index=0, y index=6]
        {distributions/CCwhole-small-step.dat};
        \addplot table [x index=0, y index=8]
        {distributions/CCwhole-small-step.dat};        
      \end{axis}
    \end{tikzpicture}
  \end{subfigure}
  \caption[CC whole blood step size]{Data and fit of several
    distributions to small step size data in the CC whole blood
    experiment ($N = 54$). Parameters of the distributions are:
    $\lambda = 0.78$, $(\mu, \sigma) = (0.78, 0.43)$, and
    $(\theta, k) = (0.33, 2.37)$. $\tn{AIC}_\tn{exp} = 154.0$,
    $\tn{AIC}_\tn{unif} = 74.9$, $\tn{AIC}_\tn{norm} = 66.9$,
    $\tn{AIC}_\tn{gamma} = 66.3$.}
  \label{fig:cc-whole-step-fit}
\end{figure}

The gamma distribution had the best (i.e. the lowest) AIC value in
three of the four sets of experimental data. The uniform distribution
had the lowest AIC value in the CC PRP experiment. In all of the data,
the exponential model had the worst AIC value, and therefore it seems
reasonable to exclude it in favor of one of the other models. However,
with the data we have, it is still inconclusive which model of the
uniform, normal, or gamma distributions is the best choice for the
small steps. In order to get more data, I ran simulations of the full
rolling model and extracted small steps from the model output.

\subsubsection{Simulations of the full rolling model}
\label{sec:simul-full-roll}

To recap, the full rolling model allows for multiple bonds to form
between the platelet and surface, and tracks the position and
orientation of each bond individually. To model the motion of the
platelet, it balances the total bond force and torque with drag force
and torque from a Stokes fluid.

This model outputs the translational and angular velocity of the
platelet at regular time steps throughout the experiment. Therefore to
generate trajectories, we can simply integrate the translational
velocity to get the position of the platelet as a function of
time. Once we have trajectories, we can extract step sizes, dwell
times, and the time-averaged velocity of each platelet.

In order to define steps and dwells, I defined a threshold velocity
$V_\tn{thresh}$ to partition a trajectory into a stepping state or a
paused state. After some experimentation, I chose a $V_\tn{thresh}$
about 50\% of the free-flowing velocity. For $V > V_\tn{thresh}$, the
platelet is stepping, and for $V < V_\tn{thresh}$ the platelet is
paused. When I applied this rule to the raw data, I ended up with a
huge number of steps and dwells: $\sim 10\,000$ of each per
trajectory. In the raw trajectories, there are large and rapid
switches in velocity between free-flowing and stationary
velocities. In order to smooth out these rapid switches, I filtered
the raw trajectories by sampling the platelet's position every
$N_\tn{skip}$ raw data points. Even choosing $N_\tn{skip} = 5$ greatly
reduced the number of steps and dwells observed: from $\sim 10\,000$
to $\sim 50$.

To start with, I chose processing parameters
($V_\tn{thresh} = 0.5 V_\tn{fluid}$, $N_\tn{skip} = 5$) by eye. But
these can be chosen to match up with the spatial and temporal
resolution of the rolling data. For example, if we assume that the
spatial resolution of the microscope is $\sim 0.15 \mu\tn{m}$ and the
temporal resolution is $\Delta t = 0.08 \tn{s}$ then the minimum
velocity that can be observed over a single time step is
$\sim 2 \frac{\mu\tn{m}}{\tn{s}}$.

However, when I tried using these processing parameters on the
existing simulation data, there were only a few dwells per trajectory,
mostly lasting over only a single time interval of the filtered data. 

\begin{figure}
  \centering
  \begin{subfigure}{0.48\textwidth}
    \begin{tikzpicture}
      \begin{axis}[
        legend entries={Data, Exponential, Uniform, Normal, Gamma},
        xmin = 0, xmax = 2,
        xlabel = Step size ($\mu$m),
        ylabel = Probability density,
        ]
        \addplot+[hist=density, fill] table [y index=0]
        {simulations/expandedalgsto_M64_N64_tsteps51200_initfree_trials1024_bmax10_cflux0_vf40.4_omf20_kappa50_eta23000_d0.01_delta16_on1_off1_sat1_xiv7.36e-06_xiom9.82e-06_L2.5_T5_sbh0-small-step.dat};
        \addplot table [x index=0, y index=1]
        {distributions/expandedalgsto_M64_N64_tsteps51200_initfree_trials1024_bmax10_cflux0_vf40.4_omf20_kappa50_eta23000_d0.01_delta16_on1_off1_sat1_xiv7.36e-06_xiom9.82e-06_L2.5_T5_sbh0-small-step.dat};
        \addplot table [x index=0, y index=3]
        {distributions/expandedalgsto_M64_N64_tsteps51200_initfree_trials1024_bmax10_cflux0_vf40.4_omf20_kappa50_eta23000_d0.01_delta16_on1_off1_sat1_xiv7.36e-06_xiom9.82e-06_L2.5_T5_sbh0-small-step.dat};
        \addplot table [x index=0, y index=5]
        {distributions/expandedalgsto_M64_N64_tsteps51200_initfree_trials1024_bmax10_cflux0_vf40.4_omf20_kappa50_eta23000_d0.01_delta16_on1_off1_sat1_xiv7.36e-06_xiom9.82e-06_L2.5_T5_sbh0-small-step.dat};
        \addplot table [x index=0, y index=7]
        {distributions/expandedalgsto_M64_N64_tsteps51200_initfree_trials1024_bmax10_cflux0_vf40.4_omf20_kappa50_eta23000_d0.01_delta16_on1_off1_sat1_xiv7.36e-06_xiom9.82e-06_L2.5_T5_sbh0-small-step.dat};
      \end{axis}
    \end{tikzpicture}
  \end{subfigure}
  \hfill
  \begin{subfigure}{0.48\textwidth}
    \begin{tikzpicture}
      \begin{axis}[
        legend entries={Data, Exponential, Uniform, Normal, Gamma},
        legend pos = south east,
        xmin = 0, xmax = 2,
        xlabel = Step size ($\mu$m),
        ylabel = Cumulative Probability,
        ]
        \addplot table[x index=0, y expr=(\coordindex+1)/18298]
        {simulations/expandedalgsto_M64_N64_tsteps51200_initfree_trials1024_bmax10_cflux0_vf40.4_omf20_kappa50_eta23000_d0.01_delta16_on1_off1_sat1_xiv7.36e-06_xiom9.82e-06_L2.5_T5_sbh0-small-step.dat};
        \addplot table [x index=0, y index=2]
        {distributions/expandedalgsto_M64_N64_tsteps51200_initfree_trials1024_bmax10_cflux0_vf40.4_omf20_kappa50_eta23000_d0.01_delta16_on1_off1_sat1_xiv7.36e-06_xiom9.82e-06_L2.5_T5_sbh0-small-step.dat};
        \addplot table [x index=0, y index=4]
        {distributions/expandedalgsto_M64_N64_tsteps51200_initfree_trials1024_bmax10_cflux0_vf40.4_omf20_kappa50_eta23000_d0.01_delta16_on1_off1_sat1_xiv7.36e-06_xiom9.82e-06_L2.5_T5_sbh0-small-step.dat};
        \addplot table [x index=0, y index=6]
        {distributions/expandedalgsto_M64_N64_tsteps51200_initfree_trials1024_bmax10_cflux0_vf40.4_omf20_kappa50_eta23000_d0.01_delta16_on1_off1_sat1_xiv7.36e-06_xiom9.82e-06_L2.5_T5_sbh0-small-step.dat};
        \addplot table [x index=0, y index=8]
        {distributions/expandedalgsto_M64_N64_tsteps51200_initfree_trials1024_bmax10_cflux0_vf40.4_omf20_kappa50_eta23000_d0.01_delta16_on1_off1_sat1_xiv7.36e-06_xiom9.82e-06_L2.5_T5_sbh0-small-step.dat};
      \end{axis}
    \end{tikzpicture}
  \end{subfigure}
  \caption[$\kappa=50$ step size]{Data and fit of several
    distributions to small step size data from simulations of the full
    model with $\kappa=50$ ($N = 18\,298$). Parameters of the distributions are:
    $\lambda = 0.61$, $(\mu, \sigma) = (0.61, 0.40)$, and
    $(\theta, k) = (0.22, 2.81)$. $\tn{AIC}_\tn{exp} = 80\,377$,
    $\tn{AIC}_\tn{unif} = 25\,366$, $\tn{AIC}_\tn{norm} = 18\,542$,
    $\tn{AIC}_\tn{gamma} = 10\,171$.}
  \label{fig:k50-step-fit}
\end{figure}

\begin{figure}
  \centering
  \begin{subfigure}{0.48\textwidth}
    \begin{tikzpicture}
      \begin{axis}[
        legend entries={Data, Exponential, Uniform, Normal, Gamma},
        xmin = 0, xmax = 2,
        xlabel = Step size ($\mu$m),
        ylabel = Probability density,
        ]
        \addplot+[hist=density, fill] table [y index=0]
        {simulations/expandedalgsto_M64_N64_tsteps51200_initfree_trials1024_bmax10_cflux0_vf40.4_omf20_kappa10_eta23000_d0.01_delta16_on1_off1_sat1_xiv7.36e-06_xiom9.82e-06_L2.5_T5_sbh0-small-step.dat};
        \addplot table [x index=0, y index=1]
        {distributions/expandedalgsto_M64_N64_tsteps51200_initfree_trials1024_bmax10_cflux0_vf40.4_omf20_kappa10_eta23000_d0.01_delta16_on1_off1_sat1_xiv7.36e-06_xiom9.82e-06_L2.5_T5_sbh0-small-step.dat};
        \addplot table [x index=0, y index=3]
        {distributions/expandedalgsto_M64_N64_tsteps51200_initfree_trials1024_bmax10_cflux0_vf40.4_omf20_kappa10_eta23000_d0.01_delta16_on1_off1_sat1_xiv7.36e-06_xiom9.82e-06_L2.5_T5_sbh0-small-step.dat};
        \addplot table [x index=0, y index=5]
        {distributions/expandedalgsto_M64_N64_tsteps51200_initfree_trials1024_bmax10_cflux0_vf40.4_omf20_kappa10_eta23000_d0.01_delta16_on1_off1_sat1_xiv7.36e-06_xiom9.82e-06_L2.5_T5_sbh0-small-step.dat};
        \addplot table [x index=0, y index=7]
        {distributions/expandedalgsto_M64_N64_tsteps51200_initfree_trials1024_bmax10_cflux0_vf40.4_omf20_kappa10_eta23000_d0.01_delta16_on1_off1_sat1_xiv7.36e-06_xiom9.82e-06_L2.5_T5_sbh0-small-step.dat};
      \end{axis}
    \end{tikzpicture}
  \end{subfigure}
  \hfill
  \begin{subfigure}{0.48\textwidth}
    \begin{tikzpicture}
      \begin{axis}[
        legend entries={Data, Exponential, Uniform, Normal, Gamma},
        legend pos = south east,
        xmin = 0, xmax = 2,
        xlabel = Step size ($\mu$m),
        ylabel = Cumulative Probability,
        ]
        \addplot table[x index=0, y expr=(\coordindex+1)/1187]
        {simulations/expandedalgsto_M64_N64_tsteps51200_initfree_trials1024_bmax10_cflux0_vf40.4_omf20_kappa10_eta23000_d0.01_delta16_on1_off1_sat1_xiv7.36e-06_xiom9.82e-06_L2.5_T5_sbh0-small-step.dat};
        \addplot table [x index=0, y index=2]
        {distributions/expandedalgsto_M64_N64_tsteps51200_initfree_trials1024_bmax10_cflux0_vf40.4_omf20_kappa10_eta23000_d0.01_delta16_on1_off1_sat1_xiv7.36e-06_xiom9.82e-06_L2.5_T5_sbh0-small-step.dat};
        \addplot table [x index=0, y index=4]
        {distributions/expandedalgsto_M64_N64_tsteps51200_initfree_trials1024_bmax10_cflux0_vf40.4_omf20_kappa10_eta23000_d0.01_delta16_on1_off1_sat1_xiv7.36e-06_xiom9.82e-06_L2.5_T5_sbh0-small-step.dat};
        \addplot table [x index=0, y index=6]
        {distributions/expandedalgsto_M64_N64_tsteps51200_initfree_trials1024_bmax10_cflux0_vf40.4_omf20_kappa10_eta23000_d0.01_delta16_on1_off1_sat1_xiv7.36e-06_xiom9.82e-06_L2.5_T5_sbh0-small-step.dat};
        \addplot table [x index=0, y index=8]
        {distributions/expandedalgsto_M64_N64_tsteps51200_initfree_trials1024_bmax10_cflux0_vf40.4_omf20_kappa10_eta23000_d0.01_delta16_on1_off1_sat1_xiv7.36e-06_xiom9.82e-06_L2.5_T5_sbh0-small-step.dat};
      \end{axis}
    \end{tikzpicture}
  \end{subfigure}
  \caption[$\kappa=10$ step size]{Data and fit of several
    distributions to small step size data from simulations of the full
    model with $\kappa=10$ ($N = 1\,187$). Parameters of the distributions are:
    $\lambda = 0.72$, $(\mu, \sigma) = (0.72, 0.43)$, and
    $(\theta, k) = (0.23, 3.17)$. $\tn{AIC}_\tn{exp} = 3\,801$,
    $\tn{AIC}_\tn{unif} = 1\,646$, $\tn{AIC}_\tn{norm} = 1\,381$,
    $\tn{AIC}_\tn{gamma} = 953$.}
  \label{fig:k10-step-fit}
\end{figure}

\begin{figure}
  \centering
  \begin{subfigure}{0.48\textwidth}
    \begin{tikzpicture}
      \begin{axis}[
        legend entries={Data, Exponential, Uniform, Normal, Gamma},
        xmin = 0, xmax = 2,
        xlabel = Step size ($\mu$m),
        ylabel = Probability density,
        ]
        \addplot+[hist=density, fill] table [y index=0]
        {simulations/expandedalgsto_M64_N64_tsteps51200_initfree_trials1024_bmax10_cflux0_vf40.4_omf20_kappa5_eta23000_d0.01_delta16_on1_off1_sat1_xiv7.36e-06_xiom9.82e-06_L2.5_T5_sbh0-small-step.dat};
        \addplot table [x index=0, y index=1]
        {distributions/expandedalgsto_M64_N64_tsteps51200_initfree_trials1024_bmax10_cflux0_vf40.4_omf20_kappa5_eta23000_d0.01_delta16_on1_off1_sat1_xiv7.36e-06_xiom9.82e-06_L2.5_T5_sbh0-small-step.dat};
        \addplot table [x index=0, y index=3]
        {distributions/expandedalgsto_M64_N64_tsteps51200_initfree_trials1024_bmax10_cflux0_vf40.4_omf20_kappa5_eta23000_d0.01_delta16_on1_off1_sat1_xiv7.36e-06_xiom9.82e-06_L2.5_T5_sbh0-small-step.dat};
        \addplot table [x index=0, y index=5]
        {distributions/expandedalgsto_M64_N64_tsteps51200_initfree_trials1024_bmax10_cflux0_vf40.4_omf20_kappa5_eta23000_d0.01_delta16_on1_off1_sat1_xiv7.36e-06_xiom9.82e-06_L2.5_T5_sbh0-small-step.dat};
        \addplot table [x index=0, y index=7]
        {distributions/expandedalgsto_M64_N64_tsteps51200_initfree_trials1024_bmax10_cflux0_vf40.4_omf20_kappa5_eta23000_d0.01_delta16_on1_off1_sat1_xiv7.36e-06_xiom9.82e-06_L2.5_T5_sbh0-small-step.dat};
      \end{axis}
    \end{tikzpicture}
  \end{subfigure}
  \hfill
  \begin{subfigure}{0.48\textwidth}
    \begin{tikzpicture}
      \begin{axis}[
        legend entries={Data, Exponential, Uniform, Normal, Gamma},
        legend pos = south east,
        xmin = 0, xmax = 2,
        xlabel = Step size ($\mu$m),
        ylabel = Cumulative Probability,
        ]
        \addplot table[x index=0, y expr=(\coordindex+1)/307]
        {simulations/expandedalgsto_M64_N64_tsteps51200_initfree_trials1024_bmax10_cflux0_vf40.4_omf20_kappa5_eta23000_d0.01_delta16_on1_off1_sat1_xiv7.36e-06_xiom9.82e-06_L2.5_T5_sbh0-small-step.dat};
        \addplot table [x index=0, y index=2]
        {distributions/expandedalgsto_M64_N64_tsteps51200_initfree_trials1024_bmax10_cflux0_vf40.4_omf20_kappa5_eta23000_d0.01_delta16_on1_off1_sat1_xiv7.36e-06_xiom9.82e-06_L2.5_T5_sbh0-small-step.dat};
        \addplot table [x index=0, y index=4]
        {distributions/expandedalgsto_M64_N64_tsteps51200_initfree_trials1024_bmax10_cflux0_vf40.4_omf20_kappa5_eta23000_d0.01_delta16_on1_off1_sat1_xiv7.36e-06_xiom9.82e-06_L2.5_T5_sbh0-small-step.dat};
        \addplot table [x index=0, y index=6]
        {distributions/expandedalgsto_M64_N64_tsteps51200_initfree_trials1024_bmax10_cflux0_vf40.4_omf20_kappa5_eta23000_d0.01_delta16_on1_off1_sat1_xiv7.36e-06_xiom9.82e-06_L2.5_T5_sbh0-small-step.dat};
        \addplot table [x index=0, y index=8]
        {distributions/expandedalgsto_M64_N64_tsteps51200_initfree_trials1024_bmax10_cflux0_vf40.4_omf20_kappa5_eta23000_d0.01_delta16_on1_off1_sat1_xiv7.36e-06_xiom9.82e-06_L2.5_T5_sbh0-small-step.dat};
      \end{axis}
    \end{tikzpicture}
  \end{subfigure}
  \caption[$\kappa=5$ step size]{Data and fit of several distributions
    to small step size data from simulations of the full model with
    $\kappa=5$ ($N = 307$). Parameters of the distributions are:
    $\lambda = 0.82$, $(\mu, \sigma) = (0.82, 0.45)$, and
    $(\theta, k) = (0.22, 3.69)$. $\tn{AIC}_\tn{exp} = 789$,
    $\tn{AIC}_\tn{unif} = 426$, $\tn{AIC}_\tn{norm} = 388$,
    $\tn{AIC}_\tn{gamma} = 296$.}
  \label{fig:k5-step-fit}
\end{figure}

\begin{figure}
  \centering
  \begin{subfigure}{0.48\textwidth}
    \begin{tikzpicture}
      \begin{axis}[
        legend entries={Data, Exponential, Uniform, Normal, Gamma},
        xmin = 0, xmax = 2,
        xlabel = Step size ($\mu$m),
        ylabel = Probability density,
        ]
        \addplot+[hist=density, fill] table [y index=0]
        {simulations/expandedalgsto_M64_N64_tsteps51200_initfree_trials1024_bmax10_cflux0_vf40.4_omf20_kappa1_eta23000_d0.01_delta16_on1_off1_sat1_xiv7.36e-06_xiom9.82e-06_L2.5_T5_sbh0-small-step.dat};
        \addplot table [x index=0, y index=1]
        {distributions/expandedalgsto_M64_N64_tsteps51200_initfree_trials1024_bmax10_cflux0_vf40.4_omf20_kappa1_eta23000_d0.01_delta16_on1_off1_sat1_xiv7.36e-06_xiom9.82e-06_L2.5_T5_sbh0-small-step.dat};
        \addplot table [x index=0, y index=3]
        {distributions/expandedalgsto_M64_N64_tsteps51200_initfree_trials1024_bmax10_cflux0_vf40.4_omf20_kappa1_eta23000_d0.01_delta16_on1_off1_sat1_xiv7.36e-06_xiom9.82e-06_L2.5_T5_sbh0-small-step.dat};
        \addplot table [x index=0, y index=5]
        {distributions/expandedalgsto_M64_N64_tsteps51200_initfree_trials1024_bmax10_cflux0_vf40.4_omf20_kappa1_eta23000_d0.01_delta16_on1_off1_sat1_xiv7.36e-06_xiom9.82e-06_L2.5_T5_sbh0-small-step.dat};
        \addplot table [x index=0, y index=7]
        {distributions/expandedalgsto_M64_N64_tsteps51200_initfree_trials1024_bmax10_cflux0_vf40.4_omf20_kappa1_eta23000_d0.01_delta16_on1_off1_sat1_xiv7.36e-06_xiom9.82e-06_L2.5_T5_sbh0-small-step.dat};
      \end{axis}
    \end{tikzpicture}
  \end{subfigure}
  \hfill
  \begin{subfigure}{0.48\textwidth}
    \begin{tikzpicture}
      \begin{axis}[
        legend entries={Data, Exponential, Uniform, Normal, Gamma},
        legend pos = south east,
        xmin = 0, xmax = 2,
        xlabel = Step size ($\mu$m),
        ylabel = Cumulative Probability,
        ]
        \addplot+[const plot] table[x index=0, y expr=(\coordindex+1)/16]
        {simulations/expandedalgsto_M64_N64_tsteps51200_initfree_trials1024_bmax10_cflux0_vf40.4_omf20_kappa1_eta23000_d0.01_delta16_on1_off1_sat1_xiv7.36e-06_xiom9.82e-06_L2.5_T5_sbh0-small-step.dat};
        \addplot table [x index=0, y index=2]
        {distributions/expandedalgsto_M64_N64_tsteps51200_initfree_trials1024_bmax10_cflux0_vf40.4_omf20_kappa1_eta23000_d0.01_delta16_on1_off1_sat1_xiv7.36e-06_xiom9.82e-06_L2.5_T5_sbh0-small-step.dat};
        \addplot table [x index=0, y index=4]
        {distributions/expandedalgsto_M64_N64_tsteps51200_initfree_trials1024_bmax10_cflux0_vf40.4_omf20_kappa1_eta23000_d0.01_delta16_on1_off1_sat1_xiv7.36e-06_xiom9.82e-06_L2.5_T5_sbh0-small-step.dat};
        \addplot table [x index=0, y index=6]
        {distributions/expandedalgsto_M64_N64_tsteps51200_initfree_trials1024_bmax10_cflux0_vf40.4_omf20_kappa1_eta23000_d0.01_delta16_on1_off1_sat1_xiv7.36e-06_xiom9.82e-06_L2.5_T5_sbh0-small-step.dat};
        \addplot table [x index=0, y index=8]
        {distributions/expandedalgsto_M64_N64_tsteps51200_initfree_trials1024_bmax10_cflux0_vf40.4_omf20_kappa1_eta23000_d0.01_delta16_on1_off1_sat1_xiv7.36e-06_xiom9.82e-06_L2.5_T5_sbh0-small-step.dat};
      \end{axis}
    \end{tikzpicture}
  \end{subfigure}
  \caption[$\kappa=1$ step size]{Data and fit of several distributions
    to small step size data from simulations of the full model with
    $\kappa = 1$ ($N = 16$). Parameters of the distributions are:
    $\lambda = 0.72$, $(\mu, \sigma) = (0.72, 0.40)$, and
    $(\theta, k) = (0.17, 4.29)$. $\tn{AIC}_\tn{exp} = 53$,
    $\tn{AIC}_\tn{unif} = 22$, $\tn{AIC}_\tn{norm} = 20$,
    $\tn{AIC}_\tn{gamma} = 13$.}
  \label{fig:k1-step-fit}
\end{figure}

% \tikzexternaldisable
% \begin{figure}
%   \centering
%   \begin{subfigure}{0.48\textwidth}
%     \begin{tikzpicture}
%       \begin{axis}[
%         legend entries={Uniform step},
%         ymin=0,
%         xlabel = Step size,
%         ylabel = Probability density,
%         ]
%         \addplot+[hist=density, fill] table[y index=0]
%         {simulations/testmod-unif-step.dat};
%       \end{axis}
%     \end{tikzpicture}
%   \end{subfigure}
%   \hfill
%   \begin{subfigure}{0.48\textwidth}
%     \begin{tikzpicture}
%       \begin{axis}[
%         legend entries = {Normal step},
%         ymin=0,
%         xlabel = Step size,
%         ylabel = Probability density,
%         ]
%         \addplot+[hist=density, fill] table[y index=0]
%         {simulations/testmod-norm-step.dat};
%       \end{axis}
%     \end{tikzpicture}
%   \end{subfigure}
%   \\
%   \begin{subfigure}{0.48\textwidth}
%     \begin{tikzpicture}
%       \begin{axis}[
%         legend entries = {Gamma step},
%         ymin=0,
%         xlabel = Step size,
%         ylabel = Probability density,
%         ]
%         \addplot+[hist=density, fill] table[y index=0]
%         {simulations/testmod-gamma-step.dat};
%       \end{axis}
%     \end{tikzpicture}
%   \end{subfigure}
%   \hfill
%   \begin{subfigure}{0.48\textwidth}
%     \begin{tikzpicture}
%       \begin{axis}[
%         legend entries = {Deterministic step},
%         ymin=0,
%         xlabel = Step size,
%         ylabel = Probability density,
%         ]
%         \addplot+[hist=density, fill] table[y index=0]
%         {simulations/testmod-delta-step.dat};
%       \end{axis}
%     \end{tikzpicture}
%   \end{subfigure}
%   \caption{Distribution of ``small'' step sizes for 4 different models
%   of transition. All distributions chosen so that $\mu = 1/\lambda$.}
%   \label{fig:exact-unif-step-cmp-equal-rates}
% \end{figure}
% \tikzexternalenable

\section{Pause times}
\label{sec:pause-times}

As mentioned above, the distribution of pause times is given by the
distribution of exit times from the pause subsystem shown in Figure
\ref{fig:pause-subsystem}. The system of ODEs that describe the
probabilities of a paused platelet being in a certain state is given
in equations (\ref{eq:qu-nd})---(\ref{eq:qbf-nd}). For convenience, I
reproduce them here:
\begin{align*}
  \Der{q_U}{t} &= \alpha q_b \\
  \Der{q_T}{t} &= \eta q_{bf} \\
  \Der{q_b}{t} &= -(\alpha + \delta) q_b + \gamma q_{bf} \\
  \Der{q_{bf}}{t} &= \delta q_b - (\gamma + \eta) q_{bf}.
\end{align*}
Here $t$ is the time elapsed since the beginning of a pause event, and
$q_i(t)$ is the probability that a platelet is in state $i$ at time
$t$. Because all pause events start with the platelet in state $B$,
$q_b(0) = 1$ and $q_i(0) = 0$ for $i \neq b$. The probability that a
platelet is bound at time $t$ is given by $1 - q_b(t) - q_{bf}(t)$.

\newpage

\section{Fitting the model to experimental data}
\label{sec:fitt-model-exper}

To fit the whole model (i.e. the models of step size and dwell time
distribution) to the data available, I performed a sequence of 3 fits:
\begin{enumerate}
\item Filter out the small step sizes using a pre-defined threshold
  $\ell_\tn{thresh}$, then fit the parameters of a (truncated) gamma
  distribution to the small steps. Specifically, I assume the small
  steps have a distribution:
  \begin{equation}
    f_\tn{small}(x; k, \theta) = \frac{x^{k-1}
      \exp\left(-\frac{x}{\theta}\right)}{\theta^k
      \Gamma(k)} \bigg/ \frac{\gamma\left(k,
        \frac{\ell_\tn{thresh}}{\theta}\right)}{\Gamma(k)}, \quad
    \tn{for} \quad 0 \le x \le \ell_\tn{thresh},
    \label{eq:small-step-pdf-gamma}
  \end{equation}
  and then estimate the parameters $k$ and $\theta$ by maximum
  likelihood estimation. Here $\gamma(k, y)$ is the lower incomplete
  gamma function:
  \begin{equation*}
    \int_{0}^{y} t^{k-1} \exp(-t) dt.
  \end{equation*}
  Truncating the domain and renormalizing
  properly is important because we've processed the data by excluding
  any steps above $\ell_\tn{thresh}$.
\item Next, I look at all the step data and fit parameters in the
  model distribution of all step sizes:
  \begin{equation}
    \label{eq:step-pdf-gamma}
    f(x; \chi, \beta, k, \theta) = \chi \frac{x^{k-1}
      \exp\left(-\frac{x}{\theta}\right)}{\theta^k \Gamma(k)} + (1 -
    \chi) \beta \exp(-\beta x).
  \end{equation}
  Two of the parameters---$k$ and $\theta$---were estimated in step 1,
  so I find MLE of the parameters $\chi$ and $\beta$ next.
\item Finally, I use the distribution of dwell times specified in the
  previous section to estimate the other parameters in the model
  ($\alpha$, $\gamma$, $\delta$, and $\eta$). The PDF of the dwell
  times is given by:
  \begin{equation}
    \label{eq:dwell-pdf}
    g(t) = \alpha q_b(t) + \eta q_{bf}(t),
  \end{equation}
  where $q_b$ and $q_{bf}$ are solutions to the ODE system
  (\ref{eq:qu-nd})---(\ref{eq:qbf-nd}). Again, I use MLE to estimate
  the remaining 4 parameters, subject to the constraint
  $\frac{\alpha (\gamma + \eta)}{\alpha (\gamma + \eta) + \delta \eta}
  = \chi$, which was derived in Section \ref{sec:exper-step-size}.
\end{enumerate}

There are a couple of things to point out about the process of
fitting:
\begin{itemize}
\item Step 1 was necessary, because when I tried to estimate the 4
  parameters in equation (\ref{eq:step-pdf-gamma}) without fixing $k$
  and $\theta$, the optimizer always returned sets of parameter values
  where the exponential distribution was used to fit the small steps,
  and the gamma distribution was used to fit the large
  steps. Mathematically, this means that $\beta$ was large relative to
  $k \theta$ (the mean of the gamma distribution).
\item The fitting process of the dwell time distribution is sensitive
  to the initial guess. I originally settled on using an initial guess
  of $\gamma = \delta = \eta = 1$ and
  $\alpha = \frac{\chi}{2 - 2\chi}$ because that guess results in
  parameter values that are mostly between 0.1--10. However, starting
  with a slightly larger guess for the values of $\gamma$, $\delta$,
  and $\eta$ ($\gamma = \delta = \eta = e^1$) resulted in a slightly
  higher maximum ($\sim 1\%$) of the likelihood function in all each
  of the 4 datasets. I could try using a global optimization routine,
  or try a bunch of different initial guesses to find a global optimum
  for the likelihood function.
\item In the simulations, $V^*$ is the final parameter that hasn't yet
  been fit. I chose this parameter in two different ways. For the
  first, I set $V^*$ for each experiment to be equal to the maximum
  observed velocity of that particular experiment, that is the $V^*$
  differed across each experiment. This gave the best agreement
  between the actual and simulated velocities, but contradicts one of
  our assumptions of the model: that the unbound and transitioning
  platelets all move at the same speed.

  The other choice is to set the $V^*$ values to be the same across
  all the experiments, so I chose $V^*$ to be the maximum observed
  velocity across all of the experiments: $68 \,\mu
  \tn{m}/\tn{s}$. This resulted in much poorer agreement between the
  actual and simulated velocities.
\end{itemize}

\begin{figure}
  \centering
  \begin{subfigure}{0.48 \textwidth}
    \begin{tikzpicture}
      \begin{axis}[
        legend entries={Data, Model},
        legend pos = north east,
        xlabel = Step size ($\mu$m),
        ylabel = Probability Density,
        ]
        \addplot+[hist=density, fill] table [y index=0]
        {simulations/HC-step.dat};
        \addplot table [x index=0, y index=1]
        {ml-estimates/HC-step-dst.dat};
      \end{axis}
    \end{tikzpicture}
  \end{subfigure}
  \hfill
  \begin{subfigure}{0.48\textwidth}
    \begin{tikzpicture}
      \begin{axis}[
        legend entries={Data, Model},
        legend pos = south east,
        xlabel = Step size ($\mu$m),
        ylabel = Cumulative Probability,
        ]
        \addplot+[const plot] table[x index=0, y expr=(\coordindex+1)/21]
        {simulations/HC-step.dat};
        \addplot table [x index=0, y index=2]
        {ml-estimates/HC-step-dst.dat};
      \end{axis}
    \end{tikzpicture}
  \end{subfigure}
  \\
  \begin{subfigure}{0.48 \textwidth}
    \begin{tikzpicture}
      \begin{axis}[
        legend entries={Data, Model},
        legend pos = north east,
        xlabel = Pause time (s),
        ylabel = Probability Density,
        ]
        \addplot+[hist=density, fill] table [y index=0]
        {simulations/HC-dwell.dat};
        \addplot table [x index=0, y index=1]
        {ml-estimates/HC-dwell-dst.dat};
      \end{axis}
    \end{tikzpicture}
  \end{subfigure}
  \hfill
  \begin{subfigure}{0.48\textwidth}
    \begin{tikzpicture}
      \begin{axis}[
        legend entries={Data, Model},
        legend pos = south east,
        xlabel = Pause time (s),
        ylabel = Cumulative Probability,
        ]
        \addplot+[const plot] table[x index=0, y expr=(\coordindex+1)/38]
        {simulations/HC-dwell.dat};
        \addplot table [x index=0, y index=2]
        {ml-estimates/HC-dwell-dst.dat};
      \end{axis}
    \end{tikzpicture}
  \end{subfigure}
  \\
  \begin{subfigure}{0.48 \textwidth}
    \begin{tikzpicture}
      \begin{axis}[
        legend entries={Data, Model},
        legend pos = north east,
        xlabel = Average velocity,
        ylabel = Probability Density,
        fill opacity = 0.75,
        ]
        \addplot+[hist=density, fill] table [y index=0]
        {simulations/HC-vel.dat};
        \addplot+[hist=density, fill] table [y expr=\thisrowno{0}*52]
        {simulations/HCsamp-gamma-sim.dat};
      \end{axis}
    \end{tikzpicture}
  \end{subfigure}
  \hfill
  \begin{subfigure}{0.48\textwidth}
    \begin{tikzpicture}
      \begin{axis}[
        legend entries={Data, Model},
        legend pos = south east,
        xlabel = Average velocity,
        ylabel = Cumulative Probability,
        ]
        \addplot+[const plot] table [x index=0, y
        expr=(\coordindex+1)/38] {simulations/HC-vel.dat};
        \addplot+[const plot] table [x expr=\thisrowno{0}*52, y
        expr=(\coordindex+1)/1024]
        {simulations/HCsamp-gamma-sim.dat};
      \end{axis}
    \end{tikzpicture}
  \end{subfigure}
  \caption{Fit of step size and dwell time distributions to HC PRP
    data $\chi = 0.22$, $\beta = 0.12$, $k = 4.31$, $\theta = 0.15$,
    $\alpha = 0.41$, $\gamma = 0.04$, $\delta = 1.58$, $\eta =
    0.41$. $V^* = 52$}
  \label{fig:HC-whole-model-fit}
\end{figure}

\begin{figure}
  \centering
  \begin{subfigure}{0.48 \textwidth}
    \begin{tikzpicture}
      \begin{axis}[
        legend entries={Data, Model},
        legend pos = north east,
        xlabel = Step size ($\mu$m),
        ylabel = Probability Density,
        ]
        \addplot+[hist=density, fill] table [y index=0]
        {simulations/CC-step.dat};
        \addplot table [x index=0, y index=1]
        {ml-estimates/CC-step-dst.dat};
      \end{axis}
    \end{tikzpicture}
  \end{subfigure}
  \hfill
  \begin{subfigure}{0.48\textwidth}
    \begin{tikzpicture}
      \begin{axis}[
        legend entries={Data, Model},
        legend pos = south east,
        xlabel = Step size ($\mu$m),
        ylabel = Cumulative Probability,
        ]
        \addplot+[const plot] table[x index=0, y expr=(\coordindex+1)/40]
        {simulations/CC-step.dat};
        \addplot table [x index=0, y index=2]
        {ml-estimates/CC-step-dst.dat};
      \end{axis}
    \end{tikzpicture}
  \end{subfigure}
  \\
  \begin{subfigure}{0.48 \textwidth}
    \begin{tikzpicture}
      \begin{axis}[
        legend entries={Data, Model},
        legend pos = north east,
        xlabel = Pause time (s),
        ylabel = Probability Density,
        ]
        \addplot+[hist=density, fill] table [y index=0]
        {simulations/CC-dwell.dat};
        \addplot table [x index=0, y index=1]
        {ml-estimates/CC-dwell-dst.dat};
      \end{axis}
    \end{tikzpicture}
  \end{subfigure}
  \hfill
  \begin{subfigure}{0.48\textwidth}
    \begin{tikzpicture}
      \begin{axis}[
        legend entries={Data, Model},
        legend pos = south east,
        xlabel = Pause time (s),
        ylabel = Cumulative Probability,
        ]
        \addplot+[const plot] table[x index=0, y expr=(\coordindex+1)/54]
        {simulations/CC-dwell.dat};
        \addplot table [x index=0, y index=2]
        {ml-estimates/CC-dwell-dst.dat};
      \end{axis}
    \end{tikzpicture}
  \end{subfigure}
  \\
  \begin{subfigure}{0.48 \textwidth}
    \begin{tikzpicture}
      \begin{axis}[
        legend entries={Data, Model},
        legend pos = north east,
        xlabel = Average velocity,
        ylabel = Probability Density,
        fill opacity = 0.75,
        ]
        \addplot+[hist=density, fill] table [y index=0]
        {simulations/CC-vel.dat};
        \addplot+[hist=density, fill] table [y expr=\thisrowno{0}*53]
        {simulations/CCsamp-gamma-sim.dat};
      \end{axis}
    \end{tikzpicture}
  \end{subfigure}
  \hfill
  \begin{subfigure}{0.48\textwidth}
    \begin{tikzpicture}
      \begin{axis}[
        legend entries={Data, Model},
        legend pos = south east,
        xlabel = Average velocity,
        ylabel = Cumulative Probability,
        ]
        \addplot+[const plot] table [x index=0, y
        expr=(\coordindex+1)/49] {simulations/CC-vel.dat};
        \addplot+[const plot] table [x expr=\thisrowno{0}*53, y
        expr=(\coordindex+1)/1024]
        {simulations/CCsamp-gamma-sim.dat};
      \end{axis}
    \end{tikzpicture}
  \end{subfigure}
  \caption{Fit of step size and dwell time distributions to CC PRP
    data $\chi = 0.27$, $\beta = 0.16$, $k = 2.13$, $\theta = 0.54$,
    $\alpha = 0.37$, $\gamma = 0.08$, $\delta = 1.25$, $\eta =
    0.37$. $V^* = 53$}
  \label{fig:CC-whole-model-fit}
\end{figure}

\begin{figure}
  \centering
  \begin{subfigure}{0.48 \textwidth}
    \begin{tikzpicture}
      \begin{axis}[
        legend entries={Data, Model},
        legend pos = north east,
        xlabel = Step size ($\mu$m),
        ylabel = Probability Density,
        ]
        \addplot+[hist=density, fill] table [y index=0]
        {simulations/HCwhole-step.dat};
        \addplot table [x index=0, y index=1]
        {ml-estimates/HCwhole-step-dst.dat};
      \end{axis}
    \end{tikzpicture}
  \end{subfigure}
  \hfill
  \begin{subfigure}{0.48\textwidth}
    \begin{tikzpicture}
      \begin{axis}[
        legend entries={Data, Model},
        legend pos = south east,
        xlabel = Step size ($\mu$m),
        ylabel = Cumulative Probability,
        ]
        \addplot+[const plot] table[x index=0, y expr=(\coordindex+1)/11]
        {simulations/HCwhole-step.dat};
        \addplot table [x index=0, y index=2]
        {ml-estimates/HCwhole-step-dst.dat};
      \end{axis}
    \end{tikzpicture}
  \end{subfigure}
  \\
  \begin{subfigure}{0.48 \textwidth}
    \begin{tikzpicture}
      \begin{axis}[
        legend entries={Data, Model},
        legend pos = north east,
        xlabel = Pause time (s),
        ylabel = Probability Density,
        ]
        \addplot+[hist=density, fill] table [y index=0]
        {simulations/HCwhole-dwell.dat};
        \addplot table [x index=0, y index=1]
        {ml-estimates/HCwhole-dwell-dst.dat};
      \end{axis}
    \end{tikzpicture}
  \end{subfigure}
  \hfill
  \begin{subfigure}{0.48\textwidth}
    \begin{tikzpicture}
      \begin{axis}[
        legend entries={Data, Model},
        legend pos = south east,
        xlabel = Pause time (s),
        ylabel = Cumulative Probability,
        ]
        \addplot+[const plot] table[x index=0, y expr=(\coordindex+1)/43]
        {simulations/HCwhole-dwell.dat};
        \addplot table [x index=0, y index=2]
        {ml-estimates/HCwhole-dwell-dst.dat};
      \end{axis}
    \end{tikzpicture}
  \end{subfigure}
  \\
  \begin{subfigure}{0.48 \textwidth}
    \begin{tikzpicture}
      \begin{axis}[
        legend entries={Data, Model},
        legend pos = north east,
        xlabel = Average velocity,
        ylabel = Probability Density,
        fill opacity = 0.75,
        ]
        \addplot+[hist=density, fill] table [y index=0]
        {simulations/HCwhole-vel.dat};
        \addplot+[hist=density, fill] table [y expr=\thisrowno{0}*68]
        {simulations/HCwholesamp-gamma-sim.dat};
      \end{axis}
    \end{tikzpicture}
  \end{subfigure}
  \hfill
  \begin{subfigure}{0.48\textwidth}
    \begin{tikzpicture}
      \begin{axis}[
        legend entries={Data, Model},
        legend pos = south east,
        xlabel = Average velocity,
        ylabel = Cumulative Probability,
        ]
        \addplot+[const plot] table [x index=0, y
        expr=(\coordindex+1)/30] {simulations/HCwhole-vel.dat};
        \addplot+[const plot] table [x expr=\thisrowno{0}*68, y
        expr=(\coordindex+1)/1024]
        {simulations/HCwholesamp-gamma-sim.dat};
      \end{axis}
    \end{tikzpicture}
  \end{subfigure}
  \caption{Fit of step size and dwell time distributions to HC whole
    blood data $\chi = 0.10$, $\beta = 0.07$, $k = 1.89$, $\theta =
    0.36$, $\alpha = 0.34$, $\gamma = 0.002$, $\delta = 2.96$, $\eta =
    0.34$. $V^* = 68$}
  \label{fig:HCwhole-whole-model-fit}
\end{figure}

\begin{figure}
  \centering
  \begin{subfigure}{0.48 \textwidth}
    \begin{tikzpicture}
      \begin{axis}[
        legend entries={Data, Model},
        legend pos = north east,
        xlabel = Step size ($\mu$m),
        ylabel = Probability Density,
        ]
        \addplot+[hist=density, fill] table [y index=0]
        {simulations/CCwhole-step.dat};
        \addplot table [x index=0, y index=1]
        {ml-estimates/CCwhole-step-dst.dat};
      \end{axis}
    \end{tikzpicture}
  \end{subfigure}
  \hfill
  \begin{subfigure}{0.48\textwidth}
    \begin{tikzpicture}
      \begin{axis}[
        legend entries={Data, Model},
        legend pos = south east,
        xlabel = Step size ($\mu$m),
        ylabel = Cumulative Probability,
        ]
        \addplot+[const plot] table[x index=0, y expr=(\coordindex+1)/76]
        {simulations/CCwhole-step.dat};
        \addplot table [x index=0, y index=2]
        {ml-estimates/CCwhole-step-dst.dat};
      \end{axis}
    \end{tikzpicture}
  \end{subfigure}
  \\
  \begin{subfigure}{0.48 \textwidth}
    \begin{tikzpicture}
      \begin{axis}[
        legend entries={Data, Model},
        legend pos = north east,
        xlabel = Pause time (s),
        ylabel = Probability Density,
        ]
        \addplot+[hist=density, fill] table [y index=0]
        {simulations/CCwhole-dwell.dat};
        \addplot table [x index=0, y index=1]
        {ml-estimates/CCwhole-dwell-dst.dat};
      \end{axis}
    \end{tikzpicture}
  \end{subfigure}
  \hfill
  \begin{subfigure}{0.48\textwidth}
    \begin{tikzpicture}
      \begin{axis}[
        legend entries={Data, Model},
        legend pos = south east,
        xlabel = Pause time (s),
        ylabel = Cumulative Probability,
        ]
        \addplot+[const plot] table[x index=0, y expr=(\coordindex+1)/122]
        {simulations/CCwhole-dwell.dat};
        \addplot table [x index=0, y index=2]
        {ml-estimates/CCwhole-dwell-dst.dat};
      \end{axis}
    \end{tikzpicture}
  \end{subfigure}
  \\
  \begin{subfigure}{0.48 \textwidth}
    \begin{tikzpicture}
      \begin{axis}[
        legend entries={Data, Model},
        legend pos = north east,
        xlabel = Average velocity,
        ylabel = Probability Density,
        fill opacity = 0.75,
        ]
        \addplot+[hist=density, fill] table [y index=0]
        {simulations/CCwhole-vel.dat};
        \addplot+[hist=density, fill] table [y expr=\thisrowno{0}*25]
        {simulations/CCwholesamp-gamma-sim.dat};
      \end{axis}
    \end{tikzpicture}
  \end{subfigure}
  \hfill
  \begin{subfigure}{0.48\textwidth}
    \begin{tikzpicture}
      \begin{axis}[
        legend entries={Data, Model},
        legend pos = south east,
        xlabel = Average velocity,
        ylabel = Cumulative Probability,
        ]
        \addplot+[const plot] table [x index=0, y
        expr=(\coordindex+1)/43] {simulations/CCwhole-vel.dat};
        \addplot+[const plot] table [x expr=\thisrowno{0}*25, y
        expr=(\coordindex+1)/1024]
        {simulations/CCwholesamp-gamma-sim.dat};
      \end{axis}
    \end{tikzpicture}
  \end{subfigure}
  \caption{Fit of step size and dwell time distributions to CC whole
    blood data $\chi = 0.40$, $\beta = 0.21$, $k = 2.033$, $\theta =
    0.43$, $\alpha = 0.62$, $\gamma = 0.09$, $\delta = 1.05$, $\eta =
    0.62$. $V^* = 25$}
  \label{fig:CCwhole-whole-model-fit}
\end{figure}

\begin{figure}
  \centering
  \begin{subfigure}{0.48 \textwidth}
    \begin{tikzpicture}
      \begin{axis}[
        legend entries={Data, Model},
        legend pos = north east,
        xlabel = Step size ($\mu$m),
        ylabel = Probability Density,
        ]
        \addplot+[hist=density, fill] table [y index=0]
        {simulations/HC-step.dat};
        \addplot table [x index=0, y index=1]
        {ml-estimates/HC-step-dst.dat};
      \end{axis}
    \end{tikzpicture}
  \end{subfigure}
  \hfill
  \begin{subfigure}{0.48\textwidth}
    \begin{tikzpicture}
      \begin{axis}[
        legend entries={Data, Model},
        legend pos = south east,
        xlabel = Step size ($\mu$m),
        ylabel = Cumulative Probability,
        ]
        \addplot+[const plot] table[x index=0, y expr=(\coordindex+1)/21]
        {simulations/HC-step.dat};
        \addplot table [x index=0, y index=2]
        {ml-estimates/HC-step-dst.dat};
      \end{axis}
    \end{tikzpicture}
  \end{subfigure}
  \\
  \begin{subfigure}{0.48 \textwidth}
    \begin{tikzpicture}
      \begin{axis}[
        legend entries={Data, Model},
        legend pos = north east,
        xlabel = Pause time (s),
        ylabel = Probability Density,
        ]
        \addplot+[hist=density, fill] table [y index=0]
        {simulations/HC-dwell.dat};
        \addplot table [x index=0, y index=1]
        {ml-estimates/HC-dwell-dst.dat};
      \end{axis}
    \end{tikzpicture}
  \end{subfigure}
  \hfill
  \begin{subfigure}{0.48\textwidth}
    \begin{tikzpicture}
      \begin{axis}[
        legend entries={Data, Model},
        legend pos = south east,
        xlabel = Pause time (s),
        ylabel = Cumulative Probability,
        ]
        \addplot+[const plot] table[x index=0, y expr=(\coordindex+1)/38]
        {simulations/HC-dwell.dat};
        \addplot table [x index=0, y index=2]
        {ml-estimates/HC-dwell-dst.dat};
      \end{axis}
    \end{tikzpicture}
  \end{subfigure}
  \\
  \begin{subfigure}{0.48 \textwidth}
    \begin{tikzpicture}
      \begin{axis}[
        legend entries={Data, Model},
        legend pos = north east,
        xlabel = Average velocity,
        ylabel = Probability Density,
        fill opacity = 0.75,
        ]
        \addplot+[hist=density, fill] table [y index=0]
        {simulations/HC-vel.dat};
        \addplot+[hist=density, fill] table [y expr=\thisrowno{0}*68]
        {simulations/HCsamp2-gamma-sim.dat};
      \end{axis}
    \end{tikzpicture}
  \end{subfigure}
  \hfill
  \begin{subfigure}{0.48\textwidth}
    \begin{tikzpicture}
      \begin{axis}[
        legend entries={Data, Model},
        legend pos = south east,
        xlabel = Average velocity,
        ylabel = Cumulative Probability,
        ]
        \addplot+[const plot] table [x index=0, y
        expr=(\coordindex+1)/38] {simulations/HC-vel.dat};
        \addplot+[const plot] table [x expr=\thisrowno{0}*68, y
        expr=(\coordindex+1)/1024]
        {simulations/HCsamp2-gamma-sim.dat};
      \end{axis}
    \end{tikzpicture}
  \end{subfigure}
  \caption{Fit of step size and dwell time distributions to HC PRP
    data $\chi = 0.22$, $\beta = 0.12$, $k = 4.31$, $\theta = 0.15$,
    $\alpha = 0.41$, $\gamma = 0.04$, $\delta = 1.58$, $\eta =
    0.41$. $V^* = 68$}
  \label{fig:HC-whole-model-fit}
\end{figure}

\begin{figure}
  \centering
  \begin{subfigure}{0.48 \textwidth}
    \begin{tikzpicture}
      \begin{axis}[
        legend entries={Data, Model},
        legend pos = north east,
        xlabel = Step size ($\mu$m),
        ylabel = Probability Density,
        ]
        \addplot+[hist=density, fill] table [y index=0]
        {simulations/CC-step.dat};
        \addplot table [x index=0, y index=1]
        {ml-estimates/CC-step-dst.dat};
      \end{axis}
    \end{tikzpicture}
  \end{subfigure}
  \hfill
  \begin{subfigure}{0.48\textwidth}
    \begin{tikzpicture}
      \begin{axis}[
        legend entries={Data, Model},
        legend pos = south east,
        xlabel = Step size ($\mu$m),
        ylabel = Cumulative Probability,
        ]
        \addplot+[const plot] table[x index=0, y expr=(\coordindex+1)/40]
        {simulations/CC-step.dat};
        \addplot table [x index=0, y index=2]
        {ml-estimates/CC-step-dst.dat};
      \end{axis}
    \end{tikzpicture}
  \end{subfigure}
  \\
  \begin{subfigure}{0.48 \textwidth}
    \begin{tikzpicture}
      \begin{axis}[
        legend entries={Data, Model},
        legend pos = north east,
        xlabel = Pause time (s),
        ylabel = Probability Density,
        ]
        \addplot+[hist=density, fill] table [y index=0]
        {simulations/CC-dwell.dat};
        \addplot table [x index=0, y index=1]
        {ml-estimates/CC-dwell-dst.dat};
      \end{axis}
    \end{tikzpicture}
  \end{subfigure}
  \hfill
  \begin{subfigure}{0.48\textwidth}
    \begin{tikzpicture}
      \begin{axis}[
        legend entries={Data, Model},
        legend pos = south east,
        xlabel = Pause time (s),
        ylabel = Cumulative Probability,
        ]
        \addplot+[const plot] table[x index=0, y expr=(\coordindex+1)/54]
        {simulations/CC-dwell.dat};
        \addplot table [x index=0, y index=2]
        {ml-estimates/CC-dwell-dst.dat};
      \end{axis}
    \end{tikzpicture}
  \end{subfigure}
  \\
  \begin{subfigure}{0.48 \textwidth}
    \begin{tikzpicture}
      \begin{axis}[
        legend entries={Data, Model},
        legend pos = north east,
        xlabel = Average velocity,
        ylabel = Probability Density,
        fill opacity = 0.75,
        ]
        \addplot+[hist=density, fill] table [y index=0]
        {simulations/CC-vel.dat};
        \addplot+[hist=density, fill] table [y expr=\thisrowno{0}*68]
        {simulations/CCsamp2-gamma-sim.dat};
      \end{axis}
    \end{tikzpicture}
  \end{subfigure}
  \hfill
  \begin{subfigure}{0.48\textwidth}
    \begin{tikzpicture}
      \begin{axis}[
        legend entries={Data, Model},
        legend pos = south east,
        xlabel = Average velocity,
        ylabel = Cumulative Probability,
        ]
        \addplot+[const plot] table [x index=0, y
        expr=(\coordindex+1)/49] {simulations/CC-vel.dat};
        \addplot+[const plot] table [x expr=\thisrowno{0}*68, y
        expr=(\coordindex+1)/1024]
        {simulations/CCsamp2-gamma-sim.dat};
      \end{axis}
    \end{tikzpicture}
  \end{subfigure}
  \caption{Fit of step size and dwell time distributions to CC PRP
    data $\chi = 0.27$, $\beta = 0.16$, $k = 2.13$, $\theta = 0.54$,
    $\alpha = 0.37$, $\gamma = 0.08$, $\delta = 1.25$, $\eta =
    0.37$. $V^* = 68$}
  \label{fig:CC-whole-model-fit}
\end{figure}

\begin{figure}
  \centering
  \begin{subfigure}{0.48 \textwidth}
    \begin{tikzpicture}
      \begin{axis}[
        legend entries={Data, Model},
        legend pos = north east,
        xlabel = Step size ($\mu$m),
        ylabel = Probability Density,
        ]
        \addplot+[hist=density, fill] table [y index=0]
        {simulations/HCwhole-step.dat};
        \addplot table [x index=0, y index=1]
        {ml-estimates/HCwhole-step-dst.dat};
      \end{axis}
    \end{tikzpicture}
  \end{subfigure}
  \hfill
  \begin{subfigure}{0.48\textwidth}
    \begin{tikzpicture}
      \begin{axis}[
        legend entries={Data, Model},
        legend pos = south east,
        xlabel = Step size ($\mu$m),
        ylabel = Cumulative Probability,
        ]
        \addplot+[const plot] table[x index=0, y expr=(\coordindex+1)/11]
        {simulations/HCwhole-step.dat};
        \addplot table [x index=0, y index=2]
        {ml-estimates/HCwhole-step-dst.dat};
      \end{axis}
    \end{tikzpicture}
  \end{subfigure}
  \\
  \begin{subfigure}{0.48 \textwidth}
    \begin{tikzpicture}
      \begin{axis}[
        legend entries={Data, Model},
        legend pos = north east,
        xlabel = Pause time (s),
        ylabel = Probability Density,
        ]
        \addplot+[hist=density, fill] table [y index=0]
        {simulations/HCwhole-dwell.dat};
        \addplot table [x index=0, y index=1]
        {ml-estimates/HCwhole-dwell-dst.dat};
      \end{axis}
    \end{tikzpicture}
  \end{subfigure}
  \hfill
  \begin{subfigure}{0.48\textwidth}
    \begin{tikzpicture}
      \begin{axis}[
        legend entries={Data, Model},
        legend pos = south east,
        xlabel = Pause time (s),
        ylabel = Cumulative Probability,
        ]
        \addplot+[const plot] table[x index=0, y expr=(\coordindex+1)/43]
        {simulations/HCwhole-dwell.dat};
        \addplot table [x index=0, y index=2]
        {ml-estimates/HCwhole-dwell-dst.dat};
      \end{axis}
    \end{tikzpicture}
  \end{subfigure}
  \\
  \begin{subfigure}{0.48 \textwidth}
    \begin{tikzpicture}
      \begin{axis}[
        legend entries={Data, Model},
        legend pos = north east,
        xlabel = Average velocity,
        ylabel = Probability Density,
        fill opacity = 0.75,
        ]
        \addplot+[hist=density, fill] table [y index=0]
        {simulations/HCwhole-vel.dat};
        \addplot+[hist=density, fill] table [y expr=\thisrowno{0}*68]
        {simulations/HCwholesamp2-gamma-sim.dat};
      \end{axis}
    \end{tikzpicture}
  \end{subfigure}
  \hfill
  \begin{subfigure}{0.48\textwidth}
    \begin{tikzpicture}
      \begin{axis}[
        legend entries={Data, Model},
        legend pos = south east,
        xlabel = Average velocity,
        ylabel = Cumulative Probability,
        ]
        \addplot+[const plot] table [x index=0, y
        expr=(\coordindex+1)/30] {simulations/HCwhole-vel.dat};
        \addplot+[const plot] table [x expr=\thisrowno{0}*68, y
        expr=(\coordindex+1)/1024]
        {simulations/HCwholesamp2-gamma-sim.dat};
      \end{axis}
    \end{tikzpicture}
  \end{subfigure}
  \caption{Fit of step size and dwell time distributions to HC whole
    blood data $\chi = 0.10$, $\beta = 0.07$, $k = 1.89$, $\theta =
    0.36$, $\alpha = 0.34$, $\gamma = 0.002$, $\delta = 2.96$, $\eta =
    0.34$. $V^* = 68$}
  \label{fig:HCwhole-whole-model-fit}
\end{figure}

\begin{figure}
  \centering
  \begin{subfigure}{0.48 \textwidth}
    \begin{tikzpicture}
      \begin{axis}[
        legend entries={Data, Model},
        legend pos = north east,
        xlabel = Step size ($\mu$m),
        ylabel = Probability Density,
        ]
        \addplot+[hist=density, fill] table [y index=0]
        {simulations/CCwhole-step.dat};
        \addplot table [x index=0, y index=1]
        {ml-estimates/CCwhole-step-dst.dat};
      \end{axis}
    \end{tikzpicture}
  \end{subfigure}
  \hfill
  \begin{subfigure}{0.48\textwidth}
    \begin{tikzpicture}
      \begin{axis}[
        legend entries={Data, Model},
        legend pos = south east,
        xlabel = Step size ($\mu$m),
        ylabel = Cumulative Probability,
        ]
        \addplot+[const plot] table[x index=0, y expr=(\coordindex+1)/76]
        {simulations/CCwhole-step.dat};
        \addplot table [x index=0, y index=2]
        {ml-estimates/CCwhole-step-dst.dat};
      \end{axis}
    \end{tikzpicture}
  \end{subfigure}
  \\
  \begin{subfigure}{0.48 \textwidth}
    \begin{tikzpicture}
      \begin{axis}[
        legend entries={Data, Model},
        legend pos = north east,
        xlabel = Pause time (s),
        ylabel = Probability Density,
        ]
        \addplot+[hist=density, fill] table [y index=0]
        {simulations/CCwhole-dwell.dat};
        \addplot table [x index=0, y index=1]
        {ml-estimates/CCwhole-dwell-dst.dat};
      \end{axis}
    \end{tikzpicture}
  \end{subfigure}
  \hfill
  \begin{subfigure}{0.48\textwidth}
    \begin{tikzpicture}
      \begin{axis}[
        legend entries={Data, Model},
        legend pos = south east,
        xlabel = Pause time (s),
        ylabel = Cumulative Probability,
        ]
        \addplot+[const plot] table[x index=0, y expr=(\coordindex+1)/122]
        {simulations/CCwhole-dwell.dat};
        \addplot table [x index=0, y index=2]
        {ml-estimates/CCwhole-dwell-dst.dat};
      \end{axis}
    \end{tikzpicture}
  \end{subfigure}
  \\
  \begin{subfigure}{0.48 \textwidth}
    \begin{tikzpicture}
      \begin{axis}[
        legend entries={Data, Model},
        legend pos = north east,
        xlabel = Average velocity,
        ylabel = Probability Density,
        fill opacity = 0.75,
        ]
        \addplot+[hist=density, fill] table [y index=0]
        {simulations/CCwhole-vel.dat};
        \addplot+[hist=density, fill] table [y expr=\thisrowno{0}*68]
        {simulations/CCwholesamp3-gamma-sim.dat};
      \end{axis}
    \end{tikzpicture}
  \end{subfigure}
  \hfill
  \begin{subfigure}{0.48\textwidth}
    \begin{tikzpicture}
      \begin{axis}[
        legend entries={Data, Model},
        legend pos = south east,
        xlabel = Average velocity,
        ylabel = Cumulative Probability,
        ]
        \addplot+[const plot] table [x index=0, y
        expr=(\coordindex+1)/43] {simulations/CCwhole-vel.dat};
        \addplot+[const plot] table [x expr=\thisrowno{0}*68, y
        expr=(\coordindex+1)/1024]
        {simulations/CCwholesamp2-gamma-sim.dat};
      \end{axis}
    \end{tikzpicture}
  \end{subfigure}
  \caption{Fit of step size and dwell time distributions to CC whole
    blood data $\chi = 0.40$, $\beta = 0.21$, $k = 2.033$, $\theta =
    0.43$, $\alpha = 0.62$, $\gamma = 0.09$, $\delta = 1.05$, $\eta =
    0.62$. $V^* = 68$}
  \label{fig:CCwhole-whole-model-fit}
\end{figure}

\begin{figure}
  \centering
  \begin{subfigure}{0.48 \textwidth}
    \begin{tikzpicture}
      \begin{axis}[
        legend entries={Data, Model},
        legend pos = north east,
        xlabel = Step size,
        ylabel = Probability Density,
        ]
        \addplot+[hist=density, fill] table [y index=0]
        {simulations/expandedalgsto_M64_N64_tsteps51200_initfree_trials1024_bmax10_cflux0_vf40.4_omf20_kappa10_eta23000_d0.01_delta16_on1_off1_sat1_xiv7.36e-06_xiom9.82e-06_L2.5_T5_sbh0-step.dat};
        \addplot table [x index=0, y index=1]
        {ml-estimates/expandedalgsto_M64_N64_tsteps51200_initfree_trials1024_bmax10_cflux0_vf40.4_omf20_kappa10_eta23000_d0.01_delta16_on1_off1_sat1_xiv7.36e-06_xiom9.82e-06_L2.5_T5_sbh0-step-dst.dat};
      \end{axis}
    \end{tikzpicture}
  \end{subfigure}
  \hfill
  \begin{subfigure}{0.48\textwidth}
    \begin{tikzpicture}
      \begin{axis}[
        legend entries={Data, Model},
        legend pos = south east,
        xlabel = Step size,
        ylabel = Cumulative Probability,
        ]
        \addplot+[const plot] table[x index=0, y expr=(\coordindex+1)/6818]
        {simulations/expandedalgsto_M64_N64_tsteps51200_initfree_trials1024_bmax10_cflux0_vf40.4_omf20_kappa10_eta23000_d0.01_delta16_on1_off1_sat1_xiv7.36e-06_xiom9.82e-06_L2.5_T5_sbh0-step.dat};
        \addplot table [x index=0, y index=2]
        {ml-estimates/expandedalgsto_M64_N64_tsteps51200_initfree_trials1024_bmax10_cflux0_vf40.4_omf20_kappa10_eta23000_d0.01_delta16_on1_off1_sat1_xiv7.36e-06_xiom9.82e-06_L2.5_T5_sbh0-step-dst.dat};
      \end{axis}
    \end{tikzpicture}
  \end{subfigure}
  \\
  \begin{subfigure}{0.48 \textwidth}
    \begin{tikzpicture}
      \begin{axis}[
        legend entries={Data, Model},
        legend pos = north east,
        xlabel = Pause time,
        ylabel = Probability Density,
        ]
        \addplot+[hist=density, fill] table [y index=0]
        {simulations/expandedalgsto_M64_N64_tsteps51200_initfree_trials1024_bmax10_cflux0_vf40.4_omf20_kappa10_eta23000_d0.01_delta16_on1_off1_sat1_xiv7.36e-06_xiom9.82e-06_L2.5_T5_sbh0-dwell.dat};
        \addplot table [x index=0, y index=1]
        {ml-estimates/expandedalgsto_M64_N64_tsteps51200_initfree_trials1024_bmax10_cflux0_vf40.4_omf20_kappa10_eta23000_d0.01_delta16_on1_off1_sat1_xiv7.36e-06_xiom9.82e-06_L2.5_T5_sbh0-dwell-dst.dat};
      \end{axis}
    \end{tikzpicture}
  \end{subfigure}
  \hfill
  \begin{subfigure}{0.48\textwidth}
    \begin{tikzpicture}
      \begin{axis}[
        legend entries={Data, Model},
        legend pos = south east,
        xlabel = Pause time,
        ylabel = Cumulative Probability,
        ]
        \addplot+[const plot] table[x index=0, y expr=(\coordindex+1)/7711]
        {simulations/expandedalgsto_M64_N64_tsteps51200_initfree_trials1024_bmax10_cflux0_vf40.4_omf20_kappa10_eta23000_d0.01_delta16_on1_off1_sat1_xiv7.36e-06_xiom9.82e-06_L2.5_T5_sbh0-dwell.dat};
        \addplot table [x index=0, y index=2]
        {ml-estimates/expandedalgsto_M64_N64_tsteps51200_initfree_trials1024_bmax10_cflux0_vf40.4_omf20_kappa10_eta23000_d0.01_delta16_on1_off1_sat1_xiv7.36e-06_xiom9.82e-06_L2.5_T5_sbh0-dwell-dst.dat};
      \end{axis}
    \end{tikzpicture}
  \end{subfigure}
  \\
  \begin{subfigure}{0.48 \textwidth}
    \begin{tikzpicture}
      \begin{axis}[
        legend entries={Data, Model},
        legend pos = north east,
        xlabel = Average velocity,
        ylabel = Probability Density,
        fill opacity = 0.75,
        ]
        \addplot+[hist=density, fill] table [y index=0]
        {simulations/expandedalgsto_M64_N64_tsteps51200_initfree_trials1024_bmax10_cflux0_vf40.4_omf20_kappa10_eta23000_d0.01_delta16_on1_off1_sat1_xiv7.36e-06_xiom9.82e-06_L2.5_T5_sbh0-vel.dat};
        \addplot+[hist=density, fill] table [y expr=\thisrowno{0}*40.4]
        {simulations/expandedalgsto_M64_N64_tsteps51200_initfree_trials1024_bmax10_cflux0_vf40.4_omf20_kappa10_eta23000_d0.01_delta16_on1_off1_sat1_xiv7.36e-06_xiom9.82e-06_L2.5_T5_sbh0samp-gamma-sim.dat};
      \end{axis}
    \end{tikzpicture}
  \end{subfigure}
  \hfill
  \begin{subfigure}{0.48\textwidth}
    \begin{tikzpicture}
      \begin{axis}[
        legend entries={Data, Model},
        legend pos = south east,
        xlabel = Average velocity,
        ylabel = Cumulative Probability,
        ]
        \addplot+[const plot] table [x index=0, y
        expr=(\coordindex+1)/1024] {simulations/expandedalgsto_M64_N64_tsteps51200_initfree_trials1024_bmax10_cflux0_vf40.4_omf20_kappa10_eta23000_d0.01_delta16_on1_off1_sat1_xiv7.36e-06_xiom9.82e-06_L2.5_T5_sbh0-vel.dat};
        \addplot+[const plot] table [x expr=\thisrowno{0}*40.4, y
        expr=(\coordindex+1)/1024]
        {simulations/expandedalgsto_M64_N64_tsteps51200_initfree_trials1024_bmax10_cflux0_vf40.4_omf20_kappa10_eta23000_d0.01_delta16_on1_off1_sat1_xiv7.36e-06_xiom9.82e-06_L2.5_T5_sbh0samp-gamma-sim.dat};
      \end{axis}
    \end{tikzpicture}
  \end{subfigure}
  \caption{Fit of step size and dwell time distributions to full model
    simulations $\kappa = 10$. In this plot, all quantities are
    nondimensional, with the scalings $L = 1 \mu\tn{m}$, $T = 5
    \tn{s}$, and $V = L/T$. $\chi = 0.89$, $\beta = 0.05$, $k = 5.76$, $\theta =
    0.10$, $\alpha = 14.9$, $\gamma = 46.5$, $\delta = 7.77$, $\eta =
    14.9$. $V^* = 40.3$}
  \label{fig:full-model-fit}
\end{figure}

\begin{figure}
  \centering
  \includegraphics[width=0.6\textwidth]{likelihood-plot.png}
  \caption{Heatmap of $-\ln(L)$ in $\ln(\alpha)$ and $\ln(\gamma)$}
  \label{fig:likelihood-plot}
\end{figure}
% To do:
% \begin{itemize}
% \item Finish deriving the PDF of dwell times, and verify with simulations
% \item Fit the dwell time distribution to simulated dwell time data to
%   estimate parameters
% \item Write a script to fit both dwell time and step size models to
%   data (simultaneously or sequentially?)
% \end{itemize}

\bibliographystyle{plain}
\bibliography{/Users/andrewwork/Documents/grad-school/thesis/library}

\end{document}




