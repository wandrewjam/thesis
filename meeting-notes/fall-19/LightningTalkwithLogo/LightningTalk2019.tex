\documentclass[10pt]{beamer}

\usetheme[progressbar=frametitle]{metropolis}
\makeatletter
\usepackage[caption=false]{subfig}
\usepackage{float}
\newlength\beamerleftmargin
\setlength\beamerleftmargin{\Gm@lmargin}
\usepackage[absolute,overlay]{textpos}
\usepackage{appendixnumberbeamer}
\setbeamertemplate{section in toc}[sections numbered]
\setbeamertemplate{subsection in toc}[subsections numbered]
\usepackage{booktabs}
\usepackage[scale=2]{ccicons}
\definecolor{myred}{RGB}{204,0,0}
\usepackage{tikz}
\usepackage{graphicx}
\usepackage{pgfplots}
\usepackage{caption}
\captionsetup[figure]{labelformat=empty}
\setbeamertemplate{subsection in toc}
{\leavevmode\leftskip=2em\rlap{\hskip-2em$\quad$\inserttocsectionnumber.\inserttocsubsectionnumber}$\quad$\inserttocsubsection\par}
%%%%%%
\usepgfplotslibrary{dateplot}

\usepackage{xspace}
\newcommand{\themename}{\textbf{\textsc{metropolis}}\xspace}

\title{Understanding the role of fibrinogen in fibrin gelation}
% \date{\today}
\date{}
\author{Anna Nelson\\ Advisor: Aaron Fogelson and James Keener}
\institute{University of Utah}
% \titlegraphic{\hfill\includegraphics[height=1.5cm]{logo.pdf}}

\setbeamercolor{progress bar in head/foot}{fg=gray,bg=gray}
\setbeamercolor{progress bar in section page}{fg=myred,bg=myred}
\setbeamercolor{progress bar in title separator}{fg=gray,bg=gray}
\setbeamercolor{frametitle}{bg=myred, fg=white}
\setbeamercolor{block title alerted}{fg=myred}
\setbeamercolor{alerted text}{fg = myred}
\setlength{\metropolis@titleseparator@linewidth}{1.5pt}
\setlength{\metropolis@progressonsectionpage@linewidth}{1.5pt}
 \setlength{\metropolis@progressinheadfoot@linewidth}{3pt}
\setlength{\metropolis@progressonsectionpage@linewidth}{1.5pt}

\setbeamerfont{bibliography item}{size=\scriptsize}
\setbeamerfont{bibliography entry author}{size=\scriptsize}
\setbeamerfont{bibliography entry title}{size=\scriptsize}
\setbeamerfont{bibliography entry location}{size=\scriptsize}
\setbeamerfont{bibliography entry note}{size=\scriptsize}



\begin{document}

\maketitle

\addtobeamertemplate{frametitle}{}{%
\begin{tikzpicture}[remember picture,overlay]
\node[anchor=north east,yshift=2pt] at (current page.north east) {\includegraphics[height=0.7cm]{ulogo@2x.png}};
\end{tikzpicture}}


\begin{frame}[plain]{Biological motivation}

\noindent
\hspace{-4cm}
\begin{columns}
\begin{column}{0.5\textwidth}
\includegraphics[width = 1.05\textwidth]{health.pdf}
\begin{itemize}
\item Under normal conditions, unactivated platelets and fibrinogen circulate in the blood
\end{itemize}
\end{column}
\begin{column}{0.5\textwidth}
\includegraphics[width = 1.05\textwidth]{clot_bw.pdf}
\begin{itemize}
\item Injury triggers coagulation response, generating thrombin and a fibrin clot
\end{itemize}
\end{column}
\end{columns}
\end{frame}

\begin{frame}[fragile]{Fibrinogen and fibrin}
\begin{figure}
\includegraphics[width=\textwidth]{cleavage.pdf}\\
\includegraphics[width = 0.7\textwidth]{fibrin(ogen.pdf}
\end{figure}
\begin{itemize}
\item{\textbf{Goal}: Investigate effect of fibrin-fibrinogen interactions on clot time, structure}
\end{itemize}
\end{frame}

\subsection{Fibrin branching model with fibrinogen}

\begin{frame}[fragile]{Fibrin polymerization}
\includegraphics[width = 0.9\textwidth]{protofibril.png}
\end{frame}

\begin{frame}[fragile]{Fibrin-fibrinogen branching model}
\begin{columns}
\begin{column}{0.5\textwidth}
\begin{itemize}
\item Allow only fibrinogen monomer to bind to free fibrin reaction sites
\item $C_{m,g,k}$ - oligomer with $m + 2(k+g-2)$ total fibrin monomer, $g$ total fibrinogen monomer, $k$ free fibrin reaction sites
\end{itemize}
\end{column}
\begin{column}{0.5\textwidth}
\vspace{-0.5cm}



\begin{figure}
\centering
\includegraphics[width=\textwidth]{reactions.pdf}
\vspace*{-5mm}
\caption{Reactions in fibrin branching model}
\end{figure}
\end{column}
\end{columns}
\end{frame}

\begin{frame}[fragile]{Fibrin-fibrinogen branching model}
\begin{itemize}
\item Let $R_f$ denote free fibrin reaction sites
\end{itemize}
\begin{multline*}
\scriptsize \frac{d {c}_{mgk} }{dt} = \overbrace{\frac{k_{{l}}}{2} \sum_{\substack{ m_1+m_2 = m \\ g_1+g_2= g\\ k_1+k_2=k+2}} k_1k_2 c_{m_1g_1k_1}{c}_{m_2g_2k_2} - k_{{l}}k{c}_{mgk} R_f}^{\text{link formation}} \\ \scriptsize + \overbrace{\frac{k_{{b}}}{6} \sum_{\substack{m_1+m_2+m_3= m+2 \\ g_1+g_2+g_3= g -1\\k_1+k_2+k_3  = k+3}} k_1k_2k_3 {c}_{m_1g_1k_1}c_{m_2g_2k_2}c_{m_3g_3k_3} - \frac{k_b}{2}kc_{mgk}R_f^2}^{\text{branch formation}}\\
\scriptsize+ \overbrace{\alert{2k_{g}c_{010} (k+1)c_{mg(k+1)} - 2 k_{g}c_{010} kc_{mgk}}}^{\text{fibrinogen reaction}}
\scriptsize- \overbrace{\delta_{m,0}\delta_{g,1}\delta_{k,0}\left(2kgc_{010}R_f + k_s c_{010}\right) + \delta_{m,1}\delta_{g,0}\delta_{k,2}k_s c_{010}}^{\text{sources}}\\
\end{multline*}
\begin{equation*}
\scriptsize \frac{d {c}_{010} }{dt} = -2k_g c_{010} R_f}, \qquad \qquad \frac{dR_f}{dt} = -k_l R_f^2-\frac{k_b}{3} R_f^3 - 2k_g c_{010} R_f
\end{equation*}
\end{frame}

\begin{frame}{Results compared with experiments}
\begin{center}
\includegraphics[width = 0.5\textwidth]{histogram.png}
\includegraphics[width = 0.5\textwidth]{histo_fbfg_k3g07s2.pdf}
\end{center}
\end{frame}

\begin{frame}{Life in Utah!}
\vspace{-3cm}
\includegraphics[width = \textwidth]{hike.pdf}
\end{frame}


\end{document}
