\documentclass{article}

\newcommand{\ep}{\rule{.06in}{.1in}}
\textheight 9.5in

\usepackage{amssymb, bm}
\usepackage{amsmath}
\usepackage{amsthm}
\usepackage{graphicx, subcaption, booktabs}

\usepackage{tikz, pgfplots, pgfplotstable, chemfig, xcolor}

% \usepgfplotslibrary{colorbrewer, statistics}
% \pgfplotsset{
%   exact axis/.style={grid=major, minor tick num=4, xlabel=$v^*$,
%     legend entries={PDF, CDF},},
%   every axis plot post/.append style={thick},
%   table/search
%   path={/Users/andrewwork/thesis/jump-velocity/dat-files},
%   colormap/YlGnBu,
%   cycle list/Set1-5,
%   legend style={legend cell align=left,},
% }

% \usepgfplotslibrary{external}
% \tikzexternalize

\renewcommand{\arraystretch}{1.2}
\pagestyle{empty} 
\oddsidemargin -0.25in
\evensidemargin -0.25in 
\topmargin -0.75in 
\parindent 0pt
\parskip 12pt
\textwidth 7in
%\font\cj=msbm10 at 12pt

\newcommand{\tn}{\textnormal}
\newcommand{\stiff}{\frac{k_f}{\gamma}}
\newcommand{\dd}{d}
\newcommand{\Der}[2]{\frac{\dd #1}{\dd #2}}
\newcommand{\Pder}[2]{\frac{\partial #1}{\partial #2}}
\newcommand{\Integral}[4]{\int_{#3}^{#4} {#1} \dd #2}
\newcommand{\vect}[1]{\boldsymbol{\mathbf{#1}}}
\newcommand{\mat}[1]{\underline{\underline{#1}}}
\DeclareMathOperator{\Exp}{Exp}

% Text width is 7 inches

\def\R{\mathbb{R}}
\def\N{\mathbb{N}}
\def\C{\mathbb{C}}
\def\Z{\mathbb{Z}}
\def\Q{\mathbb{Q}}
\def\H{\mathbb{H}}
\def\B{\mathcal{B}} 
%\topmargin -.5in 

\setcounter{secnumdepth}{2}
\begin{document}
\pagestyle{plain}

\begin{center}
  {\Large Meeting Notes (\today)}
\end{center}

\begin{enumerate}
\item I talked to Suraj last Thursday to see how he is solving his RS
  systems. For systems of $\sim 1000$ Stokeslets, he is using MATLAB
  backslash, so I ran some timing experiments comparing the solve times
  of MATLAB backslash to my Python code. They take basically the same
  time (Python actually does slightly better, 50 seconds on the
  compute servers vs 54 seconds, $6 (24)^2 + 2 = 3458$ stokeslets).
\item I read the paper \cite{Barrero-Gil2013} on the method of
  auxiliary Stokeslets.
\item Started working on adding receptors and binding dynamics to the
  model.
\end{enumerate}

\textbf{Auxiliary Stokeslets}

\begin{figure}
  \centering
  \includegraphics[width=.5\textwidth]{aux-stks.jpg}
  \caption{Sketch of primary and auxiliary Stokeslets from
    Barrero-Gil \cite{Barrero-Gil2013}}
  \label{fig:aux-stks}
\end{figure}

\begin{itemize}
\item The main idea behind this method is to generate a set of
  ``primary'' Stokeslets on the surface (these are always present, so
  in the original R.S. method, all Stokeslets are primary Stokeslets),
  and then generate a grid of temporary/auxiliary Stokeslets around
  your current evaluation point (see Figure \ref{fig:aux-stks}). This
  allows us to choose an $\epsilon$ on the order of the auxiliary
  Stokeslet separation distance, instead of the primary Stokeslet
  separation. 
\item In the usual method of R.S., the fluid velocity $\vect{u}$ at a
  point $\vect{x}$ is computed by
  \begin{equation}
    u_i(\vect{x}) = \frac{1}{8\pi\mu} \sum_{n=1}^{N_s} \sum_{j=1}^3
    S_{ij}^\epsilon (\vect{x}, \vect{x}_n) f_{n, j}
    w_n. \label{eq:org-mrs} 
  \end{equation}
\item Then, enforcing a specified velocity at each Stokeslet center
  gives us a $3N_s \times 3N_s$ system that we can solve for the
  Stokeslet strengths $f_{n, j}$.
\item In the method of auxiliary Stokeslets proposed by
  \cite{Barrero-Gil2013}, the fluid velocity at Stokeslet center $m$
  is found by
  \begin{equation}
    \label{eq:eq:mod-mrs}
    u_i(\vect{x}_m) = \frac{1}{8\pi\mu} \sum_{n \neq m} \sum_{j=1}^3
    S_{ij}^\epsilon (\vect{x}_m, \vect{x}_n) f_{n, j} w_n +
    \frac{1}{8\pi\mu} \sum_{j=1}^3 \langle S_{ij}^\epsilon
    (\vect{x}_m, \vect{x}_a) \rangle f_{m, j} w_m,
  \end{equation}
  where $\langle S_{ij}^\epsilon (\vect{x_m}, \vect{x_a}) \rangle$ is
  the averaged regularized Stokeslet value over all auxiliary
  Stokeslets $\vect{x}_a$.
\item By adding the auxiliary Stokeslets, the regularization parameter
  can be taken on the order of $h$, the separation distance of the
  auxiliary Stokeslets. They don't do any rigorous error analysis of
  this, but they test it on a variety of numerical experiments. In the
  original method of R.S., $\epsilon$ needs to be on the order of the
  separation distance of the primary Stokeslets.
\item Their numerical experiments are convincing: they show better
  accuracy for less computational work solving for Stokes' drag
  coefficient, and they demonstrate the method works for ellipsoidal
  and toroidal bodies.
\item However, all of these tests are performed in an unbounded
  domain, so we'd have to test it ourselves near a wall.
\item They also don't say much about how to distribute the auxiliary
  Stokeslets, apart from mentioning that you should distribute them
  across a diameter on the order of $H$---the separation distance of
  the primary Stokeslets. In Figure \ref{fig:aux-stks} they show the
  $\vect{x}_a$s distributed in a rectangular mesh, although my
  instinct is to distribute them radially, since the stokeslet is
  radially symmetric (however my mesh of primary Stokeslets is
  rectangular).
\item Finally, they don't do any analysis on choosing $N_s$ and $N_a$,
  they just mention that it is computationally easier to increase $N_a$.
\end{itemize}

\textbf{Binding}

\begin{itemize}
\item There are two common models for length-dependent binding and
  unbinding used in the rolling literature: the Bell model and the
  Dembo model. However, one review \cite{Pospieszalska2009} notes that
  in the parameter ranges typically encountered in cell rolling, the
  models are interchangeable.
\item In the Bell model, the formation $k_f$ and dissociation $k_d$
  rates are (let $r$ be the distance between the receptor and ligand): 
  \begin{align*}
    k_f(r) &= k_f^0 \exp\left(\frac{\sigma |r - \lambda | (\delta - 0.5
    |r - \lambda)}{k_B T}\right) \\
    k_d(r) &= k_d^0 \exp\left(\frac{\delta \sigma |r - \lambda|}{k_b
             T}\right).
  \end{align*}
\item In the Dembo model, the rates are:
  \begin{align*}
    k_f(r) &= k_f^0 \exp\left(\frac{-\sigma_\tn{ts}(r -
             \lambda)^2}{2k_B T}\right) \\
    k_d(r) &= k_d^0 \exp\left(\frac{(\sigma - \sigma_\tn{ts})(r -
             \lambda)^2}{2k_b T}\right). 
  \end{align*}
\item In both of these models, the dissociation constant is the
  following function of the length:
  \begin{equation*}
    \frac{k_d}{k_f} = \left(\frac{k_d^0}{k_f^0}\right) \exp \left(
      \frac{-\sigma(r - \lambda)^2}{2 k_B T} \right).
  \end{equation*}
\item In order to model binding between receptors on the platelet and
  ligands on the wall, we have to first place receptors on the
  platelet surface.
\item I initially tried placing them as I described on Wednesday
  (generate a uniform rectangular mesh on a parallelpiped with a 3:3:1
  aspect ratio and project to the ellipsoid). However, Andy sent me an example
  of a set of 1000 nodes generated by Poisson disc sampling, which
  looks more uniform to me.
\item For ligands on the wall, in the 2D model we assumed there was a
  continuum of ligands, which allowed us to generate
  $\mathcal{O}(N_R)$ random numbers for the bond formation part of
  each time step ($N_R$---number of receptors) instead of
  $\mathcal{O}(N_R N_L)$ random numbers. However, I think this depends
  on which model of bond formation we use. Specifically, if we use the
  Dembo model with $\lambda = 0$, then $k_f$ is a Gaussian function
  and the probablility of a bond forming from a recptor to \emph{any}
  point on the wall is a an integral of $k_f$ (which is easy to find
  analytically), and the wall-attachment point of the bond is
  distributed normally.
\item This same logic applies to the 3D model (using the Dembo model
  with $\lambda = 0$), but if $\lambda > 0$ then I don't think there
  is an analytic expression for the integral of $k_f$ over the wall,
  and generating a random variable with the right probability
  distribution is harder.
\item If we do end up distributing discrete ligands on the wall, there
  may be clever ways to distribute random numbers so that we don't
  have to generate $N_R N_L$ new ones at each time step.
\end{itemize}

% \begin{figure}
%   \centering
%   \includegraphics[width=\textwidth]{f_test4_16}
%   \caption{Motion integration of a sphere (radius $=1$) in a shear
%     flow, with no applied force and an applied torque of
%     $\vec{e}_z$. Mesh parameter $N = 16$.}
%   \label{fig:f416_plot}
% \end{figure}

% \begin{figure}
%   \centering
%   \includegraphics[width=\textwidth]{f_test0_8w}
%   \caption{Motion integration of a sphere (radius $=1$) in a shear
%     flow and at a distance of $1.54$ from the wall, with an applied
%     force of $\vec{e}_y$ and no applied torque. Mesh parameter $N =
%     8$.}
%   \label{fig:f08w_plot}
% \end{figure}

% \begin{figure}
%   \centering
%   \includegraphics[width=.5\textwidth]{timings}
%   \caption{Timings for generating a resistance matrix with a mesh with
%     mesh parameter $N$. There are $6N^2 + 2$ nodes/Stokeslets on the surface of
%     the platelet. Therefore the Stokeslets matrix contains
%     $\mathcal{O}(N^4)$ elements, and factoring the matrix requires
%     $\mathcal{O}(N^6)$ operations.}
%   \label{fig:timings_plot}
% \end{figure}

% \begin{figure}[b]
%   \centering
%   \begin{subfigure}{0.49\textwidth}
%     \includegraphics[width=\textwidth]{orient_plot71_2nd}
%   \end{subfigure}
%   \hfill
%   \begin{subfigure}{0.49\textwidth}
%     \includegraphics[width=\textwidth]{orient_err_plot71_2nd}
%   \end{subfigure}
%   \\
%   \begin{subfigure}{0.49\textwidth}
%     \includegraphics[width=\textwidth]{com_plot71_2nd}
%   \end{subfigure}
%   \hfill
%   \begin{subfigure}{0.49\textwidth}
%     \includegraphics[width=\textwidth]{nsep_plot71}
%   \end{subfigure}  
%   \caption{Plots of the ellipsoid orientation, orientation error,
%     center of mass, and the $N$ value with the plt-wall
%     separation. The height of the center of mass is initialized at
%     $1.5$. Orientation is initialized at $\vect{e}_m =
%     \vect{e}_x$.}
%   \label{fig:plt71}
% \end{figure}

% \begin{figure}[b]
%   \centering
%   \begin{subfigure}{0.49\textwidth}
%     \includegraphics[width=\textwidth]{orient_plot72_2nd}
%   \end{subfigure}
%   \hfill
%   \begin{subfigure}{0.49\textwidth}
%     \includegraphics[width=\textwidth]{orient_err_plot72_2nd}
%   \end{subfigure}
%   \\
%   \begin{subfigure}{0.49\textwidth}
%     \includegraphics[width=\textwidth]{com_plot72_2nd}
%   \end{subfigure}
%   \hfill
%   \begin{subfigure}{0.49\textwidth}
%     \includegraphics[width=\textwidth]{nsep_plot72}
%   \end{subfigure}  
%   \caption{Plots of the ellipsoid orientation, orientation error,
%     center of mass, and the $N$ value with the plt-wall
%     separation. The height of the center of mass is initialized at
%     $1.2$. Orientation is initialized at $\vect{e}_m =
%     \vect{e}_x$.}
%   \label{fig:plt72}
% \end{figure}

% \begin{figure}
%   \centering
%   \begin{subfigure}{0.49\textwidth}
%     \includegraphics[width=\textwidth]{orient_plot74_2nd}
%   \end{subfigure}
%   \hfill
%   \begin{subfigure}{0.49\textwidth}
%     \includegraphics[width=\textwidth]{orient_err_plot74_2nd}
%   \end{subfigure}
%   \\
%   \begin{subfigure}{0.49\textwidth}
%     \includegraphics[width=\textwidth]{com_plot74_2nd}
%   \end{subfigure}
%   \hfill
%   \begin{subfigure}{0.49\textwidth}
%     \includegraphics[width=\textwidth]{nsep_plot74}
%   \end{subfigure}  
%   \caption{Plots of the ellipsoid orientation, orientation error,
%     center of mass, and the $N$ value with the plt-wall
%     separation. The height of the center of mass is initialized at
%     $0.8$. Orientation is initialized at $\vect{e}_m =
%     \vect{e}_x$.}
%   \label{fig:plt74}
% \end{figure}

% \begin{figure}
%   \centering
%   \includegraphics[width=\textwidth]{f_test0_8}
%   \caption{Motion integration of an ellipsoid in a shear flow, with an
%     applied force of $\vec{e}_y$ and no applied torque. Mesh parameter
%   $N = 8$.}
%   \label{fig:f08_plot}
% \end{figure}

% \begin{figure}
%   \centering
%   \includegraphics[width=\textwidth]{f_test0_16}
%   \caption{Motion integration of an ellipsoid in a shear flow, with an
%     applied force of $\vec{e}_y$ and no applied torque. Mesh parameter
%   $N = 16$.}
%   \label{fig:f016_plot}
% \end{figure}

% \begin{figure}
%   \centering
%   \includegraphics[width=\textwidth]{f_test4_16}
%   \caption{Motion integration of a sphere (radius $=1$) in a shear
%     flow, with no applied force and an applied torque of $\vec{e}_z$.}
%   \label{fig:f4_plot}
% \end{figure}

% \begin{figure}
%   \centering
%   \includegraphics[width=\textwidth]{f_test6_16}
%   \caption{Motion integration of an ellipsoid in a shear flow, with an
%     applied force of $\vec{e}_z$ and no applied torque.}
%   \label{fig:f6_plot}
% \end{figure}

\bibliographystyle{plain}
\bibliography{/Users/andrewwork/Documents/grad-school/thesis/library}

\end{document}




