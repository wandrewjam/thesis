\documentclass{article}

\newcommand{\ep}{\rule{.06in}{.1in}}
\textheight 9.5in

\usepackage{amssymb, bm}
\usepackage{amsmath}
\usepackage{amsthm}
\usepackage{graphicx, subcaption, booktabs}

\usepackage{tikz, pgfplots, pgfplotstable, chemfig, xcolor}

% \usepgfplotslibrary{colorbrewer, statistics}
% \pgfplotsset{
%   exact axis/.style={grid=major, minor tick num=4, xlabel=$v^*$,
%     legend entries={PDF, CDF},},
%   every axis plot post/.append style={thick},
%   table/search
%   path={/Users/andrewwork/thesis/jump-velocity/dat-files},
%   colormap/YlGnBu,
%   cycle list/Set1-5,
%   legend style={legend cell align=left,},
% }

% \usepgfplotslibrary{external}
% \tikzexternalize

\renewcommand{\arraystretch}{1.2}
\pagestyle{empty} 
\oddsidemargin -0.25in
\evensidemargin -0.25in 
\topmargin -0.75in 
\parindent 0pt
\parskip 12pt
\textwidth 7in
%\font\cj=msbm10 at 12pt

\newcommand{\tn}{\textnormal}
\newcommand{\stiff}{\frac{k_f}{\gamma}}
\newcommand{\dd}{d}
\newcommand{\Der}[2]{\frac{\dd #1}{\dd #2}}
\newcommand{\Pder}[2]{\frac{\partial #1}{\partial #2}}
\newcommand{\Integral}[4]{\int_{#3}^{#4} {#1} \dd #2}
\newcommand{\vect}[1]{\boldsymbol{\mathbf{#1}}}
\newcommand{\mat}[1]{\underline{\underline{#1}}}
\DeclareMathOperator{\Exp}{Exp}

% Text width is 7 inches

\def\R{\mathbb{R}}
\def\N{\mathbb{N}}
\def\C{\mathbb{C}}
\def\Z{\mathbb{Z}}
\def\Q{\mathbb{Q}}
\def\H{\mathbb{H}}
\def\B{\mathcal{B}} 
%\topmargin -.5in 

\setcounter{secnumdepth}{2}
\begin{document}
\pagestyle{plain}

\begin{center}
  {\Large Meeting Notes (\today)}
\end{center}

\begin{itemize}
\item Ran longer experiments on CHPC servers of a couple of the
  platelet rolling experiments (Figures \ref{fig:plt72} and \ref{fig:plt74})
\item I miscalculated some run times, and so don't have results with
  an initial height of $1.0$ micron. However, I checked on a shorter
  experiment and verified that the adaptive scheme is working
  correctly now (this was the experiment that was breaking before.
\item These longer experiments confirm the trends that we noticed in
  the shorter experiments (and match the results of the Mody paper)
\item I also ran tests with a constant applied force (Figures
  \ref{fig:f0_plot}---\ref{fig:f9_plot}).
\end{itemize}

\begin{figure}[b]
  \centering
  \begin{subfigure}{0.49\textwidth}
    \includegraphics[width=\textwidth]{orient_plot72_2nd}
  \end{subfigure}
  \hfill
  \begin{subfigure}{0.49\textwidth}
    \includegraphics[width=\textwidth]{orient_err_plot72_2nd}
  \end{subfigure}
  \\
  \begin{subfigure}{0.49\textwidth}
    \includegraphics[width=\textwidth]{com_plot72_2nd}
  \end{subfigure}
  \hfill
  \begin{subfigure}{0.49\textwidth}
    \includegraphics[width=\textwidth]{nsep_plot72}
  \end{subfigure}  
  \caption{Plots of the ellipsoid orientation, orientation error,
    center of mass, and the $N$ value with the plt-wall
    separation. The height of the center of mass is initialized at
    $1.2$. Orientation is initialized at $\vect{e}_m =
    \vect{e}_x$.}
  \label{fig:plt72}
\end{figure}

\begin{figure}
  \centering
  \begin{subfigure}{0.49\textwidth}
    \includegraphics[width=\textwidth]{orient_plot74_2nd}
  \end{subfigure}
  \hfill
  \begin{subfigure}{0.49\textwidth}
    \includegraphics[width=\textwidth]{orient_err_plot74_2nd}
  \end{subfigure}
  \\
  \begin{subfigure}{0.49\textwidth}
    \includegraphics[width=\textwidth]{com_plot74_2nd}
  \end{subfigure}
  \hfill
  \begin{subfigure}{0.49\textwidth}
    \includegraphics[width=\textwidth]{nsep_plot74}
  \end{subfigure}  
  \caption{Plots of the ellipsoid orientation, orientation error,
    center of mass, and the $N$ value with the plt-wall
    separation. The height of the center of mass is initialized at
    $0.8$. Orientation is initialized at $\vect{e}_m =
    \vect{e}_x$.}
  \label{fig:plt74}
\end{figure}

\begin{figure}
  \centering
  \includegraphics[width=\textwidth]{f_test0}
  \caption{Motion integration of a sphere (radius $=1$) in a shear
    flow, with applied force vector $\vec{e}_y$ and no applied torque.}
  \label{fig:f0_plot}
\end{figure}

\begin{figure}
  \centering
  \includegraphics[width=\textwidth]{f_test3}
  \caption{Motion integration of a sphere (radius $=1$) in a shear
    flow, with no applied force and an applied torque of $\vec{e}_y$.}
  \label{fig:f3_plot}
\end{figure}

\begin{figure}
  \centering
  \includegraphics[width=\textwidth]{f_test4}
  \caption{Motion integration of a sphere (radius $=1$) in a shear
    flow, with no applied force and an applied torque of $\vec{e}_z$.}
  \label{fig:f4_plot}
\end{figure}

\begin{figure}
  \centering
  \includegraphics[width=\textwidth]{f_test6}
  \caption{Motion integration of an ellipsoid in a shear flow, with an
    applied force of $\vec{e}_z$ and no applied torque.}
  \label{fig:f6_plot}
\end{figure}

\begin{figure}
  \centering
  \includegraphics[width=\textwidth]{f_test9}
  \caption{Motion integration of an ellipsoid in a shear flow, with no
    applied force and an applied torque of $\vec{e}_z$.}
  \label{fig:f9_plot}
\end{figure}

% \begin{figure}[h!]
%   \centering
%   \begin{subfigure}{0.49\textwidth}
%     \includegraphics[width=\textwidth]{orient_plot12_2nd}
%   \end{subfigure}
%   \hfill
%   \begin{subfigure}{0.49\textwidth}
%     \includegraphics[width=\textwidth]{orient_err_plot12_2nd}
%   \end{subfigure}
%   \\
%   \begin{subfigure}{0.49\textwidth}
%     \includegraphics[width=\textwidth]{com_plot12_2nd}
%   \end{subfigure}
%   \caption{Plots of the ellipsoid orientation, orientation error, and
%      center of mass in a shear flow near a wall. The height of the
%      center of mass is initialized at $1.5$. Orientation is
%      initialized at $\vect{e}_m = \vect{e}_x$. The coarse solution
%      used the same number of nodes as the minimum node size in the
%      adaptive code.}
%   \label{fig:plt12}
% \end{figure}

\bibliographystyle{plain}
\bibliography{/Users/andrewwork/Documents/grad-school/thesis/library}

\end{document}




