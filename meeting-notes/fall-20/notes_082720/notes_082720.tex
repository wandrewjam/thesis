\documentclass{article}

\newcommand{\ep}{\rule{.06in}{.1in}}
\textheight 9.5in

\usepackage{amssymb, bm}
\usepackage{amsmath}
\usepackage{amsthm}
\usepackage{graphicx, subcaption, booktabs}

\usepackage{tikz, pgfplots, pgfplotstable, chemfig, xcolor}

% \usepgfplotslibrary{colorbrewer, statistics}
% \pgfplotsset{
%   exact axis/.style={grid=major, minor tick num=4, xlabel=$v^*$,
%     legend entries={PDF, CDF},},
%   every axis plot post/.append style={thick},
%   table/search
%   path={/Users/andrewwork/thesis/jump-velocity/dat-files},
%   colormap/YlGnBu,
%   cycle list/Set1-5,
%   legend style={legend cell align=left,},
% }

% \usepgfplotslibrary{external}
% \tikzexternalize

\renewcommand{\arraystretch}{1.2}
\pagestyle{empty} 
\oddsidemargin -0.25in
\evensidemargin -0.25in 
\topmargin -0.75in 
\parindent 0pt
\parskip 12pt
\textwidth 7in
%\font\cj=msbm10 at 12pt

\newcommand{\tn}{\textnormal}
\newcommand{\stiff}{\frac{k_f}{\gamma}}
\newcommand{\dd}{d}
\newcommand{\Der}[2]{\frac{\dd #1}{\dd #2}}
\newcommand{\Pder}[2]{\frac{\partial #1}{\partial #2}}
\newcommand{\Integral}[4]{\int_{#3}^{#4} {#1} \dd #2}
\newcommand{\vect}[1]{\boldsymbol{\mathbf{#1}}}
\newcommand{\mat}[1]{\underline{\underline{#1}}}
\DeclareMathOperator{\Exp}{Exp}

% Text width is 7 inches

\def\R{\mathbb{R}}
\def\N{\mathbb{N}}
\def\C{\mathbb{C}}
\def\Z{\mathbb{Z}}
\def\Q{\mathbb{Q}}
\def\H{\mathbb{H}}
\def\B{\mathcal{B}} 
%\topmargin -.5in 

\setcounter{secnumdepth}{2}
\begin{document}
\pagestyle{plain}

\begin{center}
  {\Large Meeting Notes (\today)}
\end{center}

\begin{itemize}
\item No new experiments---math compute servers have been down. I've
  been setting up my environment on the CHPC servers, learning
  about/testing out the job submission process. I've run small tests,
  but not any full experiments yet.
\item Almost everything is in place now to run some longer experiments
\item From last time, there was an issue in one of the adaptive
  experiments (see Figure \ref{fig:plt73}) at time $t=21$.
\item The source of the error is in the time-adaptive part of the
  solver. The time-adaptive part senses if the platelet will intersect
  the wall at the end of the step.
  \begin{itemize}
  \item Problem: at the mesh size I was using, none of the nodes
    crossed the wall, but after refining the mesh further, nodes did
    appear on the other side of the wall
  \item Solution: use a finer mesh to sense platelet-wall
    interactions. There isn't a significant penalty for using a finer
    mesh in this step.
  \end{itemize}
\item To give a sense of the relationship between the platelet-wall
  separation, and the mesh size chosen, I plotted the $N$ that would
  be chosen by the adaptive solver for a range of separations (Figure
  \ref{fig:sep_N_plot}).
\item The range of separations over which the solver ``adapts'' is
  pretty small, sort of what we'd expect given the bottom-right plot
  in Figure \ref{fig:plt73}. The adaptivity mostly acts as a switch
  between a coarse mesh and a fine mesh.
\item Finally, I found a paper by Mody and King where they perform the
  same experiments that we are (with a platelet initially flat). I was
  skimming through it this morning, and their results are very similar
  to ours. They see the same three behaviors that we've seen, and the
  starting center of mass height is what differentiates between the
  three behaviors.
\end{itemize}

\begin{figure}
  \centering
  \includegraphics[width=.6\textwidth]{sep_N_plot}
  \caption{Relationship between platelet-wall separation distance and
    the $N$ chosen by the adaptive solver}
  \label{fig:sep_N_plot}
\end{figure}

\begin{figure}
  \centering
  \begin{subfigure}{0.49\textwidth}
    \includegraphics[width=\textwidth]{orient_plot73_2nd}
  \end{subfigure}
  \hfill
  \begin{subfigure}{0.49\textwidth}
    \includegraphics[width=\textwidth]{orient_err_plot73_2nd}
  \end{subfigure}
  \\
  \begin{subfigure}{0.49\textwidth}
    \includegraphics[width=\textwidth]{com_plot73_2nd}
  \end{subfigure}
  \hfill
  \begin{subfigure}{0.49\textwidth}
    \includegraphics[width=\textwidth]{nsep_plot73}
  \end{subfigure}  
  \caption{Plots of the ellipsoid orientation, orientation error,
    center of mass, and the $N$ value with the plt-wall
    separation. The height of the center of mass is initialized at
    $1.0$. Orientation is initialized at $\vect{e}_m =
    \vect{e}_x$. Similar to the previous figure, the adaptive solver
    used the minimum mesh size for most of the simulation, apart from
    a few short intervals. However something strange is going on in
    the adaptive solver in this experiment: there's a weird hitch in
    the graph around $t = 20$, and the ellipsoid stays in the vertical
    orientation for much longer than in any other run.}
  \label{fig:plt73}
\end{figure}

% \begin{figure}[h!]
%   \centering
%   \begin{subfigure}{0.49\textwidth}
%     \includegraphics[width=\textwidth]{orient_plot12_2nd}
%   \end{subfigure}
%   \hfill
%   \begin{subfigure}{0.49\textwidth}
%     \includegraphics[width=\textwidth]{orient_err_plot12_2nd}
%   \end{subfigure}
%   \\
%   \begin{subfigure}{0.49\textwidth}
%     \includegraphics[width=\textwidth]{com_plot12_2nd}
%   \end{subfigure}
%   \caption{Plots of the ellipsoid orientation, orientation error, and
%      center of mass in a shear flow near a wall. The height of the
%      center of mass is initialized at $1.5$. Orientation is
%      initialized at $\vect{e}_m = \vect{e}_x$. The coarse solution
%      used the same number of nodes as the minimum node size in the
%      adaptive code.}
%   \label{fig:plt12}
% \end{figure}

\bibliographystyle{plain}
\bibliography{/Users/andrewwork/Documents/grad-school/thesis/library}

\end{document}




