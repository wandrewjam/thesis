\documentclass{article}

\newcommand{\ep}{\rule{.06in}{.1in}}
\textheight 9.5in

\usepackage{amssymb, bm}
\usepackage{amsmath}
\usepackage{amsthm}
\usepackage{graphicx, subcaption, booktabs}

\usepackage{tikz, pgfplots, pgfplotstable, chemfig, xcolor}

% \usepgfplotslibrary{colorbrewer, statistics}
% \pgfplotsset{
%   exact axis/.style={grid=major, minor tick num=4, xlabel=$v^*$,
%     legend entries={PDF, CDF},},
%   every axis plot post/.append style={thick},
%   table/search
%   path={/Users/andrewwork/thesis/jump-velocity/dat-files},
%   colormap/YlGnBu,
%   cycle list/Set1-5,
%   legend style={legend cell align=left,},
% }

% \usepgfplotslibrary{external}
% \tikzexternalize

\renewcommand{\arraystretch}{1.2}
\pagestyle{empty} 
\oddsidemargin -0.25in
\evensidemargin -0.25in 
\topmargin -0.75in 
\parindent 0pt
\parskip 12pt
\textwidth 7in
%\font\cj=msbm10 at 12pt

\newcommand{\tn}{\textnormal}
\newcommand{\stiff}{\frac{k_f}{\gamma}}
\newcommand{\dd}{d}
\newcommand{\Der}[2]{\frac{\dd #1}{\dd #2}}
\newcommand{\Pder}[2]{\frac{\partial #1}{\partial #2}}
\newcommand{\Integral}[4]{\int_{#3}^{#4} {#1} \dd #2}
\newcommand{\vect}[1]{\boldsymbol{\mathbf{#1}}}
\newcommand{\mat}[1]{\underline{\underline{#1}}}
\DeclareMathOperator{\Exp}{Exp}

% Text width is 7 inches

\def\R{\mathbb{R}}
\def\N{\mathbb{N}}
\def\C{\mathbb{C}}
\def\Z{\mathbb{Z}}
\def\Q{\mathbb{Q}}
\def\H{\mathbb{H}}
\def\B{\mathcal{B}} 
%\topmargin -.5in 

\setcounter{secnumdepth}{2}
\begin{document}
\pagestyle{plain}

\begin{center}
  {\Large Meeting Notes (\today)}
\end{center}

\begin{itemize}
\item Short range repulsive force (Figure \ref{fig:sep}):
  $\displaystyle{F_\tn{rep} = F_0 \frac{\tau e^{-\tau \delta}}{1 -
      e^{-\tau \delta}}}$ with $F_0 = 500 \tn{pN} \cdot \tn{m}$ and
  $\tau = 2000 \mu\tn{m}^{-1}$
  \begin{itemize}
  \item I fixed the way I was applying this force (before, I was
    using the distance from the wall to the center of mass as the
    separation distance $\delta$)
  \item Also, I was previously applying this (only) as a body force on
    the platelet. I think a slightly more accurate assumption is to
    apply the force \emph{at} the point on the platelet closest to the
    wall, so that there is a contribution to the body torque as well.
  \item Because this force grows rapidly as the separation distance
    gets small, this introduced a problem where once the platelet got
    close to the wall, the large repulsive force pushed the platelet
    100s of nm away from the wall in a single time step. I didn't see
    any binding in any simulations.
  \item In successive time steps as the platelet approached the wall,
    the separation distance went from $\sim 30$nm with a repulsive
    force of $\sim 10^{-19}$, to $\sim 20$nm with a repulsive force of
    $\sim 10^{-6}$, to $\sim 11$nm with a repulsive force of
    $\sim 500$. With this force, the separation distance jumped back
    to $\sim 50$nm.
  \item Fix---implement a heuristic restriction on the time step. I
    set it up so that if the repulsive force grows or shrinks by more
    than a factor of 10 within a time step, I take two half
    time-steps (if the repulsive force is $< 0.1$, then I don't apply
    this restriction)
  \end{itemize}
\item Also, I modified the bond force function to eliminate
  compression forces. This reduced the oscillations we had been seeing
  previously in some tests, though not all of them
\item I implemented a similar time step restriction where I took half
  time steps if the bond forces differed too much at the beginning and
  end of a time step
\end{itemize}

\begin{figure}[h]
  \centering
  \includegraphics[width=.8\textwidth]{rep_force}
  \caption{Repulsive force as a function of the minimum separation
    distance}
  \label{fig:sep}
\end{figure}

\begin{figure}
  \centering
  \includegraphics[width=.8\textwidth]{bd_expt000}
  \caption{1st random seed, two-sided bond force, no step restriction
    for the bond force. Top subplot shows the $x$ and $z$ components
    of the center of mass. 2nd subplot shows the $z$ component of the
    orientation vector. 3rd subplot shows the bond lengths. 4th
    subplot shows the difference between $z$ components of the bound
    receptor and the center of mass. Bottom subplot shows the $z$
    component of each bond.}
  \label{fig:fig1}
\end{figure}

\begin{figure}
  \centering
  \includegraphics[width=.8\textwidth]{bd_expt001}
  \caption{1st random seed, one-sided bond force, no step restriction
    for the bond force. Top subplot shows the $x$ and $z$ components
    of the center of mass. 2nd subplot shows the $z$ component of the
    orientation vector. 3rd subplot shows the bond lengths. 4th
    subplot shows the difference between $z$ components of the bound
    receptor and the center of mass. Bottom subplot shows the $z$
    component of each bond.}
  \label{fig:fig2}
\end{figure}

\begin{figure}
  \centering
  \includegraphics[width=.8\textwidth]{bd_expt002}
  \caption{1st random seed, one-sided bond force, plus step
    restriction for the bond force. Top subplot shows the $x$ and $z$
    components of the center of mass. 2nd subplot shows the $z$
    component of the orientation vector. 3rd subplot shows the bond
    lengths. 4th subplot shows the difference between $z$ components
    of the bound receptor and the center of mass. Bottom subplot shows
    the $z$ component of each bond.}
  \label{fig:fig3}
\end{figure}

\begin{figure}
  \centering
  \includegraphics[width=.8\textwidth]{bd_expt003}
  \caption{2nd random seed, two-sided bond force, no step restriction
    for the bond force. Top subplot shows the $x$ and $z$ components
    of the center of mass. 2nd subplot shows the $z$ component of the
    orientation vector. 3rd subplot shows the bond lengths. 4th
    subplot shows the difference between $z$ components of the bound
    receptor and the center of mass. Bottom subplot shows the $z$
    component of each bond.}
  \label{fig:fig4}
\end{figure}

\begin{figure}
  \centering
  \includegraphics[width=.8\textwidth]{bd_expt004}
  \caption{2nd random seed, one-sided bond force, no step restriction
    for the bond force. Top subplot shows the $x$ and $z$ components
    of the center of mass. 2nd subplot shows the $z$ component of the
    orientation vector. 3rd subplot shows the bond lengths. 4th
    subplot shows the difference between $z$ components of the bound
    receptor and the center of mass. Bottom subplot shows the $z$
    component of each bond.}
  \label{fig:fig5}
\end{figure}

\begin{figure}
  \centering
  \includegraphics[width=.8\textwidth]{bd_expt005}
  \caption{2nd random seed, one-sided bond force, plus step
    restriction for the bond force. Top subplot shows the $x$ and $z$
    components of the center of mass. 2nd subplot shows the $z$
    component of the orientation vector. 3rd subplot shows the bond
    lengths. 4th subplot shows the difference between $z$ components
    of the bound receptor and the center of mass. Bottom subplot shows
    the $z$ component of each bond.}
  \label{fig:fig6}
\end{figure}

\begin{figure}
  \centering
  \includegraphics[width=.8\textwidth]{bd_expt006}
  \caption{3rd random seed, two-sided bond force, no step restriction
    for the bond force. Top subplot shows the $x$ and $z$ components
    of the center of mass. 2nd subplot shows the $z$ component of the
    orientation vector. 3rd subplot shows the bond lengths. 4th
    subplot shows the difference between $z$ components of the bound
    receptor and the center of mass. Bottom subplot shows the $z$
    component of each bond.}
  \label{fig:fig7}
\end{figure}

\begin{figure}
  \centering
  \includegraphics[width=.8\textwidth]{bd_expt007}
  \caption{3rd random seed, one-sided bond force, no step restriction
    for the bond force. Top subplot shows the $x$ and $z$ components
    of the center of mass. 2nd subplot shows the $z$ component of the
    orientation vector. 3rd subplot shows the bond lengths. 4th
    subplot shows the difference between $z$ components of the bound
    receptor and the center of mass. Bottom subplot shows the $z$
    component of each bond.}
  \label{fig:fig8}
\end{figure}

\begin{figure}
  \centering
  \includegraphics[width=.8\textwidth]{bd_expt008}
  \caption{3rd random seed, one-sided bond force, plus step
    restriction for the bond force. Top subplot shows the $x$ and $z$
    components of the center of mass. 2nd subplot shows the $z$
    component of the orientation vector. 3rd subplot shows the bond
    lengths. 4th subplot shows the difference between $z$ components
    of the bound receptor and the center of mass. Bottom subplot shows
    the $z$ component of each bond.}
  \label{fig:fig9}
\end{figure}

\bibliographystyle{plain}
\bibliography{/Users/andrewwork/Documents/grad-school/thesis/library}

\end{document}




