\documentclass{article}

\newcommand{\ep}{\rule{.06in}{.1in}}
\textheight 9.5in

\usepackage{amssymb, bm}
\usepackage{amsmath}
\usepackage{amsthm}
\usepackage{graphicx, subcaption, booktabs}

\usepackage{tikz, pgfplots, pgfplotstable, chemfig, xcolor}

% \usepgfplotslibrary{colorbrewer, statistics}
% \pgfplotsset{
%   exact axis/.style={grid=major, minor tick num=4, xlabel=$v^*$,
%     legend entries={PDF, CDF},},
%   every axis plot post/.append style={thick},
%   table/search
%   path={/Users/andrewwork/thesis/jump-velocity/dat-files},
%   colormap/YlGnBu,
%   cycle list/Set1-5,
%   legend style={legend cell align=left,},
% }

% \usepgfplotslibrary{external}
% \tikzexternalize

\renewcommand{\arraystretch}{1.2}
\pagestyle{empty} 
\oddsidemargin -0.25in
\evensidemargin -0.25in 
\topmargin -0.75in 
\parindent 0pt
\parskip 12pt
\textwidth 7in
%\font\cj=msbm10 at 12pt

\newcommand{\tn}{\textnormal}
\newcommand{\stiff}{\frac{k_f}{\gamma}}
\newcommand{\dd}{d}
\newcommand{\Der}[2]{\frac{\dd #1}{\dd #2}}
\newcommand{\Pder}[2]{\frac{\partial #1}{\partial #2}}
\newcommand{\Integral}[4]{\int_{#3}^{#4} {#1} \dd #2}
\newcommand{\vect}[1]{\boldsymbol{\mathbf{#1}}}
\newcommand{\mat}[1]{\underline{\underline{#1}}}
\DeclareMathOperator{\Exp}{Exp}

% Text width is 7 inches

\def\R{\mathbb{R}}
\def\N{\mathbb{N}}
\def\C{\mathbb{C}}
\def\Z{\mathbb{Z}}
\def\Q{\mathbb{Q}}
\def\H{\mathbb{H}}
\def\B{\mathcal{B}} 
%\topmargin -.5in 

\setcounter{secnumdepth}{2}
\begin{document}
\pagestyle{plain}

\begin{center}
  {\Large Meeting Notes (\today)}
\end{center}

\begin{itemize}
\item I made videos of the three experiments shown below
\item Fixed sign errors in the constant force/torque tests
\item Also, I ran the force/torque tests at several different mesh
  sizes: the experiments show convergence in the center-of-mass
  position, but not in the orientation. 
\item Looking at components of the orientation vector may not be the
  ``correct'' thing to look at. I will re-do the orientation plots to
  show the orientation angles as a function of time.
\item Also working on applied force and torque tests near a wall
\item Some notes from Mody and King paper:
  \begin{itemize}
  \item They don't see the same migration away from the wall in the
    experiments with flipping (Figures \ref{fig:plt71} and
    \ref{fig:plt72}) that we do. I suspect this is due to the
    regularization of forces.
  \item With their method, they can run $\sim300,000$ time steps per
    24 hours. This is about 1,000$\times$ faster than my code. They
    are vague on the exact numerical parameters used to get this
    result, so it might not be too valid to compare my timing results
    to theirs, but this is still a huge difference.
  \item One reason for the difference: I am using a much finer mesh
    than they are (I have $16\times$ as many nodes in my ``coarse''
    mesh). I suspect one reason they can get away with a coarser mesh:
    there is no regularization in their method.
  \end{itemize}
\end{itemize}

\begin{figure}[b]
  \centering
  \begin{subfigure}{0.49\textwidth}
    \includegraphics[width=\textwidth]{orient_plot71_2nd}
  \end{subfigure}
  \hfill
  \begin{subfigure}{0.49\textwidth}
    \includegraphics[width=\textwidth]{orient_err_plot71_2nd}
  \end{subfigure}
  \\
  \begin{subfigure}{0.49\textwidth}
    \includegraphics[width=\textwidth]{com_plot71_2nd}
  \end{subfigure}
  \hfill
  \begin{subfigure}{0.49\textwidth}
    \includegraphics[width=\textwidth]{nsep_plot71}
  \end{subfigure}  
  \caption{Plots of the ellipsoid orientation, orientation error,
    center of mass, and the $N$ value with the plt-wall
    separation. The height of the center of mass is initialized at
    $1.5$. Orientation is initialized at $\vect{e}_m =
    \vect{e}_x$.}
  \label{fig:plt71}
\end{figure}

\begin{figure}[b]
  \centering
  \begin{subfigure}{0.49\textwidth}
    \includegraphics[width=\textwidth]{orient_plot72_2nd}
  \end{subfigure}
  \hfill
  \begin{subfigure}{0.49\textwidth}
    \includegraphics[width=\textwidth]{orient_err_plot72_2nd}
  \end{subfigure}
  \\
  \begin{subfigure}{0.49\textwidth}
    \includegraphics[width=\textwidth]{com_plot72_2nd}
  \end{subfigure}
  \hfill
  \begin{subfigure}{0.49\textwidth}
    \includegraphics[width=\textwidth]{nsep_plot72}
  \end{subfigure}  
  \caption{Plots of the ellipsoid orientation, orientation error,
    center of mass, and the $N$ value with the plt-wall
    separation. The height of the center of mass is initialized at
    $1.2$. Orientation is initialized at $\vect{e}_m =
    \vect{e}_x$.}
  \label{fig:plt72}
\end{figure}

\begin{figure}
  \centering
  \begin{subfigure}{0.49\textwidth}
    \includegraphics[width=\textwidth]{orient_plot74_2nd}
  \end{subfigure}
  \hfill
  \begin{subfigure}{0.49\textwidth}
    \includegraphics[width=\textwidth]{orient_err_plot74_2nd}
  \end{subfigure}
  \\
  \begin{subfigure}{0.49\textwidth}
    \includegraphics[width=\textwidth]{com_plot74_2nd}
  \end{subfigure}
  \hfill
  \begin{subfigure}{0.49\textwidth}
    \includegraphics[width=\textwidth]{nsep_plot74}
  \end{subfigure}  
  \caption{Plots of the ellipsoid orientation, orientation error,
    center of mass, and the $N$ value with the plt-wall
    separation. The height of the center of mass is initialized at
    $0.8$. Orientation is initialized at $\vect{e}_m =
    \vect{e}_x$.}
  \label{fig:plt74}
\end{figure}

\begin{figure}
  \centering
  \includegraphics[width=\textwidth]{f_test0_8}
  \caption{Motion integration of an ellipsoid in a shear flow, with an
    applied force of $\vec{e}_y$ and no applied torque. Mesh parameter
  $N = 8$.}
  \label{fig:f08_plot}
\end{figure}

\begin{figure}
  \centering
  \includegraphics[width=\textwidth]{f_test0_16}
  \caption{Motion integration of an ellipsoid in a shear flow, with an
    applied force of $\vec{e}_y$ and no applied torque. Mesh parameter
  $N = 16$.}
  \label{fig:f016_plot}
\end{figure}

\begin{figure}
  \centering
  \includegraphics[width=\textwidth]{f_test4_16}
  \caption{Motion integration of a sphere (radius $=1$) in a shear
    flow, with no applied force and an applied torque of $\vec{e}_z$.}
  \label{fig:f4_plot}
\end{figure}

\begin{figure}
  \centering
  \includegraphics[width=\textwidth]{f_test6_16}
  \caption{Motion integration of an ellipsoid in a shear flow, with an
    applied force of $\vec{e}_z$ and no applied torque.}
  \label{fig:f6_plot}
\end{figure}

% \begin{figure}[h!]
%   \centering
%   \begin{subfigure}{0.49\textwidth}
%     \includegraphics[width=\textwidth]{orient_plot12_2nd}
%   \end{subfigure}
%   \hfill
%   \begin{subfigure}{0.49\textwidth}
%     \includegraphics[width=\textwidth]{orient_err_plot12_2nd}
%   \end{subfigure}
%   \\
%   \begin{subfigure}{0.49\textwidth}
%     \includegraphics[width=\textwidth]{com_plot12_2nd}
%   \end{subfigure}
%   \caption{Plots of the ellipsoid orientation, orientation error, and
%      center of mass in a shear flow near a wall. The height of the
%      center of mass is initialized at $1.5$. Orientation is
%      initialized at $\vect{e}_m = \vect{e}_x$. The coarse solution
%      used the same number of nodes as the minimum node size in the
%      adaptive code.}
%   \label{fig:plt12}
% \end{figure}

\bibliographystyle{plain}
\bibliography{/Users/andrewwork/Documents/grad-school/thesis/library}

\end{document}




