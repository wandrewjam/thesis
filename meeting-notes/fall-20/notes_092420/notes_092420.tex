\documentclass{article}

\newcommand{\ep}{\rule{.06in}{.1in}}
\textheight 9.5in

\usepackage{amssymb, bm}
\usepackage{amsmath}
\usepackage{amsthm}
\usepackage{graphicx, subcaption, booktabs}

\usepackage{tikz, pgfplots, pgfplotstable, chemfig, xcolor}

% \usepgfplotslibrary{colorbrewer, statistics}
% \pgfplotsset{
%   exact axis/.style={grid=major, minor tick num=4, xlabel=$v^*$,
%     legend entries={PDF, CDF},},
%   every axis plot post/.append style={thick},
%   table/search
%   path={/Users/andrewwork/thesis/jump-velocity/dat-files},
%   colormap/YlGnBu,
%   cycle list/Set1-5,
%   legend style={legend cell align=left,},
% }

% \usepgfplotslibrary{external}
% \tikzexternalize

\renewcommand{\arraystretch}{1.2}
\pagestyle{empty} 
\oddsidemargin -0.25in
\evensidemargin -0.25in 
\topmargin -0.75in 
\parindent 0pt
\parskip 12pt
\textwidth 7in
%\font\cj=msbm10 at 12pt

\newcommand{\tn}{\textnormal}
\newcommand{\stiff}{\frac{k_f}{\gamma}}
\newcommand{\dd}{d}
\newcommand{\Der}[2]{\frac{\dd #1}{\dd #2}}
\newcommand{\Pder}[2]{\frac{\partial #1}{\partial #2}}
\newcommand{\Integral}[4]{\int_{#3}^{#4} {#1} \dd #2}
\newcommand{\vect}[1]{\boldsymbol{\mathbf{#1}}}
\newcommand{\mat}[1]{\underline{\underline{#1}}}
\DeclareMathOperator{\Exp}{Exp}

% Text width is 7 inches

\def\R{\mathbb{R}}
\def\N{\mathbb{N}}
\def\C{\mathbb{C}}
\def\Z{\mathbb{Z}}
\def\Q{\mathbb{Q}}
\def\H{\mathbb{H}}
\def\B{\mathcal{B}} 
%\topmargin -.5in 

\setcounter{secnumdepth}{2}
\begin{document}
\pagestyle{plain}

\begin{center}
  {\Large Meeting Notes (\today)}
\end{center}

\begin{itemize}
\item I corrected an error in my applied force/torque tests, and now
  all of them are showing convergence in $N$ (I also changed the
  orientation plot to show orientation angles). Examples are shown in
  Figures \ref{fig:f48_plot} and \ref{fig:f416_plot}.
\item I also ran the applied force and torque tests on a sphere near
  the wall, these seem to be working (Figure \ref{fig:f08w_plot}).
\item I profiled the resistance matrix code. At mesh sizes $N = 4$ and
  $N = 16$, assembling the Stokeslets matrix was more expensive than
  solving, and at mesh sizes $N = 24$ and $N = 32$ solving was more
  expensive (Figure \ref{fig:timings_plot})
\item I wrote an RK4 solver, but still need to fix the error handling
  close to the wall. It does work in an unbounded domain.
\end{itemize}

\begin{figure}
  \centering
  \includegraphics[width=\textwidth]{f_test4_8}
  \caption{Motion integration of a sphere (radius $=1$) in a shear
    flow, with no applied force and an applied torque of
    $\vec{e}_z$. Mesh parameter $N = 8$}
  \label{fig:f48_plot}
\end{figure}

\begin{figure}
  \centering
  \includegraphics[width=\textwidth]{f_test4_16}
  \caption{Motion integration of a sphere (radius $=1$) in a shear
    flow, with no applied force and an applied torque of
    $\vec{e}_z$. Mesh parameter $N = 16$.}
  \label{fig:f416_plot}
\end{figure}

\begin{figure}
  \centering
  \includegraphics[width=\textwidth]{f_test0_8w}
  \caption{Motion integration of a sphere (radius $=1$) in a shear
    flow and at a distance of $1.54$ from the wall, with an applied
    force of $\vec{e}_y$ and no applied torque. Mesh parameter $N =
    8$.}
  \label{fig:f08w_plot}
\end{figure}

\begin{figure}
  \centering
  \includegraphics[width=.5\textwidth]{timings}
  \caption{Timings for generating a resistance matrix with a mesh with
    mesh parameter $N$. There are $6N^2 + 2$ nodes/Stokeslets on the surface of
    the platelet. Therefore the Stokeslets matrix contains
    $\mathcal{O}(N^4)$ elements, and factoring the matrix requires
    $\mathcal{O}(N^6)$ operations.}
  \label{fig:timings_plot}
\end{figure}

% \begin{figure}[b]
%   \centering
%   \begin{subfigure}{0.49\textwidth}
%     \includegraphics[width=\textwidth]{orient_plot71_2nd}
%   \end{subfigure}
%   \hfill
%   \begin{subfigure}{0.49\textwidth}
%     \includegraphics[width=\textwidth]{orient_err_plot71_2nd}
%   \end{subfigure}
%   \\
%   \begin{subfigure}{0.49\textwidth}
%     \includegraphics[width=\textwidth]{com_plot71_2nd}
%   \end{subfigure}
%   \hfill
%   \begin{subfigure}{0.49\textwidth}
%     \includegraphics[width=\textwidth]{nsep_plot71}
%   \end{subfigure}  
%   \caption{Plots of the ellipsoid orientation, orientation error,
%     center of mass, and the $N$ value with the plt-wall
%     separation. The height of the center of mass is initialized at
%     $1.5$. Orientation is initialized at $\vect{e}_m =
%     \vect{e}_x$.}
%   \label{fig:plt71}
% \end{figure}

% \begin{figure}[b]
%   \centering
%   \begin{subfigure}{0.49\textwidth}
%     \includegraphics[width=\textwidth]{orient_plot72_2nd}
%   \end{subfigure}
%   \hfill
%   \begin{subfigure}{0.49\textwidth}
%     \includegraphics[width=\textwidth]{orient_err_plot72_2nd}
%   \end{subfigure}
%   \\
%   \begin{subfigure}{0.49\textwidth}
%     \includegraphics[width=\textwidth]{com_plot72_2nd}
%   \end{subfigure}
%   \hfill
%   \begin{subfigure}{0.49\textwidth}
%     \includegraphics[width=\textwidth]{nsep_plot72}
%   \end{subfigure}  
%   \caption{Plots of the ellipsoid orientation, orientation error,
%     center of mass, and the $N$ value with the plt-wall
%     separation. The height of the center of mass is initialized at
%     $1.2$. Orientation is initialized at $\vect{e}_m =
%     \vect{e}_x$.}
%   \label{fig:plt72}
% \end{figure}

% \begin{figure}
%   \centering
%   \begin{subfigure}{0.49\textwidth}
%     \includegraphics[width=\textwidth]{orient_plot74_2nd}
%   \end{subfigure}
%   \hfill
%   \begin{subfigure}{0.49\textwidth}
%     \includegraphics[width=\textwidth]{orient_err_plot74_2nd}
%   \end{subfigure}
%   \\
%   \begin{subfigure}{0.49\textwidth}
%     \includegraphics[width=\textwidth]{com_plot74_2nd}
%   \end{subfigure}
%   \hfill
%   \begin{subfigure}{0.49\textwidth}
%     \includegraphics[width=\textwidth]{nsep_plot74}
%   \end{subfigure}  
%   \caption{Plots of the ellipsoid orientation, orientation error,
%     center of mass, and the $N$ value with the plt-wall
%     separation. The height of the center of mass is initialized at
%     $0.8$. Orientation is initialized at $\vect{e}_m =
%     \vect{e}_x$.}
%   \label{fig:plt74}
% \end{figure}

% \begin{figure}
%   \centering
%   \includegraphics[width=\textwidth]{f_test0_8}
%   \caption{Motion integration of an ellipsoid in a shear flow, with an
%     applied force of $\vec{e}_y$ and no applied torque. Mesh parameter
%   $N = 8$.}
%   \label{fig:f08_plot}
% \end{figure}

% \begin{figure}
%   \centering
%   \includegraphics[width=\textwidth]{f_test0_16}
%   \caption{Motion integration of an ellipsoid in a shear flow, with an
%     applied force of $\vec{e}_y$ and no applied torque. Mesh parameter
%   $N = 16$.}
%   \label{fig:f016_plot}
% \end{figure}

% \begin{figure}
%   \centering
%   \includegraphics[width=\textwidth]{f_test4_16}
%   \caption{Motion integration of a sphere (radius $=1$) in a shear
%     flow, with no applied force and an applied torque of $\vec{e}_z$.}
%   \label{fig:f4_plot}
% \end{figure}

% \begin{figure}
%   \centering
%   \includegraphics[width=\textwidth]{f_test6_16}
%   \caption{Motion integration of an ellipsoid in a shear flow, with an
%     applied force of $\vec{e}_z$ and no applied torque.}
%   \label{fig:f6_plot}
% \end{figure}

\bibliographystyle{plain}
\bibliography{/Users/andrewwork/Documents/grad-school/thesis/library}

\end{document}




