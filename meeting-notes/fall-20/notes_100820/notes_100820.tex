\documentclass{article}

\newcommand{\ep}{\rule{.06in}{.1in}}
\textheight 9.5in

\usepackage{amssymb, bm}
\usepackage{amsmath}
\usepackage{amsthm}
\usepackage{graphicx, subcaption, booktabs}

\usepackage{tikz, pgfplots, pgfplotstable, chemfig, xcolor}

% \usepgfplotslibrary{colorbrewer, statistics}
% \pgfplotsset{
%   exact axis/.style={grid=major, minor tick num=4, xlabel=$v^*$,
%     legend entries={PDF, CDF},},
%   every axis plot post/.append style={thick},
%   table/search
%   path={/Users/andrewwork/thesis/jump-velocity/dat-files},
%   colormap/YlGnBu,
%   cycle list/Set1-5,
%   legend style={legend cell align=left,},
% }

% \usepgfplotslibrary{external}
% \tikzexternalize

\renewcommand{\arraystretch}{1.2}
\pagestyle{empty} 
\oddsidemargin -0.25in
\evensidemargin -0.25in 
\topmargin -0.75in 
\parindent 0pt
\parskip 12pt
\textwidth 7in
%\font\cj=msbm10 at 12pt

\newcommand{\tn}{\textnormal}
\newcommand{\stiff}{\frac{k_f}{\gamma}}
\newcommand{\dd}{d}
\newcommand{\Der}[2]{\frac{\dd #1}{\dd #2}}
\newcommand{\Pder}[2]{\frac{\partial #1}{\partial #2}}
\newcommand{\Integral}[4]{\int_{#3}^{#4} {#1} \dd #2}
\newcommand{\vect}[1]{\boldsymbol{\mathbf{#1}}}
\newcommand{\mat}[1]{\underline{\underline{#1}}}
\DeclareMathOperator{\Exp}{Exp}

% Text width is 7 inches

\def\R{\mathbb{R}}
\def\N{\mathbb{N}}
\def\C{\mathbb{C}}
\def\Z{\mathbb{Z}}
\def\Q{\mathbb{Q}}
\def\H{\mathbb{H}}
\def\B{\mathcal{B}} 
%\topmargin -.5in 

\setcounter{secnumdepth}{2}
\begin{document}
\pagestyle{plain}

\begin{center}
  {\Large Meeting Notes (\today)}
\end{center}

\begin{itemize}
\item I've primarily been working on exploring the Stokeslets matrices
  for speeding up the Regularized Stokeslets computations. Currently,
  the time it takes to compute and invert the Stokeslets matrices is
  roughly 10s for $N = 16$ and rougly 60s for $N = 24$.
\item For $N = 24$, this is about 500$\times$ slower than what Mody
  and King report in \cite{Mody2005} (they say a simulation with
  300,000 time steps completes in under 24 hrs)
\item One key difference: they don't regularize forces, and use a
  much smaller mesh size---equivalent to $N = 4$. If we took a mesh of
  this size, the regularization parameter would be 0.25.
\item I think there are a couple of options on how to deal with this:
  \begin{itemize}
  \item Option 1: Continue using very fine meshes, and do cleverer
    linear algebra
  \item Option 2: Modify the regularized Stokeslets method to decouple
    the choice of the regularization parameter from the average node
    distance (I found a couple published examples of this in the
    literature)
  \item Option 3: Suck up the computational cost and move on
  \end{itemize}
\end{itemize}

\textbf{Option 1: Linear algebra}
\begin{itemize}
\item I saved Stokeslets matrices from 5 representative time points in
  the platelet-wall experiments, and iterated forward 2 time steps
  using the 2nd order method (resulting in 4 stokeslets matrices)
\item Two methods for speeding up the linear algebra come to mind: one
  is doing a low-rank update of the Cholesky decomposition, and the
  other is using the conjugate gradient method to compute an
  approximate solution.
\item For the Cholesky update, the accuracy depends on the
  distribution of singular values of the update matrix (i.e. the
  difference between successive Stokeslets matrices). If you construct
  an approximation of rank $r$, then $\|A - \hat{A}\|_F =
  \sqrt{\sigma_{r+1}^2 + \cdots + \sigma_{n}^2}$.
\item I computed and plotted the singular values for the difference
  matrices in Figure \ref{fig:svals} and \ref{fig:svals_log}.
\item For the conjugate gradient method, the rate of convergence is
  bounded by the condition number ($\kappa$) in the following way:
  \[
    \tn{Rate of convergence} = \frac{\sqrt{\kappa} - 1}{\sqrt{\kappa} + 1}.
  \]
\item The matrices had condition numbers mostly in the 10--50
  range. However, at $t=11$ in the first plot, the matrices were much
  more ill-conditioned, with condition numbers in the $10^4$ to $10^5$
  range.
\end{itemize}

\begin{figure}[b]
  \centering
  \begin{subfigure}{0.49\textwidth}
    \includegraphics[width=\textwidth]{orient_plot72_2nd}
  \end{subfigure}
  \hfill
  \begin{subfigure}{0.49\textwidth}
    \includegraphics[width=\textwidth]{orient_err_plot72_2nd}
  \end{subfigure}
  \\
  \begin{subfigure}{0.49\textwidth}
    \includegraphics[width=\textwidth]{com_plot72_2nd}
  \end{subfigure}
  \hfill
  \begin{subfigure}{0.49\textwidth}
    \includegraphics[width=\textwidth]{nsep_plot72}
  \end{subfigure}  
  \caption{Plots of the ellipsoid orientation, orientation error,
    center of mass, and the $N$ value with the plt-wall
    separation. I chose time points $t=0$, $t=11$, and $t=20$ from
    this simulation.}
  \label{fig:plt72}
\end{figure}

\begin{figure}
  \centering
  \begin{subfigure}{0.49\textwidth}
    \includegraphics[width=\textwidth]{orient_plot74_2nd}
  \end{subfigure}
  \hfill
  \begin{subfigure}{0.49\textwidth}
    \includegraphics[width=\textwidth]{orient_err_plot74_2nd}
  \end{subfigure}
  \\
  \begin{subfigure}{0.49\textwidth}
    \includegraphics[width=\textwidth]{com_plot74_2nd}
  \end{subfigure}
  \hfill
  \begin{subfigure}{0.49\textwidth}
    \includegraphics[width=\textwidth]{nsep_plot74}
  \end{subfigure}  
  \caption{Plots of the ellipsoid orientation, orientation error,
    center of mass, and the $N$ value with the plt-wall
    separation. I chose time points $t=0$ and $t = 55$ from this
    simulation.} 
  \label{fig:plt74}
\end{figure}

\begin{figure}
  \centering
  \includegraphics[width=\textwidth]{svals}
  \caption{Singular values of the difference matrices (within a time step). The first two
    columns show the singular values for the two time steps starting
    at $t=0$ in Figure 1, the next two columns show the singular values for the
    two time steps starting at $t = 11$ in Figure 1, and so on.}
  \label{fig:svals}
\end{figure}

\begin{figure}
  \centering
  \includegraphics[width=\textwidth]{svals_log}
  \caption{}
  \label{fig:svals_log}
\end{figure}

\textbf{Option 2: Modified regularized Stokeslets method}
\begin{itemize}
\item No error analysis has been carried out for the method of
  regularized stokeslets with images, however for method of
  regularized stokeslets in an unbounded domain, Cortez
  et. al. \cite{Cortez2005} found that the error is
  $\mathcal{O}\left(\dfrac{h^2}{\epsilon^3}\right) +
  \mathcal{O}\left(\epsilon\right)$.
\item Therefore, if we want to take $\epsilon$ small to reduce the
  regularization error, we also need to use a fine mesh to prevent
  blow-up of the first term
\item One proposed method for decoupling the discretization error from
  $\epsilon$ is the Method of Auxiliary Regularized Stokeslets
  \cite{Barrero-Gil2013}.
\end{itemize}


% \begin{figure}
%   \centering
%   \includegraphics[width=\textwidth]{f_test4_16}
%   \caption{Motion integration of a sphere (radius $=1$) in a shear
%     flow, with no applied force and an applied torque of
%     $\vec{e}_z$. Mesh parameter $N = 16$.}
%   \label{fig:f416_plot}
% \end{figure}

% \begin{figure}
%   \centering
%   \includegraphics[width=\textwidth]{f_test0_8w}
%   \caption{Motion integration of a sphere (radius $=1$) in a shear
%     flow and at a distance of $1.54$ from the wall, with an applied
%     force of $\vec{e}_y$ and no applied torque. Mesh parameter $N =
%     8$.}
%   \label{fig:f08w_plot}
% \end{figure}

% \begin{figure}
%   \centering
%   \includegraphics[width=.5\textwidth]{timings}
%   \caption{Timings for generating a resistance matrix with a mesh with
%     mesh parameter $N$. There are $6N^2 + 2$ nodes/Stokeslets on the surface of
%     the platelet. Therefore the Stokeslets matrix contains
%     $\mathcal{O}(N^4)$ elements, and factoring the matrix requires
%     $\mathcal{O}(N^6)$ operations.}
%   \label{fig:timings_plot}
% \end{figure}

% \begin{figure}[b]
%   \centering
%   \begin{subfigure}{0.49\textwidth}
%     \includegraphics[width=\textwidth]{orient_plot71_2nd}
%   \end{subfigure}
%   \hfill
%   \begin{subfigure}{0.49\textwidth}
%     \includegraphics[width=\textwidth]{orient_err_plot71_2nd}
%   \end{subfigure}
%   \\
%   \begin{subfigure}{0.49\textwidth}
%     \includegraphics[width=\textwidth]{com_plot71_2nd}
%   \end{subfigure}
%   \hfill
%   \begin{subfigure}{0.49\textwidth}
%     \includegraphics[width=\textwidth]{nsep_plot71}
%   \end{subfigure}  
%   \caption{Plots of the ellipsoid orientation, orientation error,
%     center of mass, and the $N$ value with the plt-wall
%     separation. The height of the center of mass is initialized at
%     $1.5$. Orientation is initialized at $\vect{e}_m =
%     \vect{e}_x$.}
%   \label{fig:plt71}
% \end{figure}

% \begin{figure}[b]
%   \centering
%   \begin{subfigure}{0.49\textwidth}
%     \includegraphics[width=\textwidth]{orient_plot72_2nd}
%   \end{subfigure}
%   \hfill
%   \begin{subfigure}{0.49\textwidth}
%     \includegraphics[width=\textwidth]{orient_err_plot72_2nd}
%   \end{subfigure}
%   \\
%   \begin{subfigure}{0.49\textwidth}
%     \includegraphics[width=\textwidth]{com_plot72_2nd}
%   \end{subfigure}
%   \hfill
%   \begin{subfigure}{0.49\textwidth}
%     \includegraphics[width=\textwidth]{nsep_plot72}
%   \end{subfigure}  
%   \caption{Plots of the ellipsoid orientation, orientation error,
%     center of mass, and the $N$ value with the plt-wall
%     separation. The height of the center of mass is initialized at
%     $1.2$. Orientation is initialized at $\vect{e}_m =
%     \vect{e}_x$.}
%   \label{fig:plt72}
% \end{figure}

% \begin{figure}
%   \centering
%   \begin{subfigure}{0.49\textwidth}
%     \includegraphics[width=\textwidth]{orient_plot74_2nd}
%   \end{subfigure}
%   \hfill
%   \begin{subfigure}{0.49\textwidth}
%     \includegraphics[width=\textwidth]{orient_err_plot74_2nd}
%   \end{subfigure}
%   \\
%   \begin{subfigure}{0.49\textwidth}
%     \includegraphics[width=\textwidth]{com_plot74_2nd}
%   \end{subfigure}
%   \hfill
%   \begin{subfigure}{0.49\textwidth}
%     \includegraphics[width=\textwidth]{nsep_plot74}
%   \end{subfigure}  
%   \caption{Plots of the ellipsoid orientation, orientation error,
%     center of mass, and the $N$ value with the plt-wall
%     separation. The height of the center of mass is initialized at
%     $0.8$. Orientation is initialized at $\vect{e}_m =
%     \vect{e}_x$.}
%   \label{fig:plt74}
% \end{figure}

% \begin{figure}
%   \centering
%   \includegraphics[width=\textwidth]{f_test0_8}
%   \caption{Motion integration of an ellipsoid in a shear flow, with an
%     applied force of $\vec{e}_y$ and no applied torque. Mesh parameter
%   $N = 8$.}
%   \label{fig:f08_plot}
% \end{figure}

% \begin{figure}
%   \centering
%   \includegraphics[width=\textwidth]{f_test0_16}
%   \caption{Motion integration of an ellipsoid in a shear flow, with an
%     applied force of $\vec{e}_y$ and no applied torque. Mesh parameter
%   $N = 16$.}
%   \label{fig:f016_plot}
% \end{figure}

% \begin{figure}
%   \centering
%   \includegraphics[width=\textwidth]{f_test4_16}
%   \caption{Motion integration of a sphere (radius $=1$) in a shear
%     flow, with no applied force and an applied torque of $\vec{e}_z$.}
%   \label{fig:f4_plot}
% \end{figure}

% \begin{figure}
%   \centering
%   \includegraphics[width=\textwidth]{f_test6_16}
%   \caption{Motion integration of an ellipsoid in a shear flow, with an
%     applied force of $\vec{e}_z$ and no applied torque.}
%   \label{fig:f6_plot}
% \end{figure}

\bibliographystyle{plain}
\bibliography{/Users/andrewwork/Documents/grad-school/thesis/library}

\end{document}




