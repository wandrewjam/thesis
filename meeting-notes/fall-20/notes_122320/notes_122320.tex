\documentclass{article}

\newcommand{\ep}{\rule{.06in}{.1in}}
\textheight 9.5in

\usepackage{amssymb, bm}
\usepackage{amsmath}
\usepackage{amsthm}
\usepackage{graphicx, subcaption, booktabs}

\usepackage{tikz, pgfplots, pgfplotstable, chemfig, xcolor}

% \usepgfplotslibrary{colorbrewer, statistics}
% \pgfplotsset{
%   exact axis/.style={grid=major, minor tick num=4, xlabel=$v^*$,
%     legend entries={PDF, CDF},},
%   every axis plot post/.append style={thick},
%   table/search
%   path={/Users/andrewwork/thesis/jump-velocity/dat-files},
%   colormap/YlGnBu,
%   cycle list/Set1-5,
%   legend style={legend cell align=left,},
% }

% \usepgfplotslibrary{external}
% \tikzexternalize

\renewcommand{\arraystretch}{1.2}
\pagestyle{empty} 
\oddsidemargin -0.25in
\evensidemargin -0.25in 
\topmargin -0.75in 
\parindent 0pt
\parskip 12pt
\textwidth 7in
%\font\cj=msbm10 at 12pt

\newcommand{\tn}{\textnormal}
\newcommand{\stiff}{\frac{k_f}{\gamma}}
\newcommand{\dd}{d}
\newcommand{\Der}[2]{\frac{\dd #1}{\dd #2}}
\newcommand{\Pder}[2]{\frac{\partial #1}{\partial #2}}
\newcommand{\Integral}[4]{\int_{#3}^{#4} {#1} \dd #2}
\newcommand{\vect}[1]{\boldsymbol{\mathbf{#1}}}
\newcommand{\mat}[1]{\underline{\underline{#1}}}
\DeclareMathOperator{\Exp}{Exp}

% Text width is 7 inches

\def\R{\mathbb{R}}
\def\N{\mathbb{N}}
\def\C{\mathbb{C}}
\def\Z{\mathbb{Z}}
\def\Q{\mathbb{Q}}
\def\H{\mathbb{H}}
\def\B{\mathcal{B}} 
%\topmargin -.5in 

\setcounter{secnumdepth}{2}
\begin{document}
\pagestyle{plain}

\begin{center}
  {\Large Meeting Notes (\today)}
\end{center}

\begin{itemize}
\item Ran simulations with 4 different $k_\tn{on}$ values: $1 / s$, $5
  / s$, $10 / s$, and $25 / s$. These values were chosen pretty
  arbitrarily, $10 / s$ is the $k_\tn{on}$ value I have been using in
  my previous experiments
\item I plotted trajectories from these experiments, found the
  proportion of platelets that bind at any point, the proportion of
  platelets that are bound at the end of the experiment, as well as
  average velocities, steps, and dwells
\end{itemize}

\begin{figure}[h]
  \centering
  \includegraphics[width=.8\textwidth]{trajectories03}
  \caption{Platelet trajectories with $k_\tn{on} = 1 \ s$. 15.6\% of
    platelets bound at some point, and 12.5\% were bound at the end of
  the simulation.}
  \label{fig:traj03}
\end{figure}

\begin{figure}
  \centering
  \includegraphics[width=.8\textwidth]{trajectories02}
  \caption{Platelet trajectories with $k_\tn{on} = 5 \ s$. 79.7\% of
    platelets bound at some point, and 43.0\% were bound at the end of
  the simulation.}
  \label{fig:traj02}
\end{figure}

\begin{figure}
  \centering
  \includegraphics[width=.8\textwidth]{trajectories01}
  \caption{Platelet trajectories with $k_\tn{on} = 10 \ s$. 93.8\% of
    platelets bound at some point, and 51.6\% were bound at the end of
  the simulation.}
  \label{fig:traj01}
\end{figure}

\begin{figure}
  \centering
  \includegraphics[width=.8\textwidth]{trajectories04}
  \caption{Platelet trajectories with $k_\tn{on} = 25 \ s$. 100\% of
    platelets bound at some point, and 47.7\% were bound at the end of
  the simulation.}
  \label{fig:traj04}
\end{figure}

% \begin{figure}
%   \centering
%   \includegraphics[width=.8\textwidth]{avg_vels}
%   \caption{}
%   \label{fig:avg_vels}
% \end{figure}

% \begin{figure}
%   \centering
%   \includegraphics[width=.8\textwidth]{steps_plot}
%   \caption{}
%   \label{fig:steps}
% \end{figure}

% \begin{figure}
%   \centering
%   \includegraphics[width=.8\textwidth]{dwells_plot}
%   \caption{}
%   \label{fig:dwells}
% \end{figure}

% \begin{figure}
%   \centering
%   \includegraphics[width=.8\textwidth]{bd_expt006}
%   \caption{}
%   \label{fig:fig7}
% \end{figure}

% \begin{figure}
%   \centering
%   \includegraphics[width=.8\textwidth]{bd_expt007}
%   \caption{}
%   \label{fig:fig8}
% \end{figure}

% \begin{figure}
%   \centering
%   \includegraphics[width=.8\textwidth]{bd_expt008}
%   \caption{}
%   \label{fig:fig9}
% \end{figure}

\bibliographystyle{plain}
\bibliography{/Users/andrewwork/Documents/grad-school/thesis/library}

\end{document}




