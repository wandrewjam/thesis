\documentclass{article}

\newcommand{\ep}{\rule{.06in}{.1in}}
\textheight 9.5in

\usepackage{amssymb, bm}
\usepackage{amsmath}
\usepackage{amsthm}
\usepackage{graphicx, subcaption, booktabs}

\usepackage{tikz, pgfplots, pgfplotstable, chemfig, xcolor}
\graphicspath{{/Users/andrewwork/thesis/reg_stokeslets/plots/}}

% \usepgfplotslibrary{colorbrewer, statistics}
% \pgfplotsset{
%   exact axis/.style={grid=major, minor tick num=4, xlabel=$v^*$,
%     legend entries={PDF, CDF},},
%   every axis plot post/.append style={thick},
%   table/search
%   path={/Users/andrewwork/thesis/jump-velocity/dat-files},
%   colormap/YlGnBu,
%   cycle list/Set1-5,
%   legend style={legend cell align=left,},
% }

% \usepgfplotslibrary{external}
% \tikzexternalize

\renewcommand{\arraystretch}{1.2}
\pagestyle{empty} 
\oddsidemargin -0.25in
\evensidemargin -0.25in 
\topmargin -0.75in 
\parindent 0pt
\parskip 12pt
\textwidth 7in
%\font\cj=msbm10 at 12pt

\newcommand{\tn}{\textnormal}
\newcommand{\stiff}{\frac{k_f}{\gamma}}
\newcommand{\dd}{d}
\newcommand{\Der}[2]{\frac{\dd #1}{\dd #2}}
\newcommand{\Pder}[2]{\frac{\partial #1}{\partial #2}}
\newcommand{\Integral}[4]{\int_{#3}^{#4} {#1} \dd #2}
\newcommand{\mat}[1]{\underline{\underline{#1}}}
\DeclareMathOperator{\Exp}{Exp}

% Text width is 7 inches

\def\R{\mathbb{R}}
\def\N{\mathbb{N}}
\def\C{\mathbb{C}}
\def\Z{\mathbb{Z}}
\def\Q{\mathbb{Q}}
\def\H{\mathbb{H}}
\def\B{\mathcal{B}} 
%\topmargin -.5in 

\newcommand{\radius}{R}
\newcommand{\separation}{d}
\newcommand{\stiffness}{k_f}
\newcommand{\boltzmann}{k_B}
\newcommand{\temp}{T}
\newcommand{\onConst}{k_\text{on}}
\newcommand{\offConst}{k_\text{off}}
\newcommand{\refForce}{f_0}
\newcommand{\receptorDensity}{N_T}
\newcommand{\receptorNumber}{N_R}
\newcommand{\appliedRot}{\Omega_f}
\newcommand{\appliedVel}{V_f}
\newcommand{\velFriction}{\xi_V}
\newcommand{\rotFriction}{\xi_\omega}
\newcommand{\compliance}{\Gamma}
\newcommand{\width}{w}
\newcommand{\viscosity}{\mu}

\newcommand{\ndSeparation}{d'}
\newcommand{\ndAppliedRot}{\omega_f}
\newcommand{\ndAppliedVel}{v_f}
\newcommand{\ndOnConst}{\kappa}
\newcommand{\onForceScale}{\eta}
\newcommand{\offForceScale}{\delta}
\newcommand{\ndVelFriction}{\eta_v}
\newcommand{\ndRotFriction}{\eta_\omega}

\newcommand{\ITA}[1]{\textalpha\textsubscript{#1}}
\newcommand{\ITB}[1]{\textbeta\textsubscript{#1}}

\setcounter{secnumdepth}{2}
\begin{document}
\pagestyle{plain}

\begin{center}
  {\Large Meeting Notes (\today)}
\end{center}

\begin{itemize}
\item Added a second set of bonds to the rolling model using the
  catch-slip model in Wang 2013 \cite{Wang2013}:
  \begin{align}
    k_\tn{off}(f) &= \frac{1}{1 + \Phi} k_N(f) + \frac{\Phi}{1 + \Phi}
                    k_I(f) \\
    \Phi(f) &=
              \frac{k_{\tn{NG}\rightarrow\tn{IG}}^0}
              {k_{\tn{IG}\rightarrow\tn{NG}}^0}
              \exp \left[\frac{\gamma' f}{k_B T} \right] \\
    k_N(f) &= k_{N,\tn{off}}^0 \exp \left[\frac{y_N f}{k_B T}\right]
    \\
    k_I(f) &= k_{I,\tn{off}}^0 \exp \left[\frac{y_I f}{k_B T}\right]
  \end{align}
\item This model is based on the assumption that the GP1b-vWF can
  exist in two states: a native folded state (NG) which acts like a
  catch bond, and an intermediate open state (IG) which acts like a
  slip bond ($y_N < 0$ and $y_I > 0$). The fraction of bonds in each
  state is dependent on the force, so as the force increases, more
  bonds are in the slip-bond state.
\item Also, pure catch bonds and slip bonds are special cases of this
  model. $\Phi = 0$ gives a catch bond and $\Phi = \infty$ gives a
  slip bond
\item For receptors, I think it is okay to use the same locations for
  both types of receptors. With a rest length of 100 nm, the
  ``binding area'' of a single receptor is $3.1 \times 10^4 nm^2$, and
  each receptor has an area of around $1.6 \times 10^4 nm^2$ to
  itself.
\item I calculated the average number of bonds per dwell
  (weighted by time), and the location that bonds form on the platelet
  surface
\end{itemize}

\begin{figure}[h]
  \centering
  \begin{subfigure}{0.49\textwidth}
    \includegraphics[width=\textwidth]{avel_avg_10.png}
  \end{subfigure}
  \hfill
  \begin{subfigure}{0.49\textwidth}
    \includegraphics[width=\textwidth]{step_avg_10.png}
  \end{subfigure}
  \\
  \begin{subfigure}{0.49\textwidth}
    \includegraphics[width=\textwidth]{dwell_avg_10.png}
  \end{subfigure}
  \label{fig:stats10}
  \caption{Means of average velocity (top left), step size (top
    right), and dwell time (bottom) for different $k_\tn{on}$
    values ($k_\tn{off} = 5$)}
\end{figure}

\begin{figure}[h]
  \centering
  \includegraphics[width=0.6\textwidth]{bdav_10.png}
  \caption{Location of bond formation on the platelet surface}
  \label{fig:bd-loc}
\end{figure}

% \begin{figure}[h]
%   \centering
%   \begin{subfigure}{0.49\textwidth}
%     \includegraphics[width=\textwidth]{avel_avg_8.png}
%   \end{subfigure}
%   \hfill
%   \begin{subfigure}{0.49\textwidth}
%     \includegraphics[width=\textwidth]{step_avg_8.png}
%   \end{subfigure}
%   \\
%   \begin{subfigure}{0.49\textwidth}
%     \includegraphics[width=\textwidth]{dwell_avg_8.png}
%   \end{subfigure}
%   \label{fig:stats8}
%   \caption{Means of average velocity (top left), step size (top
%     right), and dwell time (bottom) for 3 different $k_\tn{off}$
%     values ($k_\tn{on} = 5$). For average velocities, there are
%     significant differences between the $k_\tn{off} = 1$ and
%     $k_\tn{off} = 10$ and $k_\tn{off} = 5$ and $k_\tn{off} = 10$
%     cases. For step sizes, there is a significant difference between
%     the $k_\tn{off} = 2$ and $k_\tn{off} = 10$ cases. For dwells,
%     all differences are significant except between $k_\tn{off} = 5$
%     and $k_\tn{off} = 10$}
% \end{figure}

% \begin{figure}[h]
%   \centering
%   \begin{subfigure}{0.49\textwidth}
%     \includegraphics[width=\textwidth]{avel_avg_9.png}
%   \end{subfigure}
%   \hfill
%   \begin{subfigure}{0.49\textwidth}
%     \includegraphics[width=\textwidth]{step_avg_9.png}
%   \end{subfigure}
%   \\
%   \begin{subfigure}{0.49\textwidth}
%     \includegraphics[width=\textwidth]{dwell_avg_9.png}
%   \end{subfigure}
%   \label{fig:stats9}
%   \caption{Means of average velocity (top left), step size (top
%     right), and dwell time (bottom).}
% \end{figure}

% \begin{figure}[h]
%   \centering
%   \begin{subfigure}{0.49\textwidth}
%     \includegraphics[width=\textwidth]{avg_vels_9.png}
%   \end{subfigure}
%   \hfill
%   \begin{subfigure}{0.49\textwidth}
%     \includegraphics[width=\textwidth]{steps_9.png}
%   \end{subfigure}
%   \caption{Distributions of average velocities and step sizes for
%     different receptor numbers}
%   \label{fig:avg-vels9}
% \end{figure}

% \begin{figure}[h]
%   \centering
%   \begin{subfigure}{0.49\textwidth}
%     \includegraphics[width=\textwidth]{bdbk_8.png}
%   \end{subfigure}
%   \hfill
%   \begin{subfigure}{0.49\textwidth}
%     \includegraphics[width=\textwidth]{bdbk_10.png}
%   \end{subfigure}
%   \caption{Distributions of bond lengths at breaking for a range of
%     $k_\tn{on}$ and $k_\tn{off}$ values}
%   \label{fig:bdbk}
% \end{figure}

% \begin{figure}[h]
%   \centering
%   \begin{subfigure}{0.49\textwidth}
%     \includegraphics[width=\textwidth]{bdlf_8.png}
%   \end{subfigure}
%   \hfill
%   \begin{subfigure}{0.49\textwidth}
%     \includegraphics[width=\textwidth]{bdlf_10.png}
%   \end{subfigure}
%   \caption{Distributions of bond lifetimes for a range of $k_\tn{on}$
%     and $k_\tn{off}$ values}
%   \label{fig:bdlf}
% \end{figure}

% \bibliographystyle{plain}
% \bibliography{/Users/andrewwork/Documents/grad-school/thesis/library}

\end{document}




