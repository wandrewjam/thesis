\documentclass{article}

\newcommand{\ep}{\rule{.06in}{.1in}}
\textheight 9.5in

\usepackage{amssymb, bm}
\usepackage{amsmath}
\usepackage{amsthm}
\usepackage{graphicx, subcaption, booktabs}

\usepackage{tikz, pgfplots, pgfplotstable, chemfig, xcolor}
\graphicspath{{/Users/andrewwork/thesis/reg_stokeslets/plots/}}

% \usepgfplotslibrary{colorbrewer, statistics}
% \pgfplotsset{
%   exact axis/.style={grid=major, minor tick num=4, xlabel=$v^*$,
%     legend entries={PDF, CDF},},
%   every axis plot post/.append style={thick},
%   table/search
%   path={/Users/andrewwork/thesis/jump-velocity/dat-files},
%   colormap/YlGnBu,
%   cycle list/Set1-5,
%   legend style={legend cell align=left,},
% }

% \usepgfplotslibrary{external}
% \tikzexternalize

\renewcommand{\arraystretch}{1.2}
\pagestyle{empty} 
\oddsidemargin -0.25in
\evensidemargin -0.25in 
\topmargin -0.75in 
\parindent 0pt
\parskip 12pt
\textwidth 7in
%\font\cj=msbm10 at 12pt

\newcommand{\tn}{\textnormal}
\newcommand{\stiff}{\frac{k_f}{\gamma}}
\newcommand{\dd}{d}
\newcommand{\Der}[2]{\frac{\dd #1}{\dd #2}}
\newcommand{\Pder}[2]{\frac{\partial #1}{\partial #2}}
\newcommand{\Integral}[4]{\int_{#3}^{#4} {#1} \dd #2}
\newcommand{\vect}[1]{\boldsymbol{\mathbf{#1}}}
\newcommand{\mat}[1]{\underline{\underline{#1}}}
\DeclareMathOperator{\Exp}{Exp}

% Text width is 7 inches

\def\R{\mathbb{R}}
\def\N{\mathbb{N}}
\def\C{\mathbb{C}}
\def\Z{\mathbb{Z}}
\def\Q{\mathbb{Q}}
\def\H{\mathbb{H}}
\def\B{\mathcal{B}} 
%\topmargin -.5in 

\newcommand{\radius}{R}
\newcommand{\separation}{d}
\newcommand{\stiffness}{k_f}
\newcommand{\boltzmann}{k_B}
\newcommand{\temp}{T}
\newcommand{\onConst}{k_\text{on}}
\newcommand{\offConst}{k_\text{off}}
\newcommand{\refForce}{f_0}
\newcommand{\receptorDensity}{N_T}
\newcommand{\receptorNumber}{N_R}
\newcommand{\appliedRot}{\Omega_f}
\newcommand{\appliedVel}{V_f}
\newcommand{\velFriction}{\xi_V}
\newcommand{\rotFriction}{\xi_\omega}
\newcommand{\compliance}{\Gamma}
\newcommand{\width}{w}
\newcommand{\viscosity}{\mu}

\newcommand{\ndSeparation}{d'}
\newcommand{\ndAppliedRot}{\omega_f}
\newcommand{\ndAppliedVel}{v_f}
\newcommand{\ndOnConst}{\kappa}
\newcommand{\onForceScale}{\eta}
\newcommand{\offForceScale}{\delta}
\newcommand{\ndVelFriction}{\eta_v}
\newcommand{\ndRotFriction}{\eta_\omega}

\newcommand{\ITA}[1]{\textalpha\textsubscript{#1}}
\newcommand{\ITB}[1]{\textbeta\textsubscript{#1}}

\setcounter{secnumdepth}{2}
\begin{document}
\pagestyle{plain}

\begin{center}
  {\Large Meeting Notes (\today)}
\end{center}

\begin{itemize}
\item I have data for a range of $k_\tn{on}$ and $k_\tn{off}$ values,
  and for double the number of receptors (all with the same random
  ICs)
\item Plots of average velocities, step sizes, and dwell times are
  given below, as well as some other plots with interesting results
\item For the estimate of $k_\tn{on}$ from the Litvinov paper, last
  week I had estimated an on-rate of $0.31 /s$ for a receptor-ligand
  pair from their 2nd order rate constant. I still think this is the
  correct constant, but have a bit more justification
\item Assume you have 1 ligand and an excess of receptors (they assume
  an excess of receptors in their paper), and let's model the
  probability $p_1$ that a bond is present at time $t$ within the
  contact area. This can be written as an ode system:
  \begin{align*}
    \frac{d p_0}{dt} &= -c R p_0 + d p_1 \\
    \frac{d p_1}{dt} &= c R p_0 - d p_1
  \end{align*}
\item $R$ is a constant, and is the number of receptors present in the
  area of contact $A_c$. $R = d_R A_c$
\item Before, I had estimated $c = k_\tn{on} / A_c$. The off
  ``propensity'' $d$ is the same as $k_\tn{off}$, because it is a 1st
  order rate constant. The system above equilibrates to
  $\frac{c R}{c R + d}$, which is equivalent to the formula they give
  for the equilibrium in their paper:
  $\frac{k_\tn{on} d_R}{k_\tn{on} d_R + k_\tn{off}}$
\end{itemize}

\begin{figure}[h]
  \centering
  \begin{subfigure}{0.49\textwidth}
    \includegraphics[width=\textwidth]{avel_avg_10.png}
  \end{subfigure}
  \hfill
  \begin{subfigure}{0.49\textwidth}
    \includegraphics[width=\textwidth]{step_avg_10.png}
  \end{subfigure}
  \\
  \begin{subfigure}{0.49\textwidth}
    \includegraphics[width=\textwidth]{dwell_avg_10.png}
  \end{subfigure}
  \label{fig:stats10}
  \caption{Means of average velocity (top left), step size (top
    right), and dwell time (bottom) for different $k_\tn{on}$
    values ($k_\tn{off} = 5$)}
\end{figure}

\begin{figure}[h]
  \centering
  \begin{subfigure}{0.49\textwidth}
    \includegraphics[width=\textwidth]{avel_avg_8.png}
  \end{subfigure}
  \hfill
  \begin{subfigure}{0.49\textwidth}
    \includegraphics[width=\textwidth]{step_avg_8.png}
  \end{subfigure}
  \\
  \begin{subfigure}{0.49\textwidth}
    \includegraphics[width=\textwidth]{dwell_avg_8.png}
  \end{subfigure}
  \label{fig:stats8}
  \caption{Means of average velocity (top left), step size (top
    right), and dwell time (bottom) for 3 different $k_\tn{off}$
    values ($k_\tn{on} = 5$).}
\end{figure}

\begin{figure}[h]
  \centering
  \begin{subfigure}{0.49\textwidth}
    \includegraphics[width=\textwidth]{avel_avg_9.png}
  \end{subfigure}
  \hfill
  \begin{subfigure}{0.49\textwidth}
    \includegraphics[width=\textwidth]{step_avg_9.png}
  \end{subfigure}
  \\
  \begin{subfigure}{0.49\textwidth}
    \includegraphics[width=\textwidth]{dwell_avg_9.png}
  \end{subfigure}
  \label{fig:stats9}
  \caption{Means of average velocity (top left), step size (top
    right), and dwell time (bottom).}
\end{figure}

\begin{figure}[h]
  \centering
  \begin{subfigure}{0.49\textwidth}
    \includegraphics[width=\textwidth]{avg_vels_9.png}
  \end{subfigure}
  \hfill
  \begin{subfigure}{0.49\textwidth}
    \includegraphics[width=\textwidth]{steps_9.png}
  \end{subfigure}
  \caption{Distributions of average velocities and step sizes for
    different receptor numbers}
  \label{fig:avg-vels9}
\end{figure}

\begin{figure}[h]
  \centering
  \begin{subfigure}{0.49\textwidth}
    \includegraphics[width=\textwidth]{bdbk_8.png}
  \end{subfigure}
  \hfill
  \begin{subfigure}{0.49\textwidth}
    \includegraphics[width=\textwidth]{bdbk_10.png}
  \end{subfigure}
  \caption{Distributions of bond lengths at breaking for a range of
    $k_\tn{on}$ and $k_\tn{off}$ values}
  \label{fig:bdbk}
\end{figure}

\begin{figure}[h]
  \centering
  \begin{subfigure}{0.49\textwidth}
    \includegraphics[width=\textwidth]{bdlf_8.png}
  \end{subfigure}
  \hfill
  \begin{subfigure}{0.49\textwidth}
    \includegraphics[width=\textwidth]{bdlf_10.png}
  \end{subfigure}
  \caption{Distributions of bond lifetimes for a range of $k_\tn{on}$
    and $k_\tn{off}$ values}
  \label{fig:bdlf}
\end{figure}

% \bibliographystyle{plain}
% \bibliography{/Users/andrewwork/Documents/grad-school/thesis/library}

\end{document}




