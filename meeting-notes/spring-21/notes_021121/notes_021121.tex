\documentclass{article}

\newcommand{\ep}{\rule{.06in}{.1in}}
\textheight 9.5in

\usepackage{amssymb, bm}
\usepackage{amsmath}
\usepackage{amsthm}
\usepackage{graphicx, subcaption, booktabs}

\usepackage{tikz, pgfplots, pgfplotstable, chemfig, xcolor}
\graphicspath{{/Users/andrewwork/thesis/reg_stokeslets/plots/}}

% \usepgfplotslibrary{colorbrewer, statistics}
% \pgfplotsset{
%   exact axis/.style={grid=major, minor tick num=4, xlabel=$v^*$,
%     legend entries={PDF, CDF},},
%   every axis plot post/.append style={thick},
%   table/search
%   path={/Users/andrewwork/thesis/jump-velocity/dat-files},
%   colormap/YlGnBu,
%   cycle list/Set1-5,
%   legend style={legend cell align=left,},
% }

% \usepgfplotslibrary{external}
% \tikzexternalize

\renewcommand{\arraystretch}{1.2}
\pagestyle{empty} 
\oddsidemargin -0.25in
\evensidemargin -0.25in 
\topmargin -0.75in 
\parindent 0pt
\parskip 12pt
\textwidth 7in
%\font\cj=msbm10 at 12pt

\newcommand{\tn}{\textnormal}
\newcommand{\stiff}{\frac{k_f}{\gamma}}
\newcommand{\dd}{d}
\newcommand{\Der}[2]{\frac{\dd #1}{\dd #2}}
\newcommand{\Pder}[2]{\frac{\partial #1}{\partial #2}}
\newcommand{\Integral}[4]{\int_{#3}^{#4} {#1} \dd #2}
\newcommand{\vect}[1]{\boldsymbol{\mathbf{#1}}}
\newcommand{\mat}[1]{\underline{\underline{#1}}}
\DeclareMathOperator{\Exp}{Exp}

% Text width is 7 inches

\def\R{\mathbb{R}}
\def\N{\mathbb{N}}
\def\C{\mathbb{C}}
\def\Z{\mathbb{Z}}
\def\Q{\mathbb{Q}}
\def\H{\mathbb{H}}
\def\B{\mathcal{B}} 
%\topmargin -.5in 

\newcommand{\radius}{R}
\newcommand{\separation}{d}
\newcommand{\stiffness}{k_f}
\newcommand{\boltzmann}{k_B}
\newcommand{\temp}{T}
\newcommand{\onConst}{k_\text{on}}
\newcommand{\offConst}{k_\text{off}}
\newcommand{\refForce}{f_0}
\newcommand{\receptorDensity}{N_T}
\newcommand{\receptorNumber}{N_R}
\newcommand{\appliedRot}{\Omega_f}
\newcommand{\appliedVel}{V_f}
\newcommand{\velFriction}{\xi_V}
\newcommand{\rotFriction}{\xi_\omega}
\newcommand{\compliance}{\Gamma}
\newcommand{\width}{w}
\newcommand{\viscosity}{\mu}

\newcommand{\ndSeparation}{d'}
\newcommand{\ndAppliedRot}{\omega_f}
\newcommand{\ndAppliedVel}{v_f}
\newcommand{\ndOnConst}{\kappa}
\newcommand{\onForceScale}{\eta}
\newcommand{\offForceScale}{\delta}
\newcommand{\ndVelFriction}{\eta_v}
\newcommand{\ndRotFriction}{\eta_\omega}

\newcommand{\ITA}[1]{\textalpha\textsubscript{#1}}
\newcommand{\ITB}[1]{\textbeta\textsubscript{#1}}

\setcounter{secnumdepth}{2}
\begin{document}
\pagestyle{plain}

\begin{center}
  {\Large Meeting Notes (\today)}
\end{center}

\begin{itemize}
\item \textbf{Power/Sample size analysis}:
  \begin{itemize}
  \item First, we need to specify precisely \emph{what} quantity we
    are estimating, and are testing for differences in. For our tests,
    we are looking at average velocities, steps, and dwells. We are
    looking at a number of different statistics of these quantities,
    but the easiest one to estimate the power for is the mean.
  \item In statistical jargon, assume we have two sets of samples from
    two experiments with a different $k_\tn{on}$. (For the sake of
    this analysis) we are testing against the null hypothesis $H_0:
    \mu_1 = \mu_2$, and the alternative hypothesis is $H_1: \mu_1 \neq
    \mu_2$.
  \item $\beta$ is the probability of accepting $H_0$ when $H_1$ is
    true, and $1-\beta$ is the power of the statistic. (This is like
    the converse of $\alpha$, which is the more familiar error of
    rejecting $H_0$ when it is true).
  \item According to the book I was working from, conventional choices
    for the power are $80\%$ or $90\%$
  \item $\Delta$ is the difference between the two means $\Delta =
    \mu_1 - \mu_2$. For a fixed sample size, the power of a
    statistical test decreases as $\Delta$ decreases. And to achieve
    certain power, you need more samples as $\Delta$ decreases
  \item The analysis simplifies a lot in the limit $n\rightarrow
    \infty$, so I used this approximation when computing statistical
    power. Under this assumption, the number of samples required is
    given by the formula: $$n = \frac{(z_{\alpha/2} +
        z_\beta)\sigma^2}{\epsilon^2}$$
  \item In order to choose a reasonable $\epsilon$, I found the means
    (and standard deviations) of the average velocities, step sizes,
    and dwell times from my previous simulations (figure on next page).
  \item For average velocities, the means differed by roughly $20 \mu
    m / s$, with standard deviations between 20 and 30.
  \item For step sizes, the means differed by roughly $3 \mu m$ with
    standard deviations of about 10 to 15
  \item For dwell times, the means differed by roughly 10--20 ms, with
    standard deviations of about 100 ms
  \item Plugging these in to the formula above gave me about $n=30$
    for average velocities, $n=80$ for step sizes, and $n=80$ for
    dwell times.
  \item For comparing the average velocities, it seems like we already have
    more than enough samples. However, the steps and dwells are a
    little tougher. Most simulations don't produce a step: in the sets
    of 128 runs, the number of steps ranged from 2 to 24 for the
    entire set of experiments, depending on $k_\tn{on}$. For dwells,
    the experiments produced between 6 to 79 dwells for a set of
    experiments.
  \item This suggests that 500--1000 runs are needed to spot
    differences in the mean step size across experiments. Running this
    many experiments with the largest $k_\tn{on}$ value I'm currently
    using would take 500--1000 core-hours
  \end{itemize}
\end{itemize}

\begin{figure}[h]
  \centering
  \begin{subfigure}{0.49\textwidth}
    \includegraphics[width=\textwidth]{avel_avg_1.png}
  \end{subfigure}
  \hfill
  \begin{subfigure}{0.49\textwidth}
    \includegraphics[width=\textwidth]{step_avg_1.png}
  \end{subfigure}
  \\
  \begin{subfigure}{0.49\textwidth}
    \includegraphics[width=\textwidth]{dwell_avg_1.png}
  \end{subfigure}
  \label{fig:stats}
  \caption{Means of average velocity (top left), step size (top
    right), and dwell time (bottom) for 4 different $k_\tn{on}$
    values.}
\end{figure}

\bibliographystyle{plain}
\bibliography{/Users/andrewwork/Documents/grad-school/thesis/library}

\end{document}




