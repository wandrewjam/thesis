\documentclass{article}

\newcommand{\ep}{\rule{.06in}{.1in}}
\textheight 9.5in

\usepackage{amssymb, bm}
\usepackage{amsmath}
\usepackage{amsthm}
\usepackage{graphicx, subcaption, booktabs}

\usepackage{tikz, pgfplots, pgfplotstable, chemfig, xcolor}
% \graphicspath{{/Users/andrewwork/thesis/reg_stokeslets/3d/plots}}

% \usepgfplotslibrary{colorbrewer, statistics}
% \pgfplotsset{
%   exact axis/.style={grid=major, minor tick num=4, xlabel=$v^*$,
%     legend entries={PDF, CDF},},
%   every axis plot post/.append style={thick},
%   table/search
%   path={/Users/andrewwork/thesis/jump-velocity/dat-files},
%   colormap/YlGnBu,
%   cycle list/Set1-5,
%   legend style={legend cell align=left,},
% }

% \usepgfplotslibrary{external}
% \tikzexternalize

\renewcommand{\arraystretch}{1.2}
\pagestyle{empty} 
\oddsidemargin -0.25in
\evensidemargin -0.25in 
\topmargin -0.75in 
\parindent 0pt
\parskip 12pt
\textwidth 7in
%\font\cj=msbm10 at 12pt

\newcommand{\tn}{\textnormal}
\newcommand{\stiff}{\frac{k_f}{\gamma}}
\newcommand{\dd}{d}
\newcommand{\Der}[2]{\frac{\dd #1}{\dd #2}}
\newcommand{\Pder}[2]{\frac{\partial #1}{\partial #2}}
\newcommand{\Integral}[4]{\int_{#3}^{#4} {#1} \dd #2}
\newcommand{\vect}[1]{\boldsymbol{\mathbf{#1}}}
\newcommand{\mat}[1]{\underline{\underline{#1}}}
\DeclareMathOperator{\Exp}{Exp}

% Text width is 7 inches

\def\R{\mathbb{R}}
\def\N{\mathbb{N}}
\def\C{\mathbb{C}}
\def\Z{\mathbb{Z}}
\def\Q{\mathbb{Q}}
\def\H{\mathbb{H}}
\def\B{\mathcal{B}} 
%\topmargin -.5in 

%%% This is model-defs.tex
%%%
%%% Define symbols for model parameters here so that it is
%%% straightforward to change notation if necessary.

%% Coordinates
\newcommand{\wallDist}{x}
\newcommand{\ndWallDist}{z}
\newcommand{\recAngle}{\theta}
\newcommand{\dTime}{t}
\newcommand{\ndTime}{s}

%% Unknown Functions
\newcommand{\velocity}{V}
\newcommand{\rotation}{\Omega}
\newcommand{\ndVelocity}{v}
\newcommand{\ndRotation}{\omega}
\newcommand{\bondDensity}{n}
\newcommand{\ndBondDensity}{m}

%% Model Parameters (Dimensional)
\newcommand{\radius}{R}
\newcommand{\separation}{d}
\newcommand{\height}{h}
\newcommand{\length}{L}
\newcommand{\domLength}{A}
\newcommand{\shear}{\gamma}
\newcommand{\stiffness}{k_f}
\newcommand{\boltzmann}{k_B}
\newcommand{\temp}{T}
\newcommand{\onRate}{k_\tn{on}}
\newcommand{\offRate}{k_\tn{off}}
\newcommand{\onConst}{k_\tn{on}^0}
\newcommand{\offConst}{k_\tn{off}^0}
\newcommand{\refForce}{f_0}
\newcommand{\receptorDensity}{N_T}
\newcommand{\receptorNumber}{N_R}
\newcommand{\appliedVel}{V_f}
\newcommand{\appliedRot}{\Omega_f}
\newcommand{\velFriction}{\xi_V}
\newcommand{\rotFriction}{\xi_\Omega}
\newcommand{\compliance}{\Gamma}
\newcommand{\width}{w}
\newcommand{\viscosity}{\mu}

%% Force and Torque Functions
\newcommand{\horzForce}{f_h}
\newcommand{\torque}{\tau_s}
\newcommand{\horzTotalForce}{F_h}
\newcommand{\totalTorque}{\tau}

%% Force, Velocity, and Resistance Tensors
\newcommand{\forceVec}{\mathbf{F}}
\newcommand{\velVec}{\mathbf{U}}
\newcommand{\resMatrix}{\underline{\underline{R}}}

%% Model Parameters (Nondimensional)
\newcommand{\ndSeparation}{d'}
\newcommand{\ndLength}{\ell}
\newcommand{\ndAppliedRot}{\omega_f}
\newcommand{\ndAppliedVel}{v_f}
\newcommand{\ndOnConst}{\kappa}
\newcommand{\newOnConst}{\kappa_\textnormal{new}}
\newcommand{\onForceScale}{\eta}
\newcommand{\offForceScale}{\delta}
\newcommand{\ndVelFriction}{\eta_v}
\newcommand{\ndRotFriction}{\eta_\omega}

%% Nondimensional Force and Torque Functions
\newcommand{\ndHorzForce}{f_h'}
\newcommand{\ndTorque}{\tau_s'}
\newcommand{\ndHorzTotalForce}{F_h'}
\newcommand{\ndTotalTorque}{\tau'}

%% Shorthands for Chemical Species
\newcommand{\ITA}[1]{\textalpha\textsubscript{#1}}
\newcommand{\ITB}[1]{\textbeta\textsubscript{#1}}
\newcommand{\Ca}{$\tn{Ca}^{++}$}

%% Reynolds Number
\newcommand{\Reynolds}{\mathrm{Re}}

%% Bin Midpoint
\newcommand{\binMidpoint}[1]{\theta^*_{#1}}

\setcounter{secnumdepth}{2}
\begin{document}
\pagestyle{plain}

\begin{center}
  {\Large Meeting Notes (\today)}
\end{center}

\begin{itemize}
\item Investigated the singular matrix ``issue'' in one of the
  trials---the singularity was in the Stokeslets matrix. Cortez
  mentions that this matrix is not invertible in general (in the case
  of a sphere in free-space and 3D). His argument is that any
  pressure-like force acting normal to the surface of the sphere will
  produce no motion in the fluid, due to incompressibility.
\item My solution at the moment is to assume this is also true for
  our case, and so I solve for the Stokeslets strengths using a
  least-squares routine when the Stokeslets matrix is singular
\item I ran trials with randomized initial conditions, and encountered
  errors in some trials
\item I also ran trials using the Runge-Kutta integrator, and
  encountered errors in some of these runs as well. However, these
  trials ran several times faster in the experiments with high binding
  rates.
\end{itemize}

\begin{figure}[h]
  \centering
  \begin{subfigure}{0.49\textwidth}
    \includegraphics[width=\textwidth]{bd_runner03_plot5.png}
  \end{subfigure}
  \hfill
  \begin{subfigure}{0.49\textwidth}
    \includegraphics[width=\textwidth]{bd_runner09_plot2.png}
  \end{subfigure}
  \label{fig:traj03}
\end{figure}

\begin{figure}
  \centering
  \begin{subfigure}{0.49\textwidth}
    \includegraphics[width=\textwidth]{bd_runner02_plot1}
  \end{subfigure}
  \hfill
  \begin{subfigure}{0.49\textwidth}
    \includegraphics[width=\textwidth]{bd_runner10_plot7}
  \end{subfigure}  
  \label{fig:traj02}
\end{figure}

\begin{figure}
  \centering
  \begin{subfigure}{0.49\textwidth}
    \includegraphics[width=\textwidth]{bd_runner01_plot2}
  \end{subfigure}
  \hfill
  \begin{subfigure}{0.49\textwidth}
    \includegraphics[width=\textwidth]{bd_runner01_plot10}
  \end{subfigure}  
  \caption{Platelet trajectories with $k_\tn{on} = 10 \ s$. 93.8\% of
    platelets bound at some point, and 51.6\% were bound at the end of
    the simulation.}
  \label{fig:traj14}
\end{figure}

\begin{figure}
  \centering
  \begin{subfigure}{0.49\textwidth}
    \includegraphics[width=\textwidth]{steps}
  \end{subfigure}
  \hfill
 \begin{subfigure}{0.49\textwidth}
   \includegraphics[width=\textwidth]{dwells}
 \end{subfigure}
 \\
 \begin{subfigure}{0.49\textwidth}
   \includegraphics[width=\textwidth]{avg_vels}
 \end{subfigure}
  \caption{Density histograms of step sizes, dwell times, and average
    velocities. The $x$-axis ticks aren't too informative; each group
    of 4 bars show the densities for each of the 4 $k_\tn{on}$ levels
    within the same range of values.} 
  \label{fig:other-data}
\end{figure}

\begin{figure}[h]
  \centering
  \begin{subfigure}{0.49\textwidth}
    \includegraphics[width=\textwidth]{rk_plot}
  \end{subfigure}
  \hfill
  \begin{subfigure}{0.49\textwidth}
    \includegraphics[width=\textwidth]{rk_plot2}
  \end{subfigure}
  \caption{Sample platelet trajectories integrated with a Runge-Kutta
    integrator, and updating bonds at intervals of 0.002}
  \label{fig:traj03}
\end{figure}

\bibliographystyle{plain}
\bibliography{/Users/andrewwork/Documents/grad-school/thesis/library}

\end{document}




