\documentclass{article}

\newcommand{\ep}{\rule{.06in}{.1in}}
\textheight 9.5in

\usepackage{amssymb, bm}
\usepackage{amsmath}
\usepackage{amsthm}
\usepackage{graphicx, subcaption, booktabs}

\usepackage{tikz, pgfplots, pgfplotstable, chemfig, xcolor}
% \graphicspath{{/Users/andrewwork/thesis/reg_stokeslets/3d/plots}}

% \usepgfplotslibrary{colorbrewer, statistics}
% \pgfplotsset{
%   exact axis/.style={grid=major, minor tick num=4, xlabel=$v^*$,
%     legend entries={PDF, CDF},},
%   every axis plot post/.append style={thick},
%   table/search
%   path={/Users/andrewwork/thesis/jump-velocity/dat-files},
%   colormap/YlGnBu,
%   cycle list/Set1-5,
%   legend style={legend cell align=left,},
% }

% \usepgfplotslibrary{external}
% \tikzexternalize

\renewcommand{\arraystretch}{1.2}
\pagestyle{empty} 
\oddsidemargin -0.25in
\evensidemargin -0.25in 
\topmargin -0.75in 
\parindent 0pt
\parskip 12pt
\textwidth 7in
%\font\cj=msbm10 at 12pt

\newcommand{\tn}{\textnormal}
\newcommand{\stiff}{\frac{k_f}{\gamma}}
\newcommand{\dd}{d}
\newcommand{\Der}[2]{\frac{\dd #1}{\dd #2}}
\newcommand{\Pder}[2]{\frac{\partial #1}{\partial #2}}
\newcommand{\Integral}[4]{\int_{#3}^{#4} {#1} \dd #2}
\newcommand{\vect}[1]{\boldsymbol{\mathbf{#1}}}
\newcommand{\mat}[1]{\underline{\underline{#1}}}
\DeclareMathOperator{\Exp}{Exp}

% Text width is 7 inches

\def\R{\mathbb{R}}
\def\N{\mathbb{N}}
\def\C{\mathbb{C}}
\def\Z{\mathbb{Z}}
\def\Q{\mathbb{Q}}
\def\H{\mathbb{H}}
\def\B{\mathcal{B}} 
%\topmargin -.5in 

\newcommand{\radius}{R}
\newcommand{\separation}{d}
\newcommand{\stiffness}{k_f}
\newcommand{\boltzmann}{k_B}
\newcommand{\temp}{T}
\newcommand{\onConst}{k_\text{on}}
\newcommand{\offConst}{k_\text{off}}
\newcommand{\refForce}{f_0}
\newcommand{\receptorDensity}{N_T}
\newcommand{\receptorNumber}{N_R}
\newcommand{\appliedRot}{\Omega_f}
\newcommand{\appliedVel}{V_f}
\newcommand{\velFriction}{\xi_V}
\newcommand{\rotFriction}{\xi_\omega}
\newcommand{\compliance}{\Gamma}
\newcommand{\width}{w}
\newcommand{\viscosity}{\mu}

\newcommand{\ndSeparation}{d'}
\newcommand{\ndAppliedRot}{\omega_f}
\newcommand{\ndAppliedVel}{v_f}
\newcommand{\ndOnConst}{\kappa}
\newcommand{\onForceScale}{\eta}
\newcommand{\offForceScale}{\delta}
\newcommand{\ndVelFriction}{\eta_v}
\newcommand{\ndRotFriction}{\eta_\omega}

\newcommand{\ITA}[1]{\textalpha\textsubscript{#1}}
\newcommand{\ITB}[1]{\textbeta\textsubscript{#1}}

\setcounter{secnumdepth}{2}
\begin{document}
\pagestyle{plain}

\begin{center}
  {\Large Meeting Notes (\today)}
\end{center}

\textbf{Thesis Outline}
\begin{enumerate}
\item 2D Rolling Model
  \begin{itemize}
  \item Studied 2D deterministic and stochastic models of platelet
    rolling, representing a narrow reactive strip on a spherical
    platelet rolling at a constant distance above a wall.
  \item The deterministic model represents the mean behavior of the
    stochastic simulations.
  \item Using a simplification of the deterministic model where we
    assume the platelet is strictly rolling directly on top of the
    wall (i.e. the plt-wall separation is 0, and there is no slip
    between the platelet and wall surfaces), we solved for bond
    distribution at steady state.
  \item Identified regions of parameter space where stochasticity may
    be relevant (i.e. where the total number of steady-state bonds is
    between 0.1 and 10).
  \item Found switch-like behavior in the $\ndOnConst$ (unstressed on
    rate), $\ndAppliedRot$ (background fluid rotation rate; proxy for
    shear rate), and $\receptorDensity$ (receptor density)
    parameters. That is, platelet behavior switched from mostly
    unbound (at any particular time point) to mostly bound.
  \item This switch-like behavior was observed in solutions of the
    simplified steady-state equation, and in time-dependent
    simulations.
  \end{itemize}
\item Jump-Velocity Model
  \begin{itemize}
  \item Motivation: simulating individual bonds and balancing forces
    is expensive, even in the highly idealized case  of the previous
    2D model. To estimate parameters from experimental rolling data,
    we need a simpler model.
  \item Developed a jump-velocity model for a platelet with two types
    of bonds with different on/off rates, and derived a Fokker-Planck
    equation for the probability density of the position of a platelet
    along a flow channel as a function of time. We also derived the
    probability densities for (time) average velocities, step times,
    dwell times, and number of dwells in a trajectory.
  \item With the model, we can estimate effective platelet binding
    parameters from data, as well as ``capture'' and ``escape''
    rates. A separate capture rate is motivated by the experimental
    observation that platelets are traveling much faster downstream
    when they enter the domain than after the first pause, therefore
    we assume that there is some ``capture'' process distinct from the
    binding dynamics considered in the jump-velocity model. A similar
    phenomenon is observed after the final dwell, motivating a
    separate ``escape'' process.
  \item In their fibrinogen experiments the effective on rate
    increases approximately 50-fold with priming, the off rate is
    unchanged, capture rate increases slightly with priming, and
    escape rate increases with priming (although this could be an
    artifact of assuming the domain is infinite)
  \end{itemize}
\item 3D Ellipsoid Model
  \begin{itemize}
  \item Developed a full 3D model of rolling with an ellipsoidal
    platelet geometry, which considers individual bond forces, fluid
    forces, and some colloidal forces.
  \item We observed distinct free rolling behaviors in simulations
    without binding (``pole-vaulting'' vs. ``surfing'')
  \item Open questions:
    \begin{itemize}
    \item Over what range of $k_\tn{on}$ values (or \# of receptors)
      does the rolling behavior shift from primarily free rolling with
      periodic binding, to longer dwells with shorter steps?
    \item Does this model predict a ``switch'' in behaviors similar to
      what is seen in the 2D model?
    \item Can the 3D model predict the capture and escape observed in
      the experimental data? (Or, to what extent can it predict these?
      I would guess that RBCs play a role in the capture and escape
      processes, which isn't considered in this model.)
    \item Can we connect the 3D model to the jump velocity model
      somehow? I'm thinking something like using the JV model to
      estimate effective platelet binding/unbinding rates for a range
      of different receptor on rates. I think this would help unify
      the story somewhat.
    \end{itemize}
  \end{itemize}
\end{enumerate}

\textbf{3D trajectories and analysis}

\begin{itemize}
\item I got the trajectory processing code working, and plotted data
  from individual trajectories (I plotted subsets of 8 at a time, to
  try to maintain some readability), and density histograms of average
  velocity, step size, and dwell time
\end{itemize}

\begin{figure}[h]
  \centering
  \begin{subfigure}{0.49\textwidth}
    \includegraphics[width=\textwidth]{bd_runner03_plot5.png}
  \end{subfigure}
  \hfill
  \begin{subfigure}{0.49\textwidth}
    \includegraphics[width=\textwidth]{bd_runner03_plot3}
  \end{subfigure}
  \caption{Platelet trajectories with $k_\tn{on} = 1 \ s$. 15.6\% of
    platelets bound at some point, and 12.5\% were bound at the end of
    the simulation.}
  \label{fig:traj03}
\end{figure}

\begin{figure}
  \centering
  \begin{subfigure}{0.49\textwidth}
    \includegraphics[width=\textwidth]{bd_runner02_plot1}
  \end{subfigure}
  \hfill
  \begin{subfigure}{0.49\textwidth}
    \includegraphics[width=\textwidth]{bd_runner02_plot13}
  \end{subfigure}  
  \caption{Platelet trajectories with $k_\tn{on} = 5 \ s$. 79.7\% of
    platelets bound at some point, and 43.0\% were bound at the end of
  the simulation.}
  \label{fig:traj02}
\end{figure}

\begin{figure}
  \centering
  \begin{subfigure}{0.49\textwidth}
    \includegraphics[width=\textwidth]{bd_runner01_plot2}
  \end{subfigure}
  \hfill
  \begin{subfigure}{0.49\textwidth}
    \includegraphics[width=\textwidth]{bd_runner04_plot4}
  \end{subfigure}  
  \caption{Platelet trajectories with $k_\tn{on} = 10 \ s$ (left) and
    $k_\tn{on} = 25 / s$ (right). 93.8\% of platelets bound at some
    point, and 51.6\% were bound at the end of the simulation at
    $k_\tn{on} = 10 \ s$, and 100\% of platelets bound at some point,
    and 47.7\% were bound at the end at $k_\tn{on} = 25 / s$.}
  \label{fig:traj14}
\end{figure}

\begin{figure}
  \centering
  \begin{subfigure}{0.49\textwidth}
    \includegraphics[width=\textwidth]{steps}
  \end{subfigure}
  \hfill
 \begin{subfigure}{0.49\textwidth}
   \includegraphics[width=\textwidth]{dwells}
 \end{subfigure}
 \\
 \begin{subfigure}{0.49\textwidth}
   \includegraphics[width=\textwidth]{avg_vels}
 \end{subfigure}
  \caption{Density histograms of step sizes, dwell times, and average
    velocities. The $x$-axis ticks aren't too informative; each group
    of 4 bars show the densities for each of the 4 $k_\tn{on}$ levels
    within the same range of values.} 
  \label{fig:other-data}
\end{figure}

\bibliographystyle{plain}
\bibliography{/Users/andrewwork/Documents/grad-school/thesis/library}

\end{document}




