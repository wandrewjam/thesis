\documentclass{article}

\newcommand{\ep}{\rule{.06in}{.1in}}
\textheight 9.5in

\usepackage{amssymb}
\usepackage{amsmath}
\usepackage{amsthm}
\usepackage{graphicx, subcaption, algorithmic}
\graphicspath{{/Users/andrewwork/thesis/jump-velocity/plots/}}

\usepackage{tikz, pgfplots, chemfig}
\usepgfplotslibrary{colorbrewer, statistics}
\pgfplotsset{
  exact axis/.style={grid=major, minor tick num=4, xlabel=$v^*$,
    legend entries={PDF, CDF},},
  every axis plot post/.append style={thick},
  table/search
  path={/Users/andrewwork/thesis/jump-velocity/dat-files},
  colormap/YlGnBu,
  cycle list/Set1-5,
  legend style={legend cell align=left,},
}

\usepgfplotslibrary{external}
\tikzexternalize

\renewcommand{\arraystretch}{1.2}
\pagestyle{empty} 
\oddsidemargin -0.25in
\evensidemargin -0.25in 
\topmargin -0.75in 
\parindent 0pt
\parskip 12pt
\textwidth 7in
%\font\cj=msbm10 at 12pt

\newcommand{\tn}{\textnormal}
\newcommand{\stiff}{\frac{k_f}{\gamma}}
\newcommand{\dd}{d}
\newcommand{\Der}[2]{\frac{\dd #1}{\dd #2}}
\newcommand{\Pder}[2]{\frac{\partial #1}{\partial #2}}
\newcommand{\Integral}[4]{\int_{#3}^{#4} {#1} \dd #2}
\newcommand{\F}{\mathbf{F}}
\DeclareMathOperator{\Exp}{Exp}

% Text width is 7 inches

\def\R{\mathbb{R}}
\def\N{\mathbb{N}}
\def\C{\mathbb{C}}
\def\Z{\mathbb{Z}}
\def\Q{\mathbb{Q}}
\def\H{\mathbb{H}}
\def\B{\mathcal{B}} 
%\topmargin -.5in 

\setcounter{secnumdepth}{2}
\begin{document}
\pagestyle{plain}

\begin{center}
  {\Large Notes on the 3D Adhesion Dynamics Model with Regularized
    Stokeslets}
\end{center}

\section{Adhesion Dynamics Model}
\label{sec:adhes-dynam-model}

The adhesion dynamics models describe the motion of a cell in a Stokes
flow due to binding and unbinding with a ligand-coated wall. In
the simplest case, three generalized forces are considered:
$\F_H$---the hydrodynamic force on the cell, $\F_R$---the force due to
binding with ligands on the wall, and $\F_\tn{body}$---various
body forces acting on the cell (e.g. electrostatic and steric
forces). Because of Stokes flow, we assume that all of these forces
balance: $\F_H + \F_R + \F_\tn{body} = \boldsymbol{0}$.

$\F_R$ depends on the location and orientation of bonds between the
cell and wall at time $t$, and $\F_\tn{body}$ depends on the location
and orientation of the cell. The hydrodynamic force depends both on
the location and orientation of the cell $\mathbf{x}$, and also the
translational and angular velocities $\mathbf{x}'$. Therefore we can
use this relation to find $\mathbf{x}'(t)$ at each time step, and
update the position and orientation of the cell accordingly.

Right now, I am working on using regularized Stokeslets to find $\F_H$
given a position and velocity $\mathbf{x}$ and $\mathbf{x}'$.

\section{Solving Stokes flow around a rigid body using regularized
  Stokeslets}
\label{sec:solving-stokes-flow}

We want to solve Stokes equations (with a background linear shear
rate) near a plane wall, and around a rigid body $S \subset \R^3$:
\begin{align}
  &\mu \Delta \mathbf{u} - \nabla p =
    \boldsymbol{0} \label{eq:laplace} \\
  &\nabla \cdot \mathbf{u} = 0 \label{eq:cons-of-mass} \\
  &\mathbf{u}: \{\mathbf{x} \in
    \R^3 | x_3 \geq 0\} \rightarrow \R^3 \nonumber
\end{align}
with the boundary conditions
$\mathbf{u}(\mathbf{x}) = \mathbf{U} + \mathbf{x} \times
\mathbf{\Omega}$ for $\mathbf{x} \in \partial S$, and the far-field
conditions $\mathbf{u} \rightarrow \gamma x_3$ as
$\|\mathbf{x}\| \rightarrow \infty$. Here $\mathbf{U}$ and
$\mathbf{\Omega}$ are the translational and angular velocities of the
cell, respectively. Then because Stokes equations are linear, there is
a linear relationship between the body force and torque exerted by the
fluid, and $\mathbf{U}$ and $\mathbf{\Omega}$. If we define
$\mathbf{V}$ to be the $6 \times 1$ vector of generalized velocities
and $\mathbf{F}$ to be the vector of generalized forces, then
\begin{equation}
  \label{eq:resistance-eq}
  \mathbf{F} = \underline{\underline{R}} \mathbf{V} 
\end{equation}
where $\underline{\underline{R}}$ is the resistance matrix. In
practice, the resistance matrix can be found by solving equations
(\ref{eq:laplace}) and (\ref{eq:cons-of-mass}), for
$\mathbf{V} = \mathbf{e}_i$, $i = 1, \hdots, 6$.

Our goal is to use the method of regularized Stokeslets to solve the
fluid problem (\ref{eq:laplace}) and (\ref{eq:cons-of-mass}) and
derive the resistance matrix for a given platelet configuration
$\mathbf{x}$. The remaining subsections run through tests of
increasing complexity to validate our regularized Stokeslets solver.

\subsection{Prescribed force on a unit sphere, ignoring wall effects}
\label{sec:prescr-force-unit}

In the first (and simplest) test case, we want to apply a uniform
force to the surface of a unit sphere in an infinite fluid domain. We
know that in order to generate a translational velocity of
$\mathbf{U}$, we must apply a uniform (pointwise) force to the surface
of the sphere of $\frac{3\mu}{2}\mathbf{U}$.

The method of regularized stokeslets requires integration of:
\begin{equation}
  \label{eq:reg-stokeslets}
  u_j(\mathbf{x}_0) = \frac{1}{8\pi \mu} \int_{\partial D}
  S_{ij}^\epsilon (\mathbf{x}, \mathbf{x}_0) g_i(\mathbf{x}) ds(\mathbf{x}).
\end{equation}
where $\mu$ is the viscosity of the fluid, $\partial D$ is the surface
of the body (a sphere in the simplest case), $S_{ij}^\epsilon
(\mathbf{x}, \mathbf{x}_0)$ is the regularized stokeslet with blob
parameter $\epsilon$, and $g_i$ is the $i$th component of the traction
on the surface of the body.

In order to integrate this, I used the spherical quadrature rule
derived in \cite{Portelenelle2018} (basically a 2D trapezoid rule
adapted to the sphere), which is based on a 6-patch grid that divides
the sphere into 6 equal-area patches (Figure
\ref{fig:discretization}). Local coordinates are defined on each patch
to avoid singularities in the transformation between Cartesian and
surface coordinates. 

This case is handled in Cortez et. al., 2005 \cite{Cortez2005}. The
first test case they examine is to apply a point force of $\mathbf{f}
= -\frac{3\mu}{2} \mathbf{e_3}$ everywhere on the surface of the
sphere. Analytically, this generates a velocity of $\mathbf{u} =
-\mathbf{e}_3$. The error of the numerical integration of equation
(\ref{eq:reg-stokeslets}) depends both on the size of the mesh $N
\times N$, and the regularization parameter $\epsilon$. Cortez
et. al. provide plots of the L2 error while refining each of these
parameters in turn, which we can compare our integration scheme to.

As shown in Figures \ref{fig:cortez-conv} and \ref{fig:our-conv}, the
convergence results of our numerical scheme is similar to their
published results.

\begin{figure}
  \centering
  \includegraphics[width=0.5\textwidth]{cubed_sphere}
  \caption{An example of the discretization scheme on the surface of
    the sphere (from \cite{Portelenelle2018})}
  \label{fig:discretization}
\end{figure}

\begin{figure}
  \centering
  \begin{subfigure}{0.6\textwidth}
    \includegraphics[width=\textwidth]{reg_refine_cortez}
  \end{subfigure}
  \hfill
  \begin{subfigure}{0.35\textwidth}
    \includegraphics[width=\textwidth]{grid_refine_cortez}
  \end{subfigure}
  \caption{Convergence in the $\Delta s$ and $\epsilon$ parameters
    published in \cite{Cortez2005}.}
  \label{fig:cortez-conv}
\end{figure}

\begin{figure}
  \centering
  \begin{subfigure}{0.48\textwidth}
    \includegraphics[width=\textwidth]{reg_refine}
  \end{subfigure}
  \hfill
  \begin{subfigure}{0.48\textwidth}
    \includegraphics[width=\textwidth]{grid_refine}
  \end{subfigure}
  \caption{Convergence in the $\Delta s$ and $\epsilon$ parameters in
    our numerical scheme.}
  \label{fig:our-conv}
\end{figure}

\subsection{Solving for the force generated by a prescribed velocity}
\label{sec:solv-force-gener}

The next test is to verify that the method of regularized Stokeslets
can find the fluid force generated by prescribed translational
$\mathbf{U}$ and angular $\mathbf{\Omega}$ velocities. Note that this
is ultimately the problem we need to solve in the adhesion dynamics
models, where we want to solve for the body force given boundary
conditions on $\mathbf{u}$ on $\partial S$.

With the quadrature rule defined in the previous section, equation
(\ref{eq:reg-stokeslets}) is approximated with
\begin{equation}
  \label{eq:approx-stokeslets}
  u_{m, j} = \frac{1}{8\pi \mu} \sum_{n=1}^{N_\tn{nodes}} \sum_{i=1}^3
  S_{ij}^\epsilon(\mathbf{x}_n, \mathbf{x}_m) g_{n,i} w_n,
\end{equation}
where $j = 1, 2, 3$ and $m = 1, \hdots, N_\tn{nodes}$. This can be
written as the product of a matrix and vector:
\begin{equation}
  \label{eq:quadrature-matrix}
  \hat{\mathbf{u}} = \mathcal{A} \hat{\mathbf{g}}
\end{equation}
where $\hat{\mathbf{u}}$ and $\hat{\mathbf{g}}$ are
$3N_\tn{nodes} \times 1$ vectors of the 3 components of the velocity
and traction at each node. Then $\hat{\mathbf{g}}$ for some prescribed
$\hat{\mathbf{u}}$ is given by inverting $\mathcal{A}$. Then the body
force and torque can be found by integrating the following functions
over the sphere:
\begin{align}
  \mathbf{f} &= \int_{\partial S} \mathbf{g}(\mathbf{x}) d\mathbf{x}
  \\
  \mathbf{l} &= \int_{\partial S} \mathbf{x} \times
               \mathbf{g}(\mathbf{x}) d\mathbf{x}
\end{align}

As a first check that we are assembling the matrix $\mathcal{A}$
correctly, we can set the velocity at each node equal to $(0, 0,
1)^T$, and we should get a body force of $6\pi \mu \mathbf{e}_3$. The
convergence in the estimate of the body force is shown in Figure
\ref{fig:force_error}. 

\begin{figure}
  \centering
  \includegraphics[width=0.5\textwidth]{force_error}
  \caption{Absolute error in the body force $\mathbf{f}$ with a
    prescribed velocity of $(0, 0, 1)^T$}
  \label{fig:force_error}
\end{figure}

\subsubsection{Estimation of resistance matrix of a sphere}
\label{sec:estim-resist-matr}

The next test problem performed in \cite{Cortez2005} is estimating the
resistance matrix for a sphere with a 0 background flow. To do this,
we really just have to repeat the process outlined above for pointwise
velocities $\mathbf{e}_i$ for $i = 1, 2, 3$ and $\mathbf{x} \times
\mathbf{e}_i$ for $i = 1, 2, 3$.

That is, we are solving Stokes equations with boundary conditions
$\mathbf{u}(\mathbf{x}) = \mathbf{U} + \mathbf{x} \times
\mathbf{\Omega}$, $\mathbf{x} \in \partial S$ for $\mathbf{U} =
\mathbf{e}_i$ and $\mathbf{\Omega} = \mathbf{e}_i$, and with far-field
conditions $\mathbf{u} \rightarrow \boldsymbol{0}$ as $\|\mathbf{x}\|
\rightarrow \infty$. Then we can integrate the pointwise forces to get
the body force and torques, which gives successive columns of
$\underline{\underline{R}}$.

Because of the superposition principle, the flow around a sphere with
a background linear shear flow can be decomposed into the flow around
a moving sphere with a background stationary flow (say $\mathbf{u}_0$)
and the flow around a stationary sphere with a background shear flow
(say $\mathbf{u}_\gamma$). Therefore to find the forces on a sphere
due to a background shear flow, we only need to solve one additional
Stokes problem.

First, we want to verify the resistance matrix generated by motion of
a sphere through a stationary background flow. Analytically,
\begin{equation}
  \underline{\underline{R}} =
  \begin{pmatrix}
    \mathcal{T} & 0 \\
    0 & \mathcal{R}
  \end{pmatrix} \quad \tn{where} \quad \mathcal{T} = 6\pi \mu
  \underline{\underline{I}}, \, \mathcal{R} = 8 \pi \mu
  \underline{\underline{I}}.
\end{equation}

Tables \ref{tab:t-matrix} and \ref{tab:r-matrix} show the computed
diagonal entries for $\mathcal{T}$ and $\mathcal{R}$. The values I
computed are within 0.1 of the computed values published in
\cite{Cortez2005} in all cases.

\begin{table}
  \centering
  \begin{tabular}{ccc}
    $\epsilon$ & $N = 12$ & $N = 24$ \\ \hline
    $0.1$ & 19.37 & 19.39 \\
    $0.05$ & 18.92 & 19.09 \\
    $0.01$ & 16.59 & 18.39 \\
    Exact & 18.85 & 18.85 \\ \hline
  \end{tabular}
  \caption{Computed diagonal entries of the matrix $\mathcal{T}$ for
    different regularization and discretization parameters}
  \label{tab:t-matrix}
\end{table}

\begin{table}
  \centering
  \begin{tabular}{ccc}
    $\epsilon$ & $N = 12$ & $N = 24$ \\ \hline
    $0.1$ & 27.12 & 27.16 \\
    $0.05$ & 25.65 & 26.09 \\
    $0.01$ & 19.89 & 24.06 \\
    Exact & 25.13 & 25.13 \\ \hline
  \end{tabular}
  \caption{Computed diagonal entries of the matrix $\mathcal{R}$ for
    different regularization and discretization parameters}
  \label{tab:r-matrix}
\end{table}

In addition to the resistance matrix, I estimated the vector of body
forces and torques exerted by a shear flow with $\gamma =
1$. With $\mathbf{u}^\infty = \gamma x_3$, all forces and torques on the
sphere are 0, except $\Omega_1 = -4 \pi \mu \gamma$. The estimated
nonzero component of the shear force vector for different
regularization and discretization parameters are given in Table

\begin{table}
  \centering
  \begin{tabular}{ccc}
    $\epsilon$ & $N = 12$ & $N = 24$ \\ \hline
    $0.1$ & -13.56 & -13.58 \\
    $0.05$ & -12.83 & -13.05 \\
    $0.01$ & -9.95 & -12.03 \\
    Exact & -12.57 & -12.57 \\ \hline
  \end{tabular}
  \caption{Computed diagonal entries of the shear force vector
    $\mathbf{F}_\gamma$ for different regularization and
    discretization parameters}
  \label{tab:s-vector}
\end{table}

\bibliographystyle{plain}
\bibliography{/Users/andrewwork/Documents/grad-school/thesis/library}

\end{document}




