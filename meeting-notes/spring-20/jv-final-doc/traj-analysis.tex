\documentclass{article}

\newcommand{\ep}{\rule{.06in}{.1in}}
\textheight 9.5in

\usepackage{amssymb}
\usepackage{amsmath}
\usepackage{amsthm}
\usepackage{graphicx, subcaption, algorithmic}
\graphicspath{{/Users/andrewwork/thesis/jump-velocity/plots/}}

\usepackage{tikz, pgfplots, chemfig}
% \setchemfig{scheme debug=true}

\usepgfplotslibrary{colorbrewer, statistics}
\pgfplotsset{
  exact axis/.style={grid=major, minor tick num=4, xlabel=$v^*$,
    legend entries={PDF, CDF},},
  every axis plot post/.append style={thick},
  table/search
  path={/Users/andrewwork/thesis/jump-velocity/dat-files},
  colormap/YlGnBu,
  cycle list/Set1-5,
  legend style={legend cell align=left,},
}
\usepgfplotslibrary{external}
\tikzexternalize

\renewcommand{\arraystretch}{1.2}
\pagestyle{empty} 
\oddsidemargin -0.25in
\evensidemargin -0.25in 
\topmargin -0.75in 
\parindent 0pt
\parskip 12pt
\textwidth 7in
%\font\cj=msbm10 at 12pt

\newcommand{\tn}{\textnormal}
\newcommand{\stiff}{\frac{k_f}{\gamma}}
\newcommand{\dd}{d}
\newcommand{\Der}[2]{\frac{\dd #1}{\dd #2}}
\newcommand{\Pder}[2]{\frac{\partial #1}{\partial #2}}
\newcommand{\Integral}[4]{\int_{#3}^{#4} {#1} \dd #2}
\DeclareMathOperator{\Exp}{Exp}

% Text width is 7 inches

\def\R{\mathbb{R}}
\def\N{\mathbb{N}}
\def\C{\mathbb{C}}
\def\Z{\mathbb{Z}}
\def\Q{\mathbb{Q}}
\def\H{\mathbb{H}}
\def\B{\mathcal{B}} 
%\topmargin -.5in 

\setcounter{secnumdepth}{2}
\begin{document}
\pagestyle{plain}

\begin{center}
  {\Large Notes on Platelet Trajectories and Average Velocities in the
    Jump-Velocity Model (\today)}
\end{center}

\section{A discrepancy between observed average velocities and
  simulated average velocities}
\label{sec:discr-betw-observ}

\tikzexternaldisable
\begin{figure}
  \centering
  \schemestart
  U \arrow{<=>[$\alpha$][$\beta$]}[-135]
  B \arrow{<=>[*{0}$\delta$][*{0}$\gamma$]}[-45]
  BF \arrow{->[*{0}$\eta$]}[45] T \arrow{->[$\Gamma(k, \theta)$]}[-180]
  \schemestop  
  \caption{State diagram of the modified jump-velocity model.}
  \label{fig:modified-jv-model}
\end{figure}
\tikzexternalenable

In my previous notes, I fit the models of step size and dwell time
distributions to the experimental data from each of the 4 set ups. I
did a sequential fit, where first I fit the model of step sizes, then
used the resulting parameters to do a constrained optimization fitting
the dwell times. This specifies all parameters in the jump-velocity
model, except for the free-flowing velocity $V^*$. For now, I take
this to be the maximum observed average velocity, either for each
experiment separately (i.e. there is a different $V^*$ for each data
set) or for all experiments together (i.e. the same $V^*$ for every
data set).

As shown in Figure \ref{fig:CC-whole-model-fit}, the predicted average
velocity distribution is not a good fit for the observed average
velocity distribution, in particular the model predicts average
velocities much lower than those observed. (I have only shown plots
for the collagen-collagen PRP experiment, but the results for the
other experiments are qualitatively similar). We were concerned this
discrepancy was due to some scaling error in transforming between
nondimensional and dimensional parameter space, so I ran two different
parameter estimation processes with different orders of scaling and
estimating:
\begin{enumerate}
\item In the first procedure, I estimated dimensional parameters from
  dimensional data, and then nondimensionalized those parameters to
  simulate the average velocities. Here the order of scaling and
  fitting goes:
  \begin{enumerate}
  \item Fit parameters $\hat{\beta}$, $\chi$ (the probability of
    taking an unbound step), $k$, and $\hat{\theta}$ (parameters of
    the bound step distribution) to dimensional step size data. The
    relation between the dimensional (hatted) parameters and the
    nondimensional (unhatted) parameters are
    $\hat{\beta} = \beta / L$, and $\hat{\theta} = \theta / L$. $\chi$
    and $k$ are already nondimensional.
  \item Fit the other parameters $\hat{\alpha}$, $\hat{\gamma}$,
    $\hat{\delta}$, and $\hat{\eta}$ to the dimensional dwell time
    data. $\hat{\alpha} = V \alpha / L$ and similar for the other
    parameters.
  \item Use the nondimensional parameters to generate nondimensional
    average velocities $\{v_i\}_{i=1}^{N}$, and then multiply each
    $v_i$ by $V^*$ to get dimensional average velocities.
  \end{enumerate}
\item In the second procedure, I nondimensionalized the data first,
  and then estimated nondimensional parameters:
  \begin{enumerate}
  \item Nondimensionalize step size data by dividing by $L$, and
    nondimensionalize dwell time data by dividing by $T = L/V$.
  \item Fit nondimensional parameters $\beta$, $\chi$, $k$, and
    $\theta$ to the step size data, and $\alpha$, $\gamma$, $\delta$,
    and $\eta$ to the dwell time data. 
  \item Generate nondimensional average velocities, and multiply by
    $V^*$ to get dimensional average velocities.
  \end{enumerate}
\end{enumerate}

Both of these procedures generated the same distribution of average
velocities, decreasing the likelihood that the difference between the
simulations and data is due to a coding error in scaling the
data/parameters. However, I still wasn't convinced that my simulated
velocities were correct (within the model), so I tried to think about
what kind of average velocities we would expect given the
distributions of steps and dwells below.

Here is one (very) crude estimate of the population average of the
average velocities:
\begin{align*}
  \tn{time to cross} &= \frac{L}{V^*} + \mathtt{num\_dwells} \cdot
                       \mathtt{avg\_dwells} \\
  \mathtt{num\_dwells} &\approx \frac{L}{\mathtt{avg\_step}} \\
  \implies \tn{time to cross} &\approx \frac{L}{V^*} + \frac{L
                                \mathtt{avg\_dwell}}{\mathtt{avg\_step}}
  \\
  \implies \tn{velocity} &= \frac{L}{\tn{time to cross}} =
                           \frac{1}{\frac{1}{V^*} +
                           \frac{\mathtt{avg\_dwell}}{\mathtt{avg\_step}}}.
\end{align*}

In Figure \ref{fig:CC-whole-model-fit}, the average step size seems to
be roughly $3 \,\mu\tn{m}$ and the average dwell time is roughly
$3 \,\tn{s}$. Taking $V^* \approx 50 \, \mu\tn{m}/\tn{s}$ gives us
approximately $1.02 \, \mu\tn{m}/\tn{s}$ as a rough estimate for the
expected population average of average velocities.

This is obviously only a rough estimate, but we could also estimate
average velocities by ``bootstrapping'' the step size and dwell time
data. Therefore I wrote a program to alternate picking from the
observed step sizes and dwell times to bootstrap a distribution of
expected average velocities. Specifically, I run the following
algorithm for each simulated platelet:
\begin{enumerate}
\item Specify a domain length $L = 100 \, \mu\tn{m}$, and initialize
  the starting position and starting time of the platelet to $0$.
\item While the position of the platelet is less than $L$, do the
  following
  \begin{enumerate}
  \item Pick a step size $dy$ from the observed step sizes (with
    replacement), and increase the position by $dy$ and the time by
    $dy / V$
  \item Then pick a dwell time $dt$ from the observed dwell times
    (with replacement) and increase the time by $dt$. The position
    does not change in a dwell.
  \end{enumerate}
\item Find the time $T_i$ at which the platelet's position $= L$, and
  set the average velocity to be $v_i = L/T_i$.
\end{enumerate}

The result of this bootstrap process is compared to the average
velocity distribution generated by the jump-velocity model and the
observed average velocity distribution in Figure
\ref{fig:ccp-vel-bootstrap}. The bootstrap process generates an
average velocity distribution very similar to the simulated average
velocity distribution for the collagen-collagen PRP data, with similar
results for the other data sets. All of this evidence together I think
makes a strong case that the difference between the simulated and
observed average velocities in Figure \ref{fig:CC-whole-model-fit} is
not due to a coding error.

From looking at Figure \ref{fig:CC-whole-model-fit}, an obvious
question to ask is: how can the step size and dwell time distributions
match so well with the data, yet result in average velocities which
are totally different than those observed? 

\begin{figure}
  \centering
  \begin{subfigure}{0.48 \textwidth}
    \begin{tikzpicture}
      \begin{axis}[
        legend entries={Data, Model},
        legend pos = north east,
        xlabel = Step size ($\mu$m),
        ylabel = Probability Density,
        ]
        \addplot+[hist=density, fill] table [y index=0]
        {simulations/CC-step.dat};
        \addplot table [x index=0, y index=1]
        {ml-estimates/CC-step-dst.dat};
      \end{axis}
    \end{tikzpicture}
  \end{subfigure}
  \hfill
  \begin{subfigure}{0.48\textwidth}
    \begin{tikzpicture}
      \begin{axis}[
        legend entries={Data, Model},
        legend pos = south east,
        xlabel = Step size ($\mu$m),
        ylabel = Cumulative Probability,
        ]
        \addplot+[const plot] table[x index=0, y expr=(\coordindex+1)/40]
        {simulations/CC-step.dat};
        \addplot table [x index=0, y index=2]
        {ml-estimates/CC-step-dst.dat};
      \end{axis}
    \end{tikzpicture}
  \end{subfigure}
  \\
  \begin{subfigure}{0.48 \textwidth}
    \begin{tikzpicture}
      \begin{axis}[
        legend entries={Data, Model},
        legend pos = north east,
        xlabel = Pause time (s),
        ylabel = Probability Density,
        ]
        \addplot+[hist=density, fill] table [y index=0]
        {simulations/CC-dwell.dat};
        \addplot table [x index=0, y index=1]
        {ml-estimates/CC-dwell-dst.dat};
      \end{axis}
    \end{tikzpicture}
  \end{subfigure}
  \hfill
  \begin{subfigure}{0.48\textwidth}
    \begin{tikzpicture}
      \begin{axis}[
        legend entries={Data, Model},
        legend pos = south east,
        xlabel = Pause time (s),
        ylabel = Cumulative Probability,
        ]
        \addplot+[const plot] table[x index=0, y expr=(\coordindex+1)/54]
        {simulations/CC-dwell.dat};
        \addplot table [x index=0, y index=2]
        {ml-estimates/CC-dwell-dst.dat};
      \end{axis}
    \end{tikzpicture}
  \end{subfigure}
  \\
  \begin{subfigure}{0.48 \textwidth}
    \begin{tikzpicture}
      \begin{axis}[
        legend entries={Data, Model},
        legend pos = north east,
        xlabel = Average velocity,
        ylabel = Probability Density,
        fill opacity = 0.75,
        ]
        \addplot+[hist=density, fill] table [y index=0]
        {simulations/CC-vel.dat};
        \addplot+[hist=density, fill] table [y expr=\thisrowno{0}*10]
        {simulations/CCsamp-gamma-sim.dat};
      \end{axis}
    \end{tikzpicture}
  \end{subfigure}
  \hfill
  \begin{subfigure}{0.48\textwidth}
    \begin{tikzpicture}
      \begin{axis}[
        legend entries={Data, Model},
        legend pos = south east,
        xlabel = Average velocity,
        ylabel = Cumulative Probability,
        ]
        \addplot+[const plot] table [x index=0, y
        expr=(\coordindex+1)/49] {simulations/CC-vel.dat};
        \addplot+[const plot] table [x expr=\thisrowno{0}*10, y
        expr=(\coordindex+1)/1024]
        {simulations/CCsamp-gamma-sim.dat};
      \end{axis}
    \end{tikzpicture}
  \end{subfigure}
  \caption{Fit of step size and dwell time distributions to CC PRP
    data $\chi = 0.27$, $\beta = 0.16$, $k = 2.13$, $\theta = 0.54$,
    $\alpha = 0.37$, $\gamma = 0.08$, $\delta = 1.25$, $\eta =
    0.37$. $V^* = 53$}
  \label{fig:CC-whole-model-fit}
\end{figure}

\begin{figure}
  \centering
  \includegraphics[width=\textwidth]{ccp-vel-bootstrap.png}
  \caption{Comparison of observed average velocities with simulated
    average velocities from the model, and bootstrapped average
    velocities from step and dwell data. $N = 1024$ for both the
    number of simulations, and the number of bootstrap trials.}
  \label{fig:ccp-vel-bootstrap}
\end{figure}

\section{An examination of the trajectory data}
\label{sec:an-exam-traj}

To try to find the source (or sources) of the difference in average
velocity data, I loaded the raw trajectory data into Python. The first
thing I looked at was comparing intermediate steps (i.e. those that
occur between two dwells) with the first and last steps (neither of
which are not included in the step size data, but are included in the
average velocity).

Boxplots of each of these populations of step sizes are shown in
Figure \ref{fig:ccp-step-cmp}. The steps that occur in between dwells
are much smaller than those at the beginning and end, therefore our
assumption in the jump-velocity model that the beginning and ending
steps are the same is violated. Similarly, the velocity of the
platelet within a step is different depending on whether the step is
in between dwells, or at the beginning or end of a trajectory (Figure
\ref{fig:ccp-vel-cmp}).

To eliminate this source of error, I recalculated average velocities
from the observed trajectories by only taking the average velocity
between the start of the first dwell and the end of the last
dwell. Similarly, I ran simulations of the jump-velocity model (using
parameters fit to step and dwell data), and calculated average
velocities in the same way. The results of this are shown in Figure
\ref{fig:ccp-vel-interdwell}. There is much better agreement between
the model and average velocities under the new definition, however
there is still a noticeable difference between the simulations and
data. In particular, the simulated distribution of average velocity
has a much smaller spread than the velocity data.

Another note from their data is that the velocity within steps (even
between dwells) is not constant, or even particularly close to
constant. Figure \ref{fig:step_vs_avg_vels} shows the distribution of
velocities during steps in between dwells compared to the distribution
of average velocities between the first and last dwells.

The temporal resolution of their data is only about 100 ms, so we
wanted to know the effect of this error on our parameter estimates. I
perturbed the dwell data with a random error of $N(0, 0.1 V / L)$ (the
standard deviation $0.1V/L$ is 100 ms in nondimensional units) and
then ran the fitting process on the perturbed data. Figure
\ref{fig:dwell_perturb} shows the distribution of resulting estimates
of $\alpha$ (the single-bond off rate) and $\gamma$ (the second bond
off rate). The perturbations of the dwell data seem to carry through
to the parameter estimates in a fairly linear way, as this
distribution of estimates looks pretty normal.

I ran the same process on the step data, using perturbations with a
standard deviation of $0.1 \mu m$. The resulting distribution of
parameter estimates was far less normal than with the dwell
perturbations. 

\begin{figure}
  \centering
  \includegraphics[width=0.6\textwidth]{ccp-step-cmp.png}
  \caption{Comparison of intermediate step sizes with the first and
    last steps of the experiment. Data shown is from the
    collagen-collagen PRP experiment, but other experiments have
    qualitatively similar results.}
  \label{fig:ccp-step-cmp}
\end{figure}

\begin{figure}
  \centering
  \includegraphics[width=0.6\textwidth]{ccp-vel-cmp.png}
  \caption{Comparison of intermediate free velocities with the first
    and last steps of the experiment. Again, the data shown is from
    the collagen-collagen PRP experiment, but the others are similar.}
  \label{fig:ccp-vel-cmp}
\end{figure}

\begin{figure}
  \centering
  \includegraphics[width=\textwidth]{ccp-vel-interdwell.png}
  \caption{Comparison of simulated velocities with observed average
    velocities measured from the start of the first dwell to the end
    of the last dwell. Data shown is for the collagen-collagen PRP
    experiment.}
  \label{fig:ccp-vel-interdwell}
\end{figure}

\begin{figure}
  \centering
  \includegraphics[width=0.6\textwidth]{step_vs_avg_vels.png}
  \caption{Comparison of the velocity during steps in between dwells
    and the average velocity of rolling between the (start of the) first dwell and
    the (end of the) last dwell. Data shown is for the
    collagen-collagen PRP experiment.}
  \label{fig:step_vs_avg_vels}
\end{figure}

\begin{figure}
  \centering
  \begin{subfigure}{0.48\textwidth}
    \includegraphics[width=\textwidth]{dwell_perturb_alpha.png}
    % \label{fig:dwell_perturb_alpha}
  \end{subfigure}
  \hfill
  \begin{subfigure}{0.48\textwidth}
    \includegraphics[width=\textwidth]{dwell_perturb_gamma.png}
    % \label{fig:dwell_perturb_gamma}
  \end{subfigure}  
  \caption{Results of fitting the dwell distribution to 1024 different
    perturbations of the dwell data. The vertical black lines indicate
    the parameter estimate from the original data set. Data is taken
    from the CC PRP experiment, but other experiments are similar.}
  \label{fig:dwell_perturb}
\end{figure}

\begin{figure}
  \centering
  \begin{subfigure}{0.48\textwidth}
    \includegraphics[width=\textwidth]{step_perturb_chi.png}
  \end{subfigure}
  \hfill
  \begin{subfigure}{0.48\textwidth}
    \includegraphics[width=\textwidth]{step_perturb_beta.png}
  \end{subfigure}
  \\
  \begin{subfigure}{0.48\textwidth}
    \includegraphics[width=\textwidth]{step_perturb_k.png}
  \end{subfigure}
  \hfill
  \begin{subfigure}{0.48\textwidth}
    \includegraphics[width=\textwidth]{step_perturb_theta.png}
  \end{subfigure}
  \caption{Results of fitting the step distribution to 1024 different
    perturbations of the step data. Data is taken from the CC PRP
    experiment.}
  \label{fig:step_perturb}
\end{figure}

\section{Correlations in experimental data}
\label{sec:corr-exper-data}

To look for correlations in the experimental data, I generated scatter
plots comparing data that we thought \emph{might} be correlated. I
also calculated the Pearson correlation coefficient, and the Spearman
rank correlation coefficient (a nonparametric correlation measure),
along with the associated $p$-values testing the null hypothesis that
the correlation is 0). 

\begin{figure}
  \centering
  \includegraphics[width=\textwidth]{ccp-correlations.png}
  \caption{Various scatter plots from the CC PRP data. The only plot
    where both correlation measures found $p$-values < 0.05 was the
    Step Time vs. Step Velocity correlation.}
  \label{fig:ccp-correlations}
\end{figure}

\begin{figure}
  \centering
  \includegraphics[width=\textwidth]{hcp-correlations.png}
  \caption{Various scatter plots from the HC PRP data. The only plot
    where both correlation measures found $p$-values < 0.05 was the
    Step Time vs. Step Velocity correlation.}
  \label{fig:hcp-correlations}
\end{figure}

\begin{figure}
  \centering
  \includegraphics[width=\textwidth]{ccw-correlations.png}
  \caption{Various scatter plots from the CC whole blood data. None of
  the pairwise comparisons resulted in a $p$-value < 0.05 for both of
  the tests.}
  \label{fig:ccw-correlations}
\end{figure}

\begin{figure}
  \centering
  \includegraphics[width=\textwidth]{hcw-correlations.png}
  \caption{Various scatter plots from the HC whole blood data. None of
  the pairwise comparisons resulted in a $p$-value < 0.05 for both of
  the tests.}
  \label{fig:hcw-correlations}
\end{figure}

\begin{figure}
  \centering
  \begin{subfigure}{0.48\textwidth}
    \includegraphics[width=\textwidth]{ccp-initial-step.png}
    \caption{CC PRP: $k_\tn{capture} = 0.52 / s$}
  \end{subfigure}
  \hfill
  \begin{subfigure}{0.48\textwidth}
    \includegraphics[width=\textwidth]{hcp-initial-step.png}
    \caption{HC PRP: $k_\tn{capture} = 0.63 / s$}
  \end{subfigure}
  \\
  \begin{subfigure}{0.48\textwidth}
    \includegraphics[width=\textwidth]{ccw-initial-step.png}
    \caption{CC whole blood: $k_\tn{capture} = 1.59 / s$}
  \end{subfigure}
  \hfill
  \begin{subfigure}{0.48\textwidth}
    \includegraphics[width=\textwidth]{hcw-initial-step.png}
    \caption{HC whole blood: $k_\tn{capture} = 1.12 / s$}
  \end{subfigure}
  \caption{Histograms of the initial step time. The two experiments
    with whole blood have faster ``capture'' rates than the two
    experiments in PRP.}
  \label{fig:initial-step}
\end{figure}

\begin{figure}
  \centering
  \begin{subfigure}{\textwidth}
    \includegraphics[width=\textwidth]{ccp-traj-cmp60.png}
  \end{subfigure}
  \\
  \begin{subfigure}{\textwidth}
    \includegraphics[width=\textwidth]{ccp-traj-cmp10.png}
  \end{subfigure}  
  \caption{Comparison of model fits with $V = 60$(top) and
    $V = 10$(bottom). Data is from the CC PRP experiment.}
  \label{fig:ccp-traj-cmp}
\end{figure}

\section{Comparison of free velocities with adiabatic reduction model}
\label{sec:comp-free-veloc}

My idea was to compare the observed instantaneous velocities with the
instantaneous velocities predicted by the adiabatic reduction
model. However, I don't think the instantaneous velocity in the
reduced model can be defined (since it is a scaled diffusion
process). So instead I cut out the dwells from the experimental
trajectories and computed the average ``free'' velocities and compared
those with the reduced model.

Figure \ref{fig:avg-free-vel} shows the distributions of free
velocities with two different fits of the reduced model.

The function I am fitting to the data is $u(t, y) =
\frac{1}{\sqrt{4\pi \epsilon a b t}} \left(a + \frac{y -
    at}{2t}\right) \exp \left[\frac{-(y-at)^2}{4\epsilon a b t}
\right]$, which is the result of the two-state adiabatic reduction. In
the 4 state model, if you turn off binding and unbinding of the
``slow'' receptor, it reduces to the two state model.

\begin{figure}
  \centering
  \includegraphics[width=\textwidth]{avg_free_velocity.png}
  \caption{Comparison of the average free velocity in four
    experimental conditions. For Fit 1, $a = 0.12$ and $\epsilon =
    0.65$. For Fit 2, $a = 0.13$ and $\epsilon = 0.43$.}
  \label{fig:avg-free-vel}
\end{figure}

\bibliographystyle{plain}
\bibliography{/Users/andrewwork/Documents/grad-school/thesis/library}

\end{document}




