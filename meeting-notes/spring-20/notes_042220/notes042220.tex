\documentclass{article}

\newcommand{\ep}{\rule{.06in}{.1in}}
\textheight 9.5in

\usepackage{amssymb}
\usepackage{amsmath}
\usepackage{amsthm}
\usepackage{graphicx, subcaption, algorithmic}
% \graphicspath{{/Users/andrewwork/thesis/jump-velocity/plots/}}

\usepackage{tikz, pgfplots, pgfplotstable, chemfig}
% \usepgfplotslibrary{colorbrewer, statistics}
% \pgfplotsset{
%   exact axis/.style={grid=major, minor tick num=4, xlabel=$v^*$,
%     legend entries={PDF, CDF},},
%   every axis plot post/.append style={thick},
%   table/search
%   path={/Users/andrewwork/thesis/jump-velocity/dat-files},
%   colormap/YlGnBu,
%   cycle list/Set1-5,
%   legend style={legend cell align=left,},
% }

% \usepgfplotslibrary{external}
% \tikzexternalize

\renewcommand{\arraystretch}{1.2}
\pagestyle{empty} 
\oddsidemargin -0.25in
\evensidemargin -0.25in 
\topmargin -0.75in 
\parindent 0pt
\parskip 12pt
\textwidth 7in
%\font\cj=msbm10 at 12pt

\newcommand{\tn}{\textnormal}
\newcommand{\stiff}{\frac{k_f}{\gamma}}
\newcommand{\dd}{d}
\newcommand{\Der}[2]{\frac{\dd #1}{\dd #2}}
\newcommand{\Pder}[2]{\frac{\partial #1}{\partial #2}}
\newcommand{\Integral}[4]{\int_{#3}^{#4} {#1} \dd #2}
\DeclareMathOperator{\Exp}{Exp}

% Text width is 7 inches

\def\R{\mathbb{R}}
\def\N{\mathbb{N}}
\def\C{\mathbb{C}}
\def\Z{\mathbb{Z}}
\def\Q{\mathbb{Q}}
\def\H{\mathbb{H}}
\def\B{\mathcal{B}} 
%\topmargin -.5in 

\setcounter{secnumdepth}{2}
\begin{document}
\pagestyle{plain}

\begin{center}
  {\Large Meeting Notes (\today)}
\end{center}

\large{\textbf{Regularized Stokeslets}}

\begin{itemize}
\item Repeat the convergence tests of \cite{Ainley2008} on a sphere
  moving near a bounded wall
\item I still have bugs in my code, because my results don't make sense:
  \begin{table}[h]
    \centering
    \pgfplotstabletypeset[
    col sep=comma,
    every head row/.style={after row=\hline},
    ]{par_table.dat}
    \caption{Nondimensional force required to move a sphere parallel
      to the wall}
    \label{tab:par-table}
  \end{table}

  \begin{table}[h]
    \centering
    \pgfplotstabletypeset[
    col sep=comma,
    every head row/.style={after row=\hline},
    ]{prp_table.dat}
    \caption{Nondimensional force required to move a sphere
      perpendicular to the wall}
    \label{tab:prp-table}
  \end{table}
\item Still to do:
  \begin{itemize}
  \item Search for work that uses regularized Stokeslets with
    different blob parameters
  \item Estimate errors on the ellipsoid with different blob
    parameters, mesh sizes, and height and orientations
  \end{itemize}
\end{itemize}

\large{\textbf{Jump-Velocity Model}}

\begin{figure}[h]
  \centering
  \schemestart
  $U$ \arrow(u1--vv){<=>[$k_\tn{on}$][$k_\tn{off}$]} $V$
  \arrow(@u1--ff){<=>[*{0}$k_\tn{on}^F$][*{0}$k_\tn{off}^F$]}[-90] $F$
  \arrow(--vf){<=>[$k_\tn{on}$][$k_\tn{off}$]} $VF$
  \arrow(@vv--@vf){<=>[*{0}$k_\tn{on}^F$][*{0}$k_\tn{off}^F$]}
  \arrow(@u1--ww){->[*{0}$k_\tn{escape}$]}[90]
  \arrow(@vv--vw){->[*{0}$k_\tn{escape}$]}[90]
  \schemestop
  \caption{Jump velocity model with escape}
  \label{fig:escape-diagram}
\end{figure}
  
\begin{itemize}
\item One prediction from the jump-velocity model with escape is that
  the number of dwells in a trajectory should be geometrically
  distributed. Basically, the probability that a platelet re-binds
  after unbinding is $a = \frac{k_\tn{on}}{k_\tn{on} + k_\tn{escape}}$,
  and so the number of binding events that occur before the platelet
  escapes is a geometric distribution ($p(n) = (1 - a) a^{n-1}$).
\item Still working on finding out the distribution of moving times in
  the jump-velocity model with escape. I think the moving time $M$
  should be the product of the distribution of number of steps, with
  the distribution of step times.
\item I still need to test this and derive the distribution of $M$,
  and then we can identify the $k_\tn{on}$ and $k_\tn{escape}$
  parameters.
\item Still to do:
  \begin{itemize}
  \item Figure out if it is tractable to write a distribution of
    effective platelet off rates as a function of individual receptor
    on/off rates.
  \end{itemize}

\end{itemize}

% \begin{figure}[h]
%   \centering
%   \schemestart
%   $U$ \arrow(u1--vv){<=>[$k_\tn{on}$][$k_\tn{off}$]} $V$
%   \arrow(@u1--ff){<=>[*{0}$k_\tn{on}^F$][*{0}$k_\tn{off}^F$]}[-90] $F$
%   \arrow(--vf){<=>[$k_\tn{on}$][$k_\tn{off}$]} $VF$
%   \arrow(@vv--@vf){<=>[*{0}$k_\tn{on}^F$][*{0}$k_\tn{off}^F$]}
%   \schemestop
%   \label{fig:primed-states}
% \end{figure}

% \begin{itemize}
% \item Last time: step times in simulations only converge to the
%   fit distribution of step times when the length $L$ of the simulation
%   domain is long. Running simulations through a shorter domain biases
%   the simulated distribution of step times to be smaller than expected.
% \item Figure \ref{fig:step-time}---Figure \ref{fig:traj-plots-200} show step
%   time distributions and trajectories from simulations run through
%   domains of various lengths, as well as simulated step time
%   distributions when only looking at trajectories in a $2.5 \, \mu m$
%   window. Restricting the ``viewing window'' in the longer
%   stochastic experiments seems equivalent (and I suspect is) to
%   running the simulations through a short domain.
% \item I think that estimating the effective platelet on rate as the
%   inverse of the mean step time \emph{overestimates} the effective on
%   rate. In our experiments, longer steps are more likely to get cut
%   off (and therefore not counted) because platelets are more likely to
%   leave the domain on a long step. Are longer steps more likely to get
%   cut off in observed trajectories? If they are, then our estimate of
%   effective on-rate is an over-estimate of the true effective on rate.
% \item I also imported Emma's data, and it is plotted in Figure
%   \ref{fig:emma-data}. Platelets are moving much faster in these
%   experiments than in their previous experiments. I plotted the new
%   trajectories alongside the whole blood trajectories, but they didn't
%   mention whether these experiments were done with whole blood or
%   PRP. In either case, the new trajectories have higher velocities
%   than the old trajectories. I've asked them whether these are whole
%   blood or PRP trajectories, and what the shear rate on these
%   experiments are (the old ones are at 100/s wall shear rate.)
% \item We may be able to draw a stronger connection back to
%   mechanism. Qi et. al., 2019, quote a result connecting effective
%   platelet on and off rates back to receptor on/off rates, although
%   they don't give any details on how they use that relation.
% \end{itemize}

% \begin{figure}
%   \centering
%   \includegraphics[width=\textwidth]{fbg_step_sim}
%   \caption{Step time distributions with $L = 2.5 \, \mu m$}
%   \label{fig:step-time}
% \end{figure}

% \begin{figure}
%   \centering
%   \includegraphics[width=\textwidth]{fbg_step_sim_window}
%   \caption{Step time distributions in a $L = 2.5 \, \mu m$ window}
%   \label{fig:step-time-window}
% \end{figure}

% \begin{figure}
%   \centering
%   \includegraphics[width=\textwidth]{fbg_step_long}
%   \caption{Step time distributions with $L = 50 \, \mu m$.}
%   \label{fig:step-time-long}
% \end{figure}

% \begin{figure}
%   \centering
%   \includegraphics[width=\textwidth]{fbg_step_sim_100}
%   \caption{Step time distributions with $L = 100 \, \mu m$}
%   \label{fig:step-time-100}
% \end{figure}

% \begin{figure}
%   \centering
%   \includegraphics[width=\textwidth]{fbg_step_sim_200}
%   \caption{Step time distributions with $L = 200 \, \mu m$.}
%   \label{fig:step-time-200}
% \end{figure}

% \begin{figure}
%   \centering
%   \begin{subfigure}{\textwidth}
%     \includegraphics[width=\textwidth]{fbg_prp_traj_sim}
%     \caption{Observed (left) and simulated (right) (with $L=2.5 \, \mu m$)
%       trajectories in PRP.}
%   \end{subfigure}
%   \\
%   \begin{subfigure}{\textwidth}
%     \includegraphics[width=\textwidth]{fbg_whl_traj_sim}
%     \caption{Observed (left) and simulated (right) (with $L=2.5 \, \mu m$)
%       trajectories in whole blood.}
%   \end{subfigure}
%   \caption{Trajectories of primed (red) vs unprimed (blue) platelets
%     in experiments and simulations.}
%   \label{fig:traj-plots}
% \end{figure}

% \begin{figure}
%   \centering
%   \begin{subfigure}{\textwidth}
%     \includegraphics[width=\textwidth]{fbg_prp_traj_sim_window}
%     \caption{Observed (left) and simulated (right) (in a
%       $L=2.5 \, \mu m$ window) trajectories in PRP.}
%   \end{subfigure}
%   \\
%   \begin{subfigure}{\textwidth}
%     \includegraphics[width=\textwidth]{fbg_whl_traj_sim_window}
%     \caption{Observed (left) and simulated (right) (in a
%       $L=2.5 \, \mu m$ window) trajectories in whole blood.}
%   \end{subfigure}
%   \caption{Trajectories of primed (red) vs unprimed (blue) platelets
%     in experiments and simulations.}
%   \label{fig:traj-plots-window}
% \end{figure}

% \begin{figure}
%   \centering
%   \begin{subfigure}{\textwidth}
%     \includegraphics[width=\textwidth]{fbg_prp_traj_sim_long}
%     \caption{Observed (left) and simulated (right) (with $L=50 \, \mu m$)
%       trajectories in PRP.}
%   \end{subfigure}
%   \\
%   \begin{subfigure}{\textwidth}
%     \includegraphics[width=\textwidth]{fbg_whl_traj_sim_long}
%     \caption{Observed (left) and simulated (right) (with $L=50 \, \mu m$)
%       trajectories in whole blood.}
%   \end{subfigure}
%   \caption{Trajectories of primed (red) vs unprimed (blue) platelets
%     in experiments and simulations.}
%   \label{fig:traj-plots-long}
% \end{figure}

% \begin{figure}
%   \centering
%   \begin{subfigure}{\textwidth}
%     \includegraphics[width=\textwidth]{fbg_prp_traj_sim_100}
%     \caption{Observed (left) and simulated (right) (with $L=100 \, \mu m$)
%       trajectories in PRP.}
%   \end{subfigure}
%   \\
%   \begin{subfigure}{\textwidth}
%     \includegraphics[width=\textwidth]{fbg_whl_traj_sim_100}
%     \caption{Observed (left) and simulated (right) (with $L=100 \, \mu m$)
%       trajectories in whole blood.}
%   \end{subfigure}
%   \caption{Trajectories of primed (red) vs unprimed (blue) platelets
%     in experiments and simulations.}
%   \label{fig:traj-plots-100}
% \end{figure}

% \begin{figure}
%   \centering
%   \begin{subfigure}{\textwidth}
%     \includegraphics[width=\textwidth]{fbg_prp_traj_sim_200}
%     \caption{Observed (left) and simulated (right) (with $L=200 \, \mu m$)
%       trajectories in PRP.}
%   \end{subfigure}
%   \\
%   \begin{subfigure}{\textwidth}
%     \includegraphics[width=\textwidth]{fbg_whl_traj_sim_200}
%     \caption{Observed (left) and simulated (right) (with $L=200 \, \mu m$)
%       trajectories in whole blood.}
%   \end{subfigure}
%   \caption{Trajectories of primed (red) vs unprimed (blue) platelets
%     in experiments and simulations.}
%   \label{fig:traj-plots-200}
% \end{figure}

% \begin{figure}
%   \centering
%   \includegraphics[width=\textwidth]{emma_data_traj}
%   \caption{Trajectory plots of Emma's data (plotted in magenta
%     (primed) and cyan (unprimed)), along with the previous
%     fibrinogen whole blood trajectories}
%   \label{fig:emma-data}
% \end{figure}

\bibliographystyle{plain}
\bibliography{/Users/andrewwork/Documents/grad-school/thesis/library}

\end{document}




