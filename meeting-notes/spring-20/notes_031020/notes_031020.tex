\documentclass{article}

\newcommand{\ep}{\rule{.06in}{.1in}}
\textheight 9.5in

\usepackage{amssymb}
\usepackage{amsmath}
\usepackage{amsthm}
\usepackage{graphicx, subcaption, algorithmic}
\graphicspath{{/Users/andrewwork/thesis/jump-velocity/plots/}}

\usepackage{tikz, pgfplots, chemfig}
% \usepgfplotslibrary{colorbrewer, statistics}
% \pgfplotsset{
%   exact axis/.style={grid=major, minor tick num=4, xlabel=$v^*$,
%     legend entries={PDF, CDF},},
%   every axis plot post/.append style={thick},
%   table/search
%   path={/Users/andrewwork/thesis/jump-velocity/dat-files},
%   colormap/YlGnBu,
%   cycle list/Set1-5,
%   legend style={legend cell align=left,},
% }

% \usepgfplotslibrary{external}
% \tikzexternalize

\renewcommand{\arraystretch}{1.2}
\pagestyle{empty} 
\oddsidemargin -0.25in
\evensidemargin -0.25in 
\topmargin -0.75in 
\parindent 0pt
\parskip 12pt
\textwidth 7in
%\font\cj=msbm10 at 12pt

\newcommand{\tn}{\textnormal}
\newcommand{\stiff}{\frac{k_f}{\gamma}}
\newcommand{\dd}{d}
\newcommand{\Der}[2]{\frac{\dd #1}{\dd #2}}
\newcommand{\Pder}[2]{\frac{\partial #1}{\partial #2}}
\newcommand{\Integral}[4]{\int_{#3}^{#4} {#1} \dd #2}
\DeclareMathOperator{\Exp}{Exp}

% Text width is 7 inches

\def\R{\mathbb{R}}
\def\N{\mathbb{N}}
\def\C{\mathbb{C}}
\def\Z{\mathbb{Z}}
\def\Q{\mathbb{Q}}
\def\H{\mathbb{H}}
\def\B{\mathcal{B}} 
%\topmargin -.5in 

\setcounter{secnumdepth}{2}
\begin{document}
\pagestyle{plain}

\begin{center}
  {\Large Write-up of the Jump-Velocity Model, and Analysis of
    Trajectory Data (\today)}
\end{center}

\section{Brief Description of the Jump-Velocity Model}
\label{sec:jump-vel}

Suppose we model platelets as particles in a moving fluid, with two
sets of ``effective'' binding and unbinding rates: one for fast
receptors (e.g. GP1b) and one for slow receptors
(e.g. $\alpha_\tn{IIb}\beta_3$). Assume then that platelets can exist
in 4 possible states: unbound (U), bound through fast receptors (V),
bound through slow receptors (F), and bound through both fast and slow
receptors (VF) (Figure \ref{fig:primed-states}). The effective binding
and unbinding rates are not individual receptor binding and unbinding
rates, but an overall binding and unbinding rate for the platelet as a
whole, which may have several of each type of bond to the surface.

\begin{figure}[h]
  \centering
  \schemestart
  $U$ \arrow(u1--vv){<=>[$k_\tn{on}$][$k_\tn{off}$]} $V$
  \arrow(@u1--ff){<=>[*{0}$k_\tn{on}^F$][*{0}$k_\tn{off}^F$]}[-90] $F$
  \arrow(--vf){<=>[$k_\tn{on}$][$k_\tn{off}$]} $VF$
  \arrow(@vv--@vf){<=>[*{0}$k_\tn{on}^F$][*{0}$k_\tn{off}^F$]}
  \schemestop
  \caption[Possible states of primed platelets]{A primed platelet can
    exist in four states: (U) unbound from the surface and advecting
    in the fluid, (V) bound through fast bonds to the surface, (F)
    bound through slow bonds to the surface, or (VF) bound through
    both fast bonds and slow bonds. In all three bound states, the
    platelet is immobilized on the surface.}
  \label{fig:primed-states}
\end{figure}

This is a stochastic model, where the platelet moves at a constant
velocity when it is in state $U$, and then can randomly bind and
unbind to the surface at constant rates $k_\tn{on}, k_\tn{off},
k_\tn{on}^F, k_\tn{off}^F$.

\subsection{Comparison of velocities within different steps}
\label{sec:comp-veloc-with}

One of the key assumptions in this model is that all unbound platelets
advect at a constant velocity, so we wanted to check this assumption
against the trajectory data. I extracted ``step velocities''---the
average velocity a platelet travels within a step---and compared step
velocities among (1) initial steps, where a platelet enters the field
of view, (2) intermediate steps occuring between two dwells, and (3)
final steps where a platelet leaves the field of view. Figure
\ref{fig:ccp-vel-cmp} compares velocities within each of these three
types of steps. One observation is that platelets are moving
much faster on average when they are entering and leaving the
domain. Another observation is that the velocities are nowhere near
constant, and platelets in intermediate steps are moving much slower
than we'd expect if they were totally unbound from the surface. At a
wall shear rate of 100 / s, a platelet whose center is 1 $\mu$m off
the wall should be moving roughly at 100 $\mu$m / s, but the
intermediate step velocities are in the 0---30 $\mu$m / s range. 

\begin{figure}
  \centering
  \includegraphics[width=0.6\textwidth]{ccp-vel-cmp.png}
  \caption{Comparison of intermediate free velocities with the first
    and last steps of the experiment. The data shown is from
    the collagen-collagen PRP experiment, but the others are similar.}
  \label{fig:ccp-vel-cmp}
\end{figure}

I also compared intermediate step sizes with initial and final step
sizes, and this data shows the same trend as the velocity data, but
this of course is at least partly due to the differences in velocity.

\begin{figure}
  \centering
  \includegraphics[width=0.6\textwidth]{ccp-step-cmp.png}
  \caption{Comparison of intermediate step sizes with the first and
    last steps of the experiment. Again, the data shown is from the
    collagen-collagen PRP experiment, but other experiments have
    qualitatively similar results.}
  \label{fig:ccp-step-cmp}
\end{figure}

Our hypothesis is that the discrepancy between the intermediate step
velocities, and the initial and final step velocities (as well as the
rough estimate for the velocity of an unbound platelet) is due to fast
binding/unbinding dynamics between the platelet and wall occurring on
a faster time scale than can be observed. Therefore in the 4 state
model, we assume that the binding dynamics of the fast bonds are
causing this slow-down in the intermediate steps, and the observed
dwells are only the dwells due to the formation of slow bonds between the
platelet and surface.

We tested this hypothesis by fitting the fast binding/unbinding
parameters to processed platelet trajectories, where we cut out the
dwells from each trajectory and ``stacked'' the steps on top of each
other. The probability distributions of the average velocities of
these processed trajectories on collagen and fibrinogen are shown in
Figure \ref{fig:avg-free-vel-col} and Figure
\ref{fig:avg-free-vel-fib}. The fast effective binding parameters
predicted by each model fit are given in the captions below each figure.

\begin{figure}
  \centering
  \includegraphics[width=\textwidth]{avg_free_vel_col.png}
  \caption{Comparison of the average free velocity in the four
    collagen experimental conditions. The effective binding parameters
    found by fitting the model to velocity data are
    $k_\tn{on} = 34.6 / s$ and $k_\tn{off} = 5.18 / s$. An ANOVA test
    didn't find any significant difference in the average velocities
    among the 4 experiments, and so I fit the model to all the data
    simultaneously.}
  \label{fig:avg-free-vel-col}
\end{figure}

\begin{figure}
  \centering
  \includegraphics[width=\textwidth]{avg_free_vel_fib.png}
  \caption{Comparison of the average free velocity in the four
    fibrinogen experimental conditions. The effective binding
    parameters found by fitting the model to velocity data are
    $k_\tn{on} = 70.3 / s$ and $k_\tn{off} = 11.8 / s$. An ANOVA test
    didn't find any significant difference in the average velocities
    among the 3 experiments with data, and so I fit the model to all
    the data simultaneously.}
  \label{fig:avg-free-vel-fib}
\end{figure}

\subsection{Fitting slow binding and unbinding parameters}
\label{sec:fitting-slow-binding}

Under the assumption that the fast reactions are asymptotically fast,
the slow effective binding and unbinding rates can then just be found
by taking the mean dwell time and mean step time. Figures
\ref{fig:step-time} and \ref{fig:pause-time} show the step time
distributions, means, and exponential fits ($\pm$ 95\% CI) for each
experiment.

\begin{figure}
  \centering
  \begin{subfigure}{0.75\textwidth}
    \includegraphics[width=\textwidth]{col_step}
  \end{subfigure}
  \\
  \begin{subfigure}{0.75\textwidth}
    \includegraphics[width=\textwidth]{fbg_step}
  \end{subfigure}
  \\
  \begin{subfigure}{0.75\textwidth}
    \includegraphics[width=\textwidth]{vwf_step}
  \end{subfigure}
  \caption{Step time data from all experiments}
  \label{fig:step-time}
\end{figure}

\begin{figure}
  \centering
  \begin{subfigure}{0.75\textwidth}
    \includegraphics[width=\textwidth]{col_pause}
  \end{subfigure}
  \\
  \begin{subfigure}{0.75\textwidth}
    \includegraphics[width=\textwidth]{fbg_pause}
  \end{subfigure}
  \\
  \begin{subfigure}{0.75\textwidth}
    \includegraphics[width=\textwidth]{vwf_pause}
  \end{subfigure}
  \caption{Pause time data from all experiments}
  \label{fig:pause-time}
\end{figure}

Then I ran 250 stochastic simulations of the model for the 8
experimental cases with collagen and fibrinogen, and compared the
distribution of average velocities with the observed average
velocities (computed from the start of the first dwell to the end of
the last dwell/end of experiment). In all the collagen experiments,
the simulated distribution visually matches up reasonably well with
the data, and the simulation means are within the 95\% CI for each of
the 4 experiments. The simulations do not match up as well against the
fibrinogen data; in fibrinogen-fibrinogen whole blood, the first 80\%
of the distribution of simulated average velocities matches up well
with the data, but then has a long tail which is not present in the
data. The simulations for the hfw and ffp data don't match up well
against the data at all, and I'm not really sure what the source of this
discrepancy is yet.

\begin{figure}
  \centering
  \includegraphics[width=\textwidth]{avg_vel_col.png}
  \caption{Comparison of the overall average velocity in the four
    collagen experimental conditions, along with distributions of
    average velocities from simulations with the parameters found
    above.}
  \label{fig:avg-vel-col}
\end{figure}

\begin{figure}
  \centering
  \includegraphics[width=\textwidth]{avg_vel_fib.png}
  \caption{Comparison of the overall average velocity in the four
    fibrinogen experimental conditions, along with distributions of
    average velocities from simulations with the parameters found
    above.}
  \label{fig:avg-vel-fib}
\end{figure}

Finally, in an attempt to (crudely) describe the initial steps in the platelet
trajectories, we again computed a ``capture rate'' from the time
between the start of a trajectory and the time to first binding. In
the collagen experiments, where there is the most data, there is a
clear increase in the capture rate in the whole blood experiments
relative to PRP, but not a clear relationship between primed and
unprimed platelets.

\begin{figure}
  \centering
  \begin{subfigure}{0.48\textwidth}
    \includegraphics[width=\textwidth]{ccp_capture.png}
    \caption{CC PRP: $k_\tn{capture} = 0.52 / s$}
  \end{subfigure}
  \hfill
  \begin{subfigure}{0.48\textwidth}
    \includegraphics[width=\textwidth]{hcp_capture.png}
    \caption{HC PRP: $k_\tn{capture} = 0.63 / s$}
  \end{subfigure}
  \\
  \begin{subfigure}{0.48\textwidth}
    \includegraphics[width=\textwidth]{ccw_capture.png}
    \caption{CC whole blood: $k_\tn{capture} = 1.59 / s$}
  \end{subfigure}
  \hfill
  \begin{subfigure}{0.48\textwidth}
    \includegraphics[width=\textwidth]{hcw_capture.png}
    \caption{HC whole blood: $k_\tn{capture} = 1.12 / s$}
  \end{subfigure}
  \caption{Histograms of the initial step time. The two experiments
    with whole blood have faster ``capture'' rates than the two
    experiments in PRP.}
  \label{fig:col-capture}
\end{figure}

\begin{figure}
  \centering
  \begin{subfigure}{0.48\textwidth}
    \includegraphics[width=\textwidth]{ffp_capture.png}
    \caption{FF PRP: $k_\tn{capture} = 0.67 / s$}
  \end{subfigure}
  \hfill
  \begin{subfigure}{0.48\textwidth}
    \includegraphics[width=\textwidth]{hfp_capture.png}
    \caption{HF PRP: $k_\tn{capture} = 1.42 / s$}
  \end{subfigure}
  \\
  \begin{subfigure}{0.48\textwidth}
    \includegraphics[width=\textwidth]{ffw_capture.png}
    \caption{FF whole blood: $k_\tn{capture} = 1.17 / s$}
  \end{subfigure}
  \hfill
  \begin{subfigure}{0.48\textwidth}
    \includegraphics[width=\textwidth]{hfw_capture.png}
    \caption{HF whole blood: $k_\tn{capture} = 0.81 / s$}
  \end{subfigure}
  \caption{Histograms of the initial step time on fibrinogen.}
  \label{fig:fbg-capture}
\end{figure}

\begin{figure}
  \centering
  \begin{subfigure}{0.48\textwidth}
    \includegraphics[width=\textwidth]{vvp_capture.png}
    \caption{VV PRP: $k_\tn{capture} = 1.89 / s$}
  \end{subfigure}
  \hfill
  \begin{subfigure}{0.48\textwidth}
    \includegraphics[width=\textwidth]{hvp_capture.png}
    \caption{HV PRP: $k_\tn{capture} = 0.32 / s$}
  \end{subfigure}
  \caption{Histograms of the initial step time on vWF.}
  \label{fig:vwf-capture}
\end{figure}

\begin{figure}
  \centering
  \begin{subfigure}{0.48\textwidth}
    \includegraphics[width=\textwidth]{col_prp_traj}
  \end{subfigure}
  \hfill
  \begin{subfigure}{0.48\textwidth}
    \includegraphics[width=\textwidth]{col_whl_traj}
  \end{subfigure}
  \\
  \begin{subfigure}{0.48\textwidth}
    \includegraphics[width=\textwidth]{fbg_prp_traj}
  \end{subfigure}
  \hfill
  \begin{subfigure}{0.48\textwidth}
    \includegraphics[width=\textwidth]{fbg_whl_traj}
  \end{subfigure}
  \\
  \begin{subfigure}{0.48\textwidth}
    \includegraphics[width=\textwidth]{vwf_prp_traj}
  \end{subfigure}
  \caption{Trajectory data from the 10 different experiments. Unprimed
    platelets are in blue, and primed platelets are in red}
  \label{fig:traj-plots}
\end{figure}

% \bibliographystyle{plain}
% \bibliography{/Users/andrewwork/Documents/grad-school/thesis/library}

\end{document}




