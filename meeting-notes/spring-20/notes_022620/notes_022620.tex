\documentclass{article}

\newcommand{\ep}{\rule{.06in}{.1in}}
\textheight 9.5in

\usepackage{amssymb}
\usepackage{amsmath}
\usepackage{amsthm}
\usepackage{graphicx, subcaption, algorithmic}
\graphicspath{{/Users/andrewwork/thesis/jump-velocity/plots/}}

\usepackage{tikz, pgfplots, chemfig}
\usepgfplotslibrary{colorbrewer, statistics}
\pgfplotsset{
  exact axis/.style={grid=major, minor tick num=4, xlabel=$v^*$,
    legend entries={PDF, CDF},},
  every axis plot post/.append style={thick},
  table/search
  path={/Users/andrewwork/thesis/jump-velocity/dat-files},
  colormap/YlGnBu,
  cycle list/Set1-5,
  legend style={legend cell align=left,},
}

\usepgfplotslibrary{external}
\tikzexternalize

\renewcommand{\arraystretch}{1.2}
\pagestyle{empty} 
\oddsidemargin -0.25in
\evensidemargin -0.25in 
\topmargin -0.75in 
\parindent 0pt
\parskip 12pt
\textwidth 7in
%\font\cj=msbm10 at 12pt

\newcommand{\tn}{\textnormal}
\newcommand{\stiff}{\frac{k_f}{\gamma}}
\newcommand{\dd}{d}
\newcommand{\Der}[2]{\frac{\dd #1}{\dd #2}}
\newcommand{\Pder}[2]{\frac{\partial #1}{\partial #2}}
\newcommand{\Integral}[4]{\int_{#3}^{#4} {#1} \dd #2}
\DeclareMathOperator{\Exp}{Exp}

% Text width is 7 inches

\def\R{\mathbb{R}}
\def\N{\mathbb{N}}
\def\C{\mathbb{C}}
\def\Z{\mathbb{Z}}
\def\Q{\mathbb{Q}}
\def\H{\mathbb{H}}
\def\B{\mathcal{B}} 
%\topmargin -.5in 

\setcounter{secnumdepth}{2}
\begin{document}
\pagestyle{plain}

% \begin{center}
%   {\Large Notes on trajectory data and the jump-velocity model
%     (\today)}
% \end{center}

% \section{Regularized Stokeslets on a sphere}
% \label{sec:regul-stok-sphere}

% The method of regularized stokeslets requires you to integrate:
% \begin{equation}
%   \label{eq:reg-stokeslets}
%   u_j(\mathbf{x}_0) = \frac{1}{8\pi \mu} \int_{\partial D}
%   S_{ij}^\epsilon (\mathbf{x}, \mathbf{x}_0) g_i(\mathbf{x}) ds(\mathbf{x}).
% \end{equation}
% where $\mu$ is the viscosity of the fluid, $\partial D$ is the surface
% of the body (a sphere in the simplest case), $S_{ij}^\epsilon
% (\mathbf{x}, \mathbf{x}_0)$ is the regularized stokeslet with blob
% parameter $\epsilon$, and $g_i$ is the $i$th component of the traction
% on the surface of the body.

% Ultimately our goal is to use regularized stokeslets to find the
% velocity field of the fluid around an ellipsoidal platelet near a
% wall, but we want to test out the method on simpler problems
% first. The simplest case is where $\partial D$ is the surface of a
% sphere, and $S_{ij}^\epsilon$ is the free-space Stokeslet (i.e. the
% flow is unbounded in all directions away from the sphere).

% This case is handled in Cortez et. al., 2005 \cite{Cortez2005}. The
% first test case they examine is to apply a point force of $\mathbf{f}
% = -\frac{3\mu}{2} \mathbf{e_3}$ everywhere on the surface of the
% sphere. Analytically, this generates a velocity of $\mathbf{u} =
% -\mathbf{e}_3$. The error of the numerical integration of equation
% (\ref{eq:reg-stokeslets}) depends both on the size of the mesh $N
% \times N$, and the regularization parameter $\epsilon$. Cortez
% et. al. provide plots of the L2 error while refining each of these
% parameters in turn, which we can compare our integration scheme to.

% As shown in Figures \ref{fig:cortez-conv} and \ref{fig:our-conv}, the
% convergence results of our numerical scheme is similar to their
% published results.

% There are two next steps:
% \begin{enumerate}
% \item Write a function that is able to go ``the other way,'' i.e. to
%   compute a distribution of point forces that generate a prescribed
%   motion $\mathbf{u}(\mathbf{x})$. This requires building the linear
%   system
%   \begin{equation}
%     \label{eq:linear-system}
%     \tilde{\mathbf{u}} = \frac{1}{8\pi\mu} \mathcal{A} \tilde{\mathbf{f}}
%   \end{equation}
%   where $\tilde{\mathbf{u}}$ and $\tilde{\mathbf{f}}$ are vectors of
%   length $3N_\tn{node}$.
% \item The other next step is to write a routine that integrates over
%   the surface of an ellipsoid. I am thinking this won't be too much
%   extra work, we just need an additional coordinate transform to
%   convert between coordinates on the ellipsoid and coordinates on the
%   sphere. 
% \end{enumerate}

% \begin{figure}
%   \centering
%   \includegraphics[width=0.5\textwidth]{cubed_sphere}
%   \caption{An example of the discretization scheme on the surface of
%     the sphere (from \cite{Portelenelle2018})}
%   \label{fig:discretization}
% \end{figure}

% \begin{figure}
%   \centering
%   \begin{subfigure}{0.6\textwidth}
%     \includegraphics[width=\textwidth]{reg_refine_cortez}
%   \end{subfigure}
%   \hfill
%   \begin{subfigure}{0.35\textwidth}
%     \includegraphics[width=\textwidth]{grid_refine_cortez}
%   \end{subfigure}
%   \caption{Convergence in the $\Delta s$ and $\epsilon$ parameters
%     published in \cite{Cortez2005}.}
%   \label{fig:cortez-conv}
% \end{figure}

% \begin{figure}
%   \centering
%   \begin{subfigure}{0.48\textwidth}
%     \includegraphics[width=\textwidth]{reg_refine}
%   \end{subfigure}
%   \hfill
%   \begin{subfigure}{0.48\textwidth}
%     \includegraphics[width=\textwidth]{grid_refine}
%   \end{subfigure}
%   \caption{Convergence in the $\Delta s$ and $\epsilon$ parameters in
%     our numerical scheme.}
%   \label{fig:our-conv}
% \end{figure}

% \newpage

% \section{Trajectory Data}
% \label{sec:trajectory-data}

\begin{figure}
  \centering
  \begin{subfigure}{0.48\textwidth}
    \includegraphics[width=\textwidth]{col_prp_traj}
  \end{subfigure}
  \hfill
  \begin{subfigure}{0.48\textwidth}
    \includegraphics[width=\textwidth]{col_whl_traj}
  \end{subfigure}
  \\
  \begin{subfigure}{0.48\textwidth}
    \includegraphics[width=\textwidth]{fbg_prp_traj}
  \end{subfigure}
  \hfill
  \begin{subfigure}{0.48\textwidth}
    \includegraphics[width=\textwidth]{fbg_whl_traj}
  \end{subfigure}
  \\
  \begin{subfigure}{0.48\textwidth}
    \includegraphics[width=\textwidth]{vwf_prp_traj}
  \end{subfigure}
  \caption{Trajectory data from the 10 different experiments. Unprimed
    platelets are in blue, and primed platelets are in red}
  \label{fig:traj-plots}
\end{figure}

\begin{figure}
  \centering
  \begin{subfigure}{0.75\textwidth}
    \includegraphics[width=\textwidth]{col_step}
  \end{subfigure}
  \\
  \begin{subfigure}{0.75\textwidth}
    \includegraphics[width=\textwidth]{fbg_step}
  \end{subfigure}
  \\
  \begin{subfigure}{0.75\textwidth}
    \includegraphics[width=\textwidth]{vwf_step}
  \end{subfigure}
  \caption{Step time data from all experiments}
  \label{fig:step-time}
\end{figure}

\begin{figure}
  \centering
  \begin{subfigure}{0.75\textwidth}
    \includegraphics[width=\textwidth]{col_pause}
  \end{subfigure}
  \\
  \begin{subfigure}{0.75\textwidth}
    \includegraphics[width=\textwidth]{fbg_pause}
  \end{subfigure}
  \\
  \begin{subfigure}{0.75\textwidth}
    \includegraphics[width=\textwidth]{vwf_pause}
  \end{subfigure}
  \caption{Pause time data from all experiments}
  \label{fig:pause-time}
\end{figure}

\begin{figure}
  \centering
  \begin{subfigure}{0.75\textwidth}
    \includegraphics[width=\textwidth]{col_vel}
  \end{subfigure}
  \\
  \begin{subfigure}{0.75\textwidth}
    \includegraphics[width=\textwidth]{fbg_vel}
  \end{subfigure}
  \\
  \begin{subfigure}{0.75\textwidth}
    \includegraphics[width=\textwidth]{vwf_vel}
  \end{subfigure}
  \caption{Time-averaged velocity data from all experiments}
  \label{fig:velocity}
\end{figure}

\begin{figure}
  \centering
  \begin{subfigure}{0.48\textwidth}
    \includegraphics[width=\textwidth]{ccp_capture.png}
    \caption{CC PRP: $k_\tn{capture} = 0.52 / s$}
  \end{subfigure}
  \hfill
  \begin{subfigure}{0.48\textwidth}
    \includegraphics[width=\textwidth]{hcp_capture.png}
    \caption{HC PRP: $k_\tn{capture} = 0.63 / s$}
  \end{subfigure}
  \\
  \begin{subfigure}{0.48\textwidth}
    \includegraphics[width=\textwidth]{ccw_capture.png}
    \caption{CC whole blood: $k_\tn{capture} = 1.59 / s$}
  \end{subfigure}
  \hfill
  \begin{subfigure}{0.48\textwidth}
    \includegraphics[width=\textwidth]{hcw_capture.png}
    \caption{HC whole blood: $k_\tn{capture} = 1.12 / s$}
  \end{subfigure}
  \caption{Histograms of the initial step time. The two experiments
    with whole blood have faster ``capture'' rates than the two
    experiments in PRP.}
  \label{fig:col-capture}
\end{figure}

\begin{figure}
  \centering
  \begin{subfigure}{0.48\textwidth}
    \includegraphics[width=\textwidth]{ffp_capture.png}
    \caption{FF PRP: $k_\tn{capture} = 0.67 / s$}
  \end{subfigure}
  \hfill
  \begin{subfigure}{0.48\textwidth}
    \includegraphics[width=\textwidth]{hfp_capture.png}
    \caption{HF PRP: $k_\tn{capture} = 1.42 / s$}
  \end{subfigure}
  \\
  \begin{subfigure}{0.48\textwidth}
    \includegraphics[width=\textwidth]{ffw_capture.png}
    \caption{FF whole blood: $k_\tn{capture} = 1.17 / s$}
  \end{subfigure}
  \hfill
  \begin{subfigure}{0.48\textwidth}
    \includegraphics[width=\textwidth]{hfw_capture.png}
    \caption{HF whole blood: $k_\tn{capture} = 0.81 / s$}
  \end{subfigure}
  \caption{Histograms of the initial step time on fibrinogen.}
  \label{fig:fbg-capture}
\end{figure}

\begin{figure}
  \centering
  \begin{subfigure}{0.48\textwidth}
    \includegraphics[width=\textwidth]{vvp_capture.png}
    \caption{VV PRP: $k_\tn{capture} = 1.89 / s$}
  \end{subfigure}
  \hfill
  \begin{subfigure}{0.48\textwidth}
    \includegraphics[width=\textwidth]{hvp_capture.png}
    \caption{HV PRP: $k_\tn{capture} = 0.32 / s$}
  \end{subfigure}
  \caption{Histograms of the initial step time on vWF.}
  \label{fig:vwf-capture}
\end{figure}

\bibliographystyle{plain}
\bibliography{/Users/andrewwork/Documents/grad-school/thesis/library}

\end{document}




