\documentclass{article}

\newcommand{\ep}{\rule{.06in}{.1in}}
\textheight 9.5in

\usepackage{amssymb}
\usepackage{amsmath}
\usepackage{amsthm}
\usepackage{graphicx, subcaption, algorithmic}
\graphicspath{{/Users/andrewwork/thesis/jump-velocity/plots/}}

\usepackage{tikz, pgfplots, chemfig}
% \usepgfplotslibrary{colorbrewer, statistics}
% \pgfplotsset{
%   exact axis/.style={grid=major, minor tick num=4, xlabel=$v^*$,
%     legend entries={PDF, CDF},},
%   every axis plot post/.append style={thick},
%   table/search
%   path={/Users/andrewwork/thesis/jump-velocity/dat-files},
%   colormap/YlGnBu,
%   cycle list/Set1-5,
%   legend style={legend cell align=left,},
% }

% \usepgfplotslibrary{external}
% \tikzexternalize

\renewcommand{\arraystretch}{1.2}
\pagestyle{empty} 
\oddsidemargin -0.25in
\evensidemargin -0.25in 
\topmargin -0.75in 
\parindent 0pt
\parskip 12pt
\textwidth 7in
%\font\cj=msbm10 at 12pt

\newcommand{\tn}{\textnormal}
\newcommand{\stiff}{\frac{k_f}{\gamma}}
\newcommand{\dd}{d}
\newcommand{\Der}[2]{\frac{\dd #1}{\dd #2}}
\newcommand{\Pder}[2]{\frac{\partial #1}{\partial #2}}
\newcommand{\Integral}[4]{\int_{#3}^{#4} {#1} \dd #2}
\DeclareMathOperator{\Exp}{Exp}

% Text width is 7 inches

\def\R{\mathbb{R}}
\def\N{\mathbb{N}}
\def\C{\mathbb{C}}
\def\Z{\mathbb{Z}}
\def\Q{\mathbb{Q}}
\def\H{\mathbb{H}}
\def\B{\mathcal{B}} 
%\topmargin -.5in 

\setcounter{secnumdepth}{2}
\begin{document}
\pagestyle{plain}

\begin{center}
  {\Large Meeting Notes (\today)}
\end{center}

\large{\textbf{Regularized Stokeslets}}

Two ways to handle FSI with Regularized Stokeslets:
\begin{enumerate}
\item \textbf{Elastic Body and IB} Treat nodes on the surface of the
  platelet as IB points connected with stiff springs (or define some
  function that gives the local force at each point on the platelet
  surface for a given configuration
  $\underline{F}(\underline{q},t) =
  \mathcal{F}(\underline{X}(\underline{q}, t), t)$). Then introduce
  this as a forcing term in the equations of Stokes flow:
  $$0 = -\nabla p + \Delta \underline{u} + \underline{f} \quad
  \tn{where} \quad \underline{f} = \int_\Gamma
  \underline{F}(\underline{q}, t) \delta(\underline{x} -
  \underline{X}(q, t)) dq.$$

  Finally, points on the platelet surface move at the local fluid
  velocity:
  $$\frac{d\underline{X}}{dt}(q, t) =
  \underline{u}(\underline{X}(q,t), t).$$

  This is the approach used by Cortez et. al. in \cite{Cortez2005} to
  model the motion of a flagellum. The flagellum is defined with two
  helices---one enclosed within the other---and on each helix adjacent
  points are joined with springs to maintain the structure's
  shape. Then motion is driven by connecting placing springs between
  the inner and outer helices as shown in Figure
  \ref{fig:flagellum}. When the tension on the springs drops below
  some threshold, the springs detach and reattach to the next point on
  the outer ring to maintain motion.

  To determine the motion of a bound platelet in flow, any bond forces
  would be defined locally at their point of attachment, and this
  force would be communicated to the rest of the body through the
  force function $\mathcal{F}$.

  \begin{figure}[h]
    \centering
    \includegraphics[width=.4\textwidth]{flagellum}
    \caption{Cross section of the flagellum with springs connecting
      the inner and outer helices (from \cite{Cortez2005})}
    \label{fig:flagellum}
  \end{figure}
  
\item \textbf{Rigid Body} With a rigid body in Stokes flow, we write
  Stokes equations with the boundary condition
  $\underline{u}(\underline{x}, t) = \underline{V} +
  \underline{\Omega} \times \underline{x}$, which ensures no-slip and
  no-penetration of the fluid at the boundary of the
  platelet. However, in an adhesive dynamics simulation, the
  translation and angular velocities $\underline{V}$ and
  $\underline{\Omega}$ are unknown; they are determined from force and
  torque balance on the entire body of the platelet. The hydrodynamic
  force $F_h$ and $T_h$ can be found respectively by the integrals
  $\int_\Gamma \underline{\underline{\sigma}} \cdot
  \underline{n}(\underline{x}) ds(\underline{x})$ and $\int_\Gamma
  (\underline{x} \times (\underline{\underline{\sigma}} \cdot
  \underline{n}) ds(\underline{x})$.

  Because of the linearity of Stokes flow, $F_h$ and $T_h$ are
  linearly related to the velocities $\underline{V}$ and
  $\underline{\Omega}$:
  \begin{align*}
    F_h &= -(\mathcal{T} \underline{V} + \mathcal{P}
    \underline{\Omega}) \\
    T_h &= -(\mathcal{P}^T \underline{V} + \mathcal{R}
          \underline{\Omega}).
  \end{align*}

  The resistance matrices $\mathcal{T}$, $\mathcal{P}$, and
  $\mathcal{R}$ depend only on the shape of the body, and its position
  and orientation, and they can be found by solving Stokes equations
  for the following 6 boundary conditions: $\underline{V} =
  \underline{e}_i$ for $i = 1, 2, 3$ and $\underline{\Omega} =
  \underline{e}_i$ for $i = 1, 2, 3$.
\end{enumerate}

\large{\textbf{Jump-Velocity Model}}

\begin{itemize}
\item Reviewed Qi et. al. \cite{Qi2019} and Chesla
  et. al. \cite{Chesla1998} to try and understand how Qi et. al. are
  relating their effective platelet unbinding rate $\kappa_\tn{off}$
  to individual receptor on/off rates. They quote a result from
  \cite{Chesla1998}:
  \begin{equation*}
    p_n(t) = \frac{\langle n \rangle^n}{n!} \exp(-\langle n \rangle)
    \quad \tn{with} \quad \langle n \rangle = \alpha A_\alpha
    \frac{k_\tn{on}}{k_\tn{off}} [1 - \exp(-k_\tn{off}t)],
  \end{equation*}
  but they don't really say how they use this, and I'm not sure how
  this result applies. The initial condition here is $p_0(0) = 1$, so
  there are no bonds at the start. It seems like we would want an
  initial condition with $p_1(0) = 1$ instead.
\item Also, I started looking at the jump-velocity model with escape (Figure
  \ref{fig:escape-diagram}), but my estimate for the escape rate was
  wrong, so I need to fix this first.
  
  \begin{figure}[h]
    \centering
    \schemestart
    $U$ \arrow(u1--vv){<=>[$k_\tn{on}$][$k_\tn{off}$]} $V$
    \arrow(@u1--ff){<=>[*{0}$k_\tn{on}^F$][*{0}$k_\tn{off}^F$]}[-90] $F$
    \arrow(--vf){<=>[$k_\tn{on}$][$k_\tn{off}$]} $VF$
    \arrow(@vv--@vf){<=>[*{0}$k_\tn{on}^F$][*{0}$k_\tn{off}^F$]}
    \arrow(@u1--ww){->[*{0}$k_\tn{escape}$]}[90]
    \arrow(@vv--vw){->[*{0}$k_\tn{escape}$]}[90]
    \schemestop
    \caption{Jump velocity model with escape}
    \label{fig:escape-diagram}
  \end{figure}
\item One prediction from the jump-velocity model with escape is that
  the number of dwells in a trajectory should be geometrically
  distributed. Basically, the probability that a platelet re-binds
  after unbinding is $a = \frac{k_\tn{on}}{k_\tn{on} + k_\tn{escape}}$,
  and so the number of binding events that occur before the platelet
  escapes is a geometric distribution ($p(n) = (1 - a) a^{n-1}$).

  \begin{figure}[h]
    \centering
    \includegraphics[width=\textwidth]{num_dwells}
    \caption{Number of dwells per trajectory in experiments and
      predicted by a geometric distribution.}
    \label{fig:ndwells}
  \end{figure}

\end{itemize}

% \begin{figure}[h]
%   \centering
%   \schemestart
%   $U$ \arrow(u1--vv){<=>[$k_\tn{on}$][$k_\tn{off}$]} $V$
%   \arrow(@u1--ff){<=>[*{0}$k_\tn{on}^F$][*{0}$k_\tn{off}^F$]}[-90] $F$
%   \arrow(--vf){<=>[$k_\tn{on}$][$k_\tn{off}$]} $VF$
%   \arrow(@vv--@vf){<=>[*{0}$k_\tn{on}^F$][*{0}$k_\tn{off}^F$]}
%   \schemestop
%   \label{fig:primed-states}
% \end{figure}

% \begin{itemize}
% \item Last time: step times in simulations only converge to the
%   fit distribution of step times when the length $L$ of the simulation
%   domain is long. Running simulations through a shorter domain biases
%   the simulated distribution of step times to be smaller than expected.
% \item Figure \ref{fig:step-time}---Figure \ref{fig:traj-plots-200} show step
%   time distributions and trajectories from simulations run through
%   domains of various lengths, as well as simulated step time
%   distributions when only looking at trajectories in a $2.5 \, \mu m$
%   window. Restricting the ``viewing window'' in the longer
%   stochastic experiments seems equivalent (and I suspect is) to
%   running the simulations through a short domain.
% \item I think that estimating the effective platelet on rate as the
%   inverse of the mean step time \emph{overestimates} the effective on
%   rate. In our experiments, longer steps are more likely to get cut
%   off (and therefore not counted) because platelets are more likely to
%   leave the domain on a long step. Are longer steps more likely to get
%   cut off in observed trajectories? If they are, then our estimate of
%   effective on-rate is an over-estimate of the true effective on rate.
% \item I also imported Emma's data, and it is plotted in Figure
%   \ref{fig:emma-data}. Platelets are moving much faster in these
%   experiments than in their previous experiments. I plotted the new
%   trajectories alongside the whole blood trajectories, but they didn't
%   mention whether these experiments were done with whole blood or
%   PRP. In either case, the new trajectories have higher velocities
%   than the old trajectories. I've asked them whether these are whole
%   blood or PRP trajectories, and what the shear rate on these
%   experiments are (the old ones are at 100/s wall shear rate.)
% \item We may be able to draw a stronger connection back to
%   mechanism. Qi et. al., 2019, quote a result connecting effective
%   platelet on and off rates back to receptor on/off rates, although
%   they don't give any details on how they use that relation.
% \end{itemize}

% \begin{figure}
%   \centering
%   \includegraphics[width=\textwidth]{fbg_step_sim}
%   \caption{Step time distributions with $L = 2.5 \, \mu m$}
%   \label{fig:step-time}
% \end{figure}

% \begin{figure}
%   \centering
%   \includegraphics[width=\textwidth]{fbg_step_sim_window}
%   \caption{Step time distributions in a $L = 2.5 \, \mu m$ window}
%   \label{fig:step-time-window}
% \end{figure}

% \begin{figure}
%   \centering
%   \includegraphics[width=\textwidth]{fbg_step_long}
%   \caption{Step time distributions with $L = 50 \, \mu m$.}
%   \label{fig:step-time-long}
% \end{figure}

% \begin{figure}
%   \centering
%   \includegraphics[width=\textwidth]{fbg_step_sim_100}
%   \caption{Step time distributions with $L = 100 \, \mu m$}
%   \label{fig:step-time-100}
% \end{figure}

% \begin{figure}
%   \centering
%   \includegraphics[width=\textwidth]{fbg_step_sim_200}
%   \caption{Step time distributions with $L = 200 \, \mu m$.}
%   \label{fig:step-time-200}
% \end{figure}

% \begin{figure}
%   \centering
%   \begin{subfigure}{\textwidth}
%     \includegraphics[width=\textwidth]{fbg_prp_traj_sim}
%     \caption{Observed (left) and simulated (right) (with $L=2.5 \, \mu m$)
%       trajectories in PRP.}
%   \end{subfigure}
%   \\
%   \begin{subfigure}{\textwidth}
%     \includegraphics[width=\textwidth]{fbg_whl_traj_sim}
%     \caption{Observed (left) and simulated (right) (with $L=2.5 \, \mu m$)
%       trajectories in whole blood.}
%   \end{subfigure}
%   \caption{Trajectories of primed (red) vs unprimed (blue) platelets
%     in experiments and simulations.}
%   \label{fig:traj-plots}
% \end{figure}

% \begin{figure}
%   \centering
%   \begin{subfigure}{\textwidth}
%     \includegraphics[width=\textwidth]{fbg_prp_traj_sim_window}
%     \caption{Observed (left) and simulated (right) (in a
%       $L=2.5 \, \mu m$ window) trajectories in PRP.}
%   \end{subfigure}
%   \\
%   \begin{subfigure}{\textwidth}
%     \includegraphics[width=\textwidth]{fbg_whl_traj_sim_window}
%     \caption{Observed (left) and simulated (right) (in a
%       $L=2.5 \, \mu m$ window) trajectories in whole blood.}
%   \end{subfigure}
%   \caption{Trajectories of primed (red) vs unprimed (blue) platelets
%     in experiments and simulations.}
%   \label{fig:traj-plots-window}
% \end{figure}

% \begin{figure}
%   \centering
%   \begin{subfigure}{\textwidth}
%     \includegraphics[width=\textwidth]{fbg_prp_traj_sim_long}
%     \caption{Observed (left) and simulated (right) (with $L=50 \, \mu m$)
%       trajectories in PRP.}
%   \end{subfigure}
%   \\
%   \begin{subfigure}{\textwidth}
%     \includegraphics[width=\textwidth]{fbg_whl_traj_sim_long}
%     \caption{Observed (left) and simulated (right) (with $L=50 \, \mu m$)
%       trajectories in whole blood.}
%   \end{subfigure}
%   \caption{Trajectories of primed (red) vs unprimed (blue) platelets
%     in experiments and simulations.}
%   \label{fig:traj-plots-long}
% \end{figure}

% \begin{figure}
%   \centering
%   \begin{subfigure}{\textwidth}
%     \includegraphics[width=\textwidth]{fbg_prp_traj_sim_100}
%     \caption{Observed (left) and simulated (right) (with $L=100 \, \mu m$)
%       trajectories in PRP.}
%   \end{subfigure}
%   \\
%   \begin{subfigure}{\textwidth}
%     \includegraphics[width=\textwidth]{fbg_whl_traj_sim_100}
%     \caption{Observed (left) and simulated (right) (with $L=100 \, \mu m$)
%       trajectories in whole blood.}
%   \end{subfigure}
%   \caption{Trajectories of primed (red) vs unprimed (blue) platelets
%     in experiments and simulations.}
%   \label{fig:traj-plots-100}
% \end{figure}

% \begin{figure}
%   \centering
%   \begin{subfigure}{\textwidth}
%     \includegraphics[width=\textwidth]{fbg_prp_traj_sim_200}
%     \caption{Observed (left) and simulated (right) (with $L=200 \, \mu m$)
%       trajectories in PRP.}
%   \end{subfigure}
%   \\
%   \begin{subfigure}{\textwidth}
%     \includegraphics[width=\textwidth]{fbg_whl_traj_sim_200}
%     \caption{Observed (left) and simulated (right) (with $L=200 \, \mu m$)
%       trajectories in whole blood.}
%   \end{subfigure}
%   \caption{Trajectories of primed (red) vs unprimed (blue) platelets
%     in experiments and simulations.}
%   \label{fig:traj-plots-200}
% \end{figure}

% \begin{figure}
%   \centering
%   \includegraphics[width=\textwidth]{emma_data_traj}
%   \caption{Trajectory plots of Emma's data (plotted in magenta
%     (primed) and cyan (unprimed)), along with the previous
%     fibrinogen whole blood trajectories}
%   \label{fig:emma-data}
% \end{figure}

\bibliographystyle{plain}
\bibliography{/Users/andrewwork/Documents/grad-school/thesis/library}

\end{document}




