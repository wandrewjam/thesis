\documentclass{article}

\newcommand{\ep}{\rule{.06in}{.1in}}
\textheight 9.5in

\usepackage{amssymb}
\usepackage{amsmath}
\usepackage{amsthm}
\usepackage{graphicx, subcaption, algorithmic}
\graphicspath{{/Users/andrewwork/thesis/jump-velocity/plots/}}

\usepackage{tikz, pgfplots, chemfig}
% \usepgfplotslibrary{colorbrewer, statistics}
% \pgfplotsset{
%   exact axis/.style={grid=major, minor tick num=4, xlabel=$v^*$,
%     legend entries={PDF, CDF},},
%   every axis plot post/.append style={thick},
%   table/search
%   path={/Users/andrewwork/thesis/jump-velocity/dat-files},
%   colormap/YlGnBu,
%   cycle list/Set1-5,
%   legend style={legend cell align=left,},
% }

% \usepgfplotslibrary{external}
% \tikzexternalize

\renewcommand{\arraystretch}{1.2}
\pagestyle{empty} 
\oddsidemargin -0.25in
\evensidemargin -0.25in 
\topmargin -0.75in 
\parindent 0pt
\parskip 12pt
\textwidth 7in
%\font\cj=msbm10 at 12pt

\newcommand{\tn}{\textnormal}
\newcommand{\stiff}{\frac{k_f}{\gamma}}
\newcommand{\dd}{d}
\newcommand{\Der}[2]{\frac{\dd #1}{\dd #2}}
\newcommand{\Pder}[2]{\frac{\partial #1}{\partial #2}}
\newcommand{\Integral}[4]{\int_{#3}^{#4} {#1} \dd #2}
\DeclareMathOperator{\Exp}{Exp}

% Text width is 7 inches

\def\R{\mathbb{R}}
\def\N{\mathbb{N}}
\def\C{\mathbb{C}}
\def\Z{\mathbb{Z}}
\def\Q{\mathbb{Q}}
\def\H{\mathbb{H}}
\def\B{\mathcal{B}} 
%\topmargin -.5in 

\setcounter{secnumdepth}{2}
\begin{document}
\pagestyle{plain}

\begin{center}
  {\Large Meeting Notes (\today)}
\end{center}

\large{\textbf{Regularized Stokeslets}}
\begin{itemize}
\item Regularized Stokeslets on an ellipsoid in free space---working
  for the $\mathcal{T}$ and $\mathcal{P}$ matrices, probably working
  for the $\mathcal{R}$ matrix and the $\mathbf{S_F}$ and
  $\mathbf{S_G}$ vectors.

  Definitions:
  \begin{equation}
    \begin{pmatrix}
      \mathbf{F} \\
      \mathbf{G}
    \end{pmatrix}
    =
    \begin{pmatrix}
      \mathcal{T} & \mathcal{P} \\
      \mathcal{P}^T & \mathcal{R}
    \end{pmatrix}
    \begin{pmatrix}
      \mathbf{U} \\
      \mathbf{\Omega}
    \end{pmatrix}
    +
    \begin{pmatrix}
      \mathbf{S_F} \\
      \mathbf{S_G}
    \end{pmatrix}
  \end{equation}
\item Working on finding analytical solutions for $\mathcal{R}$,
  $\mathbf{S_F}$, and $\mathbf{S_G}$.
\item Working on implementing wall-bounded regularized stokeslets
\end{itemize}

\large{\textbf{Jump-Velocity Model}}

\begin{figure}[h]
  \centering
  \schemestart
  $U$ \arrow(u1--vv){<=>[$k_\tn{on}$][$k_\tn{off}$]} $V$
  \arrow(@u1--ff){<=>[*{0}$k_\tn{on}^F$][*{0}$k_\tn{off}^F$]}[-90] $F$
  \arrow(--vf){<=>[$k_\tn{on}$][$k_\tn{off}$]} $VF$
  \arrow(@vv--@vf){<=>[*{0}$k_\tn{on}^F$][*{0}$k_\tn{off}^F$]}
  \schemestop
  \caption[Possible states of primed platelets]{A primed platelet can
    exist in four states: (U) unbound from the surface and advecting
    in the fluid, (V) bound through fast bonds to the surface, (F)
    bound through slow bonds to the surface, or (VF) bound through
    both fast bonds and slow bonds. In all three bound states, the
    platelet is immobilized on the surface.}
  \label{fig:primed-states}
\end{figure}

\begin{itemize}
\item Comparing experimental and simulated trajectories on fibrinogen
  (moving velocity, step sizes, dwell times, average velocity,
  distance traveled between 1st and last dwells)
\item I ``fixed'' the velocity comparison between experimental and
  simulated trajectories by only computing the velocity of an
  experimental trajectory between the first slow-binding event and the
  last (before I had just been taking the velocity between the first
  and last binding events, fast or slow). Now the simulated velocities
  are slower than the observed velocities.
\item What is a pause in the simulations, exactly? We want a
  definition consistent with the assumption that we can only observe
  pauses from the slow bond kinetics.
\item There is a big difference between the observed step times and
  simulated step times (bug?)
\item The simulated trajectories have a much narrower distribution of
  travel distances, and can only travel a maximum of 2.5 microns (have
  probabilistic escape?)
\end{itemize}

% \begin{figure}
%   \centering
%   \includegraphics[width=\textwidth]{avg_free_vel_col.png}
%   \caption{Comparison of the average free velocity in the four
%     collagen experimental conditions. The effective binding parameters
%     found by fitting the model to velocity data are
%     $k_\tn{on} = 34.6 / s$ and $k_\tn{off} = 5.18 / s$. An ANOVA test
%     didn't find any significant difference in the average velocities
%     among the 4 experiments, and so I fit the model to all the data
%     simultaneously.}
%   \label{fig:avg-free-vel-col}
% \end{figure}

\begin{figure}
  \centering
  \includegraphics[width=\textwidth]{avg_free_vel_fib.png}
  \caption{Comparison of the average free velocity in the four
    fibrinogen experimental conditions. The effective binding
    parameters found by fitting the model to velocity data are
    $k_\tn{on} = 70.3 / s$ and $k_\tn{off} = 11.8 / s$. An ANOVA test
    didn't find any significant difference in the average velocities
    among the 3 experiments with data, and so I fit the model to all
    the data simultaneously.}
  \label{fig:avg-free-vel-fib}
\end{figure}

\begin{figure}
  \centering
  \includegraphics[width=\textwidth]{dist_fib}
  \caption{Distribution of distances between the 1st and the last
    dwell}
  \label{fig:fbg-dist}
\end{figure}

\begin{figure}
  \centering
  % \begin{subfigure}{0.75\textwidth}
  %   \includegraphics[width=\textwidth]{col_step}
  % \end{subfigure}
  % \\
  \begin{subfigure}{\textwidth}
    \includegraphics[width=\textwidth]{fbg_step_sim}
  \end{subfigure}
  % \\
  % \begin{subfigure}{0.75\textwidth}
  %   \includegraphics[width=\textwidth]{vwf_step}
  % \end{subfigure}
  \caption{Step time data from all experiments}
  \label{fig:step-time}
\end{figure}

\begin{figure}
  \centering
  % \begin{subfigure}{0.75\textwidth}
  %   \includegraphics[width=\textwidth]{col_pause}
  % \end{subfigure}
  % \\
  \begin{subfigure}{\textwidth}
    \includegraphics[width=\textwidth]{fbg_pause_sim}
  \end{subfigure}
  % \\
  % \begin{subfigure}{0.75\textwidth}
  %   \includegraphics[width=\textwidth]{vwf_pause}
  % \end{subfigure}
  \caption{Pause time data from all experiments}
  \label{fig:pause-time}
\end{figure}

% \begin{figure}
%   \centering
%   \includegraphics[width=\textwidth]{avg_vel_col.png}
%   \caption{Comparison of the overall average velocity in the four
%     collagen experimental conditions, along with distributions of
%     average velocities from simulations with the parameters found
%     above.}
%   \label{fig:avg-vel-col}
% \end{figure}

\begin{figure}
  \centering
  \includegraphics[width=\textwidth]{fbg_vel_sim.png}
  \caption{Comparison of the overall average velocity in the four
    fibrinogen experimental conditions, along with distributions of
    average velocities from simulations with the parameters found
    above.}
  \label{fig:avg-vel-fib}
\end{figure}

\begin{figure}
  \centering
%   \begin{subfigure}{0.48\textwidth}
%     \includegraphics[width=\textwidth]{col_prp_traj}
%   \end{subfigure}
%   \hfill
%   \begin{subfigure}{0.48\textwidth}
%     \includegraphics[width=\textwidth]{col_whl_traj}
%   \end{subfigure}
%   \\
  \begin{subfigure}{\textwidth}
    \includegraphics[width=\textwidth]{fbg_prp_traj_sim}
    \caption{Trajectories in PRP}
  \end{subfigure}
  \\
  \begin{subfigure}{\textwidth}
    \includegraphics[width=\textwidth]{fbg_whl_traj_sim}
    \caption{Trajectories in whole blood}
  \end{subfigure}
%   \\
%   \begin{subfigure}{0.48\textwidth}
%     \includegraphics[width=\textwidth]{vwf_prp_traj}
%   \end{subfigure}
  % \caption{Trajectory data from the 10 different experiments. Unprimed
  %   platelets are in blue, and primed platelets are in red}
  \label{fig:traj-plots}
\end{figure}

% Finally, in an attempt to (crudely) describe the initial steps in the platelet
% trajectories, we again computed a ``capture rate'' from the time
% between the start of a trajectory and the time to first binding. In
% the collagen experiments, where there is the most data, there is a
% clear increase in the capture rate in the whole blood experiments
% relative to PRP, but not a clear relationship between primed and
% unprimed platelets.

% \begin{figure}
%   \centering
%   \begin{subfigure}{0.48\textwidth}
%     \includegraphics[width=\textwidth]{ccp_capture.png}
%     \caption{CC PRP: $k_\tn{capture} = 0.52 / s$}
%   \end{subfigure}
%   \hfill
%   \begin{subfigure}{0.48\textwidth}
%     \includegraphics[width=\textwidth]{hcp_capture.png}
%     \caption{HC PRP: $k_\tn{capture} = 0.63 / s$}
%   \end{subfigure}
%   \\
%   \begin{subfigure}{0.48\textwidth}
%     \includegraphics[width=\textwidth]{ccw_capture.png}
%     \caption{CC whole blood: $k_\tn{capture} = 1.59 / s$}
%   \end{subfigure}
%   \hfill
%   \begin{subfigure}{0.48\textwidth}
%     \includegraphics[width=\textwidth]{hcw_capture.png}
%     \caption{HC whole blood: $k_\tn{capture} = 1.12 / s$}
%   \end{subfigure}
%   \caption{Histograms of the initial step time. The two experiments
%     with whole blood have faster ``capture'' rates than the two
%     experiments in PRP.}
%   \label{fig:col-capture}
% \end{figure}

% \begin{figure}
%   \centering
%   \begin{subfigure}{0.48\textwidth}
%     \includegraphics[width=\textwidth]{ffp_capture.png}
%     \caption{FF PRP: $k_\tn{capture} = 0.67 / s$}
%   \end{subfigure}
%   \hfill
%   \begin{subfigure}{0.48\textwidth}
%     \includegraphics[width=\textwidth]{hfp_capture.png}
%     \caption{HF PRP: $k_\tn{capture} = 1.42 / s$}
%   \end{subfigure}
%   \\
%   \begin{subfigure}{0.48\textwidth}
%     \includegraphics[width=\textwidth]{ffw_capture.png}
%     \caption{FF whole blood: $k_\tn{capture} = 1.17 / s$}
%   \end{subfigure}
%   \hfill
%   \begin{subfigure}{0.48\textwidth}
%     \includegraphics[width=\textwidth]{hfw_capture.png}
%     \caption{HF whole blood: $k_\tn{capture} = 0.81 / s$}
%   \end{subfigure}
%   \caption{Histograms of the initial step time on fibrinogen.}
%   \label{fig:fbg-capture}
% \end{figure}

% \begin{figure}
%   \centering
%   \begin{subfigure}{0.48\textwidth}
%     \includegraphics[width=\textwidth]{vvp_capture.png}
%     \caption{VV PRP: $k_\tn{capture} = 1.89 / s$}
%   \end{subfigure}
%   \hfill
%   \begin{subfigure}{0.48\textwidth}
%     \includegraphics[width=\textwidth]{hvp_capture.png}
%     \caption{HV PRP: $k_\tn{capture} = 0.32 / s$}
%   \end{subfigure}
%   \caption{Histograms of the initial step time on vWF.}
%   \label{fig:vwf-capture}
% \end{figure}

% \bibliographystyle{plain}
% \bibliography{/Users/andrewwork/Documents/grad-school/thesis/library}

\end{document}




