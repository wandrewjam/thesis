\documentclass{article}

\newcommand{\ep}{\rule{.06in}{.1in}}
\textheight 9.5in

\usepackage{amssymb}
\usepackage{amsmath}
\usepackage{amsthm}
\usepackage{graphicx, subcaption, algorithmic}
\graphicspath{{/Users/andrewwork/thesis/jump-velocity/plots/}}

\usepackage{tikz, pgfplots, chemfig}
% \usepgfplotslibrary{colorbrewer, statistics}
% \pgfplotsset{
%   exact axis/.style={grid=major, minor tick num=4, xlabel=$v^*$,
%     legend entries={PDF, CDF},},
%   every axis plot post/.append style={thick},
%   table/search
%   path={/Users/andrewwork/thesis/jump-velocity/dat-files},
%   colormap/YlGnBu,
%   cycle list/Set1-5,
%   legend style={legend cell align=left,},
% }

% \usepgfplotslibrary{external}
% \tikzexternalize

\renewcommand{\arraystretch}{1.2}
\pagestyle{empty} 
\oddsidemargin -0.25in
\evensidemargin -0.25in 
\topmargin -0.75in 
\parindent 0pt
\parskip 12pt
\textwidth 7in
%\font\cj=msbm10 at 12pt

\newcommand{\tn}{\textnormal}
\newcommand{\stiff}{\frac{k_f}{\gamma}}
\newcommand{\dd}{d}
\newcommand{\Der}[2]{\frac{\dd #1}{\dd #2}}
\newcommand{\Pder}[2]{\frac{\partial #1}{\partial #2}}
\newcommand{\Integral}[4]{\int_{#3}^{#4} {#1} \dd #2}
\DeclareMathOperator{\Exp}{Exp}

% Text width is 7 inches

\def\R{\mathbb{R}}
\def\N{\mathbb{N}}
\def\C{\mathbb{C}}
\def\Z{\mathbb{Z}}
\def\Q{\mathbb{Q}}
\def\H{\mathbb{H}}
\def\B{\mathcal{B}} 
%\topmargin -.5in 

\setcounter{secnumdepth}{2}
\begin{document}
\pagestyle{plain}

\begin{center}
  {\Large Meeting Notes (\today)}
\end{center}

\large{\textbf{Regularized Stokeslets}}
\begin{itemize}
\item Found analytical solutions for the rotation matrix and
  shear-generated torque on an ellipsoid, and verified convergence of
  the regularized stokeslet code
\item Wrote a function to evaluate the wall-bounded stokeslet: it
  runs, but is failing the first test case (the wall-bounded stokeslet
  should vanish on the wall).
\end{itemize}

\large{\textbf{Jump-Velocity Model}}

\begin{figure}[h]
  \centering
  \schemestart
  $U$ \arrow(u1--vv){<=>[$k_\tn{on}$][$k_\tn{off}$]} $V$
  \arrow(@u1--ff){<=>[*{0}$k_\tn{on}^F$][*{0}$k_\tn{off}^F$]}[-90] $F$
  \arrow(--vf){<=>[$k_\tn{on}$][$k_\tn{off}$]} $VF$
  \arrow(@vv--@vf){<=>[*{0}$k_\tn{on}^F$][*{0}$k_\tn{off}^F$]}
  \schemestop
  \label{fig:primed-states}
\end{figure}

\begin{itemize}
\item Last time: simulated step times were much shorter than observed
  step times in the hfw and ffp experiments, and simulated velocities
  were much slower (Figures \ref{fig:step-time} and
  \ref{fig:avg-vel-fib}).
\item This is not due to a bug, this is a consequence of the short
  simulation domain. In these plots, I take the length of the domain
  in the simulation to be $2.5 \, \mu m$, which is the average
  distance traveled between the first and last dwells in the observed
  platelet trajectories. However this is much too short to allow the
  long observed step times. With an average moving velocity of
  $5 \, \mu m / s$, a platelet would cross the entire domain in one
  $0.5 \, s$ step. Therefore because of the short domain, we can only
  see short steps.
\item To validate this assertion, I ran several simulations with
  varying domain lengths $L$, and compared the observed step time
  distribution with the theoretical step time distribution in an
  infinite domain. The results (plotted in Figure
  \ref{fig:step-size-test}) show the observed step time distribution
  converging the the infinite-domain step time distribution as the
  length of the domain increases.
\item For another validation of this, I increased the length of the simulation
  domain from $L = 2.5 \mu m$ to $L = 50 \mu m$. This reduced the
  difference between the observed and simulated step times, as well as
  the difference between the observed and simulated average velocities
  (Figures \ref{fig:step-time-long} and \ref{fig:avg-vel-fib-long}).
\item This means that estimating the effective platelet on rate as the
  inverse of the mean step time \emph{overestimates} the effective on
  rate.
\item Question: how to estimate the effective on rate?
\item We need to incorporate the bias that comes from a finite domain
  somehow, because the step times are not small relative to the time
  it takes a platelet to cross the domain.
\item Thought: add an ``escape'' term in the jump-velocity model. I
  think this assumption reflects the data better than assuming a
  small, finite domain.
  \begin{figure}[h]
  \centering
  \schemestart
  $U$ \arrow(u1--vv){<=>[$k_\tn{on}$][$k_\tn{off}$]} $V$
  \arrow(@u1--ff){<=>[*{0}$k_\tn{on}^F$][*{0}$k_\tn{off}^F$]}[-90] $F$
  \arrow(--vf){<=>[$k_\tn{on}$][$k_\tn{off}$]} $VF$
  \arrow(@vv--@vf){<=>[*{0}$k_\tn{on}^F$][*{0}$k_\tn{off}^F$]}
  \arrow(@u1--ww){->[*{0}$k_\tn{escape}$]}[90]
  \arrow(@vv--vw){->[*{0}$k_\tn{escape}$]}[90]
  \schemestop
\end{figure}
\end{itemize}

\begin{figure}
  \centering
  \includegraphics[width=\textwidth]{fbg_step_sim}
  \caption{Step time data from all experiments}
  \label{fig:step-time}
\end{figure}

\begin{figure}
  \centering
  \includegraphics[width=\textwidth]{fbg_vel_sim}
  \caption{Comparison of the overall average velocity in the four
    fibrinogen experimental conditions, along with distributions of
    average velocities from simulations with the parameters found
    above.}
  \label{fig:avg-vel-fib}
\end{figure}

\begin{figure}
  \centering
  \begin{subfigure}{\textwidth}
    \includegraphics[width=\textwidth]{fbg_prp_traj_sim}
    \caption{Observed and simulated (with $L=2.5 \, \mu m$)
      trajectories in PRP.}
  \end{subfigure}
  \\
  \begin{subfigure}{\textwidth}
    \includegraphics[width=\textwidth]{fbg_whl_traj_sim}
    \caption{Observed and simulated (with $L=2.5 \, \mu m$)
      trajectories in whole blood.}
  \end{subfigure}
  \label{fig:traj-plots}
\end{figure}

\begin{figure}
  \centering
  \begin{subfigure}{0.49\textwidth}
    \includegraphics[width=\textwidth]{{L_2.5}.png}
  \end{subfigure}
  \hfill
  \begin{subfigure}{0.49\textwidth}
    \includegraphics[width=\textwidth]{L_10.png}
  \end{subfigure}
  \\
  \begin{subfigure}{0.49\textwidth}
    \includegraphics[width=\textwidth]{L_100.png}
  \end{subfigure}
  \caption{Step sizes observed in simulations vs. the theoretical step
    size distribution in an infinite domain.}
  \label{fig:step-size-test}
\end{figure}

\begin{figure}
  \centering
  \includegraphics[width=\textwidth]{fbg_step_long}
  \caption{Step time data from all experiments, with
    $L = 50 \, \mu m$.}
  \label{fig:step-time-long}
\end{figure}

\begin{figure}
  \centering
  \includegraphics[width=\textwidth]{fbg_vel_sim_long}
  \caption{Comparison of the overall average velocity in the four
    fibrinogen experimental conditions, along with distributions of
    average velocities from simulations with the parameters found
    above ($L = 50 \, \mu m$).}
  \label{fig:avg-vel-fib-long}
\end{figure}

\begin{figure}
  \centering
  \begin{subfigure}{\textwidth}
    \includegraphics[width=\textwidth]{fbg_prp_traj_sim_long}
    \caption{Observed and simulated (with $L=50 \, \mu m$)
      trajectories in PRP.}
  \end{subfigure}
  \\
  \begin{subfigure}{\textwidth}
    \includegraphics[width=\textwidth]{fbg_whl_traj_sim_long}
    \caption{Observed and simulated (with $L=50 \, \mu m$)
      trajectories in whole blood.}
  \end{subfigure}
  \label{fig:traj-plots-long}
\end{figure}

% \bibliographystyle{plain}
% \bibliography{/Users/andrewwork/Documents/grad-school/thesis/library}

\end{document}




