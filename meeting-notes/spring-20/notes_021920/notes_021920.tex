\documentclass{article}

\newcommand{\ep}{\rule{.06in}{.1in}}
\textheight 9.5in

\usepackage{amssymb}
\usepackage{amsmath}
\usepackage{amsthm}
\usepackage{graphicx, subcaption, algorithmic}
\graphicspath{{/Users/andrewwork/thesis/jump-velocity/plots/}}

\usepackage{tikz, pgfplots, chemfig}
\usepgfplotslibrary{colorbrewer, statistics}
\pgfplotsset{
  exact axis/.style={grid=major, minor tick num=4, xlabel=$v^*$,
    legend entries={PDF, CDF},},
  every axis plot post/.append style={thick},
  table/search
  path={/Users/andrewwork/thesis/jump-velocity/dat-files},
  colormap/YlGnBu,
  cycle list/Set1-5,
  legend style={legend cell align=left,},
}

\usepgfplotslibrary{external}
\tikzexternalize

\renewcommand{\arraystretch}{1.2}
\pagestyle{empty} 
\oddsidemargin -0.25in
\evensidemargin -0.25in 
\topmargin -0.75in 
\parindent 0pt
\parskip 12pt
\textwidth 7in
%\font\cj=msbm10 at 12pt

\newcommand{\tn}{\textnormal}
\newcommand{\stiff}{\frac{k_f}{\gamma}}
\newcommand{\dd}{d}
\newcommand{\Der}[2]{\frac{\dd #1}{\dd #2}}
\newcommand{\Pder}[2]{\frac{\partial #1}{\partial #2}}
\newcommand{\Integral}[4]{\int_{#3}^{#4} {#1} \dd #2}
\DeclareMathOperator{\Exp}{Exp}

% Text width is 7 inches

\def\R{\mathbb{R}}
\def\N{\mathbb{N}}
\def\C{\mathbb{C}}
\def\Z{\mathbb{Z}}
\def\Q{\mathbb{Q}}
\def\H{\mathbb{H}}
\def\B{\mathcal{B}} 
%\topmargin -.5in 

\setcounter{secnumdepth}{2}
\begin{document}
\pagestyle{plain}

\begin{center}
  {\Large Notes on fitting the jump velocity model, and implementing
    Regularized Stokeslets (\today)}
\end{center}

\section{Regularized Stokeslets on a sphere}
\label{sec:regul-stok-sphere}

The method of regularized stokeslets requires you to integrate:
\begin{equation}
  \label{eq:reg-stokeslets}
  u_j(\mathbf{x}_0) = \frac{1}{8\pi \mu} \int_{\partial D}
  S_{ij}^\epsilon (\mathbf{x}, \mathbf{x}_0) g_i(\mathbf{x}) ds(\mathbf{x}).
\end{equation}
where $\mu$ is the viscosity of the fluid, $\partial D$ is the surface
of the body (a sphere in the simplest case), $S_{ij}^\epsilon
(\mathbf{x}, \mathbf{x}_0)$ is the regularized stokeslet with blob
parameter $\epsilon$, and $g_i$ is the $i$th component of the traction
on the surface of the body.

Ultimately our goal is to use regularized stokeslets to find the
velocity field of the fluid around an ellipsoidal platelet near a
wall, but we want to test out the method on simpler problems
first. The simplest case is where $\partial D$ is the surface of a
sphere, and $S_{ij}^\epsilon$ is the free-space Stokeslet (i.e. the
flow is unbounded in all directions away from the sphere).

This case is handled in Cortez et. al., 2005 \cite{Cortez2005}. The
first test case they examine is to apply a point force of $\mathbf{f}
= -\frac{3\mu}{2} \mathbf{e_3}$ everywhere on the surface of the
sphere. Analytically, this generates a velocity of $\mathbf{u} =
-\mathbf{e}_3$. The error of the numerical integration of equation
(\ref{eq:reg-stokeslets}) depends both on the size of the mesh $N
\times N$, and the regularization parameter $\epsilon$. Cortez
et. al. provide plots of the L2 error while refining each of these
parameters in turn, which we can compare our integration scheme to.

As shown in Figures \ref{fig:cortez-conv} and \ref{fig:our-conv}, the
convergence results of our numerical scheme is similar to their
published results.

There are two next steps:
\begin{enumerate}
\item Write a function that is able to go ``the other way,'' i.e. to
  compute a distribution of point forces that generate a prescribed
  motion $\mathbf{u}(\mathbf{x})$. This requires building the linear
  system
  \begin{equation}
    \label{eq:linear-system}
    \tilde{\mathbf{u}} = \frac{1}{8\pi\mu} \mathcal{A} \tilde{\mathbf{f}}
  \end{equation}
  where $\tilde{\mathbf{u}}$ and $\tilde{\mathbf{f}}$ are vectors of
  length $3N_\tn{node}$.
\item The other next step is to write a routine that integrates over
  the surface of an ellipsoid. I am thinking this won't be too much
  extra work, we just need an additional coordinate transform to
  convert between coordinates on the ellipsoid and coordinates on the
  sphere. 
\end{enumerate}

\begin{figure}
  \centering
  \includegraphics[width=0.5\textwidth]{cubed_sphere}
  \caption{An example of the discretization scheme on the surface of
    the sphere (from \cite{Portelenelle2018})}
  \label{fig:discretization}
\end{figure}

\begin{figure}
  \centering
  \begin{subfigure}{0.6\textwidth}
    \includegraphics[width=\textwidth]{reg_refine_cortez}
  \end{subfigure}
  \hfill
  \begin{subfigure}{0.35\textwidth}
    \includegraphics[width=\textwidth]{grid_refine_cortez}
  \end{subfigure}
  \caption{Convergence in the $\Delta s$ and $\epsilon$ parameters
    published in \cite{Cortez2005}.}
  \label{fig:cortez-conv}
\end{figure}

\begin{figure}
  \centering
  \begin{subfigure}{0.48\textwidth}
    \includegraphics[width=\textwidth]{reg_refine}
  \end{subfigure}
  \hfill
  \begin{subfigure}{0.48\textwidth}
    \includegraphics[width=\textwidth]{grid_refine}
  \end{subfigure}
  \caption{Convergence in the $\Delta s$ and $\epsilon$ parameters in
    our numerical scheme.}
  \label{fig:our-conv}
\end{figure}

\newpage

\section{Sequential fit of the 4-state jump-velocity model}
\label{sec:sequential-fit-4}

The adiabatic reduction of the 4-state model is motivated by an
observation that the ``free velocities'' of observed platelets is much
lower than expected free-flowing velocities. The experiments are run
with a wall shear rate of 100/s, and so the fluid velocity
$1 \mu\tn{m}$ off the wall is about $100 \mu\tn{m}/s$. However the
observed free velocities (i.e. during a dwell) are much lower than
this. We have also observed that the velocity of a platelet before its
first dwell, and after its last dwell, is much higher than velocity in
between dwells. Similarly, the initial and final steps of a platelet
(i.e. as it enters and leaves the field of view) are much longer than
the lengths of intermediate steps.

One possible explanation of this is that there are brief contacts
between the platelet and the wall which slow the platelet down to well
below the free-flowing velocity, but these binding dynamics occur too
quickly to be detected individually. These bonds are likely
GPVI-collagen bonds because they occur on a relatively fast time
scale. Then the observed pauses may be due to integrin-collagen
binding.

As a first check of this idea, I processed the experimental
trajectories by removing the observed dwells, and then comparing the
resulting distribution of average free velocities to the distribution
of average velocities predicted by the adiabatic reduction of the 2
state model (Figure \ref{fig:avg-free-vel}).

\begin{figure}
  \centering
  \includegraphics[width=\textwidth]{avg_free_velocity.png}
  \caption{Comparison of the average free velocity in four
    experimental conditions. For Fit 1, $a = 0.12$ and $\epsilon =
    0.65$. For Fit 2, $a = 0.13$ and $\epsilon = 0.43$.}
  \label{fig:avg-free-vel}
\end{figure}

The dimensional parameters resulting from the estimates in Fit 2 are
$k_\tn{on} = 17.2 / s$ and $k_\tn{off} = 2.57 / s$. The next step is
to estimate the ``slow'' binding and unbinding parameters. One way to
do this is to simulate the 4-state PDE and find a pair of
$k_\tn{on}^\tn{slow}$ and $k_\tn{off}^\tn{slow}$ that maximize the
likelihood function. However, I ran into an issue while trying this:
the slowest velocities are $\sim 10^{-4}$, and so evaluating the
likelihood function each time requires running the PDE for $\sim 10^5$
time steps, slowing down the optimization procedure.

There is a much easier way to estimate $k_\tn{off}^\tn{slow}$: take
the mean of the dwell times and invert it. This gives the following
estimates for $k_\tn{off}^\tn{slow}$ (in units of $1/s$):
\begin{tabular}{c|c|c}
  & PRP & Whole Blood \\ \hline
  HC & 0.19 & 0.34 \\ \hline
  CC & 0.15 & 0.52 
\end{tabular}

I also processed the trajectories on fibrinogen and vWF to estimate
the fast binding and unbinding parameters on these agonists as
well. The estimates for fibrinogen are $a = 0.084$ and $\epsilon =
0.16$, and the estimates for vWF are $a = 0.13$ and $\epsilon = 0.10$,
which are in the same ballpark as the estimates for the same
nondimensional parameters for the collagen rolling experiments.

There is also just much less data for rolling on fibrinogen and
vWF. No experiment has more than 20 recorded trajectories, and one
experiment only has 2 (HF in PRP). Additionally, many of the
trajectories don't have any recorded pauses (though some have ``near
pauses'').

\bibliographystyle{plain}
\bibliography{/Users/andrewwork/Documents/grad-school/thesis/library}

\end{document}




