\documentclass{article}

\newcommand{\ep}{\rule{.06in}{.1in}}
\textheight 9.5in

\usepackage{amssymb}
\usepackage{amsmath}
\usepackage{amsthm}
\usepackage{graphicx, subcaption, booktabs}
\graphicspath{{/Users/andrewwork/thesis/jump-velocity/plots/}}

\usepackage{tikz, pgfplots, pgfplotstable, chemfig}

% \usepgfplotslibrary{colorbrewer, statistics}
% \pgfplotsset{
%   exact axis/.style={grid=major, minor tick num=4, xlabel=$v^*$,
%     legend entries={PDF, CDF},},
%   every axis plot post/.append style={thick},
%   table/search
%   path={/Users/andrewwork/thesis/jump-velocity/dat-files},
%   colormap/YlGnBu,
%   cycle list/Set1-5,
%   legend style={legend cell align=left,},
% }

% \usepgfplotslibrary{external}
% \tikzexternalize

\renewcommand{\arraystretch}{1.2}
\pagestyle{empty} 
\oddsidemargin -0.25in
\evensidemargin -0.25in 
\topmargin -0.75in 
\parindent 0pt
\parskip 12pt
\textwidth 7in
%\font\cj=msbm10 at 12pt

\newcommand{\tn}{\textnormal}
\newcommand{\stiff}{\frac{k_f}{\gamma}}
\newcommand{\dd}{d}
\newcommand{\Der}[2]{\frac{\dd #1}{\dd #2}}
\newcommand{\Pder}[2]{\frac{\partial #1}{\partial #2}}
\newcommand{\Integral}[4]{\int_{#3}^{#4} {#1} \dd #2}
\DeclareMathOperator{\Exp}{Exp}

% Text width is 7 inches

\def\R{\mathbb{R}}
\def\N{\mathbb{N}}
\def\C{\mathbb{C}}
\def\Z{\mathbb{Z}}
\def\Q{\mathbb{Q}}
\def\H{\mathbb{H}}
\def\B{\mathcal{B}} 
%\topmargin -.5in 

\setcounter{secnumdepth}{2}
\begin{document}
\pagestyle{plain}

\begin{center}
  {\Large Meeting Notes (\today)}
\end{center}

\large{\textbf{Regularized Stokeslets}}

\textbf{Trying to exploit symmetry}
\begin{itemize}
\item Recap: Want to speed up the generation of the quadrature matrix,
  which comes from discretizing the integral equation
  $\mathbf{u}(\mathbf{x}) = \int \underline{\underline{S}}(\mathbf{x},
  \mathbf{x}_0) \mathbf{f}(\mathbf{x}_0) d\mathbf{x}_0$
\item Observation: there is symmetry in the Stokeslet; in
  particular $\underline{\underline{S}}(\mathbf{x}, \mathbf{x}_0) =
  \underline{\underline{S}}^T(\mathbf{x}_0, \mathbf{x})$. After a lot
  of trial and error, I was able to figure out an implementation that
  cuts the compute time in half (Still need to confirm that I haven't
  introduced any logical errors)
\item I've exploited symmetry to reduce the number of operations and
  memory required to compute the Stokeslet. We can also use the
  Stokeslet's symmetry in solving for the unknown forces.
\item Reminder: the unknown stokeslet strengths $\mathbf{f}$ are found
  by solving the linear system $\hat{\mathbf{u}} =
  \underline{\underline{A}} \hat{\mathbf{f}}$ (hatted vectors contain
  the vector quantities $\mathbf{u}$ and $\mathbf{f}$ evaluated at
  each stokeslet point, and are $\in \R^{3N}$)
\item $\underline{\underline{A}}$ is the quadrature matrix, which is
  not symmetric itself, but due to the symmetry of the Stokeslet, can
  be written as a product of a diagonal matrix of the quadrature
  weights and a symmetric (and apparently spd) matrix
  $\hat{S} \in \R^{3N \times 3N}$
\item So we can use Cholesky factorization to solve the system
  $\hat{\mathbf{u}} = \hat{S} \hat{\mathbf{f}}$, and then divide to
  solve the diagonal system. In order to solve the SPD system, I still
  need to allocate a full matrix, but only need to write to the upper
  triangular elements.
\end{itemize}

\textbf{Precompute helper functions}
\begin{itemize}
\item The other experiment I wanted to try was implementing a lookup
  table for the helper functions $H_1$, $H_1'$, $H_2$, $H_2'$, $D_1$,
  and $D_2$.
\item Reminder: The expression for finding the fluid velocity at a
  point $\mathbf{x}_e$ generated by $M$ reg. Stokeslets located at
  points $\mathbf{x}_{0, k}$ is the following:

  \fbox{\includegraphics[width=\textwidth]{rs_eqns}}
\item In testing, the lookup is nearly 20 times faster than direct
  evaluation of the helper function. When I tested an implementation
  of the stokeslet evaluation using a lookup, it resulted in a much
  more modest speedup ($\sim 27\%$ in single core, $\sim 10\%$ in
  multiple core)
\item 
\end{itemize}

\bibliographystyle{plain}
\bibliography{/Users/andrewwork/Documents/grad-school/thesis/library}

\end{document}




