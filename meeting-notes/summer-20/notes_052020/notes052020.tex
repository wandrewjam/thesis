\documentclass{article}

\newcommand{\ep}{\rule{.06in}{.1in}}
\textheight 9.5in

\usepackage{amssymb}
\usepackage{amsmath}
\usepackage{amsthm}
\usepackage{graphicx, subcaption, booktabs}
\graphicspath{{/Users/andrewwork/thesis/jump-velocity/plots/}}

\usepackage{tikz, pgfplots, pgfplotstable, chemfig}

% \usepgfplotslibrary{colorbrewer, statistics}
% \pgfplotsset{
%   exact axis/.style={grid=major, minor tick num=4, xlabel=$v^*$,
%     legend entries={PDF, CDF},},
%   every axis plot post/.append style={thick},
%   table/search
%   path={/Users/andrewwork/thesis/jump-velocity/dat-files},
%   colormap/YlGnBu,
%   cycle list/Set1-5,
%   legend style={legend cell align=left,},
% }

% \usepgfplotslibrary{external}
% \tikzexternalize

\renewcommand{\arraystretch}{1.2}
\pagestyle{empty} 
\oddsidemargin -0.25in
\evensidemargin -0.25in 
\topmargin -0.75in 
\parindent 0pt
\parskip 12pt
\textwidth 7in
%\font\cj=msbm10 at 12pt

\newcommand{\tn}{\textnormal}
\newcommand{\stiff}{\frac{k_f}{\gamma}}
\newcommand{\dd}{d}
\newcommand{\Der}[2]{\frac{\dd #1}{\dd #2}}
\newcommand{\Pder}[2]{\frac{\partial #1}{\partial #2}}
\newcommand{\Integral}[4]{\int_{#3}^{#4} {#1} \dd #2}
\DeclareMathOperator{\Exp}{Exp}

% Text width is 7 inches

\def\R{\mathbb{R}}
\def\N{\mathbb{N}}
\def\C{\mathbb{C}}
\def\Z{\mathbb{Z}}
\def\Q{\mathbb{Q}}
\def\H{\mathbb{H}}
\def\B{\mathcal{B}} 
%\topmargin -.5in 

\setcounter{secnumdepth}{2}
\begin{document}
\pagestyle{plain}

\begin{center}
  {\Large Meeting Notes (\today)}
\end{center}

\textbf{Trying to speed up computations of resistance matrices}
\begin{itemize}
\item The most expensive parts of finding the resistance matrices is
  generating the quadrature matrix for the mesh on the ellipsoid and
  then solving the unknown forces
\item The quadrature matrix comes from discretizing the integral
  equation $\mathbf{u}(\mathbf{x}) = \int
  \underline{\underline{S}}(\mathbf{x}, \mathbf{x}_0)
  \mathbf{f}(\mathbf{x}_0) d\mathbf{x}_0$
\item When I profiled (a smaller version of) the convergence tests
  from last week, generating the regularized Stokeslets was by far the
  most expensive part of the code: 512 sec of the 668 sec of total
  running time.
\item In last week's code, I calculate the regularized Stokeslets by
  operating on an enormous $N \times N \times 3 \times 3$ array, where
  $N$ is the number of nodes in the mesh on the ellipsoid
\item This takes advantage of Numpy's fast array operations, but Numpy
  seems to only ever use one core to do this
\item Modification: split the array into \verb|num_threads|
  sub-arrays and calculate the reg. Stokeslet on each one, then
  recombine
\item Using 16 processes on the department's compute server, this lead
  to a $\sim5 \times$ speedup in generating the reg. Stokeslets on a
  mesh with 7778 nodes (the finest mesh from last week's convergence
  experiments).
\item Another modification: re-used functions that are computed
  multiple times in evaluating the reg. Stokeslet. This lead to a
  small speedup in computing the Stokeslets
\item For reference: The expression for finding the fluid velocity at a point
  $\mathbf{x}_e$ generated by $M$ reg. Stokeslets located at points
  $\mathbf{x}_{0, k}$ is the following:

  \fbox{\includegraphics[width=\textwidth]{rs_eqns}}
\item Some other ideas:
  \begin{itemize}
  \item pre-compute the functions $H_1$, $H_2$, $D_1$, $D_2$, $H_1'$,
    $H_2'$ for different $r$ values, and then lookup the values when I
    need them (though I've tested this idea on simple experiments and
    only seen a small speed-up)
  \item Write the Stokeslet function in FORTRAN (would we expect to
    see much improvement over the vectorized Numpy implementation?)
  \end{itemize}
\end{itemize}

% \textbf{Next}
% \begin{itemize}
% \item I want to go to finer grid meshes for small distances in order
%   to better resolve the errors for those
% \item Parallelize the generation of the linear system to find
%   stokeslet strengths (accounts for a significant proportion of the
%   total computation time)
% \item Try a couple of different orientations for the small distances
%   and just exclude cases where a node is outside the domain
% \item Find the equations of motion for the ellipsoid ($d$ is easy,
%   $\theta$ and $\phi$ require a bit of thought)
% \end{itemize}


% \large{\textbf{Jump-Velocity Model}}

% \begin{figure}[h]
%   \centering
%   \schemestart
%   $U$ \arrow(u1--vv){<=>[$k_\tn{on}$][$k_\tn{off}$]} $V$
%   \arrow(@u1--ff){<=>[*{0}$k_\tn{on}^F$][*{0}$k_\tn{off}^F$]}[-90] $F$
%   \arrow(--vf){<=>[$k_\tn{on}$][$k_\tn{off}$]} $VF$
%   \arrow(@vv--@vf){<=>[*{0}$k_\tn{on}^F$][*{0}$k_\tn{off}^F$]}
%   \arrow(@u1--ww){->[*{0}$k_\tn{escape}$]}[90]
%   \arrow(@vv--vw){->[*{0}$k_\tn{escape}$]}[90]
%   \schemestop
%   \caption{Jump velocity model with escape}
%   \label{fig:escape-diagram}
% \end{figure}

% \begin{itemize}
% \item One prediction from the jump-velocity model with escape is that
%   the number of dwells in a trajectory should be geometrically
%   distributed. Basically, the probability that a platelet re-binds
%   after unbinding is $a = \frac{k_\tn{on}}{k_\tn{on} + k_\tn{escape}}$,
%   and so the number of binding events that occur before the platelet
%   escapes is a geometric distribution ($p(n) = (1 - a) a^{n-1}$).
% \item A platelet that is not bound to the surface has two possible
%   fates: it can either re-bind with probability $a$ (in which case it
%   must unbind at some point again and return to the same state), or it
%   can escape with probability $1 - a$.
%   \begin{center}
%     \includegraphics[width=.3\textwidth]{step-escape.png}
%   \end{center}
% \item Therefore we can estimate $a$ with the mean number of dwells a
%   platelet takes before escaping: $\mu_\tn{dwell num.} = (1 -
%   a)^{-1}$. Figure \ref{fig:ndwells} shows the number of dwells per
%   trajectory in the two whole blood experiments.
  
%   \begin{figure}[h]
%     \centering
%     \includegraphics[width=\textwidth]{num_dwells}
%     \caption{Number of dwells per trajectory in experiments and
%       predicted by a geometric distribution.}
%     \label{fig:ndwells}
%   \end{figure}
% \item We need another estimate to uniquely find $k_\tn{on}$ and
%   $k_\tn{escape}$.
% \item It turns out the distribution of step times is an exponential
%   distribution with rate $k_\tn{tot} = k_\tn{on} + k_\tn{escape}$, and
%   so we can estimate $k_\tn{tot}$ with the average step time:
%   $\mu_\tn{step time} = 1 / k_\tn{tot} = 1 / (k_\tn{on} +
%   k_\tn{escape})$. With these estimates for $k_\tn{on}$ and
%   $k_\tn{escape}$ (along with the dwell time estimate for
%   $k_\tn{off}$), we get the following estimates for the primed and
%   unprimed experiments:
%   \begin{table}[h]
%     \centering
%     \begin{tabular}{rccc} \toprule
%       & $k_\tn{on}$ & $k_\tn{escape}$ & $k_\tn{off}$ \\ \midrule
%       FFW & $5.24 \pm 1.25$ & $2.42 \pm 0.58$ & $0.16 \pm 0.03$ \\
%       HFW & $0.13 \pm 0.12$ & $0.48 \pm 0.47$ & $0.14 \pm 0.06$ \\
%       \bottomrule 
%     \end{tabular}
%     \caption{Table of parameter estimates for the unprimed and primed
%       experiments}
%     \label{tab:par-est}
%   \end{table}
  
%   Based on these estimates, the on rate increases in the primed
%   experiment while the off rate remains unchanged. For the escape
%   rates, we get the (maybe) counter-intuitive result that the escape
%   rate increases in the priming experiment. I don't have an
%   explanation for this; I'll have to do some more thinking about it
%   first. 
% \item Figure \ref{fig:fbg-whl-step} shows the
%   distribution of step times in simulations and experiments, and there
%   is good agreement for both primed and unprimed platelets.

%   \begin{figure}[h]
%     \centering
%     \includegraphics[width=\textwidth]{fbg_whl_step}
%     \caption{Distribution of step times in experiments and in
%       simulations}
%     \label{fig:fbg-whl-step}
%   \end{figure}
% \item There is also good agreement in the escape time (or moving time)
%   between simulations and experiments (Figure \ref{fig:fbg-whl-esc})
%   and the trajectories look qualitatively similar (Figure
%   \ref{fig:fbg-whl-traj-sim}) 

%   \begin{figure}[h]
%     \centering
%     \includegraphics[width=\textwidth]{fbg_whl_esc}
%     \caption{Escape time distributions in experiments (solid lines)
%       and simulations (dashed lines)}
%     \label{fig:fbg-whl-esc}
%   \end{figure}

%   \begin{figure}[h]
%     \centering
%     \includegraphics[width=\textwidth]{fbg_whl_traj_sim}
%     \caption{Experimental and simulated platelet trajectories}
%     \label{fig:fbg-whl-traj-sim}
%   \end{figure}
% \item The distribution of average velocities is still not capturing
%   the long tail on the average velocity distribution (Figure
%   \ref{fig:fbg-whl-vel}) 

%   \begin{figure}[h]
%     \centering
%     \includegraphics[width=\textwidth]{fbg_whl_vel}
%     \caption{Distribution of time-averaged velocities in experiments
%       (solid line) and simulations (dashed line)}
%     \label{fig:fbg-whl-vel}
%   \end{figure}
% % \item Still to do:
% %   \begin{itemize}
% %   \item Figure out if it is tractable to write a distribution of
% %     effective platelet off rates as a function of individual receptor
% %     on/off rates.
% %   \end{itemize}

% \end{itemize}

% \begin{figure}[h]
%   \centering
%   \schemestart
%   $U$ \arrow(u1--vv){<=>[$k_\tn{on}$][$k_\tn{off}$]} $V$
%   \arrow(@u1--ff){<=>[*{0}$k_\tn{on}^F$][*{0}$k_\tn{off}^F$]}[-90] $F$
%   \arrow(--vf){<=>[$k_\tn{on}$][$k_\tn{off}$]} $VF$
%   \arrow(@vv--@vf){<=>[*{0}$k_\tn{on}^F$][*{0}$k_\tn{off}^F$]}
%   \schemestop
%   \label{fig:primed-states}
% \end{figure}

% \begin{itemize}
% \item Last time: step times in simulations only converge to the
%   fit distribution of step times when the length $L$ of the simulation
%   domain is long. Running simulations through a shorter domain biases
%   the simulated distribution of step times to be smaller than expected.
% \item Figure \ref{fig:step-time}---Figure \ref{fig:traj-plots-200} show step
%   time distributions and trajectories from simulations run through
%   domains of various lengths, as well as simulated step time
%   distributions when only looking at trajectories in a $2.5 \, \mu m$
%   window. Restricting the ``viewing window'' in the longer
%   stochastic experiments seems equivalent (and I suspect is) to
%   running the simulations through a short domain.
% \item I think that estimating the effective platelet on rate as the
%   inverse of the mean step time \emph{overestimates} the effective on
%   rate. In our experiments, longer steps are more likely to get cut
%   off (and therefore not counted) because platelets are more likely to
%   leave the domain on a long step. Are longer steps more likely to get
%   cut off in observed trajectories? If they are, then our estimate of
%   effective on-rate is an over-estimate of the true effective on rate.
% \item I also imported Emma's data, and it is plotted in Figure
%   \ref{fig:emma-data}. Platelets are moving much faster in these
%   experiments than in their previous experiments. I plotted the new
%   trajectories alongside the whole blood trajectories, but they didn't
%   mention whether these experiments were done with whole blood or
%   PRP. In either case, the new trajectories have higher velocities
%   than the old trajectories. I've asked them whether these are whole
%   blood or PRP trajectories, and what the shear rate on these
%   experiments are (the old ones are at 100/s wall shear rate.)
% \item We may be able to draw a stronger connection back to
%   mechanism. Qi et. al., 2019, quote a result connecting effective
%   platelet on and off rates back to receptor on/off rates, although
%   they don't give any details on how they use that relation.
% \end{itemize}

% \begin{figure}
%   \centering
%   \includegraphics[width=\textwidth]{fbg_step_sim}
%   \caption{Step time distributions with $L = 2.5 \, \mu m$}
%   \label{fig:step-time}
% \end{figure}

% \begin{figure}
%   \centering
%   \includegraphics[width=\textwidth]{fbg_step_sim_window}
%   \caption{Step time distributions in a $L = 2.5 \, \mu m$ window}
%   \label{fig:step-time-window}
% \end{figure}

% \begin{figure}
%   \centering
%   \includegraphics[width=\textwidth]{fbg_step_long}
%   \caption{Step time distributions with $L = 50 \, \mu m$.}
%   \label{fig:step-time-long}
% \end{figure}

% \begin{figure}
%   \centering
%   \includegraphics[width=\textwidth]{fbg_step_sim_100}
%   \caption{Step time distributions with $L = 100 \, \mu m$}
%   \label{fig:step-time-100}
% \end{figure}

% \begin{figure}
%   \centering
%   \includegraphics[width=\textwidth]{fbg_step_sim_200}
%   \caption{Step time distributions with $L = 200 \, \mu m$.}
%   \label{fig:step-time-200}
% \end{figure}

% \begin{figure}
%   \centering
%   \begin{subfigure}{\textwidth}
%     \includegraphics[width=\textwidth]{fbg_prp_traj_sim}
%     \caption{Observed (left) and simulated (right) (with $L=2.5 \, \mu m$)
%       trajectories in PRP.}
%   \end{subfigure}
%   \\
%   \begin{subfigure}{\textwidth}
%     \includegraphics[width=\textwidth]{fbg_whl_traj_sim}
%     \caption{Observed (left) and simulated (right) (with $L=2.5 \, \mu m$)
%       trajectories in whole blood.}
%   \end{subfigure}
%   \caption{Trajectories of primed (red) vs unprimed (blue) platelets
%     in experiments and simulations.}
%   \label{fig:traj-plots}
% \end{figure}

% \begin{figure}
%   \centering
%   \begin{subfigure}{\textwidth}
%     \includegraphics[width=\textwidth]{fbg_prp_traj_sim_window}
%     \caption{Observed (left) and simulated (right) (in a
%       $L=2.5 \, \mu m$ window) trajectories in PRP.}
%   \end{subfigure}
%   \\
%   \begin{subfigure}{\textwidth}
%     \includegraphics[width=\textwidth]{fbg_whl_traj_sim_window}
%     \caption{Observed (left) and simulated (right) (in a
%       $L=2.5 \, \mu m$ window) trajectories in whole blood.}
%   \end{subfigure}
%   \caption{Trajectories of primed (red) vs unprimed (blue) platelets
%     in experiments and simulations.}
%   \label{fig:traj-plots-window}
% \end{figure}

% \begin{figure}
%   \centering
%   \begin{subfigure}{\textwidth}
%     \includegraphics[width=\textwidth]{fbg_prp_traj_sim_long}
%     \caption{Observed (left) and simulated (right) (with $L=50 \, \mu m$)
%       trajectories in PRP.}
%   \end{subfigure}
%   \\
%   \begin{subfigure}{\textwidth}
%     \includegraphics[width=\textwidth]{fbg_whl_traj_sim_long}
%     \caption{Observed (left) and simulated (right) (with $L=50 \, \mu m$)
%       trajectories in whole blood.}
%   \end{subfigure}
%   \caption{Trajectories of primed (red) vs unprimed (blue) platelets
%     in experiments and simulations.}
%   \label{fig:traj-plots-long}
% \end{figure}

% \begin{figure}
%   \centering
%   \begin{subfigure}{\textwidth}
%     \includegraphics[width=\textwidth]{fbg_prp_traj_sim_100}
%     \caption{Observed (left) and simulated (right) (with $L=100 \, \mu m$)
%       trajectories in PRP.}
%   \end{subfigure}
%   \\
%   \begin{subfigure}{\textwidth}
%     \includegraphics[width=\textwidth]{fbg_whl_traj_sim_100}
%     \caption{Observed (left) and simulated (right) (with $L=100 \, \mu m$)
%       trajectories in whole blood.}
%   \end{subfigure}
%   \caption{Trajectories of primed (red) vs unprimed (blue) platelets
%     in experiments and simulations.}
%   \label{fig:traj-plots-100}
% \end{figure}

% \begin{figure}
%   \centering
%   \begin{subfigure}{\textwidth}
%     \includegraphics[width=\textwidth]{fbg_prp_traj_sim_200}
%     \caption{Observed (left) and simulated (right) (with $L=200 \, \mu m$)
%       trajectories in PRP.}
%   \end{subfigure}
%   \\
%   \begin{subfigure}{\textwidth}
%     \includegraphics[width=\textwidth]{fbg_whl_traj_sim_200}
%     \caption{Observed (left) and simulated (right) (with $L=200 \, \mu m$)
%       trajectories in whole blood.}
%   \end{subfigure}
%   \caption{Trajectories of primed (red) vs unprimed (blue) platelets
%     in experiments and simulations.}
%   \label{fig:traj-plots-200}
% \end{figure}

% \begin{figure}
%   \centering
%   \includegraphics[width=\textwidth]{emma_data_traj}
%   \caption{Trajectory plots of Emma's data (plotted in magenta
%     (primed) and cyan (unprimed)), along with the previous
%     fibrinogen whole blood trajectories}
%   \label{fig:emma-data}
% \end{figure}

\bibliographystyle{plain}
\bibliography{/Users/andrewwork/Documents/grad-school/thesis/library}

\end{document}




