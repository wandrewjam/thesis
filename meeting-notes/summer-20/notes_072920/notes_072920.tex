\documentclass{article}

\newcommand{\ep}{\rule{.06in}{.1in}}
\textheight 9.5in

\usepackage{amssymb, bm}
\usepackage{amsmath}
\usepackage{amsthm}
\usepackage{graphicx, subcaption, booktabs}
\graphicspath{{/Users/andrewwork/thesis/jump-velocity/plots/}}

\usepackage{tikz, pgfplots, pgfplotstable, chemfig, xcolor}

% \usepgfplotslibrary{colorbrewer, statistics}
% \pgfplotsset{
%   exact axis/.style={grid=major, minor tick num=4, xlabel=$v^*$,
%     legend entries={PDF, CDF},},
%   every axis plot post/.append style={thick},
%   table/search
%   path={/Users/andrewwork/thesis/jump-velocity/dat-files},
%   colormap/YlGnBu,
%   cycle list/Set1-5,
%   legend style={legend cell align=left,},
% }

% \usepgfplotslibrary{external}
% \tikzexternalize

\renewcommand{\arraystretch}{1.2}
\pagestyle{empty} 
\oddsidemargin -0.25in
\evensidemargin -0.25in 
\topmargin -0.75in 
\parindent 0pt
\parskip 12pt
\textwidth 7in
%\font\cj=msbm10 at 12pt

\newcommand{\tn}{\textnormal}
\newcommand{\stiff}{\frac{k_f}{\gamma}}
\newcommand{\dd}{d}
\newcommand{\Der}[2]{\frac{\dd #1}{\dd #2}}
\newcommand{\Pder}[2]{\frac{\partial #1}{\partial #2}}
\newcommand{\Integral}[4]{\int_{#3}^{#4} {#1} \dd #2}
\newcommand{\vect}[1]{\boldsymbol{\mathbf{#1}}}
\newcommand{\mat}[1]{\underline{\underline{#1}}}
\DeclareMathOperator{\Exp}{Exp}

% Text width is 7 inches

\def\R{\mathbb{R}}
\def\N{\mathbb{N}}
\def\C{\mathbb{C}}
\def\Z{\mathbb{Z}}
\def\Q{\mathbb{Q}}
\def\H{\mathbb{H}}
\def\B{\mathcal{B}} 
%\topmargin -.5in 

\setcounter{secnumdepth}{2}
\begin{document}
\pagestyle{plain}

\begin{center}
  {\Large Meeting Notes (\today)}
\end{center}

{\large \textbf{Follow-up from last week}}

\begin{figure}[h!]
  \centering
  \begin{subfigure}{0.49\textwidth}
    \includegraphics[width=\textwidth]{orient_plot17_2nd}
  \end{subfigure}
  \hfill
  \begin{subfigure}{0.49\textwidth}
    \includegraphics[width=\textwidth]{orient_err_plot17_2nd}
  \end{subfigure}
  \\
  \begin{subfigure}{0.49\textwidth}
    \includegraphics[width=\textwidth]{com_plot17_2nd}
  \end{subfigure}
  \caption{Plots of the ellipsoid orientation, orientation error, and
     center of mass in a shear flow near a wall. The height of the
     center of mass is initialized at $1.5$. Orientation is initialized
     at $\vect{e}_m = \vect{e}_x$. In this experiment, the platelet
     completely flips over a couple of times, and after each flip the
     center of mass (nearly) returns to the starting position before starting
     the next flip. It very slowly drifts away (after each flip it
     drifts about .0004 farther away) but this well below the
     estimated error of the simulations.}
  \label{fig:plt17}
\end{figure}

\begin{figure}[h!]
  \centering
  \begin{subfigure}{0.49\textwidth}
    \includegraphics[width=\textwidth]{orient_plot18_2nd}
  \end{subfigure}
  \hfill
  \begin{subfigure}{0.49\textwidth}
    \includegraphics[width=\textwidth]{orient_err_plot18_2nd}
  \end{subfigure}
  \\
  \begin{subfigure}{0.49\textwidth}
    \includegraphics[width=\textwidth]{com_plot18_2nd}
  \end{subfigure}
  \caption{Plots of the ellipsoid orientation, orientation error, and
     center of mass in a shear flow near a wall. The height of the
     center of mass is initialized at $1.0$. Orientation is initialized
     at $\vect{e}_m = \vect{e}_x$. In this experiment, the platelet
     completely flips over a couple of times. After the 1st flip the
     platelet is much farther away from the wall, and continues to
     move farther away after each successive flip.}
  \label{fig:plt18}
\end{figure}

\begin{figure}[h!]
  \centering
  \begin{subfigure}{0.49\textwidth}
    \includegraphics[width=\textwidth]{orient_plot51_2nd}
  \end{subfigure}
  \hfill
  \begin{subfigure}{0.49\textwidth}
    \includegraphics[width=\textwidth]{orient_err_plot51_2nd}
  \end{subfigure}
  \\
  \begin{subfigure}{0.49\textwidth}
    \includegraphics[width=\textwidth]{com_plot51_2nd}
  \end{subfigure}
  \caption{Plots of the ellipsoid orientation, orientation error, and
     center of mass in a shear flow near a wall. The height of the
     center of mass is initialized at $0.8$. Orientation is initialized
     at $\vect{e}_m = \vect{e}_x$. In this simulation, the platelet
     doesn't flip over---at least within the course of this
     simulation---and in fact rotates against the background
     rotation and slides along the wall. The fine solution rotates
     more than the coarse solution. I also ran an experiment starting
     at a distance of $0.6$, which had qualitatively similar results.}
  \label{fig:plt51}
\end{figure}

\begin{figure}[h!]
  \centering
  \begin{subfigure}{0.49\textwidth}
    \includegraphics[width=\textwidth]{orient_plot28_2nd}
  \end{subfigure}
  \hfill
  \begin{subfigure}{0.49\textwidth}
    \includegraphics[width=\textwidth]{orient_err_plot28_2nd}
  \end{subfigure}
  \\
  \begin{subfigure}{0.49\textwidth}
    \includegraphics[width=\textwidth]{com_plot28_2nd}
  \end{subfigure}
  \caption{Plots of the ellipsoid orientation, orientation error, and
     center of mass in a shear flow near a wall. The height of the
     center of mass is initialized at $1.2$. Orientation is initialized
     at $\vect{e}_m = (1/\sqrt{2}, 1/\sqrt{2}, 0)^T$. In this
     experiment, the ellipsoid does not flip over, and it looks like
     the motion is truly periodic. I also ran a similar experiment
     with an intial orientation that was more flat than this one (in
     the next figure).}
  \label{fig:plt28}
\end{figure}

\begin{figure}[h!]
  \centering
  \begin{subfigure}{0.49\textwidth}
    \includegraphics[width=\textwidth]{orient_plot48_2nd}
  \end{subfigure}
  \hfill
  \begin{subfigure}{0.49\textwidth}
    \includegraphics[width=\textwidth]{orient_err_plot48_2nd}
  \end{subfigure}
  \\
  \begin{subfigure}{0.49\textwidth}
    \includegraphics[width=\textwidth]{com_plot48_2nd}
  \end{subfigure}
  \caption{Plots of the ellipsoid orientation, orientation error, and
     center of mass in a shear flow near a wall. The height of the
     center of mass is initialized at $1.2$. Orientation is initialized
     at $\vect{e}_m = (\cos(\pi/8), \sin(\pi/8), 0)^T$. Again, in this
     experiment, the ellipsoid does not flip over, and it looks like
     the motion is truly periodic, even though it is nearly flat at
     the start of the experiment.}
  \label{fig:plt48}
\end{figure}

\newpage

{\large \textbf{Adaptively choosing mesh size}}

\begin{itemize}
\item We want to improve the speed of the regularized Stokeslet motion
  integration by making it adaptive. When the ellipsoid is close to
  the wall, more Stokeslets are required to accurately resolve the
  resistance matrix than when the ellipsoid is far away from the wall.
\item Idea: pick a target regularization parameter ($\epsilon$) based
  on the shortest distance between the ellipsoid surface and the
  wall. Because we pick $\epsilon$ based on the number of nodes, we
  can just invert this to figure out the number of nodes we need to
  reach some target $\epsilon$ value.
\item My thought was to choose $\epsilon$ so that
  $\epsilon <= \mathcal{O}(s^p)$ where $s$ is the separation distance
  between the ellipsoid and wall, and $p$ is some pre-chosen value. As
  a first pass, I tried taking $\epsilon <= s$.
\item I ran two experiments with the adaptive solver: the first was
  with an initially flat ellipsoid and a CoM height of 1.5, and the
  second was with a CoM height of 1.2.
\item The first experiment is shown in Figure \ref{fig:plt71}. 
\item The second experiment is still running; either I set the maximum
  node size to be too large, or the adaptive routine is too sensitive
  to small separations.
\end{itemize}

\begin{figure}[h!]
  \centering
  \begin{subfigure}{0.49\textwidth}
    \includegraphics[width=\textwidth]{orient_plot71_2nd}
  \end{subfigure}
  \hfill
  \begin{subfigure}{0.49\textwidth}
    \includegraphics[width=\textwidth]{orient_err_plot71_2nd}
  \end{subfigure}
  \\
  \begin{subfigure}{0.49\textwidth}
    \includegraphics[width=\textwidth]{com_plot71_2nd}
  \end{subfigure}
  \hfill
  \begin{subfigure}{0.49\textwidth}
    \includegraphics[width=\textwidth]{nsep_71}
  \end{subfigure}  
  \caption{Plots of the ellipsoid orientation, orientation error,
    center of mass, and the $N$ value with the plt-wall
    separation. The height of the center of mass is initialized at
    $1.5$. Orientation is initialized at $\vect{e}_m =
    \vect{e}_x$. Throughout almost the entire simulation, the adaptive
  solver used the minimum mesh size, and the solution is nearly
  identical to the experiment with only using a coarse mesh (Figure )}
  \label{fig:plt71}
\end{figure}

\begin{figure}[h!]
  \centering
  \begin{subfigure}{0.49\textwidth}
    \includegraphics[width=\textwidth]{orient_plot12_2nd}
  \end{subfigure}
  \hfill
  \begin{subfigure}{0.49\textwidth}
    \includegraphics[width=\textwidth]{orient_err_plot12_2nd}
  \end{subfigure}
  \\
  \begin{subfigure}{0.49\textwidth}
    \includegraphics[width=\textwidth]{com_plot12_2nd}
  \end{subfigure}
  \caption{Plots of the ellipsoid orientation, orientation error, and
     center of mass in a shear flow near a wall. The height of the
     center of mass is initialized at $1.5$. Orientation is
     initialized at $\vect{e}_m = \vect{e}_x$. The coarse solution
     used the same number of nodes as the minimum node size in the
     adaptive code.}
  \label{fig:plt12}
\end{figure}

\bibliographystyle{plain}
\bibliography{/Users/andrewwork/Documents/grad-school/thesis/library}

\end{document}




