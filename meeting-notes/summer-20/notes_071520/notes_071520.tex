\documentclass{article}

\newcommand{\ep}{\rule{.06in}{.1in}}
\textheight 9.5in

\usepackage{amssymb, bm}
\usepackage{amsmath}
\usepackage{amsthm}
\usepackage{graphicx, subcaption, booktabs}
\graphicspath{{/Users/andrewwork/thesis/jump-velocity/plots/}}

\usepackage{tikz, pgfplots, pgfplotstable, chemfig, xcolor}

% \usepgfplotslibrary{colorbrewer, statistics}
% \pgfplotsset{
%   exact axis/.style={grid=major, minor tick num=4, xlabel=$v^*$,
%     legend entries={PDF, CDF},},
%   every axis plot post/.append style={thick},
%   table/search
%   path={/Users/andrewwork/thesis/jump-velocity/dat-files},
%   colormap/YlGnBu,
%   cycle list/Set1-5,
%   legend style={legend cell align=left,},
% }

% \usepgfplotslibrary{external}
% \tikzexternalize

\renewcommand{\arraystretch}{1.2}
\pagestyle{empty} 
\oddsidemargin -0.25in
\evensidemargin -0.25in 
\topmargin -0.75in 
\parindent 0pt
\parskip 12pt
\textwidth 7in
%\font\cj=msbm10 at 12pt

\newcommand{\tn}{\textnormal}
\newcommand{\stiff}{\frac{k_f}{\gamma}}
\newcommand{\dd}{d}
\newcommand{\Der}[2]{\frac{\dd #1}{\dd #2}}
\newcommand{\Pder}[2]{\frac{\partial #1}{\partial #2}}
\newcommand{\Integral}[4]{\int_{#3}^{#4} {#1} \dd #2}
\newcommand{\vect}[1]{\boldsymbol{\mathbf{#1}}}
\newcommand{\mat}[1]{\underline{\underline{#1}}}
\DeclareMathOperator{\Exp}{Exp}

% Text width is 7 inches

\def\R{\mathbb{R}}
\def\N{\mathbb{N}}
\def\C{\mathbb{C}}
\def\Z{\mathbb{Z}}
\def\Q{\mathbb{Q}}
\def\H{\mathbb{H}}
\def\B{\mathcal{B}} 
%\topmargin -.5in 

\setcounter{secnumdepth}{2}
\begin{document}
\pagestyle{plain}

\begin{center}
  {\Large Meeting Notes (\today)}
\end{center}

{\Large \textbf{Rolling tests with an ellipsoid in an unbounded shear
    flow, and near a wall}}

\begin{itemize}
\item As an extension from the previous experiments with a sphere
  rolling near a plane wall, I next ran experiments with an ellipsoid
  with a minor axis of length $b = 0.5$, and major axes with length $a
  = 1.5$.
\item I ran tests with the ellipsoid in free space with three different
  initial configurations, and two different mesh widths. The coarse
  mesh had 386 Stokeslets on the platelet surface, and the fine mesh
  width had 1538 Stokeslets. These two meshes had approximate widths
  of $0.18$ and $0.09$ respectively.
\item There are analytic solutions for the motion of an ellipsoid in
  an unbounded shear flow, I used the expressions given in
  \cite{Kim86}, which give the angular velocities as a function of the
  ellipsoid orientation. Then I numerically integrated the ODE
  $\frac{d\vect{e}_m}{dt} = \vect{\Omega} \times \vect{e}_m$ to get the
  ``exact'' solution for the orientation of the platelet as a function
  of time.
\item Using the formulas from \cite{Kim86}, I also compared the
  approximate angular velocities found by the regularized Stokeslets
  method with the exact angular velocities at each time step, and
  plotted the results.
\item Plots of the orientation vector components, the error in
  orientation as a function of time, and the error in the velocity
  estimate at each step are plotted in Figures
  \ref{fig:first-free-test}, \ref{fig:second-free-test}, and
  \ref{fig:third-free-test}.
\item In all three of these simulations, the orientation vector (the
  minor axis) rotates around the $y$-axis. These are the Jeffery
  orbits from \cite{Jeffery22}.
\item The errors are largest in Figure \ref{fig:first-free-test},
  where the orientation vector starts at $\vect{e}_x$, and is always
  in the $x$--$z$ plane. The errors in the orientation get quite large
  ($\sim 20\%$ for the coarse mesh, and $\sim 10\%$ for the fine
  mesh), however this only occurs briefly--where the angular
  velocities are the largest--and then the errors return to about 1/4
  of their maximum point once the platelet has fully flipped over.
\item Similar to the experiments with a sphere rolling near a wall,
  the regularized Stokeslets method \emph{over}estimates the angular
  velocity when the ellipsoid is ``flat'' in the shear flow ($\omega_y
  < 0$, so the negative error shown in the plots corresponds to an
  overestimate of the magnitude of the angular velocity). However,
  when the minor axis is pointing in the direction of flow (the
  platelet is ``vertical'' in the shear flow), the regularized
  Stokeslets underestimate the angular velocity.
\item When the orientation vector starts at $(1/\sqrt{2}, 1/\sqrt{2},
  0)$, the errors are more than 2$\times$ smaller than the previous
  case. There isn't much else to point out here, except that the
  accuracy again improves by a factor of roughly 2 when the mesh width
  is decreased.
\item In the third free experiment, the orientation is initialized to
  $\vect{e}_m = \vect{e}_y$. Here the orientation of the ellipsoid
  remains constant, and there is some kind of symmetry that results in
  errors canceling, and regularized Stokeslets accurately find the
  angular velocities within machine precision (Figure
  \ref{fig:third-free-test}). 
\item In the three remaining figures, I show errors from experiments
  where the ellipsoid was placed near a plane wall (all with an
  initial orientation $\vect{e}_m = \vect{e}_x$). The center of the
  ellipsoid was placed at initial heights of $1.5$, $1.2$, and $1.0$
  in Figures \ref{fig:first-wall-test}, \ref{fig:second-wall-test},
  and \ref{fig:third-wall-test} respectively.
\item In these experiments, ``exact'' solutions were found by using an
  even finer mesh ($\sim 1/3$ the width of the coarse mesh). The
  ``exact'' simulations took $\sim 4.5$ hours to run on a single core
  for 200 time steps. This means each time step takes about 80 sec.
\item In Figure \ref{fig:first-wall-test}, we can nearly see a full
  flip, and the shape of the orientation error is similar to the
  unbounded case. That is, the coarse mesh overestimates the angular
  velocity when the platelet is flat, and underestimates it when the
  platelet is vertical.
\item In the center of mass plot, we can see that the height of the
  platelet increases slightly when the platelet is vertical. This has
  to happen in order to give the platelet room to flip over.
\item In the other two experiments, we don't get a full flip. I will
  need to run a longer simulation in order to see that.
\item I am also going to work on implementing receptor binding this
  week. I'm thinking of tracking every receptor on the cell surface,
  instead of the hybrid method I used for the 2D model. 
\end{itemize}

\begin{figure}
  \centering
  \begin{subfigure}{0.49\textwidth}
    \includegraphics[width=\textwidth]{orient_plot11}
  \end{subfigure}
  \hfill
  \begin{subfigure}{0.49\textwidth}
    \includegraphics[width=\textwidth]{orient_plot16}
  \end{subfigure}
  \\
  \begin{subfigure}{0.49\textwidth}
    \includegraphics[width=\textwidth]{orient_err_plot11}
  \end{subfigure}
  \hfill
  \begin{subfigure}{0.49\textwidth}
    \includegraphics[width=\textwidth]{orient_err_plot16}
  \end{subfigure}
  \\
  \begin{subfigure}{0.49\textwidth}
    \includegraphics[width=\textwidth]{vel_err_plot11}
  \end{subfigure}
  \hfill
  \begin{subfigure}{0.49\textwidth}
    \includegraphics[width=\textwidth]{vel_err_plot16}
  \end{subfigure}
  \caption{Plots of the ellipsoid orientation, orientation error, and
    velocity error in an unbounded shear flow. Orientation is
    initialized at $\vect{e}_m = \vect{e}_x$. The three plots in the
    left column are with the coarse mesh, and the three plots in the
    right column are with the fine mesh. I didn't plot the center of
    mass, because it remains constant.}
  \label{fig:first-free-test}
\end{figure}

\begin{figure}
  \centering
  \begin{subfigure}{0.49\textwidth}
    \includegraphics[width=\textwidth]{orient_plot21}
  \end{subfigure}
  \hfill
  \begin{subfigure}{0.49\textwidth}
    \includegraphics[width=\textwidth]{orient_plot26}
  \end{subfigure}
  \\
  \begin{subfigure}{0.49\textwidth}
    \includegraphics[width=\textwidth]{orient_err_plot21}
  \end{subfigure}
  \hfill
  \begin{subfigure}{0.49\textwidth}
    \includegraphics[width=\textwidth]{orient_err_plot26}
  \end{subfigure}
  \\
  \begin{subfigure}{0.49\textwidth}
    \includegraphics[width=\textwidth]{vel_err_plot21}
  \end{subfigure}
  \hfill
  \begin{subfigure}{0.49\textwidth}
    \includegraphics[width=\textwidth]{vel_err_plot26}
  \end{subfigure}
  \caption{Plots of the ellipsoid orientation, orientation error, and
    velocity error in an unbounded shear flow. Orientation is
    initialized at $\vect{e}_m = (1/\sqrt{2}, 1/\sqrt{2}, 0)^T$.}
  \label{fig:second-free-test}
\end{figure}

\begin{figure}
  \centering
  \begin{subfigure}{0.49\textwidth}
    \includegraphics[width=\textwidth]{orient_plot31}
  \end{subfigure}
  \hfill
  \begin{subfigure}{0.49\textwidth}
    \includegraphics[width=\textwidth]{orient_plot36}
  \end{subfigure}
  \\
  \begin{subfigure}{0.49\textwidth}
    \includegraphics[width=\textwidth]{orient_err_plot31}
  \end{subfigure}
  \hfill
  \begin{subfigure}{0.49\textwidth}
    \includegraphics[width=\textwidth]{orient_err_plot36}
  \end{subfigure}
  \\
  \begin{subfigure}{0.49\textwidth}
    \includegraphics[width=\textwidth]{vel_err_plot31}
  \end{subfigure}
  \hfill
  \begin{subfigure}{0.49\textwidth}
    \includegraphics[width=\textwidth]{vel_err_plot36}
  \end{subfigure}
  \caption{Plots of the ellipsoid orientation, orientation error, and
    velocity error in an unbounded shear flow. Orientation is
    initialized at $\vect{e}_m = \vect{e}_y$.}
  \label{fig:third-free-test}
\end{figure}

\begin{figure}
  \centering
  \begin{subfigure}{0.49\textwidth}
    \includegraphics[width=\textwidth]{orient_plot12}
  \end{subfigure}
  \hfill
  \begin{subfigure}{0.49\textwidth}
    \includegraphics[width=\textwidth]{orient_plot17}
  \end{subfigure}
  \\
  \begin{subfigure}{0.49\textwidth}
    \includegraphics[width=\textwidth]{orient_err_plot12}
  \end{subfigure}
  \hfill
  \begin{subfigure}{0.49\textwidth}
    \includegraphics[width=\textwidth]{orient_err_plot17}
  \end{subfigure}
  \\
  \begin{subfigure}{0.49\textwidth}
    \includegraphics[width=\textwidth]{com_plot12}
  \end{subfigure}
  \hfill
  \begin{subfigure}{0.49\textwidth}
    \includegraphics[width=\textwidth]{com_plot17}
  \end{subfigure}
  \caption{Plots of the ellipsoid orientation, orientation error, and
    center of mass in a shear flow near a wall. The height of the
    center of mass is initialized at $1.5$. Orientation is initialized
    at $\vect{e}_m = \vect{e}_x$. The three plots in the left column
    are with the coarse mesh, and the three plots in the right column
    are with the fine mesh.}
  \label{fig:first-wall-test}
\end{figure}

\begin{figure}
  \centering
  \begin{subfigure}{0.49\textwidth}
    \includegraphics[width=\textwidth]{orient_plot13}
  \end{subfigure}
  \hfill
  \begin{subfigure}{0.49\textwidth}
    \includegraphics[width=\textwidth]{orient_plot18}
  \end{subfigure}
  \\
  \begin{subfigure}{0.49\textwidth}
    \includegraphics[width=\textwidth]{orient_err_plot13}
  \end{subfigure}
  \hfill
  \begin{subfigure}{0.49\textwidth}
    \includegraphics[width=\textwidth]{orient_err_plot18}
  \end{subfigure}
  \\
  \begin{subfigure}{0.49\textwidth}
    \includegraphics[width=\textwidth]{com_plot13}
  \end{subfigure}
  \hfill
  \begin{subfigure}{0.49\textwidth}
    \includegraphics[width=\textwidth]{com_plot18}
  \end{subfigure}
  \caption{Plots of the ellipsoid orientation, orientation error, and
    center of mass in a shear flow near a wall. The height of the
    center of mass is initialized at $1.2$. Orientation is initialized
    at $\vect{e}_m = \vect{e}_x$. The three plots in the left column
    are with the coarse mesh, and the three plots in the right column
    are with the fine mesh.}
  \label{fig:second-wall-test}
\end{figure}

\begin{figure}
  \centering
  \begin{subfigure}{0.49\textwidth}
    \includegraphics[width=\textwidth]{orient_plot14}
  \end{subfigure}
  \hfill
  \begin{subfigure}{0.49\textwidth}
    \includegraphics[width=\textwidth]{orient_plot19}
  \end{subfigure}
  \\
  \begin{subfigure}{0.49\textwidth}
    \includegraphics[width=\textwidth]{orient_err_plot14}
  \end{subfigure}
  \hfill
  \begin{subfigure}{0.49\textwidth}
    \includegraphics[width=\textwidth]{orient_err_plot19}
  \end{subfigure}
  \\
  \begin{subfigure}{0.49\textwidth}
    \includegraphics[width=\textwidth]{com_plot14}
  \end{subfigure}
  \hfill
  \begin{subfigure}{0.49\textwidth}
    \includegraphics[width=\textwidth]{com_plot19}
  \end{subfigure}
  \caption{Plots of the ellipsoid orientation, orientation error, and
    center of mass in a shear flow near a wall. The height of the
    center of mass is initialized at $1.0$. Orientation is initialized
    at $\vect{e}_m = \vect{e}_x$. The three plots in the left column
    are with the coarse mesh, and the three plots in the right column
    are with the fine mesh.}
  \label{fig:third-wall-test}
\end{figure}

\bibliographystyle{plain}
\bibliography{/Users/andrewwork/Documents/grad-school/thesis/library}

\end{document}




