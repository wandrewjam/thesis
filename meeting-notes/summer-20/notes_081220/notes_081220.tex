\documentclass{article}

\newcommand{\ep}{\rule{.06in}{.1in}}
\textheight 9.5in

\usepackage{amssymb, bm}
\usepackage{amsmath}
\usepackage{amsthm}
\usepackage{graphicx, subcaption, booktabs}

\usepackage{tikz, pgfplots, pgfplotstable, chemfig, xcolor}

% \usepgfplotslibrary{colorbrewer, statistics}
% \pgfplotsset{
%   exact axis/.style={grid=major, minor tick num=4, xlabel=$v^*$,
%     legend entries={PDF, CDF},},
%   every axis plot post/.append style={thick},
%   table/search
%   path={/Users/andrewwork/thesis/jump-velocity/dat-files},
%   colormap/YlGnBu,
%   cycle list/Set1-5,
%   legend style={legend cell align=left,},
% }

% \usepgfplotslibrary{external}
% \tikzexternalize

\renewcommand{\arraystretch}{1.2}
\pagestyle{empty} 
\oddsidemargin -0.25in
\evensidemargin -0.25in 
\topmargin -0.75in 
\parindent 0pt
\parskip 12pt
\textwidth 7in
%\font\cj=msbm10 at 12pt

\newcommand{\tn}{\textnormal}
\newcommand{\stiff}{\frac{k_f}{\gamma}}
\newcommand{\dd}{d}
\newcommand{\Der}[2]{\frac{\dd #1}{\dd #2}}
\newcommand{\Pder}[2]{\frac{\partial #1}{\partial #2}}
\newcommand{\Integral}[4]{\int_{#3}^{#4} {#1} \dd #2}
\newcommand{\vect}[1]{\boldsymbol{\mathbf{#1}}}
\newcommand{\mat}[1]{\underline{\underline{#1}}}
\DeclareMathOperator{\Exp}{Exp}

% Text width is 7 inches

\def\R{\mathbb{R}}
\def\N{\mathbb{N}}
\def\C{\mathbb{C}}
\def\Z{\mathbb{Z}}
\def\Q{\mathbb{Q}}
\def\H{\mathbb{H}}
\def\B{\mathcal{B}} 
%\topmargin -.5in 

\setcounter{secnumdepth}{2}
\begin{document}
\pagestyle{plain}

\begin{center}
  {\Large Meeting Notes (\today)}
\end{center}

{\large \textbf{3D plots of critical points in the platelet trajectories}}

\begin{itemize}
\item I used MATLAB to render 3D diagrams of the platelet height and
  orientation at some critical points in the trajectories for a couple
  different experiments. The wall is shown in gray, and the background
  shear flow is represented by the blue vectors plotted on the $z =
  -2$ plane. The way I've plotted these, the $z$ axis doesn't change,
  but if I was being more thorough, $z$ would increase as the platelet
  translates downstream. This doesn't change the appearance of the
  plots at all, apart from changing the numbers on the $z$ axis.
\item In Figure \ref{fig:plt28-3d}, the platelet was initialized at a
  height of 1.2, and an initial orientation of $(\cos(\pi/4),
  \sin(\pi/4), 0)$. 
\item In Figure \ref{fig:plt48-3d}, the platelet was initialized at a
  height of 1.2, and an initial orientation of $(\cos(\pi/8),
  \sin(\pi/8), 0)$. 
\item I also included plots of the components of the platelet
  position, and orientation vectors for the entire experiment (Figure
  \ref{fig:plt28} and Figure \ref{fig:plt48}). 
\end{itemize}

\begin{figure}[h]
  \centering
  \begin{subfigure}{0.49\textwidth}
    \includegraphics[width=\textwidth]{{plt28_t0.0}.png}
  \end{subfigure}
  \hfill
  \begin{subfigure}{0.49\textwidth}
    \includegraphics[width=\textwidth]{{plt28_t4.8}.png}
  \end{subfigure}
  \\
  \begin{subfigure}{0.49\textwidth}
    \includegraphics[width=\textwidth]{{plt28_t15.4}.png}
  \end{subfigure}
  \hfill
  \begin{subfigure}{0.49\textwidth}
    \includegraphics[width=\textwidth]{{plt28_t26.1}.png}
  \end{subfigure}
  \caption{3D diagrams of the platelet orientation and height
    throughout one period of the orientation (shown in Figure
    \ref{fig:plt28}). The time points were chosen at critical points
    in period, specifically the min and max of $e_{m,x}$ (the 1st and
    3rd plots) and the min and max of $e_{m, z}$ (the 2nd and 4th
    plots)} 
  \label{fig:plt28-3d}
\end{figure}

\begin{figure}
  \centering
  \begin{subfigure}{0.49\textwidth}
    \includegraphics[width=\textwidth]{orient_plot28_2nd}
  \end{subfigure}
  \hfill
  \begin{subfigure}{0.49\textwidth}
    \includegraphics[width=\textwidth]{orient_err_plot28_2nd}
  \end{subfigure}
  \\
  \begin{subfigure}{0.49\textwidth}
    \includegraphics[width=\textwidth]{com_plot28_2nd}
  \end{subfigure}
  \caption{Plots of the ellipsoid orientation, orientation error, and
     center of mass in a shear flow near a wall. The height of the
     center of mass is initialized at $1.2$. Orientation is initialized
     at $\vect{e}_m = (1/\sqrt{2}, 1/\sqrt{2}, 0)^T$. In this
     experiment, the ellipsoid does not flip over, and it looks like
     the motion is truly periodic.}
  \label{fig:plt28}
\end{figure}

\begin{figure}
  \centering
  \begin{subfigure}{0.49\textwidth}
    \includegraphics[width=\textwidth]{{plt48_t0.0}.png}
  \end{subfigure}
  \hfill
  \begin{subfigure}{0.49\textwidth}
    \includegraphics[width=\textwidth]{{plt48_t7.8}.png}
  \end{subfigure}
  \\
  \begin{subfigure}{0.49\textwidth}
    \includegraphics[width=\textwidth]{{plt48_t14.2}.png}
  \end{subfigure}
  \hfill
  \begin{subfigure}{0.49\textwidth}
    \includegraphics[width=\textwidth]{{plt48_t20.9}.png}
  \end{subfigure}
  \caption{3D diagrams of the platelet orientation and height
    throughout one period of the orientation (shown in Figure
    \ref{fig:plt48}). The time points were chosen at critical points
    in period, specifically the min and max of $e_{m,x}$ (the 3rd and
    1st plots, respectively) and the min and max of $e_{m, z}$ (the
    4th and 2nd plots, respectively).}
  \label{fig:plt48-3d}
\end{figure}

\begin{figure}
  \centering
  \begin{subfigure}{0.49\textwidth}
    \includegraphics[width=\textwidth]{orient_plot48_2nd}
  \end{subfigure}
  \hfill
  \begin{subfigure}{0.49\textwidth}
    \includegraphics[width=\textwidth]{orient_err_plot48_2nd}
  \end{subfigure}
  \\
  \begin{subfigure}{0.49\textwidth}
    \includegraphics[width=\textwidth]{com_plot48_2nd}
  \end{subfigure}
  \caption{Plots of the ellipsoid orientation, orientation error, and
     center of mass in a shear flow near a wall. The height of the
     center of mass is initialized at $1.2$. Orientation is initialized
     at $\vect{e}_m = (\cos(\pi/8), \sin(\pi/8), 0)^T$. Again, in this
     experiment, the ellipsoid does not flip over, and it looks like
     the motion is truly periodic.}
  \label{fig:plt48}
\end{figure}

\begin{figure}
  \centering
  \begin{subfigure}{0.49\textwidth}
    \includegraphics[width=\textwidth]{orient_plot72_2nd}
  \end{subfigure}
  \hfill
  \begin{subfigure}{0.49\textwidth}
    \includegraphics[width=\textwidth]{orient_err_plot72_2nd}
  \end{subfigure}
  \\
  \begin{subfigure}{0.49\textwidth}
    \includegraphics[width=\textwidth]{com_plot72_2nd}
  \end{subfigure}
  \hfill
  \begin{subfigure}{0.49\textwidth}
    \includegraphics[width=\textwidth]{nsep_plot72}
  \end{subfigure}  
  \caption{Plots of the ellipsoid orientation, orientation error,
    center of mass, and the $N$ value with the plt-wall
    separation. The height of the center of mass is initialized at
    $1.2$. Orientation is initialized at $\vect{e}_m =
    \vect{e}_x$. Throughout almost the entire simulation, the adaptive
    solver used the minimum mesh size ($N = 16$), however in a few
    small intervals, the maximum mesh size is used ($N = 24$) and the
    adaptive solver shows improved accuracy relative to the coarse
    mesh size.} 
  \label{fig:plt72}
\end{figure}

\begin{figure}
  \centering
  \begin{subfigure}{0.49\textwidth}
    \includegraphics[width=\textwidth]{orient_plot73_2nd}
  \end{subfigure}
  \hfill
  \begin{subfigure}{0.49\textwidth}
    \includegraphics[width=\textwidth]{orient_err_plot73_2nd}
  \end{subfigure}
  \\
  \begin{subfigure}{0.49\textwidth}
    \includegraphics[width=\textwidth]{com_plot73_2nd}
  \end{subfigure}
  \hfill
  \begin{subfigure}{0.49\textwidth}
    \includegraphics[width=\textwidth]{nsep_plot73}
  \end{subfigure}  
  \caption{Plots of the ellipsoid orientation, orientation error,
    center of mass, and the $N$ value with the plt-wall
    separation. The height of the center of mass is initialized at
    $1.0$. Orientation is initialized at $\vect{e}_m =
    \vect{e}_x$. Similar to the previous figure, the adaptive solver
    used the minimum mesh size for most of the simulation, apart from
    a few short intervals. However something strange is going on in
    the adaptive solver in this experiment: there's a weird hitch in
    the graph around $t = 20$, and the ellipsoid stays in the vertical
    orientation for much longer than in any other run.}
  \label{fig:plt73}
\end{figure}

\begin{figure}
  \centering
  \begin{subfigure}{0.49\textwidth}
    \includegraphics[width=\textwidth]{orient_plot74_2nd}
  \end{subfigure}
  \hfill
  \begin{subfigure}{0.49\textwidth}
    \includegraphics[width=\textwidth]{orient_err_plot74_2nd}
  \end{subfigure}
  \\
  \begin{subfigure}{0.49\textwidth}
    \includegraphics[width=\textwidth]{com_plot74_2nd}
  \end{subfigure}
  \hfill
  \begin{subfigure}{0.49\textwidth}
    \includegraphics[width=\textwidth]{nsep_plot74}
  \end{subfigure}
  \caption{Plots of the ellipsoid orientation, orientation error,
    center of mass, and the $N$ value with the plt-wall
    separation. The height of the center of mass is initialized at
    $0.8$. Orientation is initialized at $\vect{e}_m = \vect{e}_x$. I
    don't have much to say about this plot, it was just included as an
    example where the adaptive solver used the minimum number of nodes
    for the entire experiment.}
  \label{fig:plt74}
\end{figure}

% \begin{figure}[h!]
%   \centering
%   \begin{subfigure}{0.49\textwidth}
%     \includegraphics[width=\textwidth]{orient_plot12_2nd}
%   \end{subfigure}
%   \hfill
%   \begin{subfigure}{0.49\textwidth}
%     \includegraphics[width=\textwidth]{orient_err_plot12_2nd}
%   \end{subfigure}
%   \\
%   \begin{subfigure}{0.49\textwidth}
%     \includegraphics[width=\textwidth]{com_plot12_2nd}
%   \end{subfigure}
%   \caption{Plots of the ellipsoid orientation, orientation error, and
%      center of mass in a shear flow near a wall. The height of the
%      center of mass is initialized at $1.5$. Orientation is
%      initialized at $\vect{e}_m = \vect{e}_x$. The coarse solution
%      used the same number of nodes as the minimum node size in the
%      adaptive code.}
%   \label{fig:plt12}
% \end{figure}

\bibliographystyle{plain}
\bibliography{/Users/andrewwork/Documents/grad-school/thesis/library}

\end{document}




