\documentclass{article}

\newcommand{\ep}{\rule{.06in}{.1in}}
\textheight 9.5in
\usepackage{amssymb}
\usepackage{amsmath}
\usepackage{amsthm}
\usepackage{subcaption}
\usepackage{graphicx}
\pagestyle{empty}

\bibliographystyle{plain}

\oddsidemargin -0.25in
\evensidemargin -0.25in 
\topmargin -0.75in 
\parindent 0pt
\parskip 12pt
\textwidth 7in

%\font\cj=msbm10 at 12pt

\newcommand{\R}{\mathbb{R}}
\newcommand{\dd}{d}
\newcommand{\Der}[2]{\frac{\dd #1}{\dd #2}}
\newcommand{\Pder}[2]{\frac{\partial #1}{\partial #2}}
\newcommand{\Int}[4]{\int_{#3}^{#4} #1 \, \dd #2}
\newcommand{\tn}{\textnormal}
\newcommand{\on}{\textnormal{on}}
\DeclareMathOperator{\erf}{erf}
\renewcommand{\arraystretch}{1.5}

%%% This is model-defs.tex
%%%
%%% Define symbols for model parameters here so that it is
%%% straightforward to change notation if necessary.

%% Coordinates
\newcommand{\wallDist}{x}
\newcommand{\ndWallDist}{z}
\newcommand{\recAngle}{\theta}
\newcommand{\dTime}{t}
\newcommand{\ndTime}{s}

%% Unknown Functions
\newcommand{\velocity}{V}
\newcommand{\rotation}{\Omega}
\newcommand{\ndVelocity}{v}
\newcommand{\ndRotation}{\omega}
\newcommand{\bondDensity}{n}
\newcommand{\ndBondDensity}{m}

%% Model Parameters (Dimensional)
\newcommand{\radius}{R}
\newcommand{\separation}{d}
\newcommand{\height}{h}
\newcommand{\length}{L}
\newcommand{\domLength}{A}
\newcommand{\shear}{\gamma}
\newcommand{\stiffness}{k_f}
\newcommand{\boltzmann}{k_B}
\newcommand{\temp}{T}
\newcommand{\onRate}{k_\tn{on}}
\newcommand{\offRate}{k_\tn{off}}
\newcommand{\onConst}{k_\tn{on}^0}
\newcommand{\offConst}{k_\tn{off}^0}
\newcommand{\refForce}{f_0}
\newcommand{\receptorDensity}{N_T}
\newcommand{\receptorNumber}{N_R}
\newcommand{\appliedVel}{V_f}
\newcommand{\appliedRot}{\Omega_f}
\newcommand{\velFriction}{\xi_V}
\newcommand{\rotFriction}{\xi_\Omega}
\newcommand{\compliance}{\Gamma}
\newcommand{\width}{w}
\newcommand{\viscosity}{\mu}

%% Force and Torque Functions
\newcommand{\horzForce}{f_h}
\newcommand{\torque}{\tau_s}
\newcommand{\horzTotalForce}{F_h}
\newcommand{\totalTorque}{\tau}

%% Force, Velocity, and Resistance Tensors
\newcommand{\forceVec}{\mathbf{F}}
\newcommand{\velVec}{\mathbf{U}}
\newcommand{\resMatrix}{\underline{\underline{R}}}

%% Model Parameters (Nondimensional)
\newcommand{\ndSeparation}{d'}
\newcommand{\ndLength}{\ell}
\newcommand{\ndAppliedRot}{\omega_f}
\newcommand{\ndAppliedVel}{v_f}
\newcommand{\ndOnConst}{\kappa}
\newcommand{\newOnConst}{\kappa_\textnormal{new}}
\newcommand{\onForceScale}{\eta}
\newcommand{\offForceScale}{\delta}
\newcommand{\ndVelFriction}{\eta_v}
\newcommand{\ndRotFriction}{\eta_\omega}

%% Nondimensional Force and Torque Functions
\newcommand{\ndHorzForce}{f_h'}
\newcommand{\ndTorque}{\tau_s'}
\newcommand{\ndHorzTotalForce}{F_h'}
\newcommand{\ndTotalTorque}{\tau'}

%% Shorthands for Chemical Species
\newcommand{\ITA}[1]{\textalpha\textsubscript{#1}}
\newcommand{\ITB}[1]{\textbeta\textsubscript{#1}}
\newcommand{\Ca}{$\tn{Ca}^{++}$}

%% Reynolds Number
\newcommand{\Reynolds}{\mathrm{Re}}

%% Bin Midpoint
\newcommand{\binMidpoint}[1]{\theta^*_{#1}}


\graphicspath{{/Users/andrewwork/thesis/rolling-model/plots/}}
%\topmargin -.5in 
\begin{document}
\pagestyle{empty}


\begin{center}
{\Large Notes on ``A modified next reaction method...''
  \cite{Anderson2007} and its application to platelet rolling}
\end{center}

The old approach I've been using to simulate platelet rolling is a
naive modification of the Gillespie algorithm where at each time step
in the simulation, I calculate two random numbers: one representing
the first time a reaction occurs, and the second representing which
particular reaction occurs. The modification I made is to enforce a
maximum time step, and if no reaction occurs before that maximum time
step I update the position of the platelet and the positions of the
bonds, but I do not create or destroy any bonds (i.e. no reactions
occur). Then I repeat the process, generating a new pair of random
numbers and comparing them to the maximum time step.

There are a couple of weaknesses in this approach. First, because I
generate two random numbers for every single time step, and in my
simulations the maximum time step I enforce is much smaller than
the average time for a bond to form or break, I end up wasting most of
the random numbers I generate. In a typical simulation, I use
$\sim10^4$ time steps but only $\sim10$---$100$ reactions occur in
a simulation. That is, I generate about 20,000 random numbers, but
only use 20---200 of them and discard the rest. The second problem is
that this method is a very crude way to integrate the motion of the
platelet. Let me explain this a little more thoroughly.

The motion of a rolling platelet is piecewise deterministic: that is,
in between bond formation/breaking events, the motion of the platelet
is deterministic. With the method described above, I am really
integrating the motion of the platelet using a forward Euler method
with a constant step size. The biggest issue with this is that the
nondimensional friction coefficient for the platelet motion is very
small, and so the ODEs governing platelet motion are stiff, requiring
a very small step size in order for the Forward Euler method to be
stable. The notes that follow develop a method that more clearly
separates the simulation of the deterministic motion of the platelet
from the stochastic simulation of chemical reaction events so that
each of these separate processes can be dealt with in a more
sophisticated way.

\section{Notes on ``A modified next reaction method for simulating
  chemical systems with time dependent propensities and delays''}
\label{sec:notes-modified-nrm}

The fundamental premise of the Gillespie algorithm and related methods
is that each chemical reaction $k$ proceeds with a propensity function
$a_j(\mathbf{x})$ that depends on the state of the system $\mathbf{x}$
and defined so that
\begin{equation}
  \label{eq:prop-def}
  a_k(\mathbf{x}) \dd t \equiv \tn{the probability that reaction $k$
    takes place in a small time interval $[t, t+\dd t)$}.
\end{equation}

Based on this assumption, the time $\Delta t_k$ that it takes for
reaction $k$ to fire is exponentially distributed with rate parameter
$a_k(\mathbf{x})$. As mentioned above, for each time step in the
Gillespie algorithm, two random numbers are generated. The first is
used to find the minimum firing time over all the reactions (the
minimum firing time is exponentially distributed with rate parameter
$a_0(\mathbf{x}) = \sum_{k=1}^M a_k(\mathbf{x})$), and the second is
used to find which reaction occurred (reaction $k$ occurred with
probability $a_k(\mathbf{x}) / a_0 (\mathbf{x})$).

\subsection{Next reaction method}
\label{sec:next-reaction-method}

In the Gillespie algorithm, two random numbers must be generated at
each time step, and the number of random numbers that must be
generated is independent of the number of possible reactions. For
chemical systems where the number of time steps is large relative to
the number of possible reactions, the next reaction method improves on
the Gillespie algorithm in that fewer random numbers are required to
advance the simulation. However the feature that is most relevant for
these notes is that the next reaction method is more easily
generalized to the case where reaction rates are explicitly dependent
on time.

For the next reaction method, we represent the reaction times of each
individual reaction as the firing times of Poisson processes. Let
$R_k(t)$ be the number of times the $k$th reaction has occurred before
time $t$. $R_k(t)$ can be represented as a Poisson process with a
nondimensional firing rate related to the reaction rate. Specifically,
if we define $k$ independent Poisson processes with unit firing rate:
$Y_k(\tau)$, then
\begin{equation}
  \label{eq:R-def}
  R_k(t) = Y_k\left(\int_0^t a_k(X(s)) ds \right).
\end{equation}

Anderson defines the \emph{internal time} of reaction $k$ as $T_k(t)$:
\begin{equation}
  \label{eq:T-def}
  T_k(t) \equiv \int_0^t a_k(X(s)) ds.
\end{equation}
The internal time is a nondimensional quantity that scales time in the
$k$th Poisson process based on the reaction rate $a_k(X(t))$.

The idea of the next reaction method is to calculate the firing times
of each Poisson process independently, and then to find the minimum
firing time. The system is updated based on which reaction fired, and
the internal times of each reaction are updated appropriately. All of
the firing times of the other reactions are saved, and a new firing
time is generated only for the reaction which fired.

More explicitly, define $P_k$ to be the first firing time of $Y_k$ (in
the time scale of $Y_k$) after $T_k(t)$:
\begin{equation}
  \label{eq:P-def}
  P_k(t) = \min\{s > T_k(t) \mid Y_k(s) > Y_k(T_k(t)) \}.
\end{equation}

Then from the definition of $T_k(t)$ the absolute time required for
reaction $k$ to fire $\Delta t_k$ is given by
\begin{equation}
  \label{eq:dtk-def}
  \int_t^{t + \Delta t_k} a_k(X(s)) ds = P_k - T_k.
\end{equation}


Assuming $X(s)$ is constant in $[t, t + \Delta t_k)$, then so is $a_k$
(as long as $a_k$ doesn't explicitly depend on time) and
$\Delta t_k = (P_k - T_k)/a_k$. This gives us the following algorithm
for the next reaction method for autonomous reaction rates:
\begin{enumerate}
\item Initialize chemical species, set $t=0$, and $T_k = 0$ for
  each $k$
\item Calculate $a_k$ for each reaction.
\item Generate $M$ independent, uniform random numbers $r_k$.
\item For each $k$, set $P_k = \log(1/r_k)$ (find the internal firing
  times of each reaction).
\item For each $k$, set $\Delta t_k = (P_k - T_k)/a_k$ (convert the
  internal firing times to absolute firing times).
\item Set $\Delta = \min_k \{\Delta t_k\}$ and
  $\mu = \operatorname{argmin}_k \{\Delta t_k\}$.
\item Set $t = t + \Delta$ and update the number of molecular species
  associated with reaction $\mu$.
\item For each $k$, set $T_k = T_k + a_k \Delta$ (advance the internal
  times).
\item For reaction $\mu$, choose a new random number $r$ and set
  $P_\mu = P_\mu + \log(1/r)$.
\item Recalculate the reaction rates $a_k$.
\item Go to 5.
\end{enumerate}

In the code below, I simulate a simple enzymatic reaction
$\mathrm{A} + \mathrm{B} \rightleftharpoons \mathrm{C} \rightarrow
\mathrm{B} + \mathrm{P}$ with rates $k^+ = 1$, $k^- = 5$, and
$k^\text{cat} = 5$ using the Gillespie and next reaction methods and
compare the results with the solution of the ODE reaction rate
equation. This chemical system gives the following set of ODEs:
\begin{align}
\frac{da}{dt} &= -k^+ ab + k^- c \\
\frac{db}{dt} &= -k^+ ab + (k^- + k^\text{cat}) c \\
\frac{dc}{dt} &= k^+ ab - (k^- + k^\text{cat}) c
\end{align}
and the following set of propensity functions $a_k(X(t))$:
\begin{align}
a_1 &= k^+ ab \\
a_2 &= k^- c \\
a_3 &= k^\text{cat} c.
\end{align}

\subsection{Time-dependent reaction rates}
\label{sec:time-depend-react}

If we now assume that the reaction rates can change explicitly as a
function of time (that is $a_k = a_k(X(t), t)$), we can apply the next
reaction method described above to this new case in a fairly
straightforward way. $P_k$ and $T_k$ are defined in the same way as
above, the only difference is that now $T_k$ is not necessarily a
piecewise linear function of the absolute time $t$.

In the algorithm described above, only steps 5 and 8 change. In step 5
to find the absolute firing times of each reaction $\Delta t_k$, we
must now solve
\begin{equation}
  \int_t^{t + \Delta t_k} a_k(X(t), s) ds = P_k - T_k.
\end{equation}

In step 8 to advance each internal clock, we now have to set
\begin{equation}
T_k = T_k + \int_t^{t + \Delta} a_k(X(t), s) ds.
\end{equation}

The code below simulates the same chemical system as above, but with
exponentially decreasing reaction rates. That is,
$k^+(t) = k^+_0 \exp(-Rt)$ and similarly for the other reaction
rates. As $R \rightarrow 0$, this system approaches the system above
with constant reaction rates.

\section{The modified next reaction method applied to platelet
  rolling}
\label{sec:plt-rolling}

In the case of platelet rolling, it is clear that the reaction rates
(i.e. bond formation and breaking rates) depend on time through the
position of the platelet, and the method for simulating time-dependent
reactions can nearly be directly applied to simulate random bond
formation and breaking.

Let's look at the force-balance equations for the platelet. We have
\begin{align}
  0 &= \ndVelFriction \left(\ndAppliedVel - \ndVelocity \right) +
  \ndHorzTotalForce \\
  0 &= \ndRotFriction \left(\ndAppliedRot - \ndRotation \right) + \ndTotalTorque
\end{align}
where $\ndHorzTotalForce = \sum_{i=0}^\texttt{n\_bonds}
(\sin\recAngle_i - \ndWallDist_i )$ and $\ndTotalTorque =
-\sum_{i=0}^\texttt{n\_bonds} [(1 - \cos\recAngle_i +
\ndSeparation)\sin\recAngle_i + (\sin\recAngle_i - \ndWallDist_i)\cos\recAngle_i]$.

\bibliography{/Users/andrewwork/thesis/library.bib}

\end{document}
