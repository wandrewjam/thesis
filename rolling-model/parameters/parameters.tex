\documentclass{article}
% \let\bb\enumerate
% \let\ee\endenumerate
% \let\bd\description
% \let\ed\enddescription
% \let\ii\item
\let\ds\displaystyle
\newcommand{\ep}{\rule{.06in}{.1in}}
\textheight 9.5in
\usepackage{amssymb}
\usepackage{amsmath}
\usepackage{amsthm}
\usepackage{textgreek}
\usepackage{tabu}
\pagestyle{empty} 
\bibliographystyle{plain}

\oddsidemargin -0.25in
\evensidemargin -0.25in 
\topmargin -0.75in 
\parindent 0pt
\parskip 12pt
\textwidth 7in

%\font\cj=msbm10 at 12pt
\def\R{\mathbb{R}}
\def\N{\mathbb{N}}
\def\C{\mathbb{C}}
\def\Z{\mathbb{Z}}
\def\Q{\mathbb{Q}}
\def\H{\mathbb{H}}
\def\B{\mathcal{B}} 
\def\and{\wedge}
\newcommand{\inv}{^{-1}}

\newcommand{\radius}{R}
\newcommand{\separation}{d}
\newcommand{\stiffness}{k_f}
\newcommand{\boltzmann}{k_B}
\newcommand{\temp}{T}
\newcommand{\onConst}{k_\text{on}}
\newcommand{\offConst}{k_\text{off}}
\newcommand{\refForce}{f_0}
\newcommand{\receptorDensity}{N_T}
\newcommand{\receptorNumber}{N_R}
\newcommand{\appliedRot}{\Omega_f}
\newcommand{\appliedVel}{V_f}
\newcommand{\velFriction}{\xi_V}
\newcommand{\rotFriction}{\xi_\omega}
\newcommand{\compliance}{\Gamma}
\newcommand{\width}{w}
\newcommand{\viscosity}{\mu}

\newcommand{\ndSeparation}{d'}
\newcommand{\ndAppliedRot}{\omega_f}
\newcommand{\ndAppliedVel}{v_f}
\newcommand{\ndOnConst}{\kappa}
\newcommand{\onForceScale}{\eta}
\newcommand{\offForceScale}{\delta}
\newcommand{\ndVelFriction}{\eta_v}
\newcommand{\ndRotFriction}{\eta_\omega}

\newcommand{\ITA}[1]{\textalpha\textsubscript{#1}}
\newcommand{\ITB}[1]{\textbeta\textsubscript{#1}}


%\def\ior{\vee} %this line can cause problems (in tables, nested
                %lists, others?), comment out if necessary
%\topmargin -.5in 
\begin{document}
\pagestyle{empty}


\begin{center}
{\Large Parameter Estimates}
\end{center}

There are 13 dimensional parameters in the dimensional version of the
PDE model, listed in Table \ref{tab:dim_pars}.

\begin{table}[h]
  \centering
  \begin{tabu}{c|X|X|X}
    Symbol & Description & Estimate & Source \\ \hline
    $\radius$ & Platelet radius & 1 {\textmugreek}m & None \\
    $\separation$ & Wall separation distance & 0.1 {\textmugreek}m
                                    & None \\
    $\stiffness$ & Bond stiffness & 100 dyne/cm & \cite{Bhatia2003} \\
    $\boltzmann$ & Boltzmann constant & $1.38 \times 10^{-23}$ J/K
                                    & None \\
    $\temp$ & Blood temperature & 310 K & None \\
    $\onConst$ & Maximum bond on rate & 10--1000 s$\inv$
                                    & \cite{Bhatia2003} \\
    $\offConst$ & Unstressed bond off rate & 0.1--1 s$\inv$
                                    & \cite{Bhatia2003} (for
                                      selectin-sLe\textsuperscript{x}
                                      binding and
                                      \textbeta\textsubscript{2}-ICAM
                                      binding) \\
    $\refForce$ & Characteristic breaking force & 0.01 pN$\inv$
                                    & Estimated below \\
    $\receptorDensity$ & Angular receptor density
                         & 1,600--16,000 OR $800w$--$8000w$
                           receptors/radian & Estimated below \\
    $\appliedRot$ & Fluid-imposed rotation rate & $\gamma$ s$\inv$ &
    \\
    $\appliedVel$ & Fluid-imposed translation velocity
                         & $(R + d)\gamma$ {\textmugreek}m/s & \\
    $\velFriction$ & Translational drag coefficient
                         & $6 \times 10^{-5}$ g/s & Estimated below \\
    $\rotFriction$ & Rotational drag coefficient
                         & $8 \times 10^{-5}$
                           {\textmugreek}m\textsuperscript{2}g/s
                                    & Estimated below \\
  \end{tabu}
  \caption{Dimensional parameters in the PDE model}
  \label{tab:dim_pars}
\end{table}

\section{Estimates for model parameters}
\label{sec:estim-model-param}

\subsection{Characteristic breaking force}
\label{sec:char-break-force}

$\refForce$ is equivalent to $\compliance/(\boltzmann\temp)$ in
\cite{Pospieszalska2009} and \cite{Sundd2011}, where $\compliance$ is
the reactive compliance and is estimated at 0.4 angstroms in
\cite{Bhatia2003}. Then converting units as needed:
\begin{align*}
  &\compliance = 0.4 \text{ angstroms} = 4 \times 10^{-5}
    \text{{\textmugreek}m} \text{, and} \\
  &\boltzmann = 1.38 \times 10^{-23} \text{N}\cdot\text{m}/\text{K} =
    1.38 \times 10^{-5} \text{pN}\cdot\text{{\textmugreek}m}/\text{K}.
\end{align*}
Finally, $\refForce = \frac{\compliance}{\boltzmann\cdot\temp} =
\frac{4 \times 10^{-5}}{310 \cdot 1.38 \times 10^{-5}} \text{pN}\inv =
9.35 \times 10^{-3} \text{pN}\inv$.

\subsection{Angular receptor density}
\label{sec:ang-rec-dens}

$\receptorDensity$ is related to the number of receptors on the
surface of the platelet, but it is not clear how. Obviously in reality
the receptors are distributed over a 2D surface, but in the model the
platelet is 2D and its surface is 1D. There are between $10^4$ and $10^5$
integrin receptors on the surface of a platelet
\cite{Burkhart2012}. If we assume all of these are on the surface of
our 2D platelet, that gives a range of 1,600--16,000 for $\receptorDensity$. 

If we assume the reactive region is a narrow strip down the centerline
of the platelet of width $\width$, then $\receptorDensity$ is found by
first calculating the number of receptors in that reactive
strip. Approximating the reactive region with a cylinder of radius
$\radius$ and height $\width$ gives a surface area of
$2\pi\width\radius$. The surface density of receptors is given by
$\receptorNumber/(4\pi\radius^2)$, and so there are
$\receptorNumber\cdot\width/(2\radius)$ total receptors in the
reactive strip. After dividing by $2\pi$ to get the angular density of
receptors, we end up with $\receptorDensity =
\receptorNumber\cdot\width/(4\pi\radius)$. With our estimates of
$\receptorNumber$ and $\radius$ above, this gives us a range for
$\receptorDensity$ of $796\width$--$7960\width$.

\subsection{Drag coefficients}
\label{sec:drag-coefficients}

For simplicity, I use the Stokes drag coefficients:
$6\pi\viscosity\radius$ and $8\pi\viscosity\radius^3$. With the
viscosity of blood around 3–4 cP, this gives drag coefficients of $6
\times 10^{-5}$ g/s and $8 \times 10^{-5}$
\textmugreek\textsuperscript{2}g/s respectively.

\section{Nondimensional Parameters}
\label{sec:nd-params}

There are 8 parameters in the nondimensional PDE rolling model, listed
in Table

\begin{table}
  \centering
  \begin{tabular}{c|c|l|l}
    Parameter & Definition & Value & Description \\ \hline
    $\ndSeparation$ & $\separation = \radius\ndSeparation$ & 0.1
                                   & cell-surface separation \\
    $\ndAppliedRot$ & $\appliedRot = \ndAppliedRot\offConst$
                           & $0.1\gamma$--$\gamma$ & Fluid-imposed
                                                     rotation rate \\
    $\ndAppliedVel$ & $\appliedVel = \ndAppliedVel\radius\offConst$
                           & $0.1\gamma$--$\gamma$
                                   & Fluid-imposed translation
                                     velocity \\
    $\ndOnConst$ & $\ndOnConst = \frac{\radius\onConst}{\offConst}$
                           & $10$--$10^4$ & Maximum relative on rate \\
    $\onForceScale$ & $\onForceScale =
                    \frac{\stiffness\radius^2}{\boltzmann\temp}$
                           & $2 \times 10^7$ & Length dependence of on
                                               rate \\
    $\offForceScale$ & $\offForceScale = \stiffness\radius/\refForce$
                           & 1000 & Length dependence of off rate \\
    $\ndVelFriction$ & $\ndVelFriction =
                     \frac{\offConst}{\stiffness\receptorDensity}
                     \velFriction$ & $4 \times 10^{-12}$--$4 \times
                                     10^{-10}$
                                   & Translational drag coefficient \\
    $\ndRotFriction$ & $\ndRotFriction =
                     \frac{\offConst}{\stiffness\receptorDensity
                                      \radius^2}\rotFriction$
                           & $5 \times 10^{-12}$--$5 \times 10^{-10}$
                                   & Rotational drag coefficient
  \end{tabular}
  \caption{Nondimensional parameters}
  \label{tab:nd-params}
\end{table}

\bibliography{/Users/andrewwork/thesis/library.bib}

\end{document}