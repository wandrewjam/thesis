\documentclass{article}

\newcommand{\ep}{\rule{.06in}{.1in}}
\textheight 9.5in
\usepackage{amssymb}
\usepackage{amsmath}
\usepackage{amsthm}
\usepackage{subcaption}
\usepackage{graphicx}
\pagestyle{empty} 

\oddsidemargin -0.25in
\evensidemargin -0.25in 
\topmargin -0.75in 
\parindent 0pt
\parskip 12pt
\textwidth 7in

%\font\cj=msbm10 at 12pt

\newcommand{\R}{\mathbb{R}}
\newcommand{\dd}{d}
\newcommand{\Der}[2]{\frac{\dd #1}{\dd #2}}
\newcommand{\Pder}[2]{\frac{\partial #1}{\partial #2}}
\newcommand{\Int}[4]{\int_{#3}^{#4} #1 \, \dd #2}
\newcommand{\tn}{\textnormal}
\newcommand{\on}{\textnormal{on}}
\renewcommand{\arraystretch}{1.5}

\graphicspath{{/Users/andrewwork/thesis/rolling-model/plots/}}
%\topmargin -.5in 
\begin{document}
\pagestyle{empty}


\begin{center}
{\Large Rolling Model Write-Up}
\end{center}

\section{Reaction Rate Equation}
\label{sec:reaction-rate-equation}

Assume we have a circular platelet (of radius $R$) rolling and
translating in shear flow adjacent to a wall. The platelet translates
parallel to the wall at speed $V$, and rolls at angular velocity
$\Omega$. Because the circle is always translating parallel to the
wall, there is a fixed vertical distance $d$ between the wall and the
closest point on the circle. 

The platelet surface is covered with receptors at an angular density
of $N_T$ receptors/radian, and links can form between points on the
surface of the circle and points on the wall. We want to find a
function $n(x, \theta, t)$ that gives the density of links between a
point $x$ on the wall and a point $\theta$ on the circle, at some time
$t$. We define $x$ to be the horizontal distance from the center of
the circle, and $\theta$ to be the angle formed by the receptor, the
center of the circle, and the closest point on the circle to the wall.

Then the PDE governing the evolution of $n(x, \theta, t)$ is 
\begin{equation}
  \label{eq:master}
  \Pder{n}{t} = \Omega \Der{n}{\theta} + V \Der{n}{x} + \alpha(x,
  \theta) \left(N_T - \Int{n(x,\theta,t)}{x}{-\infty}{\infty}\right) - 
  \beta(x, \theta) n(x,\theta,t).
\end{equation}

There are a couple of things left unspecified here. We still need to
define
\begin{enumerate}
\item $\alpha(x, \theta)$---the bond formation rate between $x$ and
  $\theta$, \label{item:form-rate}
\item $\beta(x, \theta)$---the bond breaking rate between $x$ and
  $\theta$, \label{item:break-rate}
\item $V$ and $\Omega$---the platelet linear and angular
  velocities. \label{item:velocities} 
\end{enumerate}

Let's start with \ref{item:form-rate} and \ref{item:break-rate}. Let
\begin{equation}
  \label{eq:length}
  L(x,\theta)^2 \equiv (R - R\cos\theta + d)^2 + (R\sin\theta - x)^2
\end{equation}
be the length of a link between $x$ and $\theta$. We assume that
individual bonds act like linear springs with stiffness $k_f$ and rest
length 0. Then the force (magnitude) exerted on the platelet by a bond
of length $L$ is $F = k_f L$, and from Bell's Law the breaking rate of
the bond is 
\begin{equation}
  \label{eq:breaking-rate}
  \beta(x,\theta) = k_\tn{off} \exp\left( \frac{k_f L(x, \theta)}{f_0} \right). 
\end{equation}
The bond formation rate depends on $L$ as well: 
\begin{equation}
  \label{eq:formation-rate}
  \alpha(x, \theta) = k_\tn{on} \exp\left( -\frac{k_f L^2(x,
      \theta)}{2k_B T} \right).
\end{equation}

Finally, \ref{item:velocities} depends on the forces and torques
generated by bonds between the platelet and the surface. The
horizontal force exerted by a bond on the platelet is approximately
$f_h = k_f (x - R \sin\theta)$. Then integrating over all the bonds
gives the total horizontal force on the cell: 
\begin{equation}
  \label{eq:total-force}
  F_h = \Int{ \Int{f_h n(x,\theta,t)}{ \theta }{ -\frac{\pi}{2} }{
      \frac{\pi}{2}} }{ x }{ -\infty }{ \infty }.
\end{equation}

The torque exerted by an individual bond on the platelet is given by
the equation $\tau_s = -k_f R((R - R \cos\theta + d) \sin\theta + (R
\sin\theta - x) \cos\theta)$. Again the total force on the platelet is
calculated by integrating over all bonds:
\begin{equation}
  \label{eq:total-torque}
  \tau = \Int{ \Int{\tau_s n(x,\theta,t)}{ \theta }{ -\frac{\pi}{2} }{
      \frac{\pi}{2}} }{ x }{ -\infty }{ \infty }.
\end{equation}

Assume $V_f$ and $\Omega_f$ are the target linear and
rotational velocities in a shear flow with shear rate $\gamma$. Then
$V$ and $\Omega$ must balance with $F_h$ and $\tau$ in the following way:
\begin{align}
  \label{eq:vel}
  0 &= \xi_v (V_f - V) + F_h \\
  \label{eq:angvel}
  0 &= \xi_\omega (\Omega_f - \Omega) + \tau.
\end{align}

$V(t)$, $\Omega(t)$, and $n(x, \theta, t)$ are found by solving
equations \eqref{eq:master}, \eqref{eq:vel} and \eqref{eq:angvel}.

\subsection{Boundary Conditions}
\label{sec:boundary-conditions}

Along with \eqref{eq:master}, we need to specify boundary conditions
in $x$ and $\theta$ on $n(x,\theta,t)$. For both $x$ and $\theta$, we
need BCs on the upwind side. The natural assumption is that there are
no bonds attached to the wall far away from the cell
(i.e. $n(x,\theta,t) \rightarrow 0$ as $x \rightarrow \infty$). I also
assume that there are no bonds attached to the cell at $\theta <
-\pi/2$ and $\theta > \pi/2$, and so I set $n(x, \pi/2, t) =
0$. Numerically, I just set $n = 0$ at the upstream end of the $x$
interval. % Maybe run a simulation to verify this

\subsection{Nondimensionalization}
\label{sec:nond}

In order to eliminate redundant parameters in the system, we
nondimensionalize equations \eqref{eq:master}---~\eqref{eq:vel} using
the following nondimensional variables: $x \equiv R z$, $s \equiv
k_\tn{off} t$, and $n(R z, \theta, t/k_\tn{off}) \equiv N_T m(z, \theta,
s)/R$. With these substitutions, equation \eqref{eq:master} becomes 
\begin{multline}
  \label{eq:nd-master-1}
  \frac{N_T k_\tn{off}}{R} \Pder{m}{s} = \Omega \frac{N_T}{R}
  \Pder{m}{\theta} + V \frac{N_T}{R^2}\Pder{m}{z} + k_\tn{on}
  \exp\left( - \frac{k_f L^2(Rz, \theta)}{2 k_B T} \right) \left(N_T -
    N_T \Int{m(z, \theta, s)}{z}{-\infty}{\infty}\right) \\
  - k_\tn{off} \exp\left( \frac{k_f L(Rz, \theta)}{f_0} \right). 
\end{multline}
We can define a nondimensional length function $\ell$ so that $L(Rz,
\theta) \equiv R \ell(z, \theta)$. We have from equation
\eqref{eq:length} that $L^2(Rz, \theta) = (R - R \cos\theta + d)^2 +
(R \sin\theta - Rz)^2 = R^{-2} \left((1 - \cos\theta + d/R)^2 +
  (\sin\theta - z)^2 \right)$. Define $d \equiv R d'$, then
\begin{equation}
  \label{eq:nd-length}
  \ell^2(z, \theta) = (1 - \cos\theta + d')^2 + (\sin\theta - z)^2.
\end{equation}
After simplifying \eqref{eq:nd-master-1} and defining the
nondimensional parameters $\Omega \equiv \omega k_\tn{off}$, $V = v R
k_\tn{off}$, $\kappa \equiv R k_\tn{on}/k_\tn{off}$, $\eta \equiv k_f
R^2/(k_B T)$, and $\delta \equiv k_f R/f_0$ we get the following PDE
for $m$
\begin{equation}
  \label{eq:nd-master}
  \Pder{m}{s} = \omega \Pder{m}{\theta} + v \Pder{m}{z} + \kappa
  \exp\left(-\eta \frac{\ell^2}{2}\right) \left(1 - \Int{m(z, \theta,
    s)}{z}{-\infty}{\infty} \right) - \exp(\delta \ell) m.
\end{equation}

Next, let's look at the force and torque calculations. By substituting
out dimensional variables for nondimensional ones in equation
\eqref{eq:total-force}, we get
\begin{align}
  \nonumber
  F_h &= \frac{k_f N_T}{R} \Int{\Int{(Rz - R\sin\theta) m(z, \theta,
      s)}{\theta}{-\frac{\pi}{2}}{\frac{\pi}{2}}
  }{Rz}{-\infty}{\infty} \\
  \label{eq:nd-force}
      &= k_f N_T R \Int{\Int{(z - \sin\theta) m(z, \theta, s)}
        {\theta}{-\frac{\pi}{2}}{\frac{\pi}{2}}}{z}{-\infty}{\infty}
        \equiv k_f N_T R f'_h.
\end{align}
Finding the nondimensional torque follows the same procedure:
\begin{align}
  \nonumber
  \tau &= \frac{k_f N_T}{R} \Int{\Int{R \left[(R - R\cos\theta + R
         d')\sin\theta + (R\sin\theta - R z)\cos\theta \right] m(z,
         \theta, s)}{\theta}{-\frac{\pi}{2}}{\frac{\pi}{2}}}
         {Rz}{-\infty}{\infty} \\
  \label{eq:nd-torque}
       &= k_f N_T R^2 \Int{\Int{\left[(1 - \cos\theta + d')\sin\theta +
         (\sin\theta - z)\cos\theta \right] m(z, \theta,
         s)}{\theta}{-\frac{\pi}{2}}{\frac{\pi}{2}}}
         {z}{-\infty}{\infty} \equiv k_f N_T R^2 \tau'.
\end{align}

Finally, we can nondimensionalize the force and torque balance
equations. Define $\Omega_f \equiv \omega_f k_\tn{off}$ and $V_f
\equiv v_f R k_\tn{off}$. Then substituting \eqref{eq:nd-force} and
\eqref{eq:nd-torque} into equations \eqref{eq:vel} and \eqref{eq:angvel}
respectively gives us the following two equations: 
\begin{align}
  \label{eq:nd-vel-1}
  0 &= \xi_v (v_f R k_\tn{off} - v R k_\tn{off}) + k_f N_T R f'_h \\
  \label{eq:nd-angvel-1}
  0 &= \xi_\omega (\omega_f k_\tn{off} - \omega k_\tn{off}) + k_f N_T
      R^2 \tau'.
\end{align}

Then taking $~\eqref{eq:nd-vel-1}/(k_f N_T R)$ and
$~\eqref{eq:nd-angvel-1}/(k_f N_T R^2)$ gives 
\begin{align}
  \label{eq:nd-vel}
  0 &= \eta_v (v_f - v) + f'_h \\
  \label{eq:nd-angvel}
  0 &= \eta_\omega (\omega_f - \omega) + \tau'
\end{align}
where $\eta_v \equiv k_\tn{off} \xi_v / (k_f N_T)$ and $\eta_\omega
\equiv k_\tn{off} \xi_\omega / (k_f N_T R^2)$. Note a complete list of
the nondimensional variables and parameters are given in Table
\ref{tab:nd-vars}.

Then $v(s)$, $\omega(s)$, and $m(z, \theta, s)$ are found by
simultaneously solving equations \eqref{eq:nd-master},
\eqref{eq:nd-vel}, and \eqref{eq:nd-angvel}.

% The nondimensional system is given by the following equations:
% \begin{align}
%   \label{eq:nd-master1}
%   \Pder{m}{s} &= \omega \Pder{m}{\theta} + v \Pder{m}{z} + \kappa
%   \exp\left(-\eta \frac{l^2}{2}\right) \left(1 - \Int{m(z,\theta,s)}
%   {z}{-\infty}{\infty}\right) - \exp(\delta l) m(z,\theta,s) \\
%   \label{eq:nd-torq}
%   0 &= \eta_\omega(\omega_f - \omega) + \tau_s' \\
%   \label{eq:nd-vel}
%   0 &= \eta_v (v_f - v) + f_h'.
% \end{align}

% The nondimensional torque and force---$\tau_s'$ and $f_h'$
% respectively---are defined in the following way in terms of
% nondimensional parameters:
% \begin{align}
%   \label{eq:total-nd-torq}
%   \tau' &\equiv \Int{ \Int{((1 - \cos\theta + d') \sin\theta +
%           (\sin\theta - z) \cos\theta) m(z,\theta,t)}{ \theta
%           }{-\frac{\pi}{2}}{\frac{\pi}{2}} }{ z }{ -\infty }{ \infty }
%   &= \tau/(k_f N_T R^2) \\ 
%   \label{eq:total-nd-force}
%   f_h' &\equiv \Int{ \Int{(z - \sin\theta) m(z,\theta,t)}{ \theta }{
%          -\frac{\pi}{2} }{\frac{\pi}{2}} }{ z }{ -\infty }{ \infty }
%   &= F_h/(k_f N_T R).
% \end{align}

% The definitions of the nondimensional variables and parameters are
% given in Table \ref{tab:nd-vars}. 

\begin{table}
  \centering
  \begin{tabular}{c|c|l}
    ND Parameter & Parameter Definition & Description \\
    \hline
    $z$ & $x = Rz$ & ND distance coordinate \\
    $s$ & $k_\tn{off} t = s$ & ND time \\
    $m$ & $n = m N_t/R$ & ND bond density density \\
    $\omega$ & $\Omega = \omega k_\tn{off}$ & ND angular velocity \\
    $v$ & $V = R k_\tn{off} v$ & ND linear velocity \\
    $d'$ & $d = R d'$ & ND cell-surface separation \\
    $\omega_f$ & $\Omega_f = \omega_f k_\tn{off}$ & ND target angular
                                                    velocity \\ 
    $v_f$ & $V_f = R k_\tn{off} v_f$ & ND target linear velocity \\ 
    $\kappa$ & $\kappa = \frac{R k_\tn{on}}{k_\tn{off}}$ & ND max on
                                                           rate \\
    $\eta$ & $\eta = \frac{k_f R^2}{k_B T}$ & ??? \\
    $\delta$ & $\delta = k_f R/f_0$ & ??? \\
    $\eta_\omega$ & $\eta_\omega = \frac{k_\tn{off}}{k_f N_T R^2}
                    \xi_\omega$ & Nondimensional drag coefficient \\ 
    $\eta_v$ & $\eta_v = \frac{k_\tn{off}}{k_f N_T} \xi_v$ &
                                                             Nondimensional drag
                                                             coefficient
    \\
    $\tau_s'$ & $\tau = k_f N_T R^2 \tau_s'$ & ND torque \\
    $f_h'$ & $f_h = k_f N_T R f_h'$ & ND force \\
  \end{tabular}
  \caption{Nondimensional variables and parameters}
  \label{tab:nd-vars}
\end{table}

The nondimensional length $l$ between points $z$ and $\theta$ is
given by $l^2 = (1 - \cos(\theta) + d')^2 + (\sin(\theta) - z)^2$, and
is equivalent to $L/R$. 

What parameters can change with platelet activation/priming? One
possibility is that activation increases $N_T$. In nondimensional
parameters, this is represented by proportional reductions in
$\eta_\omega$ and $\eta_v$. That is, if we vary $\eta_\omega$ while
holding $\eta_\omega/\eta_v$ constant, that is equivalent to changing
$N_T$. 

It is also concievable that the on/off rates $k_\tn{on}$ and
$k_\tn{off}$ change. $k_\tn{on}$ only appears in the nondimensional
parameter $\kappa$, so we can vary this parameter while holding
everything else constant and that is equivalent to varying
$k_\tn{on}$. 

Finally, $k_\tn{off}$ appears in a bunch of nondimensional
parameters. This makes it hard to interpret its effects on
nondimensional results. It may be better to scale time by some other
parameter to avoid this...

\section{Steady State Solution}
\label{sec:steady-state-solut}

The steady solution (of the nondimensional system) is given by solving
equations \eqref{eq:nd-master}--\eqref{eq:nd-vel} with $\Pder{m}{s}
\equiv 0$ in \eqref{eq:nd-master}. This can be viewed as a 2D
nonlinear rootfinding problem
\begin{align}
  \label{eq:2D-ss-rotation}
  0 &= \eta_\omega (\omega_f - \omega) + \tau'_s(\omega, v)
  \\
  \label{eq:2D-ss-translation}
  0 &= \eta_v (v_f - v) + f'_h(\omega, v)
\end{align}
where evaluating the functions $\tau'_s$ and $f'_h$ requires solving
the steady state PDE for $m$ and then integrating the result.

\subsection{A rolling cell}
\label{sec:rolling-cell}

If we assume that a cell is sitting directly on top of the suface
(i.e. $d' = 0$), and the cell is only rolling, but not sliding ($V =
R\Omega \implies v = \omega$), then solving the steady state problem
reduces to 1D rootfinding problem. One can solve equation
\eqref{eq:2D-ss-rotation} to find the steady state value(s) of
$\omega$ for a given $\omega_f$, and then use $v = \omega$ to find
$v$. (Note: it is also possible to use equation
\eqref{eq:2D-ss-translation} to find $v$ for a given $v_f$, and in
general this will give a different value of $v$ than if you solved for
$\omega$ first and then set $v=\omega$.)

\subsubsection{Rolling results}
\label{sec:rolling-results}

\begin{figure}
  \centering
  \begin{subfigure}{.48\textwidth}
    \includegraphics[width=\textwidth]{torque-vs-rotation.png}
    \caption{Torque $\tau$ vs ND angular velocity $\omega$ at steady
      state}
    \label{fig:tau-vs-omega}
  \end{subfigure}
  \hfill
  \begin{subfigure}{.48\textwidth}
    \includegraphics[width=\textwidth]{rotation-vs-applied.png}
    \caption{Angular velocity $\omega$ vs Applied angular velocity
      $\omega_f$}
    \label{fig:omega-vs-omega_f}
  \end{subfigure}
  \caption{Relationships between torque, angular velocity, and applied
  angular velocity. ND parameter values are $\kappa = 1$, $\eta =
  0.1$, $\delta = 3$, $\eta_\omega = 1/7500$.}
  \label{fig:steady-state-rolling}
\end{figure}

In order to find steady state values of $\omega$ for a range of
$\omega_f$s, we generate a bunch of $(\omega_f, \omega)$ pairs that
solve the steady state problem. Then we can use interpolation to find
a steady state angular velocity $\omega$ for an unknown $\omega_f$. 

We first choose a set of $\omega$ values and then find the steady
state bond distribution for those $\omega$ using \eqref{eq:nd-master}$=
0$. Then we find the torque generated by that distribution of bonds
(shown in Figure \ref{fig:tau-vs-omega}), and find the necessary
$\omega_f$ using equation \eqref{eq:nd-angvel}. This gives us the
necessary set of $(\omega_f, \omega)$ pairs for interpolation. Figure
\ref{fig:omega-vs-omega_f} shows the result of linear interpolation
between these points. That is, it shows steady state angular
velocities for a range of applied angular velocities. 

Qualitatively, it is clear that there is a range of parameters for
which multiple steady state values of $\omega$ are possible for a
single $\omega_f$. The biological interpretation of Figure
\ref{fig:omega-vs-omega_f} is that for low applied angular velocity
(i.e. low shear rate) the only stable behavior is for platelets to
adhere to the surface and move very slowly with respect to the fluid
velocity. At some medium shear rate, a second stable behavior arises
where platelets roll along the surface substantially faster than the
adhered platelets. Finally, at a high shear rate, all platelets are
moving at a speed close to the fluid velocity and none are adhered to
the surface: the torque generated by the fluid on the platelet is too
large. It is worth noting that the analysis in Figure
\ref{fig:steady-state-rolling} doesn't show which solution branches
are stable and which are unstable, so more work is required to show
that the story I told above is consistent with the model. 

\begin{figure}
  \centering
  \includegraphics[width=0.45\textwidth]{rotation-vs-applied-Nt-large.png}
  \caption{Angular velocity $\omega$ vs Applied angular velocity
    $\omega_f$ for $\eta_\omega = 1/10\, 000$. All other parameters
    identical to those in Figure \ref{fig:steady-state-rolling}}
  \label{fig:rolling-large-Nt}
\end{figure}

Two parameters that could change with priming are $N_T$ and
$k_\tn{on}$. As discussed in section \ref{sec:nond}, $\eta_\omega$ and
$\eta_v$ are inversely proportional to $N_T$ (also in Table
\ref{tab:nd-vars}). In the rolling only case, $\eta_\omega$ is the
only parameter of these two that matters, and it is easy to see the
effect of changing its value. Rearranging equation \eqref{eq:nd-angvel},
we get the relation $\omega_f = \omega -
\tau'_s/\eta_\omega$. Increasing $N_T$ gives a proportional decrease
in $\eta_\omega$, which in turn increases the magnitude of the term
$\tau'_s/\eta_\omega$ in the equation above. This lengthens the
interval in $\omega_f$ in which bistability occurs, and the interval
is centered on higher $\omega_f$ as $\eta_\omega$ decreases. For
example, in Figure \ref{fig:omega-vs-omega_f} the system is bistable
for $\omega_f$ roughly in the interval 150--190, but in Figure
\ref{fig:rolling-large-Nt} the system is bistable for $\omega_f$ in
the interval 175--250. This agrees with intuition that
activation/priming should facilitate rolling at higher shear rates.

\begin{figure}
  \centering
  \begin{subfigure}{0.45\textwidth}
    \includegraphics[width=\textwidth]{torque-vs-rotation-kap-large.png}
    \caption{Torque $\tau$ vs ND angular velocity $\omega$ at steady
      state}
    \label{fig:tau-vs-omega-kappa-large}
  \end{subfigure}
  \quad
  \begin{subfigure}{0.45\textwidth}
    \includegraphics[width=\textwidth]{rotation-vs-applied-kap-large.png}
    \caption{Angular velocity $\omega$ vs Applied angular velocity
      $\omega_f$} 
    \label{fig:omega-vs-omega_f-kappa-large}
  \end{subfigure}
  \caption{Relationships between torque, angular velocity, and applied
    angular velocity. $\kappa = 3$, and all other parameters identical
    to Figure 1}
  \label{fig:rolling-large-kap}
\end{figure}

We can also look at what happens when we vary $k_\tn{on}$, by changing
$\kappa$ in the ND system. With priming we'd expect on rates to
increase, yielding an increase in $\kappa$. This should result in a
higher density of bonds at steady state, generating larger torques at
the same rotation rate $\omega$. As shown in Figure
\ref{fig:tau-vs-omega-kappa-large}, we do get larger torque
magnitudes, and the rotation rate $\omega$ at which $|\tau|$ is
maximized is larger than in Figure \ref{fig:tau-vs-omega}. In Figure
\ref{fig:omega-vs-omega_f-kappa-large} the region of bistability
exists at larger $\omega_f$, but the size of this region seems to be
smaller than it is in either Figures \ref{fig:omega-vs-omega_f} or
\ref{fig:rolling-large-Nt}. 

$k_\tn{off}$ could also change with priming, but my
choice of nondimensionalization isn't convenient for examining changes
in this parameter. I might try to nondimensionalize time in a
different way so that I can alter this parameter easily. 

\subsection{Rolling and sliding}
\label{sec:rolling-sliding}

If we allow both rolling and sliding, then we must solve equations
\eqref{eq:2D-ss-rotation} and \eqref{eq:2D-ss-translation}
simultaneously. I have code that can generate quadruples $(\omega, v,
\omega_f, v_f)$ that solve the steady state equations, but it isn't
easy to figure out how to present this data in a sensible way. 

A couple of possibilities for using this code:
\begin{enumerate}
\item Solve the steady state problem for a specified $(\omega_f, v_f)$
  pair. This could be done by interpolating from a set of precomputed
  $(\omega, v, \omega_f, v_f)$ tuples, like in the rolling only
  problem. This really only makes sense if we want to find solutions
  for a bunch of different $(\omega_f, v_f)$ pairs. If we only want to
  solve the problem for a few specific $(\omega_f, v_f)$s, I think it
  would make more sense to use a different method for rootfinding in 2D
  nonlinear systems. 
\item It may be interesting to generate figures to show the regions in $(z,
  \theta)$ space where there are multiple equilibrium
  solutions. However I'm not sure how to approach this problem, so
  it's probably better to work on other things until we're sure this
  is something we want to pursue.
  % Find regions where there are multiple steady state solutions,
  % treat it more like a bifurcation problem, 2-parameter
  % bifurcation diagrams, and others?
\end{enumerate}

\section{Time-Dependent Solution}
\label{sec:time-depend-solut}

\begin{figure}
  \centering
  \begin{subfigure}{0.45\textwidth}
    \includegraphics[width=\textwidth]{unbound-expt-ang.png}
    \caption{Platelet angular velocity (initial condition: $\omega =
      \omega_f$)}
    \label{fig:unbound-expt-ang}
  \end{subfigure}
  \quad
  \begin{subfigure}{0.45\textwidth}
    \includegraphics[width=\textwidth]{unbound-expt-vel.png}
    \caption{Platelet linear velocity (initial condition: $v = v_f$)}
    \label{fig:unbound-expt-vel}
  \end{subfigure}
  \caption{Linear and angular velocities of a rolling
    platelet. Initial condition: $m(x, \theta, 0) \equiv
    0$. Parameters identical to those in Figure
    \ref{fig:steady-state-rolling}, except $d' = 0.1$.}
  \label{fig:unbound-expt}
\end{figure}

\begin{figure}
  \centering
  \begin{subfigure}{0.45\textwidth}
    \includegraphics[width=\textwidth]{unmoving-expt-ang.png}
    \caption{Platelet angular velocity (initial condition: $\omega =
      0$)}
    \label{fig:unmoving-expt-ang}
  \end{subfigure}
  \quad
  \begin{subfigure}{0.45\textwidth}
    \includegraphics[width=\textwidth]{unmoving-expt-vel.png}
    \caption{Platelet linear velocity (initial condition: $v = 0$)}
    \label{fig:unmoving-expt-vel}
  \end{subfigure}
  \caption{Linear and angular velocities of a rolling
    platelet. Initial condition: $m(x, \theta, 0) =
    m_\tn{ss}$, where $m_\tn{ss}$ is the distribution of bonds at
    steady state with $\omega, v = 0$. Parameters identical to those
    in Figure \ref{fig:unbound-expt}.} 
  \label{fig:unmoving-expt}
\end{figure}

In order to see the dynamic behavior of a rolling platelet, we have to
solve the time-dependent problem (equations
\eqref{eq:nd-master}--\eqref{eq:nd-vel}). Below I've run a couple
different numerical simulations meant to mimic biological
experiments. 

\begin{enumerate}
\item \textbf{Unbound platelet moving freely in the fluid:} In Figure
  \ref{fig:unbound-expt}, I take $m(x, \theta, 0) \equiv 0$, $\omega =
  \omega_f$, and $v = v_f$. This shows the dynamic rolling behavior of
  a platelet which initially is completely unbound from the wall and
  moving with the fluid. As expected, bonds form and slow the platelet
  down until it reaches a steady state velocity. 
\item \textbf{Platelet firmly adhered to the surface in no flow:} In
  Figure \ref{fig:unmoving-expt}, I take $m(z, \theta, 0) \equiv
  m_\tn{ss}(z, \theta)$, and $\omega = v = 0$, where $m_\tn{ss}$ is
  the steady state bond distribution of a platelet in no flow (but one
  which is at a distance $d'$ from the wall). Experimentally, this
  could be achieved by allowing a platelet to adhere to the wall in no
  flow before starting the flow experiment. These results don't make
  much sense, because in the very first time step, the platelet
  velocity jumps from zero to the fluid velocity. This is due to
  numerically lagging the linear and angular velocities. At the
  initial condition, the net force and torque generated by the bonds
  is 0. Therefore in the step where I take $v^{i+1} = v_f + f'_{i+1} /
  \eta_v$, $f'_1 = 0$ and then $v^1 = v_f$. I probably need to do
  something a little more sophisticated in the timestep to ensure that
  the velocities don't change too quickly.
\end{enumerate}

\subsection{Numerics}
\label{sec:numerics}

The PDE \eqref{eq:nd-master} is a linear advection-reaction equation
for known $\omega$ and $v$, but the whole system including the force
balance equations (i.e. \eqref{eq:nd-master}, \eqref{eq:nd-vel}, and
\eqref{eq:nd-angvel}) is nonlinear. Numerically, I linearized the
system by lagging the angular and linear velocities in order to find
the bond density distribution at the next time step. Then I integrated
over the bond distribution to find the net force and torque generated
by the bonds, and used that to update the angular and linear
velocities. As suggested by the results above, I will need to modify
this approach to handle the case where a platelet is initially firmly
adhered and unmoving, until fluid forces are applied.

I solve the PDE using a first order upwind scheme in both $\theta$ and
$z$. These velocities are always non-negative, so I approximate the
spatial derivatives with a forward difference. The bond formation term
is treated explicitly, so that I can advance time with just a matrix
multiplication, instead of solving a linear system. The bond breaking
term essentially \emph{must} be treated implicitly, because the
breaking rate is very large over much of the domain $(z,
\theta)$. Luckily it only requires a scalar division for each
element of $m^k$ to treat that term implicitly.

In summary, for each timestep $k$, I do the following:
\begin{enumerate}
\item Advance the PDE in time using the velocities from the previous
  timestep:
  \begin{multline}
    \label{eq:pde-timestep}
    m_{i,j}^{k+1} = \frac{1}{1 + \Delta t \exp(\delta l_{ij})}
    \left(m_{ij}^k + \omega^k \Delta t \left(\frac{m_{i, j+1}^k -
          m_{ij}^k}{\nu}\right) + v^k \Delta t \left(\frac{m_{i+1, j}^k
          - m_{ij}^k}{h}\right)\right. \\
      \left. + \kappa \Delta t \exp\left(-\eta
        \frac{l_{ij}}{2}\right) \left(1 - h \sum_{q = 0}^{N-1}
        m_{qj}^k\right)\right) 
  \end{multline}
\item Calculate $f'_{k+1}$ and $\tau'_{k+1}$ from the new bond
  distribution $m^{k+1}$.
\item Find the new angular and linear velocities:
  \begin{align}
    \label{eq:ang-timestep}
    \omega^{k+1} &= \omega_f + \tau'_{k+1}/\eta_\omega \\
    \label{eq:vel-timestep}
    v^{k+1} &= v_f + f'_{k+1}/\eta_v.
  \end{align}
\item Repeat
\end{enumerate}

\section{Stochastic Model}
\label{sec:stochastic-model}

For the stochastic model, assume that there are $N_T$ receptors per
radian, and the receptors are uniformly distributed along the surface
of the platelet. Assume there is an excess of ligand on the wall
surface, so that bonds can form anywhere along the wall and there is
no saturation of bonds in the $x$ dimension.

A bond can form between a receptor located at $\theta$ and a point $x$
on the wall at ``rate'' $\alpha(x, \theta)$. More precisely, the time
$\tau$ that it takes a bond to form between $x$ and $\theta$ is
exponentially distributed with mean $\alpha^{-1}$. Similarly, the time
it takes a bond between $\theta$ and $x$ to break is exponentially
distributed with mean $1/\beta(x, \theta)$. Finally, a single bond
between $x$ and $\theta$ generates horizontal force and torque which
are given by $f_h$ and $\tau_s$ as defined in section
\ref{sec:reaction-rate-equation}. As in the continuous model, assume
that only bonds on the lower half-circle of the platelet can interact
with the wall surface.

One way to simulate this model would be to track each of the $2\pi
N_T$ receptors on the surface of the platelet, whether they are bound
or unbound. But this seems expensive to do, so I did something else. I
partitioned the interval $[-\pi/2, \pi/2)$ into $N$ subintervals of
equal length $I_j = \left[-\frac{\pi}{2} + \frac{\pi}{N}j,
  -\frac{\pi}{2} + \frac{\pi}{N} (j+1)\right)$. Then there can be at
most $b_\tn{max} \equiv N_T \pi/N$ bonds in any subinterval. I still
keep track of the exact $x$ and $\theta$ positions the bond forms
between, the subintervals are only used to limit the bond
concentration at $N_T$ per radian. 

To simulate random bond breaking, a random number is generated for
each existing bond to test if it breaks within the time step $\Delta
t$. The probability that a bond between $x$ and $\theta$ in the time
interval $[t, t + \Delta t)$ is $\mathbb{P} = 1 - \exp(-\Delta t
\beta(x, \theta))$.

Next look at bond formation. The probability of a bond forming between
points $\theta$ and $x$ can be split into two distinct probabilities:
the probability that a bond forms from a receptor at position $\theta$
to anywhere on the wall (denote this event as $A$, and the associated
probability as $\mathbb{P}(A)$), and the probability that a bond forms
to the point $x$ on the wall from any receptor (denote this event as
$B$, and the probability $\mathbb{P}(B)$). The probability of the
combined event $A \cap B$ is $\mathbb{P}(A \cap B) = dt
\alpha(x, \theta)$. 

% To simulate bond formation, first assume that at most 1 bond can
% form in a single time step in each subinterval $I_j$. One can think of
% bond formation 

% Then for each
% subinterval, calculate the probality of a bond forming within that
% subinterval in the time interval $\Delta t$. If a bond forms then an
% $x$ is chosen at random.

First, the probability that a bond forms within subinterval $I_j$ is
given by the rate of bond formation multiplied by the number of
available receptors in $I_j$. The rate that bonds form from $\theta$
to \emph{any} point on the wall, integrate $\alpha$ with respect to
$x$. In the case $\alpha(x, \theta) = k_\tn{on} \exp\left(-\frac{k_f
    L^2}{2k_B T}\right)$, we have
\begin{equation}
  \label{eq:total_fm_rate}
  \int_{-\infty}^{\infty} \alpha(x, \theta)\, dx = k_\tn{on} \exp
  \left( -\frac{k_f}{2k_B T} (R - R \cos\theta + d)^2 \right)
  \sqrt{\frac{2 \pi k_B T}{k_f}}.
\end{equation}
Therefore the probability that a bond forms somewhere in the interval
$I_j$ to any point on the wall in a time interval of length $dt$ is
given by 
\begin{equation}
  \label{eq:1}
  \mathbb{P} = dt (b_\tn{max} - n_j) \left[ k_\on \exp \left( -
      \frac{k_f}{2 k_B T} \left( R - R\cos\theta + d \right)^2 \right)
    \sqrt{\frac{2\pi k_b T}{k_f}} \right].
\end{equation}

Next, we need to find the probability that a bond forms between a
point $x$ on the wall and $\theta$, given that a bond forms from
$\theta$. From the definition 

\begin{equation}
  \label{eq:2}
  \exp \left(-\frac{k_f}{2 k_B T}
    \left(x - R \sin{\theta} \right) \right)
\end{equation}

\subsection{Algorithm}
\label{sec:algorithm}

\begin{enumerate}
\item Discretize the interval $[-\pi/2, \pi/2)$ into $N$ equal bins.
\item Initialize list of bonds and angular and linear velocities. 
\item For each time step up to $t_\tn{max}$, \label{item:stoch-iter}
  \begin{enumerate}
  \item Update the $z$ and $\theta$ values for every bond. (For the
    variable time step algorithm, this happens after the formation and
    breaking rates are calculated, but before the bond list is updated)
  \item Bin each bond in one of the $N$ $\theta$-bins. Any bond for
    which $\theta < -\pi/2$ or $\theta > \pi/2$ is flagged to be
    removed from the list.
  \item For each $\theta$-bin, compute the rate of bond formation
    within that bin from $\alpha(\theta_j) (b_\tn{max} - n_j)$ where
    $n_i$ is the number of existing bonds with endpoints in interval
    $I_j$.
  \item For each existing bond, compute the rate of bond breaking.
  \item Update the bond list:
    \begin{itemize}
    \item If using the fixed time step algorithm, generate a random
      number for each existing bond to decide which ones break, and
      randomly sample from N Poisson distributions with $\lambda = dt
      \alpha(\theta_j) (b_\tn{max} - n_j)$, $j = 1, \hdots, N$ to
      decide how many bonds form in each interval.
    \item If using the variable time step algorithm, generate a random
      number to decide when the first reaction (a bond breaking or
      forming) occurs, and generate a second random number to decide
      which reaction occurs.
    \end{itemize}
  \item Calculate the forces and torques generated by the existing
    bonds, and solve the force balance equations to calculate the new
    $(v, \omega)$ pair.
  \item Return to step \ref{item:stoch-iter}.
  \end{enumerate}
\end{enumerate}

\subsection{Results}
\label{sec:results}

In the figure below, I compare solutions to the deterministic PDE with
results from the two stochastic simulation algorithms. For the PDE, I
show solutions both with and without saturation in $\theta$. For the
stochastic simulations, I show curves for the average and average
$\pm$ standard deviation for the variable time step algorithm, the
fixed time step algorithm with saturation, and the fixed time step
algorithm without saturation. 

The average velocity for each of the three stochastic algorithms I used is
noticeably lower than the velocity given by the deterministic
algorithm. I don't think this can be related to time step size, or
long bonds lasting ``too long,'' because the variable time step
algorithm has the same behavior. Should I be concerned about this? The
stochastic simulation algorithms should ideally converge to the
deterministic model in some limit. 

\begin{figure}[h]
  \centering
  \includegraphics[width=\textwidth]{{comp_N100_bmax10_eta0.01}.png}
  \caption{Comparison of deterministic results with stochastic results
    for $\kappa = 1, \eta = 0.1, \delta=3, \eta_\omega = \eta_v =
    0.01, \gamma=20$. For the stochastic simulations, $b_\tn{max} =
    10, N = 100$. Solid lines show averages and dashed lines show
    average $\pm$ standard deviation over 100 runs.}
  \label{fig:comp_N100_bmax10_eta0.01}
\end{figure}
\end{document}
