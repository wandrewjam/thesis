%%%%%%%%%%%%%%%%%%%%%%%%%%%%%%%%%%%%%%%%%
% Article Notes
% LaTeX Template
% Version 1.0 (1/10/15)
%
% This template has been downloaded from:
% http://www.LaTeXTemplates.com
%
% Authors:
% Vel (vel@latextemplates.com)
% Christopher Eliot (christopher.eliot@hofstra.edu)
% Anthony Dardis (anthony.dardis@hofstra.edu)
%
% License:
% CC BY-NC-SA 3.0 (http://creativecommons.org/licenses/by-nc-sa/3.0/)
%
%%%%%%%%%%%%%%%%%%%%%%%%%%%%%%%%%%%%%%%%%

%----------------------------------------------------------------------------------------
%	PACKAGES AND OTHER DOCUMENT CONFIGURATIONS
%----------------------------------------------------------------------------------------

\documentclass[
10pt, % Default font size is 10pt, can alternatively be 11pt or 12pt
letterpaper, % Alternatively letterpaper for US letter
twocolumn, % Alternatively onecolumn
landscape % Alternatively portrait
]{article}

\usepackage{amsmath, textgreek}
\newcommand{\inta}[1]{\textalpha\textsubscript{#1}}
\newcommand{\intb}[1]{\textbeta\textsubscript{#1}}

%%%%%%%%%%%%%%%%%%%%%%%%%%%%%%%%%%%%%%%%%
% Article Notes
% Structure Specification File
% Version 1.0 (1/10/15)
%
% This file has been downloaded from:
% http://www.LaTeXTemplates.com
%
% Authors:
% Vel (vel@latextemplates.com)
% Christopher Eliot (christopher.eliot@hofstra.edu)
% Anthony Dardis (anthony.dardis@hofstra.edu)
%
% License:
% CC BY-NC-SA 3.0 (http://creativecommons.org/licenses/by-nc-sa/3.0/)
%
%%%%%%%%%%%%%%%%%%%%%%%%%%%%%%%%%%%%%%%%%

%----------------------------------------------------------------------------------------
%	REQUIRED PACKAGES
%----------------------------------------------------------------------------------------

\usepackage[includeheadfoot,columnsep=2cm, left=1in, right=1in, top=.5in, bottom=.5in]{geometry} % Margins

\usepackage[T1]{fontenc} % For international characters
\usepackage{XCharter} % XCharter as the main font

\usepackage{natbib} % Use natbib to manage the reference
\bibliographystyle{apalike} % Citation style

\usepackage[english]{babel} % Use english by default

%----------------------------------------------------------------------------------------
%	CUSTOM COMMANDS
%----------------------------------------------------------------------------------------

\newcommand{\articletitle}[1]{\renewcommand{\articletitle}{#1}} % Define a command for storing the article title
\newcommand{\articlecitation}[1]{\renewcommand{\articlecitation}{#1}} % Define a command for storing the article citation
\newcommand{\doctitle}{\articlecitation\ --- ``\articletitle''} % Define a command to store the article information as it will appear in the title and header

\newcommand{\datenotesstarted}[1]{\renewcommand{\datenotesstarted}{#1}} % Define a command to store the date when notes were first made
\newcommand{\docdate}[1]{\renewcommand{\docdate}{#1}} % Define a command to store the date line in the title

\newcommand{\docauthor}[1]{\renewcommand{\docauthor}{#1}} % Define a command for storing the article notes author

% Define a command for the structure of the document title
\newcommand{\printtitle}{
\begin{center}
\textbf{\Large{\doctitle}}

\docdate

\docauthor
\end{center}
}

%----------------------------------------------------------------------------------------
%	STRUCTURE MODIFICATIONS
%----------------------------------------------------------------------------------------

\setlength{\parskip}{3pt} % Slightly increase spacing between paragraphs

% Uncomment to center section titles
%\usepackage{sectsty}
%\sectionfont{\centering}

% Uncomment for Roman numerals for section numbers
%\renewcommand\thesection{\Roman{section}}
 % Input the file specifying the document layout and structure

%----------------------------------------------------------------------------------------
%	ARTICLE INFORMATION
%----------------------------------------------------------------------------------------

\articletitle{Single-molecule study on GP1b\inta{} and
  vWF-mediated platelet adhesion and signal triggering} % The title of
                                % the article 
\articlecitation{\cite{Ju2013}} % The BibTeX citation key from your
                                % bibliography 

\datenotesstarted{June 13, 2018} % The date when these notes were
                                % first made 
\docdate{\datenotesstarted; rev. \today} % The date when the notes were lasted updated (automatically the current date)

\docauthor{Andrew Watson} % Your name

%----------------------------------------------------------------------------------------

\begin{document}

\pagestyle{myheadings} % Use custom headers
\markright{\doctitle} % Place the article information into the header

%----------------------------------------------------------------------------------------
%	PRINT ARTICLE INFORMATION
%----------------------------------------------------------------------------------------

\thispagestyle{plain} % Plain formatting on the first page

\printtitle % Print the title

%----------------------------------------------------------------------------------------
%	ARTICLE NOTES
%----------------------------------------------------------------------------------------

\section{Summary}
\label{sec:summary}

\begin{itemize}
\item Platelet adhesion is initiated in a two-step cascade:
  \begin{enumerate}
  \item Fast tethering to immobilized vWF by GP1b\inta{} recruits
    platelets to the site of vascular injury.
  \item Stable adhesion is then mediated by integrins. For example,
    \inta{IIb}\intb{3} binds to immobilized vWF and fibronectin, and
    also immobilizes plasma proteins like vWF and fibrinogen.
  \end{enumerate}
\item This is the general picture of thrombus formation, however the
  details still remain to be ironed out. In particular, they are
  interested in how a mechanical stimulus is translated into
  biochemical signals in this two-step cascade.
\item They used a biomembrane force probe (BFP) to measure
  force-dependent lifetimes of molecular bonds, to estimate the
  association and dissociation rates of individual bonds.
\item They also combined the BFP with fluorescence imaging to observe
  kinetics and Ca$^{++}$ signaling together.
\item They made the following new insights on platelet adhesion:
  \begin{enumerate}
  \item The vWF-GP1b\inta{} displays a catch-bond behavior, where the
    off rate decreases with increasing shear force
  \item GP1b\inta{} has a high on rate, and is therefore critical in
    mediating cell tethering to substrate at higher shear rates.
  \item There was a strong correlation between bond lifetimes and
    Ca$^{++}$ levels. This supports a model where force and strutural
    variation regulate platelet signaling through the lifetime of the
    A1-GP1b bond.
  \item They found GP1b-triggered integrin priming. Following a
    calcium signal, \inta{IIb}\intb{3} displayed an intermediate
    binding affinity. Notably, platelet shape does not change,
    suggesting that this priming is activation independent. Thus it
    may reinforce adhesion following initial GP1b-vWF tethering.
  \end{enumerate}
\end{itemize}

%------------------------------------------------

\section{Introduction}
\label{sec:introduction}

\begin{itemize}
\item Regulation of the initial adhesion process is important, as
  insufficient adhesion cannot stop bleeding well enough to maintain
  hemostasis, while too much adhesion results in thrombosis.
\item Traditional methods of studying platelet adhesion (flow chambers
  and microfluidics) lack temporal and spatial resolution, and are
  unable to characterize kinetics of single bonds. They used
  single-bond experiments using a BFP to characterize adhesion and
  signaling functions of GP1b.
\item Three specific aims:
  \begin{enumerate}
  \item Study how vWF regions surrounding A1 regulates the
    vWF-GP1b\inta{} interaction under force.
    \begin{itemize}
    \item vWF requires an extensional force to uncover the A1 domain
      for binding with GP1b.
    \item They used a BFP to analyze the regulatory role of vWF
      regions surrounding the A1 domain, in particular looking at the
      N-terminal flanking region. They used a BFP to analyze the
      single-bond dissociation of GP1b\inta{} from vWF, and to examine
      the effect of structural variation on vWF-GP1b\inta{}
      kinetics. They also used a flow chamber to measure the rolling
      velocity of washed platelets over immobilized vWF.
    \end{itemize}
  \item Characterize physical reulation of vWF-GP1b\inta{} 2D
    association.
    \begin{itemize}
    \item Two possible mechanisms for flow-enhanced platelet adhesion
      to vWF: transport-dependent on rate, and a force-dependent off
      rate.
    \item They used a BFP to address these two hypotheses
    \end{itemize}
  \item Study how force triggers platelet calcium signaling via
    GP1b\inta{}.
    \begin{itemize}
    \item They wanted to demonstrate that mechanical force regulates
      platelet signaling. More specifically they defined the
      vWF-GP1b\inta{} catch bond as responsible for optimal Ca$^{++}$
      signaling, and verified the hypothesis that the Ca$^{++}$ signal
      triggered by vWF-GP1b\inta{} is responsible for priming integrins.
    \end{itemize}
  \end{enumerate}
\end{itemize}

%------------------------------------------------

\section{Chapter 4: The interplay of force and the A1 domain
  N-terminal flanking region on regulating the vWF-GP1b\inta{} 
  catch bond}
\label{sec:chapter4}



%------------------------------------------------

\section{Chapter 5: Transport regulation of 2D kinetics of
  vWF-GP1b\inta{} association}
\label{sec:chapter5}

\begin{itemize}
\item 
\end{itemize}

%------------------------------------------------

\section{Chapter 6: vWF-GP1b\inta{} catch bond triggers platelet
  signaling by force prolonged bond lifetimes}
\label{sec:chapter6}

\begin{itemize}
\item GP1b\inta{} transduces signals necessary to upregulate
  \inta{IIb}\intb{3}. The Ca$^{++}$ signals generated by GP1b\inta{}
  signaling cause a ``low level activation'' which primes
  \inta{IIb}\intb{3} but does not induce other platelet activation
  signals. 
\item There is ample evidence identifying GP1b\inta{} as a
  mechanosensor, meaning its dual functions of adhesion and signaling
  are directly related, although the exact mechanism of this is still
  unknown. (Perhaps unknown in 2013, but known now?)
\item The type 2B mutant of A1 causes an increased affinity of vWF for
  GP1b\inta{}, but counterintuitively results in a bleeding
  disorder. A study has shown that binding of 2B vWF to GP1b\inta{}
  did not activate \inta{IIb}\intb{3} and led to unstable thrombus
  formation. 
\end{itemize}

%------------------------------------------------

\section{Conclusions}
\label{sec:conclusions}



%-------------------------------------------------------------------------
%	BIBLIOGRAPHY
%-------------------------------------------------------------------------

\renewcommand{\refname}{Reference} % Change the default bibliography
                                % title 

\bibliography{/Users/andrewwork/thesis/library} % Input your
                                % bibliography file 

%-------------------------------------------------------------------------

\end{document}