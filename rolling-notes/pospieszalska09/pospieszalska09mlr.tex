%%%%%%%%%%%%%%%%%%%%%%%%%%%%%%%%%%%%%%%%%
% Article Notes
% LaTeX Template
% Version 1.0 (1/10/15)
%
% This template has been downloaded from:
% http://www.LaTeXTemplates.com
%
% Authors:
% Vel (vel@latextemplates.com)
% Christopher Eliot (christopher.eliot@hofstra.edu)
% Anthony Dardis (anthony.dardis@hofstra.edu)
%
% License:
% CC BY-NC-SA 3.0 (http://creativecommons.org/licenses/by-nc-sa/3.0/)
%
%%%%%%%%%%%%%%%%%%%%%%%%%%%%%%%%%%%%%%%%%

%----------------------------------------------------------------------------------------
%	PACKAGES AND OTHER DOCUMENT CONFIGURATIONS
%----------------------------------------------------------------------------------------

\documentclass[
10pt, % Default font size is 10pt, can alternatively be 11pt or 12pt
letterpaper, % Alternatively letterpaper for US letter
twocolumn, % Alternatively onecolumn
landscape % Alternatively portrait
]{article}

\usepackage{amsmath}
%%%%%%%%%%%%%%%%%%%%%%%%%%%%%%%%%%%%%%%%%
% Article Notes
% Structure Specification File
% Version 1.0 (1/10/15)
%
% This file has been downloaded from:
% http://www.LaTeXTemplates.com
%
% Authors:
% Vel (vel@latextemplates.com)
% Christopher Eliot (christopher.eliot@hofstra.edu)
% Anthony Dardis (anthony.dardis@hofstra.edu)
%
% License:
% CC BY-NC-SA 3.0 (http://creativecommons.org/licenses/by-nc-sa/3.0/)
%
%%%%%%%%%%%%%%%%%%%%%%%%%%%%%%%%%%%%%%%%%

%----------------------------------------------------------------------------------------
%	REQUIRED PACKAGES
%----------------------------------------------------------------------------------------

\usepackage[includeheadfoot,columnsep=2cm, left=1in, right=1in, top=.5in, bottom=.5in]{geometry} % Margins

\usepackage[T1]{fontenc} % For international characters
\usepackage{XCharter} % XCharter as the main font

\usepackage{natbib} % Use natbib to manage the reference
\bibliographystyle{apalike} % Citation style

\usepackage[english]{babel} % Use english by default

%----------------------------------------------------------------------------------------
%	CUSTOM COMMANDS
%----------------------------------------------------------------------------------------

\newcommand{\articletitle}[1]{\renewcommand{\articletitle}{#1}} % Define a command for storing the article title
\newcommand{\articlecitation}[1]{\renewcommand{\articlecitation}{#1}} % Define a command for storing the article citation
\newcommand{\doctitle}{\articlecitation\ --- ``\articletitle''} % Define a command to store the article information as it will appear in the title and header

\newcommand{\datenotesstarted}[1]{\renewcommand{\datenotesstarted}{#1}} % Define a command to store the date when notes were first made
\newcommand{\docdate}[1]{\renewcommand{\docdate}{#1}} % Define a command to store the date line in the title

\newcommand{\docauthor}[1]{\renewcommand{\docauthor}{#1}} % Define a command for storing the article notes author

% Define a command for the structure of the document title
\newcommand{\printtitle}{
\begin{center}
\textbf{\Large{\doctitle}}

\docdate

\docauthor
\end{center}
}

%----------------------------------------------------------------------------------------
%	STRUCTURE MODIFICATIONS
%----------------------------------------------------------------------------------------

\setlength{\parskip}{3pt} % Slightly increase spacing between paragraphs

% Uncomment to center section titles
%\usepackage{sectsty}
%\sectionfont{\centering}

% Uncomment for Roman numerals for section numbers
%\renewcommand\thesection{\Roman{section}}
 % Input the file specifying the document layout and structure

%----------------------------------------------------------------------------------------
%	ARTICLE INFORMATION
%----------------------------------------------------------------------------------------

\articletitle{Modeling Leukocyte Rolling} % The title of the article
\articlecitation{} % The BibTeX citation key from your bibliography

\datenotesstarted{April 5, 2018} % The date when these notes were first made
\docdate{\datenotesstarted; rev. \today} % The date when the notes were lasted updated (automatically the current date)

\docauthor{Andrew Watson} % Your name

%----------------------------------------------------------------------------------------

\begin{document}

\pagestyle{myheadings} % Use custom headers
\markright{\doctitle} % Place the article information into the header

%----------------------------------------------------------------------------------------
%	PRINT ARTICLE INFORMATION
%----------------------------------------------------------------------------------------

\thispagestyle{plain} % Plain formatting on the first page

\printtitle % Print the title

%----------------------------------------------------------------------------------------
%	ARTICLE NOTES
%----------------------------------------------------------------------------------------

\section{Introduction} 

\begin{itemize}
% Introduction
\item Leukocyte rolling is mediated by a continuous series of
  molecular bonds between the cell and the substrate that rapidly form
  and dissociate. Some models of rolling are direct models that track
  the states of important molecules and bonds, and others are
  semi-analytic or analytic or agent-based models (unsure what this
  means?). 
% Motivation for modeling leukocyte rolling
\item Rolling of leukocytes is necessary for firm adhesion. Rolling of
  leukocytes involves, at a minimum, a selectin and selectin
  ligand. Selectins bind carbohydrates and are expressed on leukocytes
  and other cells.
\item There are different scales of leukocyte binding/rolling, from
  the cellular level down to the molecular level ($\sim$ 4 orders of
  magnitude).
\item While the simplest description of rolling involves a single
  selectin binding to a single ligand, in reality rolling involves
  several selectins interacting with an unknown number of ligands.
\item Rolling velocity is modulated by integrins. 
\item Mechanical properties of the cell are also important. Rolling of
  rigid beads, for example, is unstable even at relatively low shear
  rates (no specific shear rates are given in this section).
\item Modeling shows that rolling requires molecular bonds with fast
  association and dissociation.
% History of modeling leukocyte rolling
\item The motion of a rigid sphere in flow next to a wall can be
  calculated (see Goldman et al., 1967), and the translational
  velocity of a sphere in flow next to a wall is the hydrodynamic
  velocity. Rolling cells travel at a much lower velocity than the
  hydrodynamic velocity. 
\item A repulsive force is necessary to prevent permanent adhesion of
  a sphere to the wall. 
\item There are four basic types of leukocyte rolling models:
  \begin{enumerate}
  \item Direct models---track receptor-ligand bonds individually,
    track positions of receptors and ligands, and their association
    state 
  \item Semi-analytic models---track bond density in the cell-substrate
    contact area using kinetic rate equations
  \item Analytic models---``describe the evolution of the rolling
    process through individual variables, the evolution of which can
    be described by mathematical equations.''
  \item Agent-based models---``multilevel, object-oriented models,
    which are typically based on available software toolkits.''
  \end{enumerate}
\item See p. 228 for a brief historical description of the development
  of leukocyte rolling models. They mention a couple of models where
  the activity of integrins is included.
% Development of a leukocyte rolling model
\item Four important classes of parameters:
  \begin{enumerate}
  \item Cellular parameters like cell radius and density, properties
    of the microvilli, and viscoelastic/mechanical properties.
  \item Molecular parameters like receptor and ligand lengths,
    densities, and bond properties
  \item Environmental parameters like temperature and fluid properties
  \item Algorithmic/Numerical parameters, like time step, mesh size, etc.
  \end{enumerate}
\item There are three classes of interactions:
  \begin{enumerate}
  \item Receptor-ligand interactions: the Bell model or the Dembo
    model and MC simulations
  \item Cell-substrate interactions: forces due to gravity, other cells,
    electrostatics, and receptor-ligand binding
  \item Cell-fluid interactions.
  \end{enumerate}
\item With the Dembo model of binding/unbinding, 
  \begin{align*}
    k_f = k_f^0 \exp\left[\frac{-\sigma_\text{ts} \left(L_\text{sep} -
    \lambda\right)^2}{2\kappa_B T}\right], \text{ and } \\
    k_d = k_d^0\exp\left[\frac{(\sigma -
    \sigma_\text{ts})\left(L_\text{sep} - \lambda\right)^2}{2\kappa_B
    T}\right]
  \end{align*}
  where $\sigma_\text{ts}$, $\sigma$, $\lambda$, and $L_\text{sep}$
  are the transition-state spring constant, the bound-state spring
  constant, the unstressed bond length, and the separation distance.
\item With the Bell model, 
  \begin{align*}
    k_f = k_f^0 \exp\left[\frac{\sigma \left| L_\text{sep} - \lambda
    \right|(\delta - 0.5\left|L_\text{sep} - \lambda\right|)}{\kappa_B
    T}\right], \text{ and } \\ 
    k_d = k_d^0\exp\left[\frac{\sigma\left|L_\text{sep} -
    \lambda\right|}{\kappa_B T}\right]
  \end{align*}
  where $\delta$ is the reactive compliance.
\item Then these reaction rates can be used to run Monte Carlo
  simulations of bond formation and breaking.
\item 
\end{itemize}

%------------------------------------------------

\section{Published Modeling Approaches} 

The authors discuss 6 models:
\begin{enumerate}
\item Adhesion dynamics model of Hammer and Apte (1992)
\item Event-tracking model of adhesion (ETMA) of Pospieszalska et
  al. (2009)
\item A semi-analytic model of Tozeren and Ley (1992)
\item A model of cell deformation in rolling by Khismatullin and
  Truskey (2004)
\item An analytic model by Zhao et al. (1995)
\item An agent-based model of rolling by Tang et al. (2007)
\end{enumerate}

\subsection{Adhesion Dynamics Model by Hammer and Apte (1992)}
\label{sec:adhes-dynam-model}

\begin{itemize}
\item Simulates the rolling of a rigid sphere with rigid microvilli
  protruding normally from the surface
\item Have a single class of receptors, randomly distributed over the
  sphere surface
\item Receptor-ligand bonds are modeled with the Dembo model, and
  bonds form perpendicularly to the substrate and are modeled as
  Hookean springs.
\item They use a fixed time step, and in each step determine whether
  each bond or potential bond breaks or forms, and update the velocity
  on the sphere
\item This model has been updated and modified several times,
  substituting the Bell model for the Dembo model, adding integrin
  receptors, adding deformable microvilli, etc.
\item One extension of the model by Krasik et al. (2006) adds
  activation of integrins
\item Another extension by Caputo et al. (2007) introduces an overall
  ligand on rate given by
  \begin{equation}
    \label{eq:ligand_onrate}
    \left(k_f^0\right)_L = k_0 P n_R
  \end{equation}
  where $k_0$ is the encounter rate, $P$ is the probability of bond
  formation, and $n_R$ is the density of substrate receptors. With
  increasing shear rate, $k_0$ increases and $P$ decreases. 
\end{itemize}

\subsection{Event-Tracking Model of Adhesion by Pospieszalska et
  al. (2009)}
\label{sec:event-tracking-model}

\begin{itemize}
\item The cell is modeled as a sphere with flexible microvilli.
\item ETMA tracks individual receptors, and so bonds can form at
  angles other than $90^\circ$, and they track individual microvilli
\item Use a variable time step for MC simulation, and pick the first
  reaction, and ignore the others.
\item They ignore translation along the $y$-axis and rotations about
  the $x$- and $z$-axes
\item Track a single type of receptor.
\item They found that $30\%$ of bonds do not last long enough to
  become load-bearing ($\gamma = 50 s^{-1}$).
\item In a later iteration, they allow microvilli to stretch as well
  as bend
\end{itemize}

\subsection{Model by Tozeren and Ley (1992)}
\label{sec:model-tozeren-ley}

\begin{itemize}
\item The first semi-analytic model of rolling
\item They model the leukocyte as a ``bumpy'' sphere. They assume
  there is a fluid layer separating the cell and the substrate with a
  thickness of the length of typical cell protrusions. 
\item The cells have a uniform rolling velocity $V_x$ and slip
  velocity $V_s$, and are separated from the surface with a constant
  distance $h_c$.
\item All bonds formed at the same time are lumped together into a
  single bond state
\item The forces on a bond can be found from the bond length, and the
  bond density as a function of $x$ in the contact zone is calculated
  according to equation (29) (it looks like the steady state of an
  advection-reaction equation).
\item The model describes steady-state rolling for a given shear rate
\end{itemize}

\subsection{Model by Khismatullin and Truskey (2004)}
\label{sec:model-khism-trusk}

\begin{itemize}
\item Model whole-cell deformation in rolling. The leukocyte is
  modeled as a viscoelastic drop composed of a viscoelastic nucleus
  surrounded by a viscoelastic cytoplasm. The membrane has a cortical
  tension, which pulls it into a sphere shape.
\item Microvilli are modeled as springs
\item The fluid is modeled with the Navier-Stokes equations, as
  opposed to the Stokes equations used in previous models
\end{itemize}

\subsection{Model by Zhao et al. (1995)}
\label{sec:model-zhao}

\begin{itemize}
\item This is an analytical model of rolling which tracks the
  distribution of velocities for rolling leukocytes
\item Assumption: leukocyte displacement is composed of random
  step-like jumps at random times
\item Assume cells are observed at intervals of $\Delta t$, then the
  observed velocity of a cell at a time $t$ and position $x$ is given
  by
  \begin{equation}
    \label{eq:random_velocity}
    V(x, t) = \frac{x(t + 0.5 \Delta t) - x(t - 0.5 \Delta t)}{\Delta
      t}.
  \end{equation}
\item Then $p(V, t)$ satisfies $\frac{\partial p}{\partial t} =
  -\frac{\partial J}{\partial V}$ where $J$ is a probability flux
  given by
  \begin{equation}
    \label{eq:prob_flux}
    J(V, t) = A(V) p(V, t) - \frac{1}{2}\frac{\partial (B(V) p(V,
      t))}{\partial V}.
  \end{equation}
  where $A(V) = \frac{l_\text{mean}/t_\text{mean} - V}{\Delta t}$ and
  $B(V) = 2V l_\text{mean}\left[\frac{1 +
      (\sigma_1/l_\text{mean})^2}{(\Delta t)^2}\right].$
\item This describes the transient evolution of leukocyte rolling. The
  steady state distribution is a $\gamma$ distribution, that depends
  on the frame rate of the observations $\Delta t$.
\end{itemize}

\subsection{In Silico White Blood Cell Model by Tang et al. (2007)}
\label{sec:silico-white-blood}

\begin{itemize}
\item This is an agent-based model
\item The cell is modeled as a rectangle with rounded edges, and the
  surface of the cell is divided into 600 subunits. The substrate
  surface is also divided into subunits of the same size.
\item The leukocyte subunits contain 3 agents representing PSGL-1,
  VLA-4, and CXCR2. The substrate subunits contain 3 agents
  representing P-selectin, VCAM-1, and GRO-$\alpha$.
\item I don't understand the geometry of this model. Something like
  rolling is simulated by removing elements from the trailing end of
  the cell and adding them to the leading end of the cell, and then
  elements on the cell and substrate which are adjacent to each other
  can form bonds.
\end{itemize}

% %------------------------------------------------

% \section{Results Overview}



% %------------------------------------------------

% \section{Discussion/Conclusions Overview}



% %------------------------------------------------

% \section*{Article Evaluation}



%----------------------------------------------------------------------------------------
%	BIBLIOGRAPHY
%----------------------------------------------------------------------------------------

\renewcommand{\refname}{Reference} % Change the default bibliography title

\bibliography{/Users/andrewwork/thesis/library} % Input your bibliography file

%----------------------------------------------------------------------------------------

\end{document}
%  LocalWords:  twocolumn onecolumn myheadings
