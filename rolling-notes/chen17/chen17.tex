%%%%%%%%%%%%%%%%%%%%%%%%%%%%%%%%%%%%%%%%%
% Article Notes
% LaTeX Template
% Version 1.0 (1/10/15)
%
% This template has been downloaded from:
% http://www.LaTeXTemplates.com
%
% Authors:
% Vel (vel@latextemplates.com)
% Christopher Eliot (christopher.eliot@hofstra.edu)
% Anthony Dardis (anthony.dardis@hofstra.edu)
%
% License:
% CC BY-NC-SA 3.0 (http://creativecommons.org/licenses/by-nc-sa/3.0/)
%
%%%%%%%%%%%%%%%%%%%%%%%%%%%%%%%%%%%%%%%%%

%----------------------------------------------------------------------------------------
%	PACKAGES AND OTHER DOCUMENT CONFIGURATIONS
%----------------------------------------------------------------------------------------

\documentclass[
10pt, % Default font size is 10pt, can alternatively be 11pt or 12pt
letterpaper, % Alternatively letterpaper for US letter
twocolumn, % Alternatively onecolumn
landscape % Alternatively portrait
]{article}

\usepackage{amsmath, textgreek}
\newcommand{\inta}[1]{\textalpha\textsubscript{#1}}
\newcommand{\intb}[1]{\textbeta\textsubscript{#1}}

\input{structure.tex} % Input the file specifying the document layout and structure

%----------------------------------------------------------------------------------------
%	ARTICLE INFORMATION
%----------------------------------------------------------------------------------------

\articletitle{Receptor-mediated cell mechanosensing} % The title of
                                % the article
\articlecitation{\cite{Chen2017}} % The BibTeX citation key from your
                                % bibliography 

\datenotesstarted{April 6, 2018} % The date when these notes were
                                % first made
\docdate{\datenotesstarted; rev. \today} % The date when the notes
                                % were lasted updated (automatically
                                % the current date)

\docauthor{Andrew Watson} % Your name

%----------------------------------------------------------------------------------------

\begin{document}

\pagestyle{myheadings} % Use custom headers
\markright{\doctitle} % Place the article information into the header

%----------------------------------------------------------------------------------------
%	PRINT ARTICLE INFORMATION
%----------------------------------------------------------------------------------------

\thispagestyle{plain} % Plain formatting on the first page

\printtitle % Print the title

%----------------------------------------------------------------------------------------
%	ARTICLE NOTES
%----------------------------------------------------------------------------------------

\section{Introduction}
\label{sec:introduction}

\begin{itemize}
\item It is important for cells to be able to sense and respond to
  mechanical signals from their surroundings. 
\item Mechanosensitive structures exist across multiple scales in
  organisms. This review is concerned with cellular and molecular
  scales, and specifically receptor-mediated mechanosensing.
\item Receptor-mediated mechanosensing is important in many cellular
  processes, including activation, differentiation, proliferation,
  apoptosis, cell spreading, migration, and others.
\item Receptors and intracellular molecules involved in mechanosensing
  can be found by altering their concentration or activity somehow,
  which affects how the cell responds to a mechanical stimulus.
\item Intracellular forces can also be transferred to an extracellular
  substrate through receptors, affecting the cell's response as well.
\item In reality, mechanosensing is a complex process that involves
  many receptors as well as crosstalk between the cell and the
  substrate. However in this review, they simplify the problem and
  only look at one direction of the signaling through a single
  receptor.
\end{itemize}

%------------------------------------------------

\section{Four-step model for mechanosensing}

\begin{itemize}
\item When a receptor binds to an immobilized ligand and force is
  exerted on the receptor-ligand bond, mechanosensing can be
  initiated.
\item There are 4 steps in transferring this mechanical signal to the
  inside of the cell:
  \begin{enumerate}
  \item Mechanopresentation---mechanical cues are presented to the
    receptor. The ligand must be bound to a surface and able to support
    a force. Soluble ligands cannot transduce force. The
    mechanopresenter is a ligand anchored to a surface.
  \item Mechanoreception---the ligand transmits force to the binding site
    of a cell surface receptor. This force may alter the bond between
    the ligand and receptor. The mechanoreceptor binds to the
    mechanopresenter.
  \item Mechanotransmission---the force is communicated from the binding
    site towards the cell interior. The mechanotransmitter is
    responsible for propagating the force signal. It may transmit
    mechanical force across the membrane, or it may propagate
    information in another way (e.g. force-induced conformational
    change). 
  \item Mechanotransduction---the mechanical cue is translated into a
    biochemical signal. Often either the receptor or a linked subunit
    undergoes a conformational change in this step. The
    mechanotransducer is the structure(s) undergoing a mechanical
    change to initiate a chemical signal. 
  \end{enumerate}
\item An important role of mechanotransduction is distinguishing
  between different types of forces. The mechanotransducer may respond
  to one type of force and not others (digital mechanism) or its
  response may scale with the strength of the mechanical force (analog
  mechanism). 
\item The difference between the mechanotransmitter and
  mechanotransducer is that the mechanotransducer translates the
  mechanical signal into a chemical one, whereas the
  mechanotransmitter does not.
\end{itemize}

%------------------------------------------------

\section{Kinetic and Mechanical Aspects}
\label{sec:kinet-mech-aspects}

\subsection{Kinetic constraints}
\label{sec:kinetic-constraints}

\begin{itemize}
\item Bond kinetics are affected by mechanical forces, and so play a
  role in mechanopresentation and mechanoreception. In particular,
  kinetics modulate the magnitude, duration, and frequency of force,
  and place a constraint on presentation and reception.
\item The off rate of a single bond can increase with applied force
  (slip bond), decrease with applied force (catch bond), or be
  independent of force (ideal bond). The type of receptor-ligand bond
  that forms influences the magnitude and lifetime of mechanical
  signals communicated to the mechanoreceptor. For example, the amount
  of Ca$^{++}$ signaling induced by TCR and GP1b depends on the amount
  of force applied to these receptors from their ligands.
\end{itemize}

\subsection{Mechanical changes}
\label{sec:mechanical-changes}

\begin{itemize}
\item Mechanically-triggered conformational changes can play an
  important role in a mechanosensor, and these changes fall into 6
  different types:
  \begin{enumerate}
  \item Deformation---e.g. membrane ion channels which open in
    response to an increase in membrane tension.
  \item Relative displacement---force can induce displacement of two
    subunits of transmembrane channels.
  \item Hinge movement---for proteins consisting of two globular
    domains connected by a hinge region (like integrins and
    selectins), pulling force can promote hinge opening.
  \item Unfolding/unmasking---force can cause proteins to unfold,
    revealing previously hidden active sites. Examples include GP1b,
    talin, vinculin, fibronectin, and vWF.
  \item Translocation/rotation---force applied to proteins
    noncovalently complexed with other proteins can lead to relative
    movement and breaking bonds on the trailing edge and forming bonds
    on the leading edge. Examples of translocation are linear
    molecular motors like myosin, kinesin, and dyneine. Examples of
    rotation are ATP synthases.
  \item Cluster rearrangement---force acting on proteins in a cluster
    trigger the proteins to change their arrangement through their
    mutual interactions. An example is integrins in a focal adhesion,
    which alter the cytoskeleton through interactions between integrin
    tails and actin.
  \end{enumerate}
\item The choice of conformational change to employ depends on the
  physiological function of the mechanotransducer. For example,
  cluster rearrangement in integrins can induce changes in the
  cytoskeleton. However, a single GP1b cannot trigger actin
  rearrangement.
\item Mechanical signals can be important in other steps of
  mechanosensing as well. For example, vWF only exposes its A1 binding
  sites when exposed to a high enough shear. This modulates the
  mechanopresentation of A1 to GP1b molecules.
\item Kinetics and mechanics can be coupled at the level of a single
  receptor. For example, force applied to the leucine-rich repeat
  domain (LRRD) in GP1b can strengthen the GP1b-A1 bond, increasing
  its lifetime. Also, only abruptly increasing force can cause this
  strengthening; constantly applied force cannot.
\end{itemize}

%------------------------------------------------

\section{Nanotools for studying mechanosensing}
\label{sec:nanot-study-mech}

\subsection{Dynamic force spectroscopy}
\label{sec:dynam-force-spectr}

\begin{itemize}
\item Dynamic force spectroscopy (DFS) uses ultrasensitive force
  probes like atomic force microscopy (AFM), optical tweezers,
  magnetic tweezers, and biomembrane force probes (BFPs). These
  instruments have very fine resolution in space, time, and force
  scales. 
\item In DFS, ligands immobilized to a substrate are brought into
  contact with a mechanoreceptor to facilitate binding. Then a
  piconewton-level pulling force is applied to the bond through one of
  the above mechanisms. Force-ramp experiments involve increasing the
  force until the bond breaks, while force-clamp experiments ramp the
  force to a pre-defined amount, and then the force is held there
  until the bond breaks. These experiments can be used to determine
  the force dependence of receptor-ligand off rates, or protein
  conformational changes or unfolding.
\item DFS experiments combined with real-time imaging of intracellular
  signaling events allows experimentalists to directly observe the
  intracellular response of mechanotransducers to force. For example,
  prolonged forces applied to the GP1b-A1 bond have been observed to
  initiate Ca$^{++}$ in platelets.
\end{itemize}

\subsection{Magnetic twisting cytometry}
\label{sec:magn-twist-cytom}

\begin{itemize}
\item This technique involves a magnetic bead coated with ligand and
  bound to multiple receptors on a cell surface. This is a
  high-throughput method, using many beads bound to the surfaces of
  many cells.
\item This allows experimentalists to measure cell stiffness. These
  kinds of experiments provided evidence of mechanosensing by
  integrins, as the cell stiffened in response to an applied torque.
\end{itemize}

\subsection{Microscopic probes for cell traction and internal force}
\label{sec:micr-prob-cell}

\begin{itemize}
\item Two aspects of mechanosensing: environment exerting force on a
  cell, and a cell producing force to sense mechanical properties of
  its surroundings.  % I read the first page of this, come back to
                     % finish taking notes (reading about traction
                     % force microscopy)
\item The force probes described in previous sections provide no
  information on if the cell is actively exerting force. 
\item Traction force microscopy measures bulk traction forces
  generated by adherent cells. Cells are seeded onto a ligand-coated
  substrate which contains markers that can be tracked
  microscopically. By observing the displacement of these markers, the
  force exerted by the cell can be calculated.
\item Traction force microscopy has been an important tool in learning
  about focal adhesions.
\item When combined with live-cell imaging, this can be used to
  visualize the dynamics of traction forces and molecular signals.
\item A second class of force probes involves inserting a polymer
  between the cell and substrate to either limit the tension
  experienced by the cell, or report the tension generated by the
  cell.
\item Tension gauge tethers (TGT) are designed to rupture above a
  certain force, and thereby limiting the amount of tension the cell
  can generate. This provides information on how much tension is
  required to support certain force-dependent functions.
\item Molecular tension-based fluorescence microscopy probes unfold
  and fluoresce when tension is applied above a certain
  threshold. This unfolding is reversible, and so is doesn't affect
  cell functions.
\item A third class of probes is similar to the DNA-based probes, but
  using flexible peptides which fluoresce under applied tension. These
  can be used to moniter both extracellular and intracellular forces,
  and have been inserted along the cytoplasmic tails of \inta{L} and
  \intb{2} to show that force is transduced through the \intb{}
  subunit in migrating cells.
\end{itemize}

%------------------------------------------------

\section{Platelet mechanosensing via single GP1b}

\begin{itemize}
\item vWF is mechanosensitive, it requires a pulling force to unfurl
  the molecule and expose the GP1b binding sites. In the vWF-GP1b
  mechanosensing system, vWF is the mechanopresenter.
\item Once vWF is immobilized on vessel walls, it is subjected to
  higher shear rates near the vessel wall. Once vWF is bound to GP1b,
  intracellular Ca$^{++}$ release is triggered, which signals integrin
  \inta{IIb}\intb{3} to activate.
\item Upon vascular injury, vWF adsorbs to the subendothelium, and
  fluid forces extend the vWF multimer, exposing the A1 binding site
  for GP1b. The GP1B-A1 bond has fast association and dissociation
  constants. Fast association allows for the capture of fast-flowing
  platelets, and fast dissociation allows for rolling along the
  surface. 
\item Physical transport in the bulk drives platelets to collide with
  the vessel wall, bringing GP1b and vWF close enough to form
  bonds. Three steps of the transport regulate GP1b-vWF association:
  \begin{enumerate}
  \item tethering of the platelet to the vascular surface,
  \item Brownian motion of the platelet, and
  \item rotational diffusion of the interacting molecules.
  \end{enumerate}
\item The GP1b-vWF bond acts as a catch bond below 22 pN, and as a
  slip bond above that. 2B mutant vWF forms slip bonds only with
  GP1b. 
\item Collagen-vWF interaction also activates vWF, in addition to
  providing an anchor point for vWF. vWF is initially captured through
  A3 binding with collagen, but further collagen binding with A1 can
  increase its affinity for GP1b binding.
\item Increased flow enhances platelet capture on vWF through three
  mechanisms:
  \begin{enumerate}
  \item Exposure of vWF-A1 binding sites on immobilized vWF
  \item Increase in the number of platelets that collide with the wall
  \item Longer association times of GP1b and vWF-A1 through catch bond
    behavior
  \end{enumerate}
\item There are two domains in GP1b that extend when the molecule is
  pulled on: the LRRD (leucine-rich repeat domain) and the MSD
  (mechanosensitive domain).
\item These two domains have different unfolding lengths, and respond
  differently to different force waveforms. Ramped force (a quick
  increase in force application) unfolds both LRRD and MSD, whereas
  clamped force (a constant force applied for a long time) only
  unfolds MSD.
\item Unfolding of the MSD allows for transmission of the mechanical
  signal along the receptor. 
\item The unfolding frequency of MSD depends on the level of applied
  force in the same way as GP1b-vWF binding kinetics. This is true in
  both normal vWF and 2B vWD mutants. This suggests a coupling between
  unbinding kinetics and unfolding kinetics.
\item Experiments show that LRRD unfolding prolongs the A1-GP1b bond
  lifetime, perhaps by exposing additional binding sites within the
  LRRD region. 
\item Unfolding of MSD requires sustained force, and it cannot occur
  after vWF unbinds and force is no longer being applied along the
  molecule. There is cooperativity between LRRD and MSD
  unfolding. LRRD unfolding lengthens the lifetime of the GP1b-vWF
  bond, which increases the probability of MSD unfolding.
\item Unfolding of MSD results in intracellular signaling. In mutation
  experiments, mutations that unfolded the MSD triggered intracellular
  signaling even in the absence of ligand binding.
\item Two types of intracellular Ca$^{++}$ signaling have been
  observed in platelets with a single GP1b-vWF bond:
  \begin{enumerate}
  \item \inta{} type, which features an initial latent phase followed
    by a high spike with a quick decay, and 
  \item \intb{} type, which features fluctuating sinals around the
    baseline or gradually increasing signals to an intermediate level
    followed by a gradual decay to baseline. 
  \end{enumerate}
\item MSD unfolding is required to trigger \inta{}-type calcium
  signaling. It has also been found that constant force at an optimum
  level triggers maximum Ca$^{++}$ release, no transient force.
\item Patients with vWD have increased bleeding relative to
  controls. This provides a possible explanation for that phenotype.
  In vWD, the GP1b-A1 bond is converted to a slip bond, therefore the
  bond lifetime is relatively shorter under force, and maximum
  Ca$^{++}$ release cannot be triggered.
\item In summary, the 4 steps of mechanosensing in the vWF-GP1b system
  are as follows:
  \begin{enumerate}
  \item Mechanopresentation: vWF undergoes local and global
    conformational changes in response to force resulting in increased
    binding affinity to GP1b.
  \item Mechanoreception: LRRD binds to the vWF-A1 region. This domain
    can then unfold under force, stabilizing the bond and increasing
    its lifetime.
  \item Mechanotransmission: Force propagates from the LRRD along the
    MP stalk, inducing unfolding of MSD.
  \item Mechanotransduction: Exposure of the Trigger sequence within
    the MSD and association of 14-3-3{\textzeta} to GP1b allows the
    mechanical signal to be converted into a chemical one.
  \item GP1b may represent a more general class of
    mechanoreceptors. For example, Notch receptors (involved in
    cell-cell communication) form catch-slip bonds with their ligands,
    they transmit mechanical signals along a polypeptide sequence, and
    force induces unfolding of a juxtamembrane MSD, similar to GP1b.
  \end{enumerate}
\end{itemize}

%------------------------------------------------

\section{Integrin-Mediated Cell Adhesion and Mechanosensing} 

\begin{itemize}
\item Integrins are involved in cellular processes involving adhesion,
  spreading migration, proliferation, and differentiation. They have a
  long-known role in force transmission from the ECM across the cell
  membrane. 
\item Integrins are heterodimers comprised an \inta{} subunit and an
  \intb{} subunit. Each of these subunits has a large head, a long
  leg, a single pass TMD (transmembrane domain), and a short CT
  (cytoplasmic tail). The heads and upper legs of the two subunits
  connect to form the headpiece.
\item Resting integrins have a bent conformation with a closed
  headpiece. When activated, a series of conformational changes happen
  resulting in extension of the headpiece.
\item In inside-out signaling, intracellular activation signals can
  trigger the integrin to adopt an open conformation. Specifically,
  talin, kindlin, and other adaptors bind to the \intb{}CT to induce a
  conformational change. 
\item These steps can also run in reverse in outside-in
  signaling. They analyze outside-in signaling in their 4 step
  framework. 
\item The ECM is a major mechanical environment that is sensed by many
  integrins. Stiffness of the ECM can affect cell functions.
\item Cells can also modify the ECM as a response of
  mechanosensing. For example, fibrin networks stiffen under
  deformation. 
\item Several integrins can form catch-slip bonds (though not
  \inta{IIb}\intb{3}) 
\item Integrins can exist in 3 different global conformations:
  \begin{enumerate}
  \item bent with a closed headpiece, 
  \item extended with a closed headpiece, and 
  \item extended with an open headpiece.
  \end{enumerate}
\item Integrin conformation can regulate force transmission across the
  cell membrane, and force also affects integrin conformation. Tension
  increases the frequency of unbending, and decreases the frequency of
  bending. 
\item Structural analysis and MD simulations suggest a ``cytoskeletal
  force model,'' where lateral pulling due to retrograde actin flow
  could pull apart the \inta{} and \intb{} tails to stabilize the
  extended-open conformation.
\item Integrin binding affinity and conformation are closely
  related. The bent conformation has a low affinity for the integrin's
  ligand, and upon activation the hybrid domain swings out, opening
  the headpiece and allowing for high-affinity binding. 
\item Opening of the integrin headpiece is allosterically associated
  with ectodomain extension. However, locking the integrin headpiece
  in a closed position suppresses high-affinity binding, independent
  of whether the ectodomain is bent or extended.
\item Many studies support this model, however some evidence
  contradicts it. For example, a high-affinity bent conformation in
  \intb{2} integrins has been found, implying that ectodomain
  extension is not necessarily required for integrin activation.
\item Given the relationship between integrin conformation, binding
  affinity, and force, there may be a regulatory feedback loop in the
  mechanoreception and mechanotransmission steps of integrin
  signaling.
\item Mechanotransduction involves the recruitment and phosphorylation
  of kinases. Two examples:
  \begin{enumerate}
  \item FAK (focal adhesion kinase) is phosphorylated by integrin
    signaling and regulates Rho-family GTPase activation, which
    mediates cell migration.
  \item SFKs (Src family of tyrosine kinases) are phosphorylated
    downstream of integrin activation, and contribute to the formation
    of integrin-mediated firm adhesion.
  \end{enumerate}
\item It was previously thought that clustering of integrins was
  required for induction of these signals, but more recent studies
  have found that single integrins can produce these signals as well.
\item They provide three models of single-integrin
  mechanotransduction:
  \begin{enumerate}
  \item \cite{Zhu2008b} found that a combination of lateral and
    tensile forces on the \intb{} leg of an extended integrin causes
    headpiece opening. Therefore it is possible that a pulling force
    on an integrin which is out of line with its cytoplasmic anchor
    could cause \inta{}\intb{} separation, which may then open up
    binding sites on the \intb{} CT.
  \item Talin binding to the integrin \intb{} CT is the final step of
    integrin activation. The talin head associates with the CT, and
    its tail associates with actin, and the sequences in between are
    susceptible to conformational change under pulling force. In
    outside-in signaling, force that is transmitted through the
    \intb{} CT can unfold the attached talin, exposing binding sites
    for vinculin. It has been found that integrin-ligand engagement on
    a soft substrate fails to extend talin, but on stiff substrates
    talin is able to unfold before ligand unbinding.
  \item Other intracellular molecules may change conformation or
    function when force is propagated to the cell interior. For
    example, vinculin unfolds to expose hidden motifs, Src is
    activated, myosin translocates on actin, as well as other
    responses. Vinculin and Src are important in focal adhesion
    maturation, and myosin translocation is important for cell
    deformation. 
  \end{enumerate}
\item Organized into the 4-step model, integrin mechanosensing looks
  like this:
  \begin{enumerate}
  \item Mechanopresentation: the integrin headpiece binds to its
    ligand, and force is applied to the bond.
  \item Mechanoreception: the ligand binding site recieves the ligand
    together with force, and binding kinetics respond accordingly. 
  \item Mechanotransmission: force propagates along the integrin, and
    can unbend the hinge region, and cause hybrid domain swing-out to
    open the headpiece. Force may propagate onto talin or other
    cytoplasmic molecules without these integrin conformational
    changes. 
  \item Mechanotransduction: head opening results in \inta{}\intb{} CT
    separation, exposing binding sites for signaling molecules. 
  \end{enumerate}
\end{itemize}

%------------------------------------------------

\section{Other Cell Mechanosensing Models}

\begin{itemize}
\item The mechanotransduction step in any model is important, because
  it converts mechanical signals into biochemical signals.
\item This review only addresses monovalent molecular interactions,
  but multivalent interactions are important as well. In particular,
  fibrinogen and vWF molecules contain multiple receptor binding
  sites, and therefore can interact with several receptors at once.
\item This complicates things. Association and dissociation of
  complexed bonds follow multistep kinetic mechanisms involving
  intermediate steps. The force may be distributed unevenly across
  different receptors. Additionally, overall bond lifetime is probably
  prolonged, because once a receptor and ligand unbind, they remain in
  close proximity to each other.
\end{itemize}

%------------------------------------------------

\section*{Summary}

This paper presents a general 4-step model of mechanosensing, and then
interprets 3 different mechanosensing systems within this model
framework. It also summarizes available biophysical probes that allow
experimentalists to test different aspects of these mechanosensing
systems. 

%-------------------------------------------------------------------------
%	BIBLIOGRAPHY
%-------------------------------------------------------------------------

\renewcommand{\refname}{Reference} % Change the default bibliography
                                % title

\bibliography{/Users/andrewwork/thesis/library.bib} % Input your
                                % bibliography file 

%-------------------------------------------------------------------------

\end{document}
