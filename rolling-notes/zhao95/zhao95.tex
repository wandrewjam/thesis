%%%%%%%%%%%%%%%%%%%%%%%%%%%%%%%%%%%%%%%%%
% Article Notes
% LaTeX Template
% Version 1.0 (1/10/15)
%
% This template has been downloaded from:
% http://www.LaTeXTemplates.com
%
% Authors:
% Vel (vel@latextemplates.com)
% Christopher Eliot (christopher.eliot@hofstra.edu)
% Anthony Dardis (anthony.dardis@hofstra.edu)
%
% License:
% CC BY-NC-SA 3.0 (http://creativecommons.org/licenses/by-nc-sa/3.0/)
%
%%%%%%%%%%%%%%%%%%%%%%%%%%%%%%%%%%%%%%%%%

%----------------------------------------------------------------------------------------
%	PACKAGES AND OTHER DOCUMENT CONFIGURATIONS
%----------------------------------------------------------------------------------------

\documentclass[
10pt, % Default font size is 10pt, can alternatively be 11pt or 12pt
letterpaper, % Alternatively letterpaper for US letter
twocolumn, % Alternatively onecolumn
landscape % Alternatively portrait
]{article}

\usepackage{amsmath, amsfonts}
%%%%%%%%%%%%%%%%%%%%%%%%%%%%%%%%%%%%%%%%%
% Article Notes
% Structure Specification File
% Version 1.0 (1/10/15)
%
% This file has been downloaded from:
% http://www.LaTeXTemplates.com
%
% Authors:
% Vel (vel@latextemplates.com)
% Christopher Eliot (christopher.eliot@hofstra.edu)
% Anthony Dardis (anthony.dardis@hofstra.edu)
%
% License:
% CC BY-NC-SA 3.0 (http://creativecommons.org/licenses/by-nc-sa/3.0/)
%
%%%%%%%%%%%%%%%%%%%%%%%%%%%%%%%%%%%%%%%%%

%----------------------------------------------------------------------------------------
%	REQUIRED PACKAGES
%----------------------------------------------------------------------------------------

\usepackage[includeheadfoot,columnsep=2cm, left=1in, right=1in, top=.5in, bottom=.5in]{geometry} % Margins

\usepackage[T1]{fontenc} % For international characters
\usepackage{XCharter} % XCharter as the main font

\usepackage{natbib} % Use natbib to manage the reference
\bibliographystyle{apalike} % Citation style

\usepackage[english]{babel} % Use english by default

%----------------------------------------------------------------------------------------
%	CUSTOM COMMANDS
%----------------------------------------------------------------------------------------

\newcommand{\articletitle}[1]{\renewcommand{\articletitle}{#1}} % Define a command for storing the article title
\newcommand{\articlecitation}[1]{\renewcommand{\articlecitation}{#1}} % Define a command for storing the article citation
\newcommand{\doctitle}{\articlecitation\ --- ``\articletitle''} % Define a command to store the article information as it will appear in the title and header

\newcommand{\datenotesstarted}[1]{\renewcommand{\datenotesstarted}{#1}} % Define a command to store the date when notes were first made
\newcommand{\docdate}[1]{\renewcommand{\docdate}{#1}} % Define a command to store the date line in the title

\newcommand{\docauthor}[1]{\renewcommand{\docauthor}{#1}} % Define a command for storing the article notes author

% Define a command for the structure of the document title
\newcommand{\printtitle}{
\begin{center}
\textbf{\Large{\doctitle}}

\docdate

\docauthor
\end{center}
}

%----------------------------------------------------------------------------------------
%	STRUCTURE MODIFICATIONS
%----------------------------------------------------------------------------------------

\setlength{\parskip}{3pt} % Slightly increase spacing between paragraphs

% Uncomment to center section titles
%\usepackage{sectsty}
%\sectionfont{\centering}

% Uncomment for Roman numerals for section numbers
%\renewcommand\thesection{\Roman{section}}
 % Input the file specifying the document layout
                      % and structure

%----------------------------------------------------------------------------------------
%	ARTICLE INFORMATION
%----------------------------------------------------------------------------------------

\articletitle{A stochastic model of leukocyte rolling} % The title of
                                % the article
\articlecitation{\cite{Zhao1995}} % The BibTeX citation key from your
                                % bibliography

\datenotesstarted{April 8, 2018} % The date when these notes were
                                % first made
\docdate{\datenotesstarted; rev. \today} % The date when the notes
                                % were lasted updated (automatically
                                % the current date)

\docauthor{Andrew Watson} % Your name

%----------------------------------------------------------------------------------------

\begin{document}

\pagestyle{myheadings} % Use custom headers
\markright{\doctitle} % Place the article information into the header

%----------------------------------------------------------------------------------------
%	PRINT ARTICLE INFORMATION
%----------------------------------------------------------------------------------------

\thispagestyle{plain} % Plain formatting on the first page

\printtitle % Print the title

%----------------------------------------------------------------------------------------
%	ARTICLE NOTES
%----------------------------------------------------------------------------------------

\section{Introduction}



%------------------------------------------------

\section{Model Overview}

They approximate the displacement of a rolling cell by a step-like
function in time, where there are rapid jumps in the cells position,
followed by a relatively long period where the cell is bound to the
substrate. Motivation: leukocytes bind to the substrate through
receptors located in the microvilli. When all the bonds on the
trailing microvillus are broken, the cell ``jumps'' forward. In
between steps, the cell is stationary, and so the bonds between
cell-surface receptors and the substrate must balance the forces
exerted on the cell by the fluid.

The displacement $h$ in a single step ``is a random variable
determined mainly by the spacing of microvilli and by the
distributions of adhesion molecules on the individual microvilli and
the ligands on the substraturm surface.'' The time of this
displacement is just $t_1 = \frac{h}{v_\text{free}}$ (they write $t_1
\sim \frac{h}{v_\text{free}}$, is this the same?) They define $t_2$ as
the time between successive jumps (i.e. $t_1$ plus the resting time
following a jump), and then claim $t_2 \sim
\frac{h}{v_\text{rolling}}$. Here $v_\text{free}$ is the velocity of a
free-flowing cell, and $v_\text{rolling}$ is the average velocity of a
rolling cell. (Note: though this isn't explicitly stated, I think that
$x \sim y$ means the distribution of $x$ is $y$. 

Define $\tau_1 \equiv \langle t_1 \rangle$ and $\tau_2 = \langle t_2
\rangle$, then 
\begin{equation*}
  \frac{\tau_1}{\tau_2} \sim \frac{v_\text{rolling}}{v_\text{free}}.
\end{equation*}
(Are the $\tau$s themselves random variables? I'm not sure if the
above equation can be interpreted as a regular equals.) Experimentally
this ratio is much less than 1. This is their justification for
idealizing cell motion as a random jumping process. 

The waiting time $t_2$ is a function of the dissociation rate of the
bonds in the rearmost microvillus (because these are the bonds
preventing the forward progress of the cell). If we assume that there
is only a single bond in this rearmost cluster, then $\tau_2 =
1/k_\text{off}$. When adhesion is mediated by multiple bonds, the
relationship is more complicated. They suggest that $h$ is related to
the spatial distribution of receptors and ligands. If the the density
of ligands on the substrate is high, then $h$ is determined by the
spacing of adhesion receptors on the cell. They want to estimate $h$
and $\tau_2$ from the distribution of rolling velocities. 

%------------------------------------------------

\section{Stochastic model of cell rolling for a homogeneous
  population}

In this section they consider a homogeneous population of cells. They
model the rolling velocity using a Fokker-Planck equation with drift
and diffusion terms. If $\Delta t \ll \tau_2$, then instantaneous
velocity can't be determined, and instead 
\begin{equation}
  \label{eq:velocity}
  v(t) \equiv \frac{x(t + \Delta t/2) - x(t - \Delta t/2)}{\Delta t}
\end{equation}
defines a moving average of the velocity. When $\Delta t/\tau_2 \gg
1$, then $v(t)$ is approximately continuous. Therefore they model the
distribution of velocities as 
\begin{equation*}
  \frac{\partial p}{\partial t}(v, t) = -\frac{\partial}{\partial
    v}[A(v)p(v,t)] + \frac{1}{2}\frac{\partial^2}{\partial
    v^2}[B(v)p(v,t)].
\end{equation*}

The drift term $A(v)$ is defined as
\begin{equation}
  \label{eq:drift_defn}
  A(v) = \lim_{t' \rightarrow t}\frac{\langle v' - v\rangle}{t' - t}.
\end{equation}
(Remark: the definitions for $A(v)$ and $B(v)$ below come from the
derivation of the Fokker-Planck equation, which is an approximation of
the Chapman-Kolmogorov equation describing the distribution of the
random jump process). Here $v'$ is the realization of the random
process at time $t'$. By substituting equation (\ref{eq:velocity})
into equation (\ref{eq:drift_defn}), you find
\begin{equation}
  \label{eq:drift_2}
  A(v) = \lim_{t' \rightarrow t}\frac{\langle x(t + \Delta t/2) - x(t
    + \Delta t/2)\rangle - \langle x(t' - \Delta t/2) - x(t - \Delta
    t/2)\rangle}{(t' - t)\Delta t}.
\end{equation}

Consider the mean of the displacement of a rolling cell in the
interval $\delta t = t' - t$. Assuming the jumps are a Poisson process
with mean $1/\langle t_2 \rangle$, then the probability of a jump
occurring in the time interval $\delta t$ is approximately
$\frac{\delta t}{\langle t_2 \rangle}$. Then it seems reasonable that
the displacement in the interval $\delta t$ should be distributed like
$h \mathbb{P}[\text{jump}]$. 
\begin{equation}
  \label{eq:average_displacement}
  \langle x(t + \delta t) - x(t) \rangle = \left\langle h
    \left[\frac{\delta t}{\langle t_2 \rangle} + o\left(\frac{\delta
          t}{\langle t_2 \rangle}\right) \right] \right\rangle.
\end{equation}

Let's look at the two averages in (\ref{eq:drift_2}). The first term
is $\langle x(t' + \Delta t/2) - x(t + \Delta t/2)\rangle$. Applying
(\ref{eq:average_displacement}) we get $\langle x(t' + \Delta t/2) -
x(t + \Delta t/2) \rangle \approx \left\langle h \frac{\delta
    t}{\tau_2}\right\rangle$ for the first term and $\langle x(t' -
\Delta t/2) - x(t - \Delta t/2)\rangle \approx \left\langle h
  \frac{\delta t}{\langle t_2 \rangle} \right\rangle$ for the second
term. We have to be careful about the $\langle t_2 \rangle$ term. In
the second term, we have prior information about the distribution of
$t_2$ in the interval $[t - \Delta t/2, t + \Delta
t/2]$. Specifically, we know that the average velocity of the cell in
this interval is $v$. Therefore $\langle t_2 \rangle = \bar{h}/v$,
not $\tau_2$. 

The diffusion coefficient is given by 
\begin{equation}
  \label{eq:diff_defn}
  B(v) = \lim_{t' \rightarrow t}\frac{\langle (v' - v)^2 \rangle}{t' - t}.
\end{equation}
Again, by substituting (\ref{eq:velocity}) into (\ref{eq:diff_defn}),
we get 
\begin{equation*}
  B(v) = \lim_{t' \rightarrow t}\frac{\left\langle ([x(t' + \Delta t/2) -
    x(t + \Delta t/2)] - [x(t' - \Delta t/2) - x(t - \Delta
    t/2)])^2\right\rangle}{(t' - t)(\Delta t)^2}. 
\end{equation*}
When you expand the numerator, you get each of the square bracket
terms squared, plus the cross term. This cross term disappears by
assuming that the terms in the square brackets are essentially
independent. This is because we've assumed that $\delta t \ll t_2 <
\Delta t$.

Then, we assume that $\left\langle [x(t'-\Delta t/2) - x(t - \Delta
  t/2)]^2\right\rangle = \left\langle h^2\left(\frac{\delta t}{\langle
      t_2 \rangle} + o\left(\frac{\delta t}{\langle t_2
        \rangle}\right)\right) \right\rangle$ and similar for the
other term. Here $\langle t_2 \rangle = \bar{h}/v$ in both terms (I
don't understand their argument why, and so $B(v)$ reduces to 
\begin{equation*}
  B(v) = \frac{2 \langle h^2 \rangle}{\bar{h}(\Delta t)^2}v = \frac{2
    \bar{h} v}{(\Delta t^2)}\left(1 + \frac{\sigma_h^2}{\bar{h}^2}\right).
\end{equation*}

%------------------------------------------------

\section{Boundary conditions for the Fokker-Planck equation}

From the conservation of probability mass, we require that
$\frac{\partial}{\partial t}p(v, t) = -\frac{\partial}{\partial v} J(v,
t)$. Therefore the probability flux is given by
\begin{equation}
  \label{eq:prob_flux}
  J(v, t) = A(v)p + \frac{1}{2}\frac{\partial}{\partial v}[B(v)p].
\end{equation}
The lower limit on the velocity of a rolling cell is 0, by the
definition of the stochastic process (no backwards jumps are
allowed). To find the flux at this boundary, plug $v = 0$ into
(\ref{eq:prob_flux}), then $J(0, t) \ge 0$ if $\Delta t \ge
\tau_2(1 + \sigma^2_h/\bar{h}^2).$ There is also no source of
probability to the left of $v = 0$, and so we must have that $J(0, t)
= 0$. The maximum possible rolling velocity is $\infty$.

The stationary solution of $p(v, t)$ is a $\gamma$ distribution:
\begin{equation}
  \label{eq:stable_distribution}
  p_s(v) = \frac{1}{\beta^\alpha\Gamma(\alpha)} v^{\alpha-1}
  \exp\left(-\frac{v}{\beta}\right),
\end{equation}
where
\begin{equation*}
  \alpha = \frac{\Delta t}{\tau_2} \left(1 +
    \frac{\sigma_h^2}{\bar{h}^2}\right)^{-1}, \quad 
  \beta = \frac{\bar{h}}{\Delta t} \left(1 +
    \frac{\sigma^2_h}{\bar{h}^2}\right). 
\end{equation*}
The steady state distribution of rolling velocities spreads out as
$\Delta t$ increases, but the mean remains the same. 

%------------------------------------------------

%------------------------------------------------

\section{Discussion/Conclusions Overview}

Their method provides a way to estimate the step size $\bar{h}$ and
average pause time $\tau_2$ when $\Delta t \gg \tau_2$. They also do
an analysis for a heterogeneous population of cells, and propose that
by varying $\Delta t$, one can separate the contribution to the
variance of heterogeneity in the cell population from the contribution
by variations in the pause time. 

%------------------------------------------------

\section{Article Evaluation}

While it may be good to keep in mind the fact that experimental
results in estimating rolling velocity can depend on the time interval
of observation, I don't know if we can do anything to improve on this
analysis for platelets, but it may be of interest to Vlado's group to
interpret their experimental results. I think it makes more sense for
us to look at more physical models.

%----------------------------------------------------------------------------------------
%	BIBLIOGRAPHY
%----------------------------------------------------------------------------------------

\renewcommand{\refname}{Reference} % Change the default bibliography title

\bibliography{/Users/andrewwork/Documents/Grad_School/thesis/library}
% Input your bibliography file

%----------------------------------------------------------------------------------------

\end{document}