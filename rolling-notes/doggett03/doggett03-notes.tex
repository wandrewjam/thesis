%%%%%%%%%%%%%%%%%%%%%%%%%%%%%%%%%%%%%%%%%
% Article Notes
% LaTeX Template
% Version 1.0 (1/10/15)
%
% This template has been downloaded from:
% http://www.LaTeXTemplates.com
%
% Authors:
% Vel (vel@latextemplates.com)
% Christopher Eliot (christopher.eliot@hofstra.edu)
% Anthony Dardis (anthony.dardis@hofstra.edu)
%
% License:
% CC BY-NC-SA 3.0 (http://creativecommons.org/licenses/by-nc-sa/3.0/)
%
%%%%%%%%%%%%%%%%%%%%%%%%%%%%%%%%%%%%%%%%%

%----------------------------------------------------------------------------------------
%	PACKAGES AND OTHER DOCUMENT CONFIGURATIONS
%----------------------------------------------------------------------------------------

\documentclass[
10pt, % Default font size is 10pt, can alternatively be 11pt or 12pt
letterpaper, % Alternatively letterpaper for US letter
twocolumn, % Alternatively onecolumn
landscape % Alternatively portrait
]{article}

\usepackage{amsmath}


\input{structure.tex} % Input the file specifying the document layout
                      % and structure
\newcommand{\kom}{k_\textit{off, mut}}
\newcommand{\kow}{k_\textit{off, wt}}

%----------------------------------------------------------------------------------------
%	ARTICLE INFORMATION
%----------------------------------------------------------------------------------------

\articletitle{Alterations in the intrinsic properties of the
  GPIb$\alpha$-vWF tether bond define the kinetics of the
  platelet-type von Willebrand disease mutation, Gly233Val} % The title of the article
\articlecitation{\cite{doggett03aip}} % The BibTeX citation key from your bibliography

\datenotesstarted{March 28, 2018} % The date when these notes were first made
\docdate{\datenotesstarted; rev. \today} % The date when the notes were lasted updated (automatically the current date)

\docauthor{Andrew Watson} % Your name

%----------------------------------------------------------------------------------------

\begin{document}

\pagestyle{myheadings} % Use custom headers
\markright{\doctitle} % Place the article information into the header

%----------------------------------------------------------------------------------------
%	PRINT ARTICLE INFORMATION
%----------------------------------------------------------------------------------------

\thispagestyle{plain} % Plain formatting on the first page

\printtitle % Print the title

%----------------------------------------------------------------------------------------
%	ARTICLE NOTES
%----------------------------------------------------------------------------------------

\section{Introduction} 

In their previous paper, Doggett et al compared the kinetic properties
of the GP1b-vWF bond with wild type vWF and with 2B mutated vWF
\cite{doggett02slk}. 
In this paper, they study adhesion of platelets donated from patients
with Platelet-type von Willebrand disease (PT-vWD). These patients are
heterozygous for a mutation in the GP1b receptors, therefore theirpa
platelets have both wild-type and mutant GP1b receptors. The authors
developed three hypothetical models of platelet binding to account for
this heterogeneous population of GP1b:
\begin{enumerate}
\item All tethering events are mediated by a single mutant GP1b-vWF
  bond. \label{item:model1}
\item All tethering events are mediated by a single GP1b-vWF bond,
  which can either be wild-type or mutant. \label{item:model2}
\item Tethering can be mediated by multiple GP1b-vWF bonds of either
  type. \label{item:model3}
\end{enumerate}

%------------------------------------------------

\section{Models} 

In Model \ref{item:model1}, the pause time represents the length of
time it takes for a single mutant GP1b and vWF to dissociate. Assume
dissociation is a 1st order reaction with rate $k_\textit{off,
  mut}$. Then the pause times are exponentially distributed with
$\langle t \rangle = 1/\kom$ (i.e. $P(t) = \kom \exp(-\kom t)$). For
this model, the statistical estimate of $\kom$ is simply $\kom =
1/\langle t \rangle$. 

For Model \ref{item:model2}, a fraction $q$ of GP1b-vWF bonds are with
wild-type GP1b and $1-q$ of GP1b-vWF bonds are with mutant GP1b. The
pause time represents the length of time it takes one of these bonds
to dissociate, and if we assume first-order kinetics for both of these
bonds, then the pause time distribution is given by $P(t) = q\kow
\exp(-\kow t) + (1-q) \kom \exp(-\kom t)$. Aside: the parameter $q$ depends
(not explicitly) on the $k_\textit{on}$ rates of both receptors, and
their relative levels of expression. They take $\kow = 3.5
\exp\left(\frac{\sigma F_b}{k T}\right)$ from 
\cite{doggett02slk}. 
Then to estimate $\kom$, they solved 
\begin{equation*}
  \sum{i=1}^N \frac{(1-q)\exp(-\kom t_i)(1 - \kom t_i)}{q \kow
    \exp(-\kow t_i) + (1-q) \kom \exp(-\kom t_i)} = 0.
\end{equation*}
Values of $q$ were sampled over the range of possible values ($0 \le q
\le 1$) and a value of $\kom$ was calculated for each value of
$q$. Then they selected the $(q, \kom)$ pair that gave the best fit to
the data.

For Model \ref{item:model3}, multiple bonds of either type can
form. In this case, the distribution of pause times doesn't have a
simple expression, so they used MC simulations for many different $(q,
\kom)$ pairs and selected the one with the best fit. 

As in \cite{doggett02slk},
the simulation starts with a single GP1b-vWF bond. The GP1b in this
bond is wild-type with probability $q$ and mutant with probability
$1-q$. Then 4 (or 5) things can happen in a time interval $dt$:
\begin{enumerate}
\item A WT GP1b-vWF bond can break ($P_1 dt = \kow n_{wt} dt$),
\item A mutant GP1b-vWF bond can break ($P_2 dt = \kom n_{mut} dt$),
\item A WT GP1b-vWF bond can form ($P_3 dt = k_\textit{on, wt} X_{A1}
  X_{wt} dt$),
\item A mutant GP1b-vWF bond can form ($P_4 dt = k_\textit{on, mut}
  X_{A1} X_{mut} dt$),
\item if there are no bonds, the bead can leave ($P_5 dt = \gamma_w
  \delta_{n, 0} dt$). It isn't clear if this is part of their
  simulation, but it is includeded in their previous paper.
\end{enumerate}

%------------------------------------------------

\section{Results Overview}

Similar to their previous paper, they collected data on platelet
accumulation, translocation velocity, frequency of bead thethering,
and pause times at variable wall shear rates. They estimated $\kom$
for each shear rate based on Model \ref{item:model1}, and then
estimated $k_\textit{off}^0$ and $\sigma$, the parameters in the Bell
model:
\begin{equation*}
  k_\textit{off}(F) = k_\textit{off}^0 \exp\left(\frac{\sigma F}{kT}\right).
\end{equation*}

In comparing their three hypothetical models, they found that Model
\ref{item:model1} was most consistent with the experimental
data. Finally they compared the kinetic parameters of the GP1b-vWF
bond in 2B vWD with PT vWD and found no significant difference between
these two bonds.

%------------------------------------------------

\section*{Summary}

This paper extends the analysis developed in \cite{doggett02slk}
to the case where there are two different receptors in a
platelet. While I'm not so interested in the biological conclusions of
this paper, this framework may be useful in modeling platelet priming
by agonists that act through multiple receptors.

%----------------------------------------------------------------------------------------
%	BIBLIOGRAPHY
%----------------------------------------------------------------------------------------

\renewcommand{\refname}{Reference} % Change the default bibliography title

\bibliography{sample} % Input your bibliography file

%----------------------------------------------------------------------------------------

\end{document}