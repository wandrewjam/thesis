%%%%%%%%%%%%%%%%%%%%%%%%%%%%%%%%%%%%%%%%%
% Article Notes
% LaTeX Template
% Version 1.0 (1/10/15)
%
% This template has been downloaded from:
% http://www.LaTeXTemplates.com
%
% Authors:
% Vel (vel@latextemplates.com)
% Christopher Eliot (christopher.eliot@hofstra.edu)
% Anthony Dardis (anthony.dardis@hofstra.edu)
%
% License:
% CC BY-NC-SA 3.0 (http://creativecommons.org/licenses/by-nc-sa/3.0/)
%
%%%%%%%%%%%%%%%%%%%%%%%%%%%%%%%%%%%%%%%%%

%------------------------------------------------------------------------
%	PACKAGES AND OTHER DOCUMENT CONFIGURATIONS
%------------------------------------------------------------------------

\documentclass[
10pt, % Default font size is 10pt, can alternatively be 11pt or 12pt
letterpaper, % Alternatively letterpaper for US letter
twocolumn, % Alternatively onecolumn
landscape % Alternatively portrait
]{article}

\usepackage{amsmath}

%%%%%%%%%%%%%%%%%%%%%%%%%%%%%%%%%%%%%%%%%
% Article Notes
% Structure Specification File
% Version 1.0 (1/10/15)
%
% This file has been downloaded from:
% http://www.LaTeXTemplates.com
%
% Authors:
% Vel (vel@latextemplates.com)
% Christopher Eliot (christopher.eliot@hofstra.edu)
% Anthony Dardis (anthony.dardis@hofstra.edu)
%
% License:
% CC BY-NC-SA 3.0 (http://creativecommons.org/licenses/by-nc-sa/3.0/)
%
%%%%%%%%%%%%%%%%%%%%%%%%%%%%%%%%%%%%%%%%%

%----------------------------------------------------------------------------------------
%	REQUIRED PACKAGES
%----------------------------------------------------------------------------------------

\usepackage[includeheadfoot,columnsep=2cm, left=1in, right=1in, top=.5in, bottom=.5in]{geometry} % Margins

\usepackage[T1]{fontenc} % For international characters
\usepackage{XCharter} % XCharter as the main font

\usepackage{natbib} % Use natbib to manage the reference
\bibliographystyle{apalike} % Citation style

\usepackage[english]{babel} % Use english by default

%----------------------------------------------------------------------------------------
%	CUSTOM COMMANDS
%----------------------------------------------------------------------------------------

\newcommand{\articletitle}[1]{\renewcommand{\articletitle}{#1}} % Define a command for storing the article title
\newcommand{\articlecitation}[1]{\renewcommand{\articlecitation}{#1}} % Define a command for storing the article citation
\newcommand{\doctitle}{\articlecitation\ --- ``\articletitle''} % Define a command to store the article information as it will appear in the title and header

\newcommand{\datenotesstarted}[1]{\renewcommand{\datenotesstarted}{#1}} % Define a command to store the date when notes were first made
\newcommand{\docdate}[1]{\renewcommand{\docdate}{#1}} % Define a command to store the date line in the title

\newcommand{\docauthor}[1]{\renewcommand{\docauthor}{#1}} % Define a command for storing the article notes author

% Define a command for the structure of the document title
\newcommand{\printtitle}{
\begin{center}
\textbf{\Large{\doctitle}}

\docdate

\docauthor
\end{center}
}

%----------------------------------------------------------------------------------------
%	STRUCTURE MODIFICATIONS
%----------------------------------------------------------------------------------------

\setlength{\parskip}{3pt} % Slightly increase spacing between paragraphs

% Uncomment to center section titles
%\usepackage{sectsty}
%\sectionfont{\centering}

% Uncomment for Roman numerals for section numbers
%\renewcommand\thesection{\Roman{section}}
 % Input the file specifying the document layout
                      % and structure 

%------------------------------------------------------------------------
%	ARTICLE INFORMATION
%------------------------------------------------------------------------

\articletitle{Experimental and numerical study of platelets rolling on
a von Willebrand factor-coated surface} % The title of the article
\articlecitation{\cite{Pujos2018}} % The BibTeX citation key from your
                                % bibliography

\datenotesstarted{April 13, 2018} % The date when these notes were
                                % first made
\docdate{\datenotesstarted; rev. \today} % The date when the notes
                                % were lasted updated (automatically
                                % the current date)

\docauthor{Andrew Watson} % Your name

%------------------------------------------------------------------------

\begin{document}

\pagestyle{myheadings} % Use custom headers
\markright{\doctitle} % Place the article information into the header

%------------------------------------------------------------------------
%	PRINT ARTICLE INFORMATION
%------------------------------------------------------------------------

\thispagestyle{plain} % Plain formatting on the first page

\printtitle % Print the title

%------------------------------------------------------------------------
%	ARTICLE NOTES
%------------------------------------------------------------------------

% \section{Introduction}

%------------------------------------------------

\section{Experimental Observations}

\begin{itemize}
\item Experimental setup: Fixed platelets were perfused through a
  microfluidic channel coated with von Willebrand factor.
\item Flow chamber dimensions: 400 $\mu$m wide, 4 cm long, and a
  height of either 14 or 63 $\mu$m (why?). The shear rate was between
  $1400$ and $1800$ s$^{-1}$, to facilitate unfolding of vWF.
\item After 95 min of perfusion, platelets adhered at a greater
  concentration close to the channel entrance, and decreasing platelet
  concentration along the flow channel. 
\item Videos showed a couple of different platelet behaviors. Over the
  17 minute time period illustrated in the paper, some platelets
  remained adhered throughout the time interval, some platelets rolled
  during the entire period, some platelets initially rolled and then
  adhered, and some platelets which were initially adhered came
  unstuck and began rolling. 
\item Note: they do not characterize the rolling at all, i.e. no
  rolling velocity, pause times, step size, etc are measured.
\end{itemize}

%------------------------------------------------

\section{Model Description}

\begin{itemize}
\item In their model, they only consider platelet dynamics on the
  vWF-coated surface, and a small boundary layer in the fluid above
  the surface, which contains platelets that are able to interact with
  the surface.
\item Platelets in the bulk outside of this boundary layer are assumed
  to have some fixed concentration $C_v^\infty$.
\item To derive PDEs for the platelet concentration, consider a small
  volume in the boundary layer with dimensions $dx \times dy \times
  h$, where $h$ is the height of the boundary layer.
\item Denote the number of platelets in the boundary layer element as
  $n_v$ and the number of platelets on the surface below this element
  as $n_s$. 
\item Platelet number can change in the volume element by
  transport in the flow into/out of the upstream/downstream ends,
  respectively, binding/unbinding to the surface, and diffusion from
  bulk. 
\item Platelet number can change on the surface element by rolling
  into/out of the region, and binding/unbinding. 
\item Then by taking limits as $dx, dy, dt \rightarrow 0$, you get an
  advection-reaction equation for platelets in the boundary layer and
  platelets on the surface:  
  \begin{align}
    \label{eq:dimensional_pde1}
    &\partial_t C_v + V_v \partial_x C_x = -J +
    \frac{D}{l_D^2}(C_v^\infty - C_v) \\
    \label{eq:dimensional_pde2}
    &\partial_t C_s + V_s \partial_x C_s = h J \\
    \label{eq:dimensional_reaction} 
    &J = k_\text{on} C_v \left(1 - \frac{C_s}{C_{s,
      \text{max}}}\right) - k_\text{off} \frac{C_s}{h}.
  \end{align}
\item Parameter definitions:\\
  \begin{tabular}{|c|c|}
    \hline
    $V_v$ & Velocity of free-flowing platelets \\
    $V_s$ & Velocity of rolling platelets \\
    $D$ & Diffusion coefficient of platelets \\
    $l_D$ & Characteristic length scale of diffusion \\
    $C_v^\infty$ & Concentration of platelets in the bulk \\
    $h$ & Height of the boundary layer \\
    $k_\text{on}$ & On rate of platelet binding to vWF \\
    $k_\text{off}$ & Off rate of platelet binding to vWF \\
    $C_{s,\text{max}}$ & Maximum concentration of platelets on the
                         surface \\ \hline
  \end{tabular}
\item They non-dimensionalized the above equations to get:
  \begin{align*}
    &v_t + v_x = -j + d(v^\infty - v) \\
    &s_t + \epsilon s_x = j \\
    &j = v(1-s) - \alpha s.
  \end{align*}
\item Non-dimensional parameter definitions: $v = \frac{C_v
    j}{C_\text{s, max}}$, $d = \frac{D}{l_D^2 k_\text{on}}$, $\epsilon
  = \frac{V_s}{V_v}$, and $\alpha =
  \frac{k_\text{off}}{k_\text{on}}$.
\end{itemize}

%------------------------------------------------

\section{Discussion/Conclusions Overview}

\begin{itemize}
\item Figure 9 in the paper compares experimental data on accumulation
  of platelets on the surface with simulations from the fitted
  model. The model seems to fit-ish, but it in particular
  underestimates platelet accumulation on the surface early in time at
  points close to the inflow and points near the outflow.
\item Other estimated parameters seem more
  reasonable. $C_{s,\text{max}}$ is $1.7 \times 10^4$ mm$^{-2}$,
  compared to the theoretical maximal 2D packing concentration $12
  \times 10^4$ mm$^{-2}$. 
\item $k_\text{on}$ is estimated at 0.24 s$^{-1}$, which is consistent
  with estimates by \cite{Fitzgibbon2014}.
\item $k_\text{off}$ is estimated at $6 \times 10^{-5}$, which they
  claim is reasonable, because the surface is densely coated with vWF.
\item The height of the boundary layer is found to be $0.1$
  $\mu$m. This is obviously quite a bit smaller than a platelet, so
  they interpret the volume concentration of platelets in the boundary
  layer as a probability that a platelet in the bulk interacts with
  the wall.
\item The estimated value of $V_v$ is 0.36 mm/s. Based on a shear rate
  of 1500 s$^{-1}$, the velocity of fluid 1 platelet radius from the
  wall is about $1.5$ mm/s. In the paper, they estimate $V_v$ to be
  0.6 based on the shear rate (somehow).
\item The estimated value of $V_s$ is effectively 0 ($\sim 10^{-10}$
  to $10^{-13}$). They speculate that this is due to the presence of
  platelets that were activated prior to fixing, and remain bound to
  the surface throughout the experiment.
\item When they included diffusion (assumed to be shear-induced
  diffusion), the curve farthest from the channel entrance fit the
  experimental data more closely. That is, platelets accumulated more
  quickly when they included diffusion.
\item Because their estimated value of $V_s$ is so low, they just
  exclude it from the model, and their new parameter estimates remain
  within the same order of magnitude.
\item They also eliminate the saturation term because they only see
  values of $C_s$ much lower than $C_\text{s, max}$.
\item In the supplementary data, they have a couple of figures showing
  model fits to data from different patients, and under different
  experimental conditions. 
\end{itemize}

%------------------------------------------------

\section*{Article Evaluation}

This model doesn't include any activation (platelets are fixed in the
experiment) and greatly simplifies diffusion and rolling. Their data
is much more suited to estimating platelet on and off rates than
rolling parameters. I am dissatisfied with their interpretation of the
concentration of platelets in the boundary layer as a probability of a
platelet from the bulk contacting the wall. It seems to me they
developed their model with the assumption that the boundary layer is
large enough to contain enough platelets to reasonably be approximated
as a continuum, but the estimated $h$ based on fitting doesn't
validate this assumption, so they change their interpretation of the
model after the fact.

%-----------------------------------------------------------------------
%	BIBLIOGRAPHY
%-----------------------------------------------------------------------

\renewcommand{\refname}{Reference} % Change the default bibliography
                                % title 

\bibliography{/Users/andrewwork/thesis/library} % Input your
                                % bibliography file 

%-----------------------------------------------------------------------

\end{document}