%%%%%%%%%%%%%%%%%%%%%%%%%%%%%%%%%%%%%%%%%
% Article Notes
% LaTeX Template
% Version 1.0 (1/10/15)
%
% This template has been downloaded from:
% http://www.LaTeXTemplates.com
%
% Authors:
% Vel (vel@latextemplates.com)
% Christopher Eliot (christopher.eliot@hofstra.edu)
% Anthony Dardis (anthony.dardis@hofstra.edu)
%
% License:
% CC BY-NC-SA 3.0 (http://creativecommons.org/licenses/by-nc-sa/3.0/)
%
%%%%%%%%%%%%%%%%%%%%%%%%%%%%%%%%%%%%%%%%%

%----------------------------------------------------------------------------------------
%	PACKAGES AND OTHER DOCUMENT CONFIGURATIONS
%----------------------------------------------------------------------------------------

\documentclass[
10pt, % Default font size is 10pt, can alternatively be 11pt or 12pt
letterpaper, % Alternatively letterpaper for US letter
twocolumn, % Alternatively onecolumn
landscape % Alternatively portrait
]{article}

\input{structure.tex} % Input the file specifying the document layout
% and structure
\newcommand{\tn}{\textnormal}

%----------------------------------------------------------------------------------------
%	ARTICLE INFORMATION
%----------------------------------------------------------------------------------------

\articletitle{Assessing transient and persistent pain in animals} % The title of the article
\articlecitation{\cite{dubner1999assessing}} % The BibTeX citation key from your bibliography

\datenotesstarted{March 28, 2018} % The date when these notes were first made
\docdate{\datenotesstarted; rev. \today} % The date when the notes were lasted updated (automatically the current date)

\docauthor{Andrew Watson} % Your name

%----------------------------------------------------------------------------------------

\begin{document}

\pagestyle{myheadings} % Use custom headers
\markright{\doctitle} % Place the article information into the header

%----------------------------------------------------------------------------------------
%	PRINT ARTICLE INFORMATION
%----------------------------------------------------------------------------------------

\thispagestyle{plain} % Plain formatting on the first page

\printtitle % Print the title

%----------------------------------------------------------------------------------------
%	ARTICLE NOTES
%----------------------------------------------------------------------------------------

\section*{Introduction} % Unnumbered section

Leukocytes use selectins to bind to sites of injury or inflammation,
and have an analogous role to GP1b in platelets. Rolling of leukocytes
is caused by rapid formation and breakage of selectin-glycoprotein
bonds and is important for firm adhesion (which requires the
activation of integrins). Estimates of typical dissociation rates of
selectins range from \textasciitilde 0.7 s$^{-1}$ to more than 10
s$^{-1}$. The authors expected to find similar rates for GP1b. 

However, previous estimates of $k_\tn{off}$ for GP1b were much lower:
0.0038 s$^{-1}$. These estimates were based on equilibrium binding of
GP1b. The authors' goal was to estimate the off-rate of GP1b-A1 bonds
using a kinetic analysis of platelets in flow.

%------------------------------------------------

\section{Results Overview}

First they confirm that platelets rapidly bind to injured endothelium
and translocate in vivo. They found that platelet-vessel wall
interaction times were less than 1 second. They also observed
translocating platelets, and found only a few firm adhesions ($<
15\%$).

Next they looked at the effects of varying wall shear rate. Platelets
transiently interacted with wild-type vWF above wall shear stresses of 0.73
dyn/cm$^2$ (73 s$^{-1}$ wall shear rate). The vWF with type 2B
mutation allowed transient binding at any shear rate, and there was no
apparent dependence on the shear rate. They found that when the shear
rate dropped below the threshold necessary to support transient
adhesions, platelets were released from the wall in less than 1 s, and
platelets began binding again almost immediately after the
above-threshold shear rate was established. 
%------------------------------------------------

\section{Methodology Overview} % Numbered section

%------------------------------------------------

\section{Discussion/Conclusions Overview}

Donec ultrices odio in rhoncus rutrum. Nunc tristique venenatis nisl in aliquam. Aenean vulputate nisl quis nibh dapibus cursus. Suspendisse ornare mauris lorem, sit amet gravida massa luctus ac. Nullam facilisis sodales erat in porttitor. Curabitur vitae leo tellus. Pellentesque fermentum, lorem id tempus blandit, massa quam condimentum dolor, et vestibulum mi eros sit amet orci. Quisque velit quam, ullamcorper eu pretium porttitor, scelerisque sit amet odio. Suspendisse quis tincidunt velit. 

%------------------------------------------------

\section*{Article Evaluation}

I found their approach of subjecting helpless animals to long-term pain stimuli and monitoring depressive behaviours afterwards both novel and interesting.

%----------------------------------------------------------------------------------------
%	BIBLIOGRAPHY
%----------------------------------------------------------------------------------------

\renewcommand{\refname}{Reference} % Change the default bibliography title

\bibliography{sample} % Input your bibliography file

%----------------------------------------------------------------------------------------

\end{document}