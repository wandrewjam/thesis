%%%%%%%%%%%%%%%%%%%%%%%%%%%%%%%%%%%%%%%%%
% Article Notes
% LaTeX Template
% Version 1.0 (1/10/15)
%
% This template has been downloaded from:
% http://www.LaTeXTemplates.com
%
% Authors:
% Vel (vel@latextemplates.com)
% Christopher Eliot (christopher.eliot@hofstra.edu)
% Anthony Dardis (anthony.dardis@hofstra.edu)
%
% License:
% CC BY-NC-SA 3.0 (http://creativecommons.org/licenses/by-nc-sa/3.0/)
%
%%%%%%%%%%%%%%%%%%%%%%%%%%%%%%%%%%%%%%%%%

%----------------------------------------------------------------------------------------
%	PACKAGES AND OTHER DOCUMENT CONFIGURATIONS
%----------------------------------------------------------------------------------------

\documentclass[
10pt, % Default font size is 10pt, can alternatively be 11pt or 12pt
letterpaper, % Alternatively letterpaper for US letter
twocolumn, % Alternatively onecolumn
landscape % Alternatively portrait
]{article}

\input{structure.tex} % Input the file specifying the document layout and structure

%----------------------------------------------------------------------------------------
%	ARTICLE INFORMATION
%----------------------------------------------------------------------------------------

\articletitle{Integrin Dynamics and platelet mechanosensing via
  surface receptors} % The title of the article
\articlecitation{} % The BibTeX citation key from your bibliography

\datenotesstarted{May 23, 2018} % The date when these notes were first made
\docdate{\datenotesstarted; rev. \today} % The date when the notes were lasted updated (automatically the current date)

\docauthor{Andrew Watson} % Your name

%----------------------------------------------------------------------------------------

\begin{document}

\pagestyle{myheadings} % Use custom headers
\markright{\doctitle} % Place the article information into the header

%----------------------------------------------------------------------------------------
%	PRINT ARTICLE INFORMATION
%----------------------------------------------------------------------------------------

\thispagestyle{plain} % Plain formatting on the first page

\printtitle % Print the title

%----------------------------------------------------------------------------------------
%	ARTICLE NOTES
%----------------------------------------------------------------------------------------

\section*{Summary}

\begin{itemize}
\item When a mechanosensor recieves a mechanical signal, the signal
  propagates along the receptor. It is usually accompanied by
  conformational changes.
\item This signal is translated into a chemical signal, which then can
  mediate a host of different physiological or pathological responses.
\item Mechanotransduction through integrins is realized by
  conformational changes that allow bidirectional signaling.
\item GP1b$\alpha$ and $\alpha_{IIb} \beta_3$ mediate early and
  mid-stages of platelet adhesion and activation.
\item Mechanosensing ultimately results in enhanced binding capacity
  of the receptors, and signal transduction leading to activation of
  platelets
\item How does the platelet and its receptors answer the following
  questions:
  \begin{enumerate}
  \item How are integrin conformational changes related to its binding
    kinetics? 
  \item How do signals in different waveforms affect signaling through
    GP1b and integrins?
  \item What are the molecular mechanisms for the signal to propagate
    across the cell membrane?
  \item What is the difference between activation through
    mechanosensors and activation through soluble ligands?
  \end{enumerate}
\item Brief answers to these questions:
  \begin{enumerate}
  \item Integrin $\alpha_V \beta_3$ undergoes bending and unbending in
    a dynamic fashion, and does not require energetic or signaling
    support from the cell. Bending dynamics are regulated by
    mechanical force, ligand, genotype, and extracellular
    environment. Force-enabled $\alpha_v \beta_3$ unbending does not
    reinforce the strength of its binding in physiological conditions
    (seems to contradict the story that unbending results in a
    high-affinity conformation).
  \item A single GP1b$\alpha$--vWF-A1 binding event can trigger
    intracellular Ca$^{++}$ release, and the mechanical
    characteristics of the bond can modulate the extent of this
    release. There are two mechanosensitive domains in GP1b$\alpha$:
    the leucine-rich-repeat domain (LRRD) and the mechanosensitive
    domain (MSD). LRRD unfolding prolongs the binding lifetime, and
    MSD unfolding affects the Ca$^{++}$ flux.
  \item Mechanotransduction of GP1b$\alpha$ leads to an intermediate
    level of integrin activation, P-selectin expression, and PS
    exposure. Up-regulation of $\alpha_{IIb} \beta_3$ is transient,
    but it allows the integrin to be stabilized in the active state
    through outside-in signaling
  \end{enumerate}
\end{itemize}

%------------------------------------------------

\section{Methodology Overview}



%------------------------------------------------

\section{Results Overview}



%------------------------------------------------

\section{Discussion/Conclusions Overview}



%------------------------------------------------

\section*{Article Evaluation}



%----------------------------------------------------------------------------------------
%	BIBLIOGRAPHY
%----------------------------------------------------------------------------------------

\renewcommand{\refname}{Reference} % Change the default bibliography title

\bibliography{/Users/andrewwork/thesis/library} % Input your
                                % bibliography file 

%----------------------------------------------------------------------------------------

\end{document}