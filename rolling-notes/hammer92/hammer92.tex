%%%%%%%%%%%%%%%%%%%%%%%%%%%%%%%%%%%%%%%%%
% Article Notes
% LaTeX Template
% Version 1.0 (1/10/15)
%
% This template has been downloaded from:
% http://www.LaTeXTemplates.com
%
% Authors:
% Vel (vel@latextemplates.com)
% Christopher Eliot (christopher.eliot@hofstra.edu)
% Anthony Dardis (anthony.dardis@hofstra.edu)
%
% License:
% CC BY-NC-SA 3.0 (http://creativecommons.org/licenses/by-nc-sa/3.0/)
%
%%%%%%%%%%%%%%%%%%%%%%%%%%%%%%%%%%%%%%%%%

%----------------------------------------------------------------------------------------
%	PACKAGES AND OTHER DOCUMENT CONFIGURATIONS
%----------------------------------------------------------------------------------------

\documentclass[
10pt, % Default font size is 10pt, can alternatively be 11pt or 12pt
letterpaper, % Alternatively letterpaper for US letter
twocolumn, % Alternatively onecolumn
landscape % Alternatively portrait
]{article}

\usepackage{amsmath}
\newcommand{\mv}{\textnormal{mv}}
\newcommand{\ch}{\textnormal{ch}}
\newcommand{\x}{\mathbf{x}}
\newcommand{\on}{\textnormal{on}}
\newcommand{\off}{\textnormal{off}}
\newcommand{\ts}{\textnormal{ts}}
\newcommand{\sep}{\textnormal{sep}}
%%%%%%%%%%%%%%%%%%%%%%%%%%%%%%%%%%%%%%%%%
% Article Notes
% Structure Specification File
% Version 1.0 (1/10/15)
%
% This file has been downloaded from:
% http://www.LaTeXTemplates.com
%
% Authors:
% Vel (vel@latextemplates.com)
% Christopher Eliot (christopher.eliot@hofstra.edu)
% Anthony Dardis (anthony.dardis@hofstra.edu)
%
% License:
% CC BY-NC-SA 3.0 (http://creativecommons.org/licenses/by-nc-sa/3.0/)
%
%%%%%%%%%%%%%%%%%%%%%%%%%%%%%%%%%%%%%%%%%

%----------------------------------------------------------------------------------------
%	REQUIRED PACKAGES
%----------------------------------------------------------------------------------------

\usepackage[includeheadfoot,columnsep=2cm, left=1in, right=1in, top=.5in, bottom=.5in]{geometry} % Margins

\usepackage[T1]{fontenc} % For international characters
\usepackage{XCharter} % XCharter as the main font

\usepackage{natbib} % Use natbib to manage the reference
\bibliographystyle{apalike} % Citation style

\usepackage[english]{babel} % Use english by default

%----------------------------------------------------------------------------------------
%	CUSTOM COMMANDS
%----------------------------------------------------------------------------------------

\newcommand{\articletitle}[1]{\renewcommand{\articletitle}{#1}} % Define a command for storing the article title
\newcommand{\articlecitation}[1]{\renewcommand{\articlecitation}{#1}} % Define a command for storing the article citation
\newcommand{\doctitle}{\articlecitation\ --- ``\articletitle''} % Define a command to store the article information as it will appear in the title and header

\newcommand{\datenotesstarted}[1]{\renewcommand{\datenotesstarted}{#1}} % Define a command to store the date when notes were first made
\newcommand{\docdate}[1]{\renewcommand{\docdate}{#1}} % Define a command to store the date line in the title

\newcommand{\docauthor}[1]{\renewcommand{\docauthor}{#1}} % Define a command for storing the article notes author

% Define a command for the structure of the document title
\newcommand{\printtitle}{
\begin{center}
\textbf{\Large{\doctitle}}

\docdate

\docauthor
\end{center}
}

%----------------------------------------------------------------------------------------
%	STRUCTURE MODIFICATIONS
%----------------------------------------------------------------------------------------

\setlength{\parskip}{3pt} % Slightly increase spacing between paragraphs

% Uncomment to center section titles
%\usepackage{sectsty}
%\sectionfont{\centering}

% Uncomment for Roman numerals for section numbers
%\renewcommand\thesection{\Roman{section}}
 % Input the file specifying the document layout and structure

%----------------------------------------------------------------------------------------
%	ARTICLE INFORMATION
%----------------------------------------------------------------------------------------

\articletitle{Simulation of cell rolling and adhesion on surfaces in
  shear flow} % The title of the article
\articlecitation{\cite{Hammer1992}} % The BibTeX citation key from your bibliography

\datenotesstarted{April 30, 2018} % The date when these notes were first made
\docdate{\datenotesstarted; rev. \today} % The date when the notes were lasted updated (automatically the current date)

\docauthor{Andrew Watson} % Your name

%----------------------------------------------------------------------------------------

\begin{document}

\pagestyle{myheadings} % Use custom headers
\markright{\doctitle} % Place the article information into the header

%----------------------------------------------------------------------------------------
%	PRINT ARTICLE INFORMATION
%----------------------------------------------------------------------------------------

\thispagestyle{plain} % Plain formatting on the first page

\printtitle % Print the title

%----------------------------------------------------------------------------------------
%	ARTICLE NOTES
%----------------------------------------------------------------------------------------

\section{Introduction Overview}

\begin{itemize}
\item Goal: come up with a physically realistic, stochastic model of
  leukocyte rolling
\item Motion of the cell is calculated by balancing fluid forces on
  the cell with bond and colloidal forces.
\item They explicitly track individual microvilli and receptors, and
  their bound states. 
\item They assume microvilli all have a fixed length, are stiff and
  oriented normal to the surface of a spherical cell, and support a
  random number of receptors.
\item All receptors are the same type, and they assume all the
  receptors are linear springs with force-dependent on and off rates
  governed by the Dembo model. 
\item They compute rolling velocities for different parameters. They
  also characterize a few qualitatively different rolling/adhesive
  behaviors (unbound, rolling, tumbling, transient adhesion, and
  adhesion) and identify important parameters for rolling.
\end{itemize}

%------------------------------------------------

\section{Model Description}

\subsection{Cell Geometry and Receptors}
\label{sec:cell-geom-recept}


\begin{itemize}
\item Assume the cell is a sphere of radius $R_c$, which is covered by
  $N_\mv$ microvilli with radius $R_\mv$ and length $L_\mv$. Assume
  the microvilli are rigid, and are oriented normal to the cell
  surface. 
\item For calculating the hydrodynamic forces on the cell, it is
  assumed that the cell is a sphere with radius $R_\ch \equiv R_c +
  L_\mv$. 
\item There are $R_T$ receptors on a cell, and the rolling surface
  has a ligand density of $N_l$.
\item Assume the adhesion molecules are linear springs.
\item Bonds are assumed to form between the center of the tip of a
  microvillus (denoted $\x_m \equiv (x_m, y_m, z_m)$) and a point on
  the surface (denoted $\x_0 \equiv (x_0, y_0, z_0)$). Define $\x_b =
  \x_0 - \x_m$ which gives the orientation and length of the bond.
\item The on and off rates of each bond depend on the bond length
  according to the Dembo model: 
\begin{align}
  k_\on &= k_\on^0 \exp\left(- \frac{\sigma_\ts
                (L_\sep - \lambda)^2}{2k_b
          T}\right) \label{eq:dembo_on} \\ 
  k_\off &= k_\off^0 \exp\left(\frac{(\sigma -
                 \sigma_\ts)(L_\sep - \lambda)^2}{2k_b
                 T}\right). \label{eq:dembo_off}
\end{align}
\item Here $\lambda$ is the equilibrium separation distance, $k_b T$
  is the thermal energy, $\sigma$ is the spring constant, and
  $\sigma_\ts$ is the transition state spring constant.
\item Note from equations (\ref{eq:dembo_on}) and
  (\ref{eq:dembo_off}), if $\sigma - \sigma_\ts > 0$ the bonds are
  slip bonds, but if $\sigma - \sigma_\ts < 0$ the bonds are catch
  bonds.
\item They find that $\sigma - \sigma_\ts$ has a large effect on
  cell adhesion in flow.
\item They assume that only receptors on microvilli tips are able to
  bind to the rolling surface, therefore they only track receptors on
  the tips of microvilli.
\item Bonds form normal to the rolling surface, so $k_\on$ is given by
  equation (\ref{eq:dembo_on}) with $L_\sep \equiv x_m$.
\item The off rate of a specific bond depends on the length of the
  existing bond, and so $k_\off$ is given by equation
  (\ref{eq:dembo_off}) with $L_\sep \equiv |\x_b|$.
\item If receptors are distributed uniformly over the surface of the
  cell, then the number of receptors on the tip of a single
  microvillus is Poisson-distributed:
  \begin{equation}
    \label{eq:receptor_distribution}
    P(n) = \frac{\left(\frac{R_T A_\mv}{A_c}\right)^n \exp
      \left(-\frac{R_T A_\mv}{A_c}\right)}{n!}.
  \end{equation}
\item $A_\mv$ is the surface area of a single microvillus tip, and
  $A_c$ is the surface area of the entire cell. So $\frac{R_T
    A_\mv}{A_c}$ is the average number of receptors on a microvillus
  tip.
\item Assume that the time it takes for a given bond to form or break is
  given by an exponential distribution with mean $1/k_\on$ or
  $1/k_\off$. Then the probability that a given bond forms in a time
  interval $\Delta t$ is given by
  \begin{equation}
    \label{eq:prob_bond_form}
    P_\on = 1 - \exp(-k_\on \Delta t),
  \end{equation}
  and the probability of a given bond breaking is the same with
  $k_\off$ in place of $k_\on$.
\item Each bond generates a force and torque on the cell body, and the
  sum of these forces along with colloidal forces (described in
  sub-section \ref{sec:bond-coll-forc}) affect the cell velocity.
\item We assume the fluid velocity is given by Stokes' equation, and
  therefore the cell velocity is linearly related to the net force on
  the cell:
  \begin{equation*}
    \mathbf{U} = \underline{\underline{M}} \mathbf{F}
  \end{equation*}
where $\mathbf{U} = (V_x, V_y, V_z, \Omega_x, \Omega_y, \Omega_z)$ and
$\mathbf{F} = (F^b_x + F^c_x, F^b_y + F^s, F^b_z, C^b_x, C^b_y, C^b_z
+ C^s)$.
\item The positions of the microvilli must be tracked as the cell
  translates and rotates. Spherical coordinates are used to track the
  microvilli. If a given microvillus has coordinates $(\theta_m,
  \phi_m)$, then its position evolves according to:
  \begin{align}
    \label{eq:phi_evol}
    \frac{d \phi_m}{dt} &= -\sin\theta_m \Omega_x + \cos\theta_m
                          \Omega_y \\
    \label{eq:thet_evol}
    \frac{d \theta_m}{dt} &= -\frac{\cos\phi_m
                            \cos\theta_m}{\sin\phi_m} \Omega_x -
                            \frac{\cos\phi_m \sin\theta_m}{\sin
                            \phi_m} \Omega_y + \Omega_z.
  \end{align}
\item Both coordinate systems use the cell's frame of reference (see
  Fig. ), so the anchor point $\x_0$ of each bond must be tracked as
  well:
  \begin{equation}
    \label{eq:anch_evol}
    \frac{d \x_0}{dt} = -\mathbf{V}
  \end{equation}
\item Cartesian and spherical coordinates are related through the
  expected equations:
  \begin{align*}
    x &= R_\ch \sin\phi \cos\theta \\
    y &= R_\ch \sin\phi \sin\theta \\
    z &= R_\ch \cos\phi
  \end{align*}
\end{itemize}

\subsection{Bond and Colloidal Forces}
\label{sec:bond-coll-forc}

\begin{itemize}
\item Assume the bonds are linear springs with spring constant $\sigma$ and
  rest length $\lambda$.
\item The force generated by a single bond is given by 
  \begin{equation}
    \label{eq:bond_forc}
    \mathbf{F}^b = \sigma (|\x_b| - \lambda) \frac{\x_b}{|\x_b|}.
  \end{equation}
\item The torque generated by a single bond is given by
  \begin{equation}
    \label{eq:bond_torq}
    \mathbf{C}^b = \x_m \times \mathbf{F}^b.
  \end{equation}
\item The force due to gravity is approximately
  \begin{equation}
    \label{eq:gravity}
    F_\textnormal{gr} = \frac{4}{3}\pi(\rho_c - \rho_m) R_\ch^3 g,
  \end{equation}
where $\rho_c$ and $rho_m$ are the densities of the cell and the
medium, and $g$ is the acceleration due to gravity.
\item Colloidal forces decay strongly with distance, and usually the
  length scale for decay is shorter than the length of the microvilli
  (sources in \cite{Hammer1992}: Bongrand and Bell, 1984;
  Israelachvili, 1985).
\item Therefore assume that all of the colloidal forces acting on the
  cell come from the microvilli (except for gravity).
\item There are three colloidal forces associated with individual
  microvilli: van der Waals forces, electrostatic forces, and steric
  stabilization forces. All of these forces only act normal to the
  rolling surface, i.e. in the $x$ direction.
\item van der Waals force:
  \begin{equation}
    \label{eq:vdw_forc}
    F_\textnormal{vdw} = \frac{A_\textnormal{H} A_\mv}{6\pi}
    \sum_\textnormal{microvilli} x_m^{-3}.
  \end{equation}
$A_\textnormal{H}$ is the Hamaker constant.
\item Electrostatic force:
  \begin{equation}
    \label{eq:elec_forc}
    F_\textnormal{el} = \frac{e^2 z_1 z_2 N_1 N_2 A_\mv}{2 \epsilon_r
      \epsilon_o \delta^2} \sinh \delta L_1 \sinh \delta L_2
    \sum_\textnormal{microvilli} e^{-\delta x_m}.
  \end{equation}
% What about the signs of these forces? These equations all seem to
% give positive forces
\item Steric stabilization force:
  \begin{equation}
    \label{eq:ster_forc}
    F_\textnormal{ss} = \left(\lambda_\textnormal{ss} A_\mv
    \sum_\textnormal{microvilli} x_m^{-2}\right) \mathcal{H}\left(L_g
    - x_m\right)
  \end{equation}
where $\mathcal{H}$ is the Heaviside function.
\item Then the colloidal force is the sum of the above 4 forces (with
  the appropriate signs)
\end{itemize}

%------------------------------------------------

\section{Results Overview}

\begin{itemize}
\item Increasing the number of receptors per microvillus leads to a
  decrease in rolling velocity (see Fig. 4 in the paper). Also, the
  population average of velocity variance reaches a minimum around 5
  receptors per microvillus.
\item Increasing $\dot{\gamma}$ results in a decrease in dimensionless
  rolling velocity (dimensionless velocity is velocity scaled by
  $\dot{\gamma} R_\ch$).
\item Increasing $F_\sigma$ (a nondimensional parameter that is
  related to the bond's response to force, in the limit $F_\sigma
  \rightarrow 0$ the off rate becomes independent of force) leads to a
  greater average rolling velocity (less adhesion). 
\item Next they examined the effect of $\sigma$ on rolling
  velocity. They found that increasing bond stiffness resulted in an
  increase of rolling velocity, primarily because the off rate of
  bonds is more sensitive to force.
\item Changing the formation rate of bonds is more important to
  rolling velocity than changing the breaking rate.
\item They classify cell trajectories into 5 qualitative groups:
  unbound, rolling at a constant speed, tumbling (rolling w/ brief
  adhesion), transient adhesion, and permanent adhesion.
\item They introduce 3 new statistics to help distinguish between the
  5 different types of motion.
\item The first is $F_a$ which is the frequency of adhesion, defined
  as the fraction of time a cell has a nondimensional velocity
  $\tilde{V}_y < 0.001$.
\item The second and third are forward and reverse kinetic constants
  $k_a$ and $k_d$, defined as the inverse of the mean time before a
  moving/adherent cell becomes attached/unattached.
  \begin{equation}
    \label{eq:forward_kin_const}
    k_a = \frac{N_{\tilde{V}_y > 0.001}}{\sum_{i = 1}^{N_{\tilde{V}_y
          > 0.001}} \Delta t_{i, N_{\tilde{V}_y > 0.001}}} 
  \end{equation}
  $k_d$ is defined in a similar way.
\item 
\end{itemize}

%------------------------------------------------

\section{Discussion/Conclusions Overview}



%------------------------------------------------

\section*{Article Evaluation}



%----------------------------------------------------------------------------------------
%	BIBLIOGRAPHY
%----------------------------------------------------------------------------------------

\renewcommand{\refname}{Reference} % Change the default bibliography title

\bibliography{/Users/andrewwork/thesis/library} % Input your
                                % bibliography file 

%----------------------------------------------------------------------------------------

\end{document}