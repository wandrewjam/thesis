%%% -*-LaTeX-*-

\chapter{Document Sections}

This chapter illustrates the typesetting and control of \LaTeX{}
sectional commands: 
    \verb=\part=,
    \verb=\chapter=,
    \verb=\section=,
    \verb=\subsection=,
    \verb=\subsubsection=,
    \verb=\paragraph=,
    and
    \verb=\subparagraph=.

\LaTeX{} keeps separate counters for each of those objects,
initialized to zero, and incremented by one inside each sectioning
command, just before the number is typeset.  The first use of each
such command normally zeros the counters for all later commands in the
structure hierarchy.

\section{Sectional numbering}

Numbering depth is controlled by \LaTeX{} document preamble commands
like these:
%
\begin{verbatim}
    \setcounter {secnumdepth} {2} % default in book.cls & uuthesis-base.sty
    \setcounter {tocdepth}    {2} % default in book.cls
\end{verbatim}
%
You can change their values elsewhere in the document, as we do later
in this chapter, but it would be highly unusual to do so.

The possible counter values are:
%
\begin{center}
    \begin{tabular}{rlrlrl}
      \boldmath
      \bf $-1$ & parts               \\
      \bf 0    & plus chapters       &
      \bf 1    & plus sections       &
      \bf 2    & plus subsections    \\
      \bf 3    & plus subsubsections &
      \bf 4    & plus paragraphs     &
      \bf 5    & plus subparagraphs
    \end{tabular}
\end{center}

Deep sectional numbering is desirable when there are external
references to a document that must precisely identify a small portion
of text.  Legal documents, such as contracts, laws, and regulations,
as well as scientific handbooks, and international standards for
manufacturing and programming languages, are examples of such texts.
Most books, dissertations, and theses should rarely need numbering
deeper than sections, and possibly, subsections.

There is a separate counter, \verb=tocdepth=, for the table of
contents, because some authors prefer to have a briefer display there,
or to supply both compact and detailed contents, like this:
%
\begin{verbatim}
    \usepackage {shorttoc}
    ...
    \shorttableofcontents {Brief Contents} {0}
    %
    \setcounter   {tocdepth}      {5}
    \renewcommand {\contentsname} {Extended Contents}
    \tableofcontents
    %
    \setcounter {secnumdepth}     {2} % for default sectional numbering in this book
\end{verbatim}

\section{Cross references in your document}

If you need to cross reference a sectional unit in your document,
\emph{never} do so with an absolute number, but rather, define a
symbolic label \emph{immediately after} the sectional command, and
then reference that label, such as in these examples:
%
\begin{verbatim}
    \chapter{Integral Inequalities}%
    \label{chap:int-ineq}
    ...
    In Chapter~\ref{chap:int-ineq}, we show that ...

    % or
    In Chapter~\ref{chap:int-ineq} on page~\pageref{chap:int-ineq}, 
    we show that ...

    % or better, with \usepackage{varioref}
    In Chapter~\vref{chap:int-ineq}, we show that ...

    % or even better, with internal tagging of label types
    %     \newcommand {\chaplabel} [1] {\label{chap:#1}}
    %     \newcommand {\chapref}   [1] {Chapter~\vref{chap:#1}}

    \chapter{Integral Inequalities}%
    \chaplabel{int-ineq}
    ...
    In \chapref{int-ineq}, we show that ...
\end{verbatim}

\section{Indentation}

The first line of text following a sectional heading is traditionally
\emph{not} indented in English-language typography, because the
heading, and its following additional vertical space, are sufficient
clues that a new paragraph begins at the text.

Paragraphs and subparagraphs have run-in headings, without a newline
and a bit of extra vertical space.

Paragraph headings are flush left, but subparagraph headings are
indented, so they can be distinguished, even when they are unnumbered.

\section{This is a section}

\blah

\subsection{This is a subsection}

\blah

\subsubsection{This is a subsubsection}

\blah

\paragraph{This is a paragraph}

\blah

\subparagraph{This is a subparagraph}

\blah

%% ----------------------------------------------------------------------

\setcounter {secnumdepth} {5}

\section{This is a section}

To illustrate deeper section numbering, in this section, we
temporarily change the depth like this:
%
\begin{verbatim}
  \setcounter {secnumdepth} {5}
\end{verbatim}
%
At the end of this chapter, we restore it to the default value of
\textbf{2}.

\blah

\subsection{This is a subsection}

\blah

\subsubsection{This is a subsubsection}

\blah

\paragraph{This is a paragraph}

\blah

\subparagraph{This is a subparagraph}

\blah

\setcounter {secnumdepth} {2}

\begin{verbatim}
  \setcounter {secnumdepth} {2}
\end{verbatim}
