%%% -*-LaTeX-*-

\chapter{Mathematical Rolling Model}

\section{Problem Description}
\label{sec:problem-description}

\subsection{Geometry and Physics}
\label{sec:geometry-physics}

Assume we have a circular rigid platelet (of radius $\radius$) rolling
and translating in shear flow with shear rate $\shear$ adjacent to a
wall. The platelet translates parallel to the wall at speed
$\velocity$, and rolls at angular velocity $\rotation$. Because the
circle is always translating parallel to the wall, there is a fixed
vertical distance $\separation$ between the wall and the closest point
on the circle. Define the height $\height$ of the platelet to be the
distance from the wall to the center of the platelet ($\height =
\separation + \radius$). The fluid forces exerted on the platelet are
a function of $\height$, $\radius$, $\shear$, and $\velocity$, as well
as the fluid viscosity ($\viscosity$) which is constant across all
experiments. We assume that the platelet is moving in a Stokes flow,
meaning that the inertial terms are negligible and force balance must
be satisfied on the platelet at all times.

Due to the linearity of Stokes equations, there is a linear
relationship between platelet velocity and fluid force. In a general
3D Stokes flow, this relationship is given by a 6x6 matrix equation
$equation$ where $U$ is a 6x1 vector containing the 3 translational
velocities and 3 rotational velocities, and $F$ is a 6x1 vector
containing the 3 forces and 3 torques. However in the current model
we've made a number of simplifying assumptions.
\begin{enumerate}
\item The flow is a 2D Stokes flow, eliminating all forces and
  translational velocities along the dimension going into and out of
  the page. This also eliminates rotations and torques about the two
  other dimensions.
\item The platelet remains at a constant separation distance from the
  wall, eliminating all vertical motion (and consequently we ignore
  all vertical forces imposed by bonds between the platelet and
  wall).
\item The translational velocity is decoupled from torque, and vice
  versa. (Is this equivalent to assuming the platelet is sitting in a
  constant flow?)
\end{enumerate}
Therefore we end up with two decoupled linear equations relating the
horizontal force ($fh$) to the translational velocity ($V$) and
relating the torque ($\tau$) to the rotational velocity ($\Omega$):
% Equations.
% NOTE: I will need to expand on this more when I add equations to
% the paper. I also forgot to discuss the applied fluid velocities.

We identify bonds by their two attachment points on the wall and on
the platelet surface, so each bond has two coordinates associated with
it for each of its endpoints. Points on the wall are given by the
coordinate $x$, defined to be the horizontal distance from the center
of the circle, and points on the platelet surface are given by the
coordinate $\theta$, defined to be the angle formed by the receptor,
the center of the circle, and the closest point on the circle to the
wall (Figure something). With this definition of $x$ and $\theta$, the
distance from a ligand at $x$ on wall and a receptor $\theta$ on the
platelet surface is given by the equation
% Equation. 

\subsection{Biology}
\label{sec:biology}

The platelet surface is covered with receptors at an angular density
of $N_T$ receptors/radian, and bonds can form between points on the
surface of the circle and points on the wall. The number of ligand
binding sites on the substrate is assumed to be in excess. We assume
that bonds act like Hookean springs with a rest length of 0, so the
force exerted by a bond is proportional to its length and the
proportionality constant is the stiffness of the bond. Formation and
dissociation rates are distance and force dependent, respectively. We
use the Bell model (source) to express the bond dissociation rate as a
function of force, and the (something) model to express the bond
formation rate as a function of the distance between a receptor and
ligand. This gives us the following equations for formation rate and
dissociation rate:
% Equations.

In order to calculate $fh$ and $\tau$ for an existing set of bonds
between the platelet and wall, we simply have to add up the individual
contributions to the force and torque from each bond. The force and
torque generated by an individual bond can be derived from the
geometry of the model along with the assumption that the force vector
$F$ points in the same direction as the bond with magnitude
proportional to the bond length. I'll leave out the derivation. The
horizontal force generated by a single bond ($fh$) is given simply by
% Equation,
while the torque generated by a single bond is given by the slightly
more complicated
% Equation.

\section{Deterministic PDE model}
\label{sec:determ-pde-model}

If we assume that there is a continuous distribution of bonds between
the platelet and the wall, we can define the function $n(x, \theta,
t)$ which gives the density of bonds at time $t$ between points $x$
and $\theta$ on the wall and platelet, respectively. Bonds advect in
$x$ with velocity $V$ and they advect in $\theta$ with velocity
$\Omega$. Bonds form at the rate given in equation (something) and
saturate in $\theta$ with a maximum density $Nt$. Finally, bonds break
at the force-dependent rate given in equation (something). Putting all
of this together gives us the following PDE definition of $n$:
% Equation.

This equation can't yet be solved for $n$, because there are still 2
unknowns in it: $V$ and $\Omega$. Recall from above that $V$ and
$\Omega$ are found by balancing the fluid and bond forces acting on
the platelet (equations (something)). First, we have to calculate the
total force $fh$ and torque $tau$ generated by the distribution of
bonds $n$:
% Equations.

Once we have $fh$ and $\tau$, the platelet velocities $V$ and $\Omega$
are given by the equations: % Equations.

Now we have 3 equations in 3 unknowns and this is a closed system that
can (almost) be solved simultaneously for $n$, $V$, and $\Omega$. We
still need to define the domains of all three variables, and give
boundary conditions for $n$ before we can solve this system. Assume
that bonds can only attach to the lower half-circle of the platelet
surface, while bonds can form at any point along the wall. Thus we are
solving $n$ in the domain $something$. We need to enforce boundary
conditions at the upstream (with respect to variables $x$ and
$\theta$) ends of the $x$ and $\theta$ domains. Because we assume
there are no bonds attached to the upper half of the platelet, and
because the equilibrium concentration of bonds $something rightarrow
0$ as $L \rightarrow \infty$, we must set homogeneous Dirichlet
boundary conditions on both $x$ and $\theta$. Finally we just set the
initial condition to be $something$.

In order to eliminate some parameters, we nondimensionalize this model
by introducing nondimensional coordinates $x = Rz$ and $s = koff t$
(note that $\theta$ is already dimensionless) and nondimensional
functions $R n = Nt m$, $V = R koff v$, and $\Omega = koff
\omega$. The details of the nondimensionalization are included in
Appendix (something), but the final nondimensional set of equations is
% Equations.

The nondimensional bond length $\ell$ is given by $something$, and the
nondimensional force $f_h'$ and torque $\tau'$ are given by
$something$. A complete list of the nondimensional coordinates,
functions, and parameters with their definitions are given in Table
(something).

\section{Stochastic model}
\label{sec:stochastic-model}

