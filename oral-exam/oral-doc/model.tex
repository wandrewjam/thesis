%%% -*-LaTeX-*-

\chapter{Mathematical Rolling Model}

\section{Problem Description}
\label{sec:problem-description}

\begin{figure}
  \centering
  \begin{tikzpicture}[scale=3]
    \newcommand{\dist}{0.5}
    \newcommand{\Length}{2.5}
    \newcommand{\slope}{0.5}
    \newcommand{\buff}{0.1}

    % Define a coordinate for the center of the platelet, and draw a
    % node there
    \coordinate (pltcenter) at (0, 1 + \dist);
    \node [circle, inner sep=1.5pt, fill=black] at (pltcenter) {};

    % Draws the wall and the platelet
    \draw (-\Length, 0) -- (\Length, 0);
    \draw (pltcenter) circle [radius = 1];
    \draw[<->] (0, 0) -- node[right] {$\separation$} (0, \dist);

    % Draws the lines showing the applied velocities
    \draw[->] (pltcenter) ++(1+\buff, 0) -- node[above] {$\velocity$}
    (\Length-\buff, 1+\dist);

    \draw[->] (0, 1+\dist) ++(135:1+\buff) arc [start angle=135, end
    angle=45, radius=1+\buff] node[midway, above] {$\rotation$};

    % Draws the arrows showing the shear flow
    \foreach \y in {0.5, 1, 1.5, 2, 2.5}
    \draw[->] (-\Length, \y) -- (-\Length + \slope*\y, \y);
    \draw[gray, very thin] (-\Length, 0) -- node[midway, right, black]
    {$V_\tn{fluid} = \shear h$} (-\Length + \slope*\Length, \Length);

    % Draws a bond between the platelet and wall
    \draw[decorate, decoration=zigzag] (pltcenter) ++(315:1)
    node[circle, inner sep=1.5pt, fill=black] {} -- (1.25,0) node
    [circle, inner sep=1.5pt, fill=black] {};

    % Labels the x and theta coordinates
    \draw[{Bar[]}->] (0, -\buff) -- node[fill=white] {$x$} (1.25,
    -\buff);

    \draw (pltcenter) -- +(0, -1);
    \draw (pltcenter) -- node[above right] {$R$} +(315:1);
    \draw[->] (pltcenter) ++(0, -3*\buff) arc [start angle=270, end
    angle=315, radius=3*\buff] node[midway, below] {$\theta$};
  \end{tikzpicture}
  \caption{Model geometry}
  \label{fig:model-geometry}
\end{figure}

\subsection{Geometry and Physics}
\label{sec:geometry-physics}

A sketch of the problem geometry is given in Figure
\ref{fig:model-geometry}. Assume we have a circular rigid platelet (of
radius $\radius$) rolling and translating in shear flow with shear
rate $\shear$ adjacent to a wall. The platelet translates parallel to
the wall at speed $\velocity$, and rolls at angular velocity
$\rotation$. Because the circle is always translating parallel to the
wall, there is a fixed vertical distance $\separation$ between the
wall and the closest point on the circle. Define the height $\height$
of the platelet to be the distance from the wall to the center of the
platelet ($\height = \separation + \radius$). The fluid forces exerted
on the platelet are a function of $\height$, $\radius$, $\shear$, and
$\velocity$, as well as the fluid viscosity ($\viscosity$) which is
constant across all experiments. We assume that the platelet is moving
in a Stokes flow, meaning that the inertial terms are negligible and
force balance must be satisfied on the platelet at all times.

Due to the linearity of Stokes equations, there is a linear
relationship between platelet velocity and fluid force. In a general
3D Stokes flow, this relationship is given by a 6x6 matrix equation
$\mathbf{F} = \underline{\underline{R}} \mathbf{U}$ where $\mathbf{U}$
is a 6x1 vector containing the 3 translational velocities and 3
rotational velocities, and $\mathbf{F}$ is a 6x1 vector containing the
3 forces and 3 torques. However in the current model we've made a
number of simplifying assumptions. 
\begin{enumerate}
\item The flow is a 2D Stokes flow, eliminating all forces and
  translational velocities along the dimension going into and out of
  the page. This also eliminates rotations and torques about the two
  other dimensions.
\item The platelet remains at a constant separation distance from the
  wall, eliminating all vertical motion (and consequently we ignore
  all vertical forces imposed by bonds between the platelet and
  wall).
\item The hydrodynamic forces on the platelet are computed assuming
  the platelet is infinitely far from the wall. In particular the
  translational velocity of the platelet is assumed to be independent
  of the torque, and vice versa.
\end{enumerate}
Therefore we end up with two decoupled linear equations relating the
horizontal force ($\horzForce$) to the translational velocity
($\velocity$) and relating the torque ($\torque$) to the rotational
velocity ($\rotation$):
% Equations.
% NOTE: I will need to expand on this more when I add equations to
% the paper. I also forgot to discuss the applied fluid velocities.

We identify bonds by their two attachment points on the wall and on
the platelet surface, so each bond has two coordinates associated with
it for each of its endpoints. Points on the wall are given by the
coordinate $x$, defined to be the horizontal distance from the center
of the circle, and points on the platelet surface are given by the
coordinate $\theta$, defined to be the angle formed by the receptor,
the center of the circle, and the closest point on the circle to the
wall (Figure \ref{fig:model-geometry}). With this definition of $x$
and $\theta$, the distance from a ligand at $x$ on wall and a receptor
$\theta$ on the platelet surface is given by the equation
\begin{equation}
  \label{eq:dim-length}
  \length^2 (x, \theta) \equiv (\radius - \radius\cos\theta +
  \separation)^2 + (\radius\sin\theta - x)^2.
\end{equation}

\subsection{Biology}
\label{sec:biology}

The platelet surface is covered with receptors at an angular density
of $\receptorDensity$ receptors/radian, and bonds can form between
points on the surface of the circle and points on the wall. The number
of ligand binding sites on the substrate is assumed to be in
excess. We assume that bonds act like Hookean springs with a rest
length of 0, so the force exerted by a bond is proportional to its
length and the proportionality constant is the stiffness of the
bond. Formation and dissociation rates are distance and force
dependent, respectively. We use the Bell model \cite{Bell1978} to
express the bond dissociation rate as a function of force, and the
(something) model to express the bond formation rate as a function of
the distance between a receptor and ligand. This gives us the
following equations for formation rate and dissociation rate:
\begin{align}
  \label{eq:binding}
  \onRate(\length) &= \onConst \exp
                     \left(-\frac{\stiffness}{2\boltzmann\temp}
                     \length^2(x, \theta) \right), \\
  \label{eq:unbinding}
  \offRate(\length) &= \offConst \exp \left( \frac{\stiffness
                      \length(x, \theta)}{\refForce} \right).
\end{align}

In order to calculate $\horzForce$ and $\torque$ for an existing set
of bonds between the platelet and wall, we simply have to add up the
individual contributions to the force and torque from each bond. The
force and torque generated by an individual bond can be derived from
the geometry of the model along with the assumption that the force
vector $\mathbf{F}$ points in the same direction as the bond with
magnitude proportional to the bond length. I'll leave out the
derivation. The horizontal force generated by a single bond
($\horzForce$) is given simply by
\begin{equation}
  \label{eq:horz-force}
  \horzForce = \stiffness (x - \radius\sin\theta),
\end{equation}
while the torque generated by a single bond ($\torque$) is given by
the slightly more complicated
\begin{equation}
  \label{eq:torque}
  \torque = -\stiffness \radius ((\radius - \radius\cos\theta +
  \separation)\sin\theta + (\radius\sin\theta - x)\cos\theta).
\end{equation}

\section{Deterministic PDE model}
\label{sec:determ-pde-model}

If we assume that there is a continuous distribution of bonds between
the platelet and the wall, we can define the function $n(x, \theta,
t)$ which gives the density of bonds at time $t$ between points $x$
and $\theta$ on the wall and platelet, respectively. Bonds advect in
$x$ with velocity $\velocity$ and they advect in $\theta$ with
velocity $\rotation$. Bonds form at the rate given in equation
(something) and saturate in $\theta$ with a maximum density
$\receptorDensity$. Finally, bonds break at the force-dependent rate
given in equation (something). Putting all of this together gives us
the following PDE definition of $n$:
\begin{equation}
  \label{eq:bond-density}
  \frac{\partial n}{\partial t} = \rotation \frac{\partial n}{\partial
    \theta} + \velocity \frac{\partial n}{\partial x} +
  \onRate(\length) \left(\receptorDensity - \int_{-\infty}^{\infty}
    n(x, \theta, t) dx \right) - \offRate(\length) n(x, \theta, t).
\end{equation}

This equation can't yet be solved for $n$, because there are still 2
unknowns in it: $\velocity$ and $\rotation$. Recall from above that
$\velocity$ and $\rotation$ are found by balancing the fluid and bond
forces acting on the platelet (equations (something)). First, we have
to calculate the total force $\horzTotalForce$ and torque
$\totalTorque$ generated by the distribution of bonds $n$:
\begin{align}
  \label{eq:horz-total-force}
  \horzTotalForce &= \int_{-\infty}^{\infty} \int_{-\pi/2}^{\pi/2}
                    \horzForce n(x, \theta, t) d\theta dx, \\
  \label{eq:total-torque}
  \totalTorque &= \int_{-\infty}^{\infty} \int_{-\pi/2}^{\pi/2}
                 \torque n(x, \theta, t) d\theta dx.
\end{align}

Once we have $\horzForce$ and $\torque$, the platelet velocities
$\velocity$ and $\rotation$ are given by the equations:
\begin{align}
  \label{eq:force-balance}
  0 &= \velFriction (\appliedVel - \velocity) + \horzTotalForce, \\
  \label{eq:torque-balance}
  0 &= \rotFriction (\appliedRot - \rotation) + \totalTorque.
\end{align}

Now we have 3 equations in 3 unknowns and this is a closed system that
can be solved simultaneously for $n$, $\velocity$, and $\rotation$. We
still need to define the domains of all three variables, and give
boundary conditions for $n$ before we can solve this system. Assume
that bonds can only attach to the lower half-circle of the platelet
surface, while bonds can form at any point along the wall. Thus we are
solving $n$ in the domain $(-\infty, \infty) \times [-\pi/2,
\pi/2]$. We need to enforce boundary conditions at the upstream (with
respect to variables $x$ and $\theta$) ends of the $x$ and $\theta$
domains. Because we assume there are no bonds attached to the upper
half of the platelet, and because the equilibrium concentration of
bonds $n \rightarrow 0$ as $L \rightarrow \infty$, we must set
homogeneous Dirichlet boundary conditions on both $x$ and
$\theta$. Finally we just set the initial condition to be $n(x,
\theta, 0) \equiv n_0(x, \theta)$.

In order to eliminate some parameters, we nondimensionalize this model
by introducing nondimensional coordinates $x = \radius z$ and $s =
\offConst t$ (note that $\theta$ is already dimensionless) and
nondimensional functions $\radius n = \receptorDensity m$, $\velocity
= \radius \offConst \ndVelocity$, and $\rotation = \offConst
\ndRotation$. The details of the nondimensionalization are included in
Appendix (something), but the final nondimensional set of equations is
\begin{align}
  \label{eq:nd-bond-density}
  \frac{\partial m}{\partial s}
  &= \ndRotation \frac{\partial m}{\partial \theta} + \ndVelocity
    \frac{\partial m}{\partial z} + \ndOnConst \exp
    \left(-\frac{\onForceScale}{2} \ndLength^2(z, \theta) \right)
    \left(1 - \int_{-\infty}^{\infty} m(z, \theta, s) dz \right) -
    \exp \left(\offForceScale\ndLength(z, \theta)\right) m(z, \theta,
    s), \\
  \label{eq:nd-force-balance}
  0 &= \ndVelFriction \left(\ndAppliedVel - \ndVelocity\right) + 
      \ndHorzTotalForce, \\
  \label{eq:nd-torque-balance}
  0 &= \ndRotFriction \left(\ndAppliedRot - \ndRotation\right) +
      \ndTotalTorque.
\end{align}

The nondimensional bond length $\ell$ is given by $something$, and the
nondimensional force $\ndHorzForce$ and torque $\ndTorque$ are given
by $something$. A complete list of the nondimensional coordinates,
functions, and parameters with their definitions are given in Table
(something).

\section{Stochastic model}
\label{sec:stochastic-model}

