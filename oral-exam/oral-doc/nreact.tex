%%% -*-LaTeX-*-

\chapter{The Modified Next Reaction Method}
\label{cha:modif-next-react}

The fundamental premise of the Gillespie algorithm and related methods
\cite{Gillespie2007} is that each chemical reaction $k$ proceeds with
a propensity function $a_j(\mathbf{x})$ that depends on the state of
the system $\mathbf{x}$ and defined so that
\begin{equation}
  \label{eq:prop-def}
  a_k(\mathbf{x}) \dd t \equiv \tn{the probability that reaction $k$
    takes place in a small time interval $[t, t+\dd t)$}.
\end{equation}

Based on this assumption, the time $\Delta t_k$ that it takes for
reaction $k$ to fire is exponentially distributed with rate parameter
$a_k(\mathbf{x})$. As mentioned above, for each time step in the
Gillespie algorithm, two random numbers are generated. The first is
used to find the minimum firing time over all the reactions (the
minimum firing time is exponentially distributed with rate parameter
$a_0(\mathbf{x}) = \sum_{k=1}^M a_k(\mathbf{x})$), and the second is
used to find which reaction occurred (reaction $k$ occurred with
probability $a_k(\mathbf{x}) / a_0 (\mathbf{x})$).

\section{Next reaction method}
\label{sec:next-reaction-method}

In the Gillespie algorithm, two random numbers must be generated at
each time step, and the number of random numbers that must be
generated is independent of the number of possible reactions. For
chemical systems where the number of time steps is large relative to
the number of possible reactions, the next reaction method improves on
the Gillespie algorithm in that fewer random numbers are required to
advance the simulation. However the feature that is most relevant for
these notes is that the next reaction method is more easily
generalized to the case where reaction rates are explicitly dependent
on time.

For the next reaction method, we represent the reaction times of each
individual reaction as the firing times of Poisson processes. Let
$R_k(t)$ be the number of times the $k$th reaction has occurred before
time $t$. $R_k(t)$ can be represented as a Poisson process with a
nondimensional firing rate related to the reaction rate. Specifically,
if we define $k$ independent Poisson processes with unit firing rate:
$Y_k(\tau)$, then
\begin{equation}
  \label{eq:R-def}
  R_k(t) = Y_k\left(\int_0^t a_k(X(s)) ds \right).
\end{equation}

Anderson defines the \emph{internal time} of reaction $k$ as $T_k(t)$:
\begin{equation}
  \label{eq:T-def}
  T_k(t) \equiv \int_0^t a_k(X(s)) ds.
\end{equation}
The internal time is a nondimensional quantity that scales time in the
$k$th Poisson process based on the reaction rate $a_k(X(t))$.

The idea of the next reaction method is to calculate the firing times
of each Poisson process independently, and then to find the minimum
firing time. The system is updated based on which reaction fired, and
the internal times of each reaction are updated appropriately. All of
the firing times of the other reactions are saved, and a new firing
time is generated only for the reaction which fired.

More explicitly, define $P_k$ to be the first firing time of $Y_k$ (in
the time scale of $Y_k$) after $T_k(t)$:
\begin{equation}
  \label{eq:P-def}
  P_k(t) = \min\{s > T_k(t) \mid Y_k(s) > Y_k(T_k(t)) \}.
\end{equation}

Then from the definition of $T_k(t)$ the absolute time required for
reaction $k$ to fire $\Delta t_k$ is given by
\begin{equation}
  \label{eq:dtk-def}
  \int_t^{t + \Delta t_k} a_k(X(s)) ds = P_k - T_k.
\end{equation}


Assuming $X(s)$ is constant in $[t, t + \Delta t_k)$, then so is $a_k$
(as long as $a_k$ doesn't explicitly depend on time) and
$\Delta t_k = (P_k - T_k)/a_k$. This gives us the following algorithm
for the next reaction method for autonomous reaction rates:
\begin{enumerate}
\item Initialize chemical species, set $t=0$, and $T_k = 0$ for
  each $k$
\item Calculate $a_k$ for each reaction.
\item Generate $M$ independent, uniform random numbers $r_k$.
\item For each $k$, set $P_k = \log(1/r_k)$ (find the internal firing
  times of each reaction).
\item For each $k$, set $\Delta t_k = (P_k - T_k)/a_k$ (convert the
  internal firing times to absolute firing times).
\item Set $\Delta = \min_k \{\Delta t_k\}$ and
  $\mu = \operatorname{argmin}_k \{\Delta t_k\}$.
\item Set $t = t + \Delta$ and update the number of molecular species
  associated with reaction $\mu$.
\item For each $k$, set $T_k = T_k + a_k \Delta$ (advance the internal
  times).
\item For reaction $\mu$, choose a new random number $r$ and set
  $P_\mu = P_\mu + \log(1/r)$.
\item Recalculate the reaction rates $a_k$.
\item Go to 5.
\end{enumerate}

\section{Time-dependent reaction rates}
\label{sec:time-depend-react}

If we now assume that the reaction rates can change explicitly as a
function of time (that is $a_k = a_k(X(t), t)$), we can apply the next
reaction method described above to this new case in a fairly
straightforward way. $P_k$ and $T_k$ are defined in the same way as
above, the only difference is that now $T_k$ is not necessarily a
piecewise linear function of the absolute time $t$.

In the algorithm described above, only steps 5 and 8 change. In step 5
to find the absolute firing times of each reaction $\Delta t_k$, we
must now solve
\begin{equation}
  \label{eq:step-5}
  \int_t^{t + \Delta t_k} a_k(X(t), s) ds = P_k - T_k.
\end{equation}

In step 8 to advance each internal clock, we now have to set
\begin{equation}
  \label{eq:step-8}
  T_k = T_k + \int_t^{t + \Delta} a_k(X(t), s) ds.
\end{equation}

\section{The modified next reaction method applied to platelet
  rolling}
\label{sec:plt-rolling}

In the case of platelet rolling, it is clear that the reaction rates
(i.e. bond formation and breaking rates) depend on time through the
position of the platelet, and the method for simulating time-dependent
reactions can nearly be directly applied to simulate random bond
formation and breaking.

Let's look at the force-balance equations for the platelet. We have
\begin{align}
  \label{eq:vel-force-bal}
  0 &= \ndVelFriction \left(\ndAppliedVel - \ndVelocity \right) +
      \ndHorzTotalForce \\
  \label{eq:rot-force-bal}
  0 &= \ndRotFriction \left(\ndAppliedRot - \ndRotation \right) +
      \ndTotalTorque
\end{align}
where $\ndHorzTotalForce = \sum_{i=1}^\texttt{numBonds}
(\sin\recAngle_i - \ndWallDist_i )$ and $\ndTotalTorque =
-\sum_{i=1}^\texttt{numBonds} [(1 - \cos\recAngle_i +
\ndSeparation)\sin\recAngle_i + (\sin\recAngle_i -
\ndWallDist_i)\cos\recAngle_i]$.

The motion of the platelet is defined so that
$\Der{\ndWallDist_i}{\dTime} = -\ndVelocity$ and
$\Der{\recAngle_i}{\dTime} = -\ndRotation$ for all $i = 1, \hdots
\texttt{numBonds}$. These differential equations have the solution
$\ndWallDist_i(\dTime) = -\ndVelocity \dTime + \ndWallDist_i^0$ and
$\recAngle_i(\dTime) = -\ndRotation \dTime + \recAngle_i^0$. The
time-dependent piece of these equations is independent of $i$, and so
we can write $\ndWallDist_i(\dTime)$ and $\recAngle_i(\dTime)$ in
terms of an initial value (that varies only over $i$) plus a reference
point that varies only over $\dTime$:
\begin{align}
  \ndWallDist_i(t) &= \ndWallDist_\tn{ref}(\dTime) + \ndWallDist_i^0
                     \quad \tn{where} \quad
                     \ndWallDist_\tn{ref}(\dTime) \equiv
                     -\ndVelocity\dTime \\
  \recAngle_i(t) &= \recAngle_\tn{ref}(\dTime) + \recAngle_i^0 \quad
                   \tn{where} \quad \recAngle_\tn{ref}(\dTime) \equiv
                   -\ndRotation \dTime.
\end{align}

From the definitions of $\ndWallDist_\tn{ref}$ and
$\recAngle_\tn{ref}$, we can rewrite equations
\eqref{eq:vel-force-bal} and \eqref{eq:rot-force-bal} as a set of
ODEs:
\begin{align}
  \label{eq:wall-evol}
  \Der{\ndWallDist_\tn{ref}}{\dTime}
  &= -\ndAppliedVel + F_h'\left(\ndWallDist_\tn{ref},
    \recAngle_\tn{ref} \right)/\ndVelFriction \\
  \label{eq:plt-evol}
  \Der{\recAngle_\tn{ref}}{\dTime}
  &= -\ndAppliedRot + \tau'\left(\ndWallDist_\tn{ref},
    \recAngle_\tn{ref} \right)/\ndRotFriction.
\end{align}

Assuming that no bonds form or break in the time interval
$[t_1, t_2]$, integrating equations \eqref{eq:wall-evol} and
\eqref{eq:plt-evol} gives the position of the platelet at the times in
that interval. Knowing the position and orientation of the platelet at
all times $\dTime$ in some interval also means that the bond formation
and breaking rates are also known for all times in that
interval. Therefore equations \eqref{eq:wall-evol} and
\eqref{eq:plt-evol} give us a way to evaluate the integral in
equations \eqref{eq:step-5} and \eqref{eq:step-8}.

One consideration we have to make is that we don't know \emph{a
  priori} how long to integrate the above system to guarantee that a
reaction fires in the time interval of integration. My solution to
this is to augment the system with equations of the form
$\Der{F_k}{\dTime} = a_k(\ndWallDist_\tn{ref}, \recAngle_\tn{ref})$
where the $a_k$s are bond formation/breaking rates. That is, to
integrate the reaction rates along with equations of motion, and
then integrate the augmented system until one of the reactions fires.

I propose the following stochastic simulation algorithm for platelet
rolling with the next reaction method:
\begin{enumerate}
\item Initialize platelet velocities $\ndVelocity$ and $\ndRotation$,
  and $\texttt{bondList}$. Set $\dTime = 0$ and $T_k = 0$ for each of the
  $2N$ bins on the platelet surface, and any bonds that exist in the
  initial condition.
\item Generate $2N + \texttt{numBonds}$ independent, uniform random
  numbers $r_k$.
\item Set $P_k = \log(1/r_k)$ for each $k$ (i.e. set the internal
  firing times for each reaction).
\item Define $\dTime_0$ to be the current time $\dTime$,
  $\ndWallDist_i^0$ to be the current $\ndWallDist$ coordinates of the
  existing bonds, and $\recAngle_i^0$ to be the current $\recAngle$
  coordinates of the existing bonds.
\item Set up the ODE system
  \begin{align}
    \Der{\ndWallDist_\tn{ref}}{\dTime}
    &= -\ndAppliedVel + F_h'\left(\ndWallDist_\tn{ref},
      \recAngle_\tn{ref} \right)/\ndVelFriction \\
    \Der{\recAngle_\tn{ref}}{\dTime}
    &= -\ndAppliedRot + \tau'\left(\ndWallDist_\tn{ref},
      \recAngle_\tn{ref} \right)/\ndRotFriction \\
    \Der{F_k}{\dTime} &= a_k\left(\ndWallDist_\tn{ref},
    \recAngle_\tn{ref} \right).
  \end{align}
  with initial conditions $\ndWallDist_\tn{ref}(\dTime_0) =
  \recAngle_\tn{ref}(\dTime_0) = 0$ and $F_k(\dTime_0) = T_k - P_k$. 
\item Integrate the ODE system until:
  \begin{enumerate}
  \item The end of the stochastic simulation $T_\tn{end}$ is reached,
  \item One of the $F_k$s reaches 0 (i.e. a reaction fires), or
  \item One of the bond endpoints exits the numerical domain.
  \end{enumerate}
\item If one of the bond endpoints exited the numerical domain, break
  the appropriate bond, update the bond positions and bin positions on
  the platelet, and return to step 4.
\item If the $k$th reaction fired, either add or remove a bond as
  needed, update the bond positions and bin positions, and return to
  step 4.
\item If $T_\tn{end}$ was reached, end the simulation. 
\end{enumerate}


% Local Variables:
% TeX-master: "oral-document"
% End:
