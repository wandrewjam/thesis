%%% -*-LaTeX-*-

\chapter{Numerical Schemes}
\label{cha:numerical-schemes}

\section{Deterministic model}
\label{sec:deterministic-model}

The PDE \eqref{eq:nd-master} is a linear advection-reaction equation
for known $\ndRotation$ and $\ndVelocity$, but the whole system including the force
balance equations (i.e. \eqref{eq:nd-master}, \eqref{eq:nd-vel}, and
\eqref{eq:nd-angvel}) is nonlinear. Numerically, I linearized the
system by lagging the angular and linear velocities in order to find
the bond density distribution at the next time step. Then I integrated
over the bond distribution to find the net force and torque generated
by the bonds, and used that to update the angular and linear
velocities. As suggested by the results above, I will need to modify
this approach to handle the case where a platelet is initially firmly
adhered and unmoving, until fluid forces are applied.

I solve the PDE using a first order upwind scheme in both $\recAngle$ and
$\ndWallDist$. These velocities are always non-negative, so I approximate the
spatial derivatives with a forward difference. The bond formation term
is treated explicitly, so that I can advance time with just a matrix
multiplication, instead of solving a linear system. The bond breaking
term essentially \emph{must} be treated implicitly, because the
breaking rate is very large over much of the domain $(\ndWallDist,
\recAngle)$. Luckily it only requires a scalar division for each
element of $\ndBondDensity^k$ to treat that term implicitly.

In summary, for each timestep $k$, I do the following:
\begin{enumerate}
\item Advance the PDE in time using the velocities from the previous
  timestep:
  \begin{multline}
    \label{eq:pde-timestep}
    \ndBondDensity_{i,j}^{k+1} = \frac{1}{1 + \Delta \dTime
      \exp(\offForceScale \ndLength_{ij})} \left(\ndBondDensity_{ij}^k +
      \ndRotation^k \Delta \dTime \left(\frac{\ndBondDensity_{i, j+1}^k -
          \ndBondDensity_{ij}^k}{\nu}\right) + \ndVelocity^k \Delta \dTime
      \left(\frac{\ndBondDensity_{i+1, j}^k -
          \ndBondDensity_{ij}^k}{h}\right)\right. \\ 
    \left. + \kappa \Delta \dTime \exp\left(-\eta
        \frac{\ndLength_{ij}}{2}\right) \left(1 - h \sum_{q = 0}^{N-1}
        \ndBondDensity_{qj}^k\right)\right) 
  \end{multline}
\item Calculate $\ndHorzTotalForce_{k+1}$ and $\ndTotalTorque_{k+1}$ from the new bond
  distribution $\ndBondDensity^{k+1}$.
\item Find the new angular and linear velocities:
  \begin{align}
    \label{eq:ang-timestep}
    \ndRotation^{k+1} &= \ndAppliedRot + \ndTotalTorque_{k+1}/\ndRotFriction \\
    \label{eq:vel-timestep}
    \ndVelocity^{k+1} &= \ndAppliedVel + \ndHorzTotalForce_{k+1}/\ndVelFriction.
  \end{align}
\item Repeat
\end{enumerate}

\section{Stochastic Model}
\label{sec:stochastic-model}

For the stochastic model, assume that there are $\receptorDensity$ receptors per
radian, and the receptors are uniformly distributed along the surface
of the platelet. Assume there is an excess of ligand on the wall
surface, so that bonds can form anywhere along the wall and there is
no saturation of bonds in the $\wallDist$ dimension.

A bond can form between a receptor located at $\recAngle$ and a point $\wallDist$
on the wall at ``rate'' $\onRate(\wallDist, \recAngle)$. More precisely, the time
$\tau$ that it takes a bond to form between $\wallDist$ and $\recAngle$ is
exponentially distributed with mean $\onRate^{-1}$. Similarly, the time
it takes a bond between $\recAngle$ and $\wallDist$ to break is exponentially
distributed with mean $1/\offRate(\wallDist, \recAngle)$. Finally, a single bond
between $\wallDist$ and $\recAngle$ generates horizontal force and torque which
are given by $f_h$ and $\tau_s$ as defined in section
\ref{sec:reaction-rate-equation}. As in the continuous model, assume
that only bonds on the lower half-circle of the platelet can interact
with the wall surface.

One way to simulate this model would be to track each of the $2\pi
\receptorDensity$ receptors on the surface of the platelet, whether they are bound
or unbound. But this seems expensive to do, so I did something else. I
partitioned the interval $[-\pi/2, \pi/2)$ into $N$ subintervals of
equal length $I_j = \left[-\frac{\pi}{2} + \frac{\pi}{N}j,
  -\frac{\pi}{2} + \frac{\pi}{N} (j+1)\right)$. Then there can be at
most $b_\tn{max} \equiv \receptorDensity \pi/N$ bonds in any subinterval. I still
keep track of the exact $\wallDist$ and $\recAngle$ positions the bond forms
between, the subintervals are only used to limit the bond
concentration at $\receptorDensity$ per radian. 

To simulate random bond breaking, a random number is generated for
each existing bond to test if it breaks within the time step $\Delta
\dTime$. The probability that a bond between $\wallDist$ and $\recAngle$ in the time
interval $[\dTime, \dTime + \Delta \dTime)$ is $\mathbb{P} = 1 - \exp(-\Delta \dTime
\offRate(\wallDist, \recAngle))$.

Next look at bond formation. The probability of a bond forming between
points $\recAngle$ and $\wallDist$ can be split into two distinct probabilities:
the probability that a bond forms from a receptor at position $\recAngle$
to anywhere on the wall (denote this event as $A$, and the associated
probability as $\mathbb{P}(A)$), and the probability that a bond forms
to the point $\wallDist$ on the wall from any receptor (denote this event as
$B$, and the probability $\mathbb{P}(B)$). The probability of the
combined event $A \cap B$ is $\mathbb{P}(A \cap B) = dt
\onRate(\wallDist, \recAngle)$. 

% To simulate bond formation, first assume that at most 1 bond can
% form in a single time step in each subinterval $I_j$. One can think of
% bond formation 

% Then for each
% subinterval, calculate the probality of a bond forming within that
% subinterval in the time interval $\Delta \dTime$. If a bond forms then an
% $\wallDist$ is chosen at random.

First, the probability that a bond forms within subinterval $I_j$ is
given by the rate of bond formation multiplied by the number of
available receptors in $I_j$. The rate that bonds form from $\recAngle$
to \emph{any} point on the wall, integrate $\onRate$ with respect to
$\wallDist$. In the case $\onRate(\wallDist, \recAngle) = k_\tn{on} \exp\left(-\frac{\stiffness
    \length^2}{2\boltzmann \temp}\right)$, we have
\begin{equation}
  \label{eq:total_fm_rate}
  \int_{-\infty}^{\infty} \onRate(\wallDist, \recAngle)\, dx = k_\tn{on} \exp
  \left( -\frac{\stiffness}{2\boltzmann \temp} (\radius - \radius \cos\recAngle + \separation)^2 \right)
  \sqrt{\frac{2 \pi \boltzmann \temp}{\stiffness}}.
\end{equation}
Therefore the probability that a bond forms somewhere in the interval
$I_j$ to any point on the wall in a time interval of length $dt$ is
given by 
\begin{equation}
  \label{eq:1}
  \mathbb{P} = dt (b_\tn{max} - \bondDensity_j) \left[ k_\on \exp \left( -
      \frac{\stiffness}{2 \boltzmann \temp} \left( \radius - \radius\cos\recAngle + \separation \right)^2 \right)
    \sqrt{\frac{2\pi \boltzmann \temp}{\stiffness}} \right].
\end{equation}

Next, we need to find the probability that a bond forms between a
point $\wallDist$ on the wall and $\recAngle$, given that a bond forms from
$\recAngle$. From the definition 

\begin{equation}
  \label{eq:2}
  \exp \left(-\frac{\stiffness}{2 \boltzmann \temp}
    \left(\wallDist - \radius \sin{\recAngle} \right) \right)
\end{equation}

\subsection{Algorithm}
\label{sec:algorithm}

\begin{enumerate}
\item Discretize the interval $[-\pi/2, \pi/2)$ into $N$ equal bins.
\item Initialize list of bonds and angular and linear velocities. 
\item For each time step up to $\dTime_\tn{max}$, \label{item:stoch-iter}
  \begin{enumerate}
  \item Update the $\ndWallDist$ and $\recAngle$ values for every bond. (For the
    variable time step algorithm, this happens after the formation and
    breaking rates are calculated, but before the bond list is updated)
  \item Bin each bond in one of the $N$ $\recAngle$-bins. Any bond for
    which $\recAngle < -\pi/2$ or $\recAngle > \pi/2$ is flagged to be
    removed from the list.
  \item For each $\recAngle$-bin, compute the rate of bond formation
    within that bin from $\onRate(\recAngle_j) (b_\tn{max} - \bondDensity_j)$ where
    $\bondDensity_i$ is the number of existing bonds with endpoints in interval
    $I_j$.
  \item For each existing bond, compute the rate of bond breaking.
  \item Update the bond list:
    \begin{itemize}
    \item If using the fixed time step algorithm, generate a random
      number for each existing bond to decide which ones break, and
      randomly sample from N Poisson distributions with $\lambda = dt
      \onRate(\recAngle_j) (b_\tn{max} - \bondDensity_j)$, $j = 1, \hdots, N$ to
      decide how many bonds form in each interval.
    \item If using the variable time step algorithm, generate a random
      number to decide when the first reaction (a bond breaking or
      forming) occurs, and generate a second random number to decide
      which reaction occurs.
    \end{itemize}
  \item Calculate the forces and torques generated by the existing
    bonds, and solve the force balance equations to calculate the new
    $(\ndVelocity, \ndRotation)$ pair.
  \item Return to step \ref{item:stoch-iter}.
  \end{enumerate}
\end{enumerate}
