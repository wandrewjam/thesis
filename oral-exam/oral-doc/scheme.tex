%%% -*-LaTeX-*-

\chapter{Numerical Schemes}
\label{cha:numerical-schemes}

\section{Deterministic model}
\label{sec:deterministic-model}

The PDE \eqref{eq:nd-bond-density} is a linear advection-reaction
equation for known $\ndRotation$ and $\ndVelocity$, but the whole
system including the force balance equations
(i.e. \eqref{eq:nd-bond-density}, \eqref{eq:nd-force-balance}, and
\eqref{eq:nd-torque-balance}) is nonlinear. Numerically, I linearized
the system by lagging the angular and linear velocities in order to
find the bond density distribution at the next time step. Then I
integrated over the bond distribution to find the net force and torque
generated by the bonds, and used that to update the angular and linear
velocities. As suggested by the results above, I will need to modify
this approach to handle the case where a platelet is initially firmly
adhered and unmoving, until fluid forces are applied.
% Discuss the issues with upwinding?

I solve the PDE using a second order Beam-Warming scheme in both
$\recAngle$ and $\ndWallDist$. The velocities $\ndVelocity$ and
$\ndRotation$ are always non-negative, so I approximate the spatial
derivatives with a forward difference. Equation
(\ref{eq:forward-difference}) below gives the forward difference in
$\ndWallDist$; the forward difference in $\recAngle$ is analogous.
\begin{equation}
  \label{eq:forward-difference}
  \Pder{m}{\ndWallDist}(\ndWallDist_i, \recAngle_j) \approx
  \frac{-3m(\ndWallDist_i, \recAngle_j) + 4m(\ndWallDist_{i+1},
    \recAngle_j) - m(\ndWallDist_{i+2},
    \recAngle_j)}{2(\ndWallDist_{i+1} - \ndWallDist_i)}.
\end{equation}

The bond formation term is treated explicitly, so that I can advance
time with just a matrix multiplication, instead of solving a linear
system. The bond breaking term essentially \emph{must} be treated
implicitly, because the breaking rate is very large over much of the
domain $(\ndWallDist, \recAngle)$. Luckily it only requires a scalar
division for each element of $\ndBondDensity^k$ to treat that term
implicitly.

In summary, for each timestep $k$, I do the following:
\begin{enumerate}
\item Advance the PDE in time using the velocities from the previous
  timestep:
  \begin{multline}
    \label{eq:pde-timestep}
    \ndBondDensity_{i,j}^{k+1} = \frac{1}{1 + \Delta \dTime
      \exp(\offForceScale \ndLength_{ij})} \left(\ndBondDensity_{ij}^k
      + \ndRotation^k \Delta \dTime
      \left(\frac{-3\ndBondDensity_{ij}^k + 4\ndBondDensity_{i, j+1}^k
          - \ndBondDensity_{i, j+2}^k}{2\nu}\right) + \right. \\
    \left. \ndVelocity^k
      \Delta \dTime \left(\frac{-3\ndBondDensity_{ij}^k +
          4\ndBondDensity_{i+1, j}^k - \ndBondDensity_{i+2,
            j}^k}{2h}\right) + \kappa \Delta \dTime \exp\left(-\eta
        \frac{\ndLength_{ij}}{2}\right) \left(1 - h \sum_{q = 0}^{N-1}
        \ndBondDensity_{qj}^k\right)\right)
  \end{multline}
\item Calculate $\ndHorzTotalForce^{k+1}$ and $\ndTotalTorque^{k+1}$
  from the new bond distribution $\ndBondDensity^{k+1}$ using the
  trapezoid rule:
  \begin{align}
    \label{eq:num-force-calc}
    \ndHorzTotalForce^{k+1} &= (h \nu)^{-1} \sum_{i=0}^{2M} w_i
                              \sum_{j=0}^{2N} w_j
                              \ndBondDensity_{ij}^{k+1}
                              \left(\ndWallDist_i -
                              \sin\recAngle_j\right) \\
    \ndTotalTorque^{k+1} &= (h \nu)^{-1} \sum_{i=0} ^{2M} w_i
                           \sum_{j=0}^{2N} w_j
                           \ndBondDensity_{ij}^{k+1} \left[ \left(1 -
                           \cos\recAngle_j + \ndSeparation \right)
                           \sin\recAngle_j + \left(\sin\recAngle_j -
                           \ndWallDist_i \right) \right].
  \end{align}
  Here the $w_i$s are the weights in the trapezoid rule: $\{w_i\} =
  \{1/2, 1, \hdots, 1, 1/2\}$.
\item Find the new angular and linear velocities:
  \begin{align}
    \label{eq:ang-timestep}
    \ndRotation^{k+1} &= \ndAppliedRot + \ndTotalTorque^{k+1}/\ndRotFriction \\
    \label{eq:vel-timestep}
    \ndVelocity^{k+1} &= \ndAppliedVel + \ndHorzTotalForce^{k+1}/\ndVelFriction.
  \end{align}
\item Repeat
\end{enumerate}

\section{Stochastic Model}
\label{sec:stochastic-model}

\begin{enumerate}
\item Discretize the interval $[-\pi/2, \pi/2)$ into $N$ equal bins.
\item Initialize list of bonds and angular and linear velocities. 
\item For each time step up to $\dTime_\tn{max}$, \label{item:stoch-iter}
  \begin{enumerate}
  \item Update the $\ndWallDist$ and $\recAngle$ values for every bond. (For the
    variable time step algorithm, this happens after the formation and
    breaking rates are calculated, but before the bond list is updated)
  \item Bin each bond in one of the $N$ $\recAngle$-bins. Any bond for
    which $\recAngle < -\pi/2$ or $\recAngle > \pi/2$ is flagged to be
    removed from the list.
  \item For each $\recAngle$-bin, compute the rate of bond formation
    within that bin from $\onRate(\recAngle_j) (b_\tn{max} - \bondDensity_j)$ where
    $\bondDensity_i$ is the number of existing bonds with endpoints in interval
    $I_j$.
  \item For each existing bond, compute the rate of bond breaking.
  \item Update the bond list:
    \begin{itemize}
    \item If using the fixed time step algorithm, generate a random
      number for each existing bond to decide which ones break, and
      randomly sample from N Poisson distributions with $\lambda = dt
      \onRate(\recAngle_j) (b_\tn{max} - \bondDensity_j)$, $j = 1, \hdots, N$ to
      decide how many bonds form in each interval.
    \item If using the variable time step algorithm, generate a random
      number to decide when the first reaction (a bond breaking or
      forming) occurs, and generate a second random number to decide
      which reaction occurs.
    \end{itemize}
  \item Calculate the forces and torques generated by the existing
    bonds, and solve the force balance equations to calculate the new
    $(\ndVelocity, \ndRotation)$ pair.
  \item Return to step \ref{item:stoch-iter}.
  \end{enumerate}
\end{enumerate}
