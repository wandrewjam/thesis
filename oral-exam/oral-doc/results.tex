%%% -*-LaTeX-*-

\chapter{Results}
\label{cha:results}

\section{Hysteresis in rolling velocities}
\label{sec:hyst-roll-veloc}

The steady solution (of the nondimensional system) is given by solving
equations \eqref{eq:nd-bond-density}--\eqref{eq:nd-torque-balance}
with $\Pder{\ndBondDensity}{\ndTime} \equiv 0$ in
\eqref{eq:nd-bond-density}. This can be viewed as a 2D nonlinear
rootfinding problem
\begin{align}
  \label{eq:2D-ss-translation}
  0 &= \ndVelFriction (\ndAppliedVel - \ndVelocity) +
      \ndHorzTotalForce(\ndVelocity, \ndRotation) \\
  \label{eq:2D-ss-rotation}
  0 &= \ndRotFriction (\ndAppliedRot - \ndRotation) +
      \ndTotalTorque(\ndVelocity, \ndRotation)
\end{align}
where evaluating the functions $\ndTotalTorque$ and
$\ndHorzTotalForce$ requires solving the steady state PDE for
$\ndBondDensity$ and then integrating the result.

\subsection{Reduced model: rolling-only}
\label{sec:reduced-model}

One use of the deterministic model is to search for regions of
parameter space in which a small number of bonds mediate platelet
rolling. We expect that if a small number of bonds are responsible for
the rolling behavior of the platelet, randomness in bond chemistry
will play a significant role in platelet motion. Based on an estimate
of $\sim 10^3$ receptors/radian, we want to find regions of parameter
space where $m$ is between $10^{-3}$ to $10^{-2}$.

If we assume that a cell is sitting directly on top of the suface
(i.e. $\ndSeparation = 0$), and the cell is only rolling, but not
sliding
($\velocity = \radius\rotation \implies \ndVelocity = \ndRotation$),
then solving the steady state problem reduces to a 1D rootfinding
problem. One can solve equation \eqref{eq:2D-ss-rotation} to find the
steady state value(s) of $\ndRotation$ for a given $\ndAppliedRot$,
and then use $\ndVelocity = \ndRotation$ to find $\ndVelocity$. (Note:
it is also possible to use equation \eqref{eq:2D-ss-translation} to
find $\ndVelocity$ for a given $\ndAppliedVel$, and in general this
will give a different value of $\ndVelocity$ than if you solved for
$\ndRotation$ first and then set $\ndVelocity = \ndRotation$.)

\subsubsection{Rolling results}
\label{sec:rolling-results}

\begin{figure}
  \centering
  \begin{subfigure}{.48\textwidth}

    \caption{Torque $\ndTotalTorque$ vs ND angular velocity
      $\ndRotation$ at steady state}
    \label{fig:tau-vs-omega}
  \end{subfigure}
  \hfill
  \begin{subfigure}{.48\textwidth}

    \caption{Angular velocity $\ndRotation$ vs Applied angular
      velocity $\ndAppliedRot$}
    \label{fig:omega-vs-omega_f}
  \end{subfigure}
  \caption{Relationships between torque, angular velocity, and applied
    angular velocity. ND parameter values are $\ndOnConst = 1$,
    $\onForceScale = 0.1$, $\delta = 3$, $\ndRotFriction = 1/7500$.}
  \label{fig:steady-state-rolling}
\end{figure}

In order to find steady state values of $\ndRotation$ for a range of
$\ndAppliedRot$s, we generate a bunch of
$(\ndAppliedRot, \ndRotation)$ pairs that solve the steady state
problem. Then we can use interpolation to find a steady state angular
velocity $\ndRotation$ for an unknown $\ndAppliedRot$.

We first choose a set of $\ndRotation$ values and then find the steady
state bond distribution for those $\ndRotation$ using
\eqref{eq:nd-bond-density}$= 0$. Then we find the torque generated by that
distribution of bonds (shown in Figure \ref{fig:tau-vs-omega}), and
find the necessary $\ndAppliedRot$ using equation
\eqref{eq:nd-torque-balance}. This gives us the necessary set of
$(\ndAppliedRot, \ndRotation)$ pairs for interpolation. Figure
\ref{fig:omega-vs-omega_f} shows the result of linear interpolation
between these points. That is, it shows steady state angular
velocities for a range of applied angular velocities.

Qualitatively, it is clear that there is a range of parameters for
which multiple steady state values of $\ndRotation$ are possible for a
single $\ndAppliedRot$. The biological interpretation of Figure
\ref{fig:omega-vs-omega_f} is that for low applied angular velocity
(i.e. low shear rate) the only stable behavior is for platelets to
adhere to the surface and move very slowly with respect to the fluid
velocity. At some medium shear rate, a second stable behavior arises
where platelets roll along the surface substantially faster than the
adhered platelets. Finally, at a high shear rate, all platelets are
moving at a speed close to the fluid velocity and none are adhered to
the surface: the torque generated by the fluid on the platelet is too
large. It is worth noting that the analysis in Figure
\ref{fig:steady-state-rolling} doesn't show which solution branches
are stable and which are unstable, so more work is required to show
that the story I told above is consistent with the model.

\begin{figure}
  \centering

  \caption{Angular velocity $\ndRotation$ vs Applied angular velocity
    $\ndAppliedRot$ for $\ndRotFriction = 1/10\, 000$. All other
    parameters identical to those in Figure
    \ref{fig:steady-state-rolling}}
  \label{fig:rolling-large-Nt}
\end{figure}

Two parameters that could change with priming are $\receptorDensity$
and $\onConst$. As discussed in Appendix \ref{app:nondim},
$\ndRotFriction$ and $\ndVelFriction$ are inversely proportional to
$\receptorDensity$ (also in Table \ref{tab:nd-vars}). In the rolling
only case, $\ndRotFriction$ is the only parameter of these two that
matters, and it is easy to see the effect of changing its
value. Rearranging equation \eqref{eq:nd-torque-balance}, we get the
relation
$\ndAppliedRot = \ndRotation -
\ndTotalTorque/\ndRotFriction$. Increasing $\receptorDensity$ gives a
proportional decrease in $\ndRotFriction$, which in turn increases the
magnitude of the term $\ndTotalTorque/\ndRotFriction$ in the equation
above. This lengthens the interval in $\ndAppliedRot$ in which
bistability occurs, and the interval is centered on higher
$\ndAppliedRot$ as $\ndRotFriction$ decreases. For example, in Figure
\ref{fig:omega-vs-omega_f} the system is bistable for $\ndAppliedRot$
roughly in the interval 150--190, but in Figure
\ref{fig:rolling-large-Nt} the system is bistable for $\ndAppliedRot$
in the interval 175--250. This agrees with intuition that
activation/priming should facilitate rolling at higher shear rates.

\begin{figure}
  \centering
  \begin{subfigure}{0.45\textwidth}

    \caption{Torque $\tau$ vs ND angular velocity $\ndRotation$ at
      steady state}
    \label{fig:tau-vs-omega-kappa-large}
  \end{subfigure}
  \quad
  \begin{subfigure}{0.45\textwidth}

    \caption{Angular velocity $\ndRotation$ vs Applied angular
      velocity $\ndAppliedRot$}
    \label{fig:omega-vs-omega_f-kappa-large}
  \end{subfigure}
  \caption{Relationships between torque, angular velocity, and applied
    angular velocity. $\ndOnConst = 3$, and all other parameters
    identical to Figure 1}
  \label{fig:rolling-large-kap}
\end{figure}

We can also look at what happens when we vary $\onConst$, by changing
$\ndOnConst$ in the ND system. With priming we'd expect on rates to
increase, yielding an increase in $\ndOnConst$. This should result in
a higher density of bonds at steady state, generating larger torques
at the same rotation rate $\ndRotation$. As shown in Figure
\ref{fig:tau-vs-omega-kappa-large}, we do get larger torque
magnitudes, and the rotation rate $\ndRotation$ at which
$|\ndTotalTorque|$ is maximized is larger than in Figure
\ref{fig:tau-vs-omega}. In Figure
\ref{fig:omega-vs-omega_f-kappa-large} the region of bistability
exists at larger $\ndAppliedRot$, but the size of this region seems to
be smaller than it is in either Figures \ref{fig:omega-vs-omega_f} or
\ref{fig:rolling-large-Nt}.

The unstressed bond breaking rate $\offConst$ could also change with
priming, but the current choice of nondimensionalization is not
convenient for examining changes in this parameter. As shown in
Appendix \ref{app:nondim}, the nondimensional time variable $\ndTime$
is scaled based on $\offConst$ and so many of the nondimensional
parameters defined in Table \ref{tab:nd-vars} depend on $\offConst$. 

\section{Bound and unbound experiments}
\label{sec:bound-unbo-exper}

\section{Parameter sweeps}
\label{sec:parameter-sweeps}

