%%% -*-LaTeX-*-

\chapter{Future Work}
\label{cha:future-work}

There are many areas where this model can be expanded, where
assumptions can be relaxed or more biological realism
incorporated. Perhaps the most obvious next step is to use the
stochastic method to fit parameters to experimental velocity
distributions obtained by Dr. Hlady's group. Some extensions I would like
to make to the model include modeling multiple receptors,
incorporating some intracellular signaling dynamics, and 3D modeling
of a nonspherical platelet. 

\section{Fit stochastic parameters to experimental data}
\label{sec:fit-stoch-param}

One advantage of the stochastic algorithm is that it is able to
model the full velocity distribution of a population of platelets all
interacting with an agonist-coated surface. Experimental observations
show that platelets go through a wide range of rolling velocities in
the course of an experiment, indicating that stochasticity is
important, and it also highlights a deficiency of the deterministic
model in that it can only reproduce the mean rolling velocity of
platelets.

Dr. Hlady's group has characterized platelet rolling on an
agonist-coated surface in several different ways that provide
different checks on the stochastic algorithm. The different statistics
that are calculated from platelet rolling may not be independent
of one another, but nonetheless the stochastic algorithm should be
able to capture the same behaviors. The rolling data collected are:
\begin{itemize}
\item platelet trajectories (i.e. displacement of individual platelets
  with respect to time),
\item platelet pause/dwell times (i.e. the length of time a platelet
  spends stationary or almost stationary on the surface before
  resuming rolling),
\item platelet instantaneous and time-averaged velocities, and
\item platelet step sizes (i.e. the distance a platelet travels in
  between pause events).
\end{itemize}

All of these features have analogs in the stochastic model which can
be tracked and recorded to compare with experimental data. In the
deterministic model, there are no pauses and therefore it produces
nothing that can be compared to experimental measurements of pause
times or step sizes, in addition to the deficiency mentioned before
that it cannot produce a distribution of platelet velocities.

What parameters can be fit to this experimental data? Many of the
parameters in this model have been estimated before in other research
on platelet rolling, both in modeling work \cite{Fitzgibbon2014,
  Mody2008b, Wang2013} and in experimental papers \cite{Litvinov2011,
  Litvinov2012}. However the confidence in these parameter estimates
varies depending on the parameter. For example, the bond formation
rate is difficult to estimate precisely and can only be estimated to
within several orders of magnitude (SOURCE). In addition, some of
these parameters with platelet priming. For example, it is known that
more adhesion receptors are recruited to the surface during platelet
activation (SOURCE). It is also known that integrins \ITA{IIb}\ITB{3}
and \ITA{2}\ITB{1} are activated as a result of platelet activation
\cite{Kee2015}, which may change any or all of $\onConst$, $\offConst$,
$\refForce$, $\stiffness$, or $\compliance$. By modifying these
parameters, we can crudely model platelet activation in the model
presented above.

\section{Multiple receptors}
\label{sec:multiple-receptors}

One fairly obvious extension of the model is to model 2 different
receptors with different parameters. For platelet agonists vWF and
collagen, there are two receptors which can bind with each of these
receptors: GP1b and \ITA{IIb}\ITB{3} for vWF, and GPIV and
\ITA{2}\ITB{1} for collagen. GP1b and GPIV are fast receptors with
relatively large binding and unbinding rates, and these receptors are
constitutively present and active on the surfaces of inactive
platelets and do not change their activity based on the platelet's
activation state.

On the other hand, the integrin receptors \ITA{IIb}\ITB{3} and
\ITA{2}\ITB{1} have smaller binding and unbinding rates, and therefore
mediate firm adhesion. In addition, while they are constitutively
expressed on the surface of unactivated platelets, these receptors are
in their low-affinity conformation until the platelet is activated and
intracellular signaling pathways signal the integrins to switch to
their high-affinity conformation.

The idea to perform rolling simulations with two different receptors
with different binding kinetics is not a new one. This was done for
rolling leukocytes in Bhatia et. al. \cite{Bhatia2003}, for
example. To my knowledge this analysis has not been done for platelets
yet, though I would not expect qualitatively different results from
theirs which show that the two receptors act synergistically to
mediate firm adhesion, whereas either receptor acting independently
can only mediate cell rolling. However this is an important step
towards a model that includes intracellular signaling dynamics, and
simulating platelets that are able to change their chemistry in order
to transition from rolling to firm adhesion. 

\section{Intracellular signaling}
\label{sec:intr-sign}

Ultimately, a model that attempts to describe both platelet rolling
\emph{and} activation must include some intracellular signaling. As
described in section \ref{sec:overview-clotting}, platelet activation
is a complex process involving dozens of chemical intermediates which
mediate a diverse suite of responses.

There has been a large amount of modeling work describing different
pathways within the platelet activation network, although most of them
include \Ca dynamics as a central point of signal integration. A
research group at Moscow State University have published a set of
models describing intracellular platelet activation as a response to
thrombin stimulation \cite{Shakhidzhanov2015, Sveshnikova2015,
  Balabin2016, Sveshnikova2016}. Other models attempt to describe
intracellular \Ca dynamics within a resting platelet \cite{Purvis2008,
  Dolan2014}. Instead of modeling the activation responses downstream
of the intracellular \Ca response, all of these models prescribe
cytosolic \Ca concentration as an output and fit parameters to
experimental data of platelet cytosol \Ca concentrations. One model
that includes pathways downstream of elevated cytosolic \Ca
(specifically \ITA{IIb}\ITB{3} activation and dense granule release)
is the model of Lenoci et. al. \cite{Lenoci2011}. They were able to
accurately predict a number of platelet responses (cytosol \Ca level,
$\tn{PIP}_2$ concentration, ATP concentration, and activation of
\ITA{IIb}\ITA{3}) after stimulation of PAR1, a platelet receptor for
thrombin.

All of these models are developed with activation by a soluble agonist
in mind. In particular, the activation signal recieved is assumed to
be both temporally and spacially constant. However in the case of
platelet rolling, the platelet transiently contacts the surface and
the activation signal is transmitted through GP1b and GPVI receptors
which are randomly bound to the surface. Therefore in order to model
platelet activation by rolling on an immobilized surface, 


