%%% -*-LaTeX-*-

\chapter{Introduction}

Some text in between the chapter header and the first section.

\section{Description of the priming project}

% In \figref{fig1}, we have a picture, and the \LaTeX{} markup
% to include it looks like this:
% %
% \begin{verbatim}
% \begin{figure}[t]
%     \centerline{\includegraphics{fig1}}
%     \caption{The first figure.}%
%     \figlabel{fig1}
% \end{figure}
% \end{verbatim}
% %
% We intentionally omitted an extension on the filename, so that this
% document can be processed with \verb=latex= to get an output
% \verb=.dvi= file, or with \verb=pdflatex= to get an output \verb=.pdf=
% file.  The first case uses the file \verb=fig1.eps=, and the second
% uses \verb=fig1.pdf=.  The \verb=distill= or \verb=ps2pdf= commands
% can be used to convert from \emph{Encapulated PostScript}%
% files to \emph{Portable Document Format}%
% files.

% \begin{figure}[t]
%     \centerline{\includegraphics{fig1}}
%     \caption{The first figure.}%
%     \figlabel{fig1}
%   \end{figure}

Some more text here to introduce.

\subsection{Why is hemocompatibility important?}
\label{sec:why-hemoc-import}

Cardiovascular disease is one of the leading causes of death in the
U.S, and a common treatment for these diseases is to implant medical
devices into the blood stream. For example, stents (an expandable
solid mesh) are often used to treat stenotic arteries. However, the
introduction of a foreign material into the blood stream will cause
thrombosis unless the material is treated somehow, and while these
devices have been effective in saving lives, they are still far from
perfect. Patients with these implanted devices still must be placed on
anticoagulants, as they have a higher risk of a thrombotic event even
when the device is functional and surgery is successful (cit: ref 9 in
Colin's thesis--Cannegieter et. al., 1994). 

\subsection{What has been done in hemocompatibility research?}
\label{sec:what-has-been}
Most hemocompatibility studies focus on local interactions and effects
of the material on platelets. However recent work (cit: Dr. Hlady's
work) has shown that nonlocal effects are also important in
understanding platelet interactions with implanted biomaterials. In
particular they have shown that platelet interactions with immobilized
agonists may not cause the platelets to adhere at the site of
interaction, but can nonetheless prime them for downstream adhesion
and full activation. One possible example of this in a vascular graft
is shown in Figure #. At either end of the implanted device, native
tissue must be joined with the artificial device. At these points
along the vessel wall the tissue is inflamed and anastomotic
(source?), which could expose platelet agonists on the surface of the
wall. Additionally the increased shear rate in the stenotic regions
could act as a platelet agonist (source?). Therefore while platelets
may not bind to a single inflamed region of the vessel, the upstream
region may prime platelets for adhesion, and then more readily bind to
the downstream inflamed region. It is also possible that the inflamed
tissue or the biomaterial may not cause platelets to bind locally, but
nonetheless partially activate platelets and cause them to adhere
downstream somewhere in the circulation.

\subsection{The second subsection}

\subsection{The third subsection}

\subsubsection{The first subsubsection}

\subsubsection{The second subsubsection}

\paragraph{The first numbered paragraph}

\paragraph{The second numbered paragraph}

\section{The second section}

\section{Summary and conclusions}

%%% Index phrases should be attached to an important word of a phrase,
%%% and are usually best kept on a separate line by terminating the
%%% previous line with a percent comment without intervening space, as
%%% in this example:
%%%
%%%     \newcommand {\X} [1] {#1\index{#1}}
%%%
%%%     African ungulates,%
%%%     \index{African ungulate}
%%%     like the \X{gnu}, \X{impala}, \X{kudu}, and \X{springbok}
%%%     live mostly in hot climate and consume vegetation.
%%%
%%% However, for this document, we only want lots of index entries to
%%% populate a sample topic index.

%%% ====================================================================
%%% Cross-references for index entries should be specified only once:
