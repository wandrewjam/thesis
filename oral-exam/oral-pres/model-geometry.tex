\begin{tikzpicture}[scale=2]
  \newcommand{\dist}{0.5}
  \newcommand{\Length}{2.5}
  \newcommand{\slope}{0.5}
  \newcommand{\buff}{0.1}

  % Define a coordinate for the center of the platelet, and draw a
  % node there
  \coordinate (pltcenter) at (0, 1 + \dist);
  \node [circle, inner sep=1.5pt, fill=black] at (pltcenter) {};

  % Draws the wall and the platelet
  \draw (-\Length, 0) -- (\Length, 0);
  \draw (pltcenter) circle [radius = 1];
  \draw[<->] (0, .05) -- node[right] {$\separation$} (0, \dist-.05);
  \draw[<->] (-.15, .05) -- node[left] {$\height$} (-.15,
  .95+\dist);
  \draw[dashed] (0, 1+\dist) -- (-.3, 1+\dist);

  % Draws the lines showing the applied velocities
  \draw[->] (pltcenter) ++(1+\buff, 0) -- node[above] {$\velocity$}
  (\Length-\buff, 1+\dist);

  \draw[->] (0, 1+\dist) ++(135:1+\buff) arc [start angle=135, end
  angle=45, radius=1+\buff] node[midway, above] {$\rotation$};

  % Draws the arrows showing the shear flow
  \foreach \y in {0.5, 1, 1.5, 2, 2.5}
     \draw[->] (-\Length, \y) -- (-\Length + \slope*\y, \y);
  
  \draw[gray, very thin] (-\Length, 0) -- node[near start, right,
  black] {$\tn{Shear rate} = \shear$} (-\Length + \slope*\Length,
  \Length);

  % Draws a bond between the platelet and wall
  \draw[decorate, decoration=zigzag] (pltcenter) ++(315:1)
  node[circle, inner sep=1.5pt, fill=black] {} -- (1.25,0) node
  [circle, inner sep=1.5pt, fill=black] {};

  % Labels the x and theta coordinates
  \draw[{Bar[]}->] (0, -\buff) -- node[fill=white] {$\wallDist$}
  (1.25, -\buff);

  \draw (pltcenter) -- +(0, -1);
  \draw (pltcenter) -- node[above right] {$R$} +(315:1);
  \draw[->] (pltcenter) ++(0, -3*\buff) arc [start angle=270, end
  angle=315, radius=3*\buff] node[midway, below] {$\recAngle$};
\end{tikzpicture}
